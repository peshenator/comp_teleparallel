%%%%%%%%%%%%%%%%%%%%%%%%%%%%%%%%%%%%%%%%%
% Arsclassica Article
% LaTeX Template
% Version 1.1 (1/8/17)
%
% This template has been downloaded from:
% http://www.LaTeXTemplates.com
%
% Original author:
% Lorenzo Pantieri (http://www.lorenzopantieri.net) with extensive modifications by:
% Vel (vel@latextemplates.com)
%
% License:
% CC BY-NC-SA 3.0 (http://creativecommons.org/licenses/by-nc-sa/3.0/)
%
%%%%%%%%%%%%%%%%%%%%%%%%%%%%%%%%%%%%%%%%%

%----------------------------------------------------------------------------------------
%	PACKAGES AND OTHER DOCUMENT CONFIGURATIONS
%----------------------------------------------------------------------------------------

\documentclass[
10pt, % Main document font size
a4paper, % Paper type, use 'letterpaper' for US Letter paper
oneside, % One page layout (no page indentation)
%twoside, % Two page layout (page indentation for binding and different headers)
headinclude,footinclude, % Extra spacing for the header and footer
BCOR5mm, % Binding correction
]{scrartcl}

%%%%%%%%%%%%%%%%%%%%%%%%%%%%%%%%%%%%%%%%%
% Arsclassica Article
% Structure Specification File
%
% This file has been downloaded from:
% http://www.LaTeXTemplates.com
%
% Original author:
% Lorenzo Pantieri (http://www.lorenzopantieri.net) with extensive modifications by:
% Vel (vel@latextemplates.com)
%
% License:
% CC BY-NC-SA 3.0 (http://creativecommons.org/licenses/by-nc-sa/3.0/)
%
%%%%%%%%%%%%%%%%%%%%%%%%%%%%%%%%%%%%%%%%%

%----------------------------------------------------------------------------------------
%	REQUIRED PACKAGES
%----------------------------------------------------------------------------------------

\usepackage[
nochapters, % Turn off chapters since this is an article        
beramono, % Use the Bera Mono font for monospaced text (\texttt)
eulermath,% Use the Euler font for mathematics
pdfspacing, % Makes use of pdftex’ letter spacing capabilities via the microtype package
dottedtoc % Dotted lines leading to the page numbers in the table of contents
]{classicthesis} % The layout is based on the Classic Thesis style
\usepackage{hyperref}

\usepackage{arsclassica} % Modifies the Classic Thesis package

\usepackage[T1]{fontenc} % Use 8-bit encoding that has 256 glyphs

\usepackage[utf8]{inputenc} % Required for including letters with accents

\usepackage{graphicx} % Required for including images
\graphicspath{{Figures/}} % Set the default folder for images

\usepackage{enumitem} % Required for manipulating the whitespace between and within lists

\usepackage{lipsum} % Used for inserting dummy 'Lorem ipsum' text into the template

\usepackage{subfig} % Required for creating figures with multiple parts (subfigures)

\usepackage{amsmath,amssymb,amsthm} % For including math equations, theorems, symbols, etc

\usepackage{varioref} % More descriptive referencing

\usepackage{accents}


%----------------------------------------------------------------------------------------
%	THEOREM STYLES
%---------------------------------------------------------------------------------------

\theoremstyle{definition} % Define theorem styles here based on the definition style (used for definitions and examples)
\newtheorem{definition}{Definition}

\theoremstyle{plain} % Define theorem styles here based on the plain style (used for theorems, lemmas, propositions)
\newtheorem{theorem}{Theorem}

\theoremstyle{remark} % Define theorem styles here based on the remark style (used for remarks and notes)

%----------------------------------------------------------------------------------------
%	HYPERLINKS
%---------------------------------------------------------------------------------------

\hypersetup{
%draft, % Uncomment to remove all links (useful for printing in black and white)
colorlinks=true, breaklinks=true, bookmarks=true,bookmarksnumbered,
urlcolor=webbrown, linkcolor=RoyalBlue, citecolor=webgreen, % Link colors
pdftitle={}, % PDF title
pdfauthor={\textcopyright}, % PDF Author
pdfsubject={}, % PDF Subject
pdfkeywords={}, % PDF Keywords
pdfcreator={pdfLaTeX}, % PDF Creator
pdfproducer={LaTeX with hyperref and ClassicThesis} % PDF producer
}


%----------------------------------------------------------------------------------------
%	BIBLATEX
%---------------------------------------------------------------------------------------

\usepackage[backend=bibtex,giveninits=true,url=false,doi=true,eprint=true,isbn=false,
backref,backrefstyle=none,maxbibnames=99]{biblatex}
\DefineBibliographyStrings{english}{%
  backrefpage = {Cited on p\adddot},%
  backrefpages = {Cited on pp\adddot}%
}

\bibliography{library}

\renewcommand*{\bibfont}{\footnotesize}

% in order to suppress 'In:'
\renewbibmacro{in:}{%
  \ifboolexpr{%
     test {\ifentrytype{article}}%
  }{}{\printtext{\bibstring{in}\intitlepunct}}%
}

%----------------------------------------------------------------------------------------
% these commands allow to put equations in a fancy boxes:
%----------------------------------------------------------------------------------------
\usepackage{empheq}
\newlength\mytemplen
\newsavebox\mytempbox
\makeatletter
\definecolor{cream}{rgb}{.81, .88, 1}
 \newcommand\mycreambox{%
     \@ifnextchar[%]
        {\@mycreambox}%
        {\@mycreambox[0pt]}}
 \def\@mycreambox[#1]{%
     \@ifnextchar[%]
        {\@@mycreambox[#1]}%
        {\@@mycreambox[#1][0pt]}}
 \def\@@mycreambox[#1][#2]#3{
     \sbox\mytempbox{#3}%
     \mytemplen\ht\mytempbox
     \advance\mytemplen #1\relax
     \ht\mytempbox\mytemplen
     \mytemplen\dp\mytempbox
     \advance\mytemplen #2\relax
     \dp\mytempbox\mytemplen
     \colorbox{cream}{\hspace{1em}\usebox{\mytempbox}\hspace{1em}}}
 \makeatother % Include the structure.tex file which specified the document structure and 
%layout

\usepackage[hmarginratio=1:1,top=25mm,left=30mm,columnsep=25pt]{geometry}
\usepackage{hyperref}
\usepackage{relsize} % e.g. used for \mathsmaller
\usepackage{bm}


\newcommand{\xx}{\mathbf{x}}
\newcommand{\XX}{\mathbf{X}}
\newcommand{\XXX}{\mathbb{X}}
\newcommand{\diff}{\mathrm{d}}
\newcommand{\Id}{\mathbf{I}}
\newcommand{\Tr}{\mathrm{Tr}}
\newcommand{\CC}{\mathbf{C}}
\newcommand{\mm}{\mathbf{m}}
\newcommand{\vv}{\mathbf{v}}
\newcommand{\MM}{\mathbf{M}}
\newcommand{\OBig}{\mathcal{O}}
\newcommand{\FF}{\mathbf{F}}
\renewcommand{\AA}{\mathbf{A}}
\newcommand{\BB}{\mathbf{B}}
\newcommand{\qq}{\mathbf{q}}
\newcommand{\QQ}{\mathbf{Q}}
\newcommand{\LL}{\mathbf{L}}
\newcommand{\Lie}{\mathfrak{L}}

\newcommand{\MP}[1]{{\color{Green}MP:\ \ #1}}
\newcommand{\IP}[1]{{\color{Red}[IP:\ \ #1]}}
\newcommand{\MH}[1]{{\color{Red}MH:\ \ #1}}

\newcommand{\sA}{\mathsmaller A}
\newcommand{\sB}{\mathsmaller B}
\newcommand{\sC}{\mathsmaller C}
\newcommand{\sD}{\mathsmaller D}
\newcommand{\sM}{\mathsmaller M}
\newcommand{\sN}{\mathsmaller N}
\newcommand{\sL}{\mathsmaller L}
\newcommand{\sr}{\mathsmaller r}

\newcommand{\pd}[1]{\partial_{#1}}
\newcommand{\F}[2]{F^{\ #1}_{\mathsmaller#2}}
\newcommand{\hatF}[2]{\hat{F}^{\ #1}_{\mathsmaller#2}}
\newcommand{\A}[2]{A^{\mathsmaller#1}_{\ #2}}

\newcommand{\pdd}[1]{{\bm{\partial}_{#1}}}
\newcommand{\dx}[1]{{\bm{\mathrm{d}x}^{#1}}}

\newcommand{\bas}[1]{\bm{h}^{#1}}
\newcommand{\cobas}[1]{\bm{h}_{#1}}

\newcommand{\itetr}[2]{F^{#1}_{\phantom{#1}#2}}
\newcommand{\tetr}[2]{A^{#1}_{\phantom{#1}#2}}
\newcommand{\deth}{A}
\newcommand{\rtetr}[2]{h^{#1}_{\mathsmaller{(r)} #2}}
\newcommand{\spin}[2]{\omega^{#1}_{\phantom{#1}#2}}
\newcommand{\Lor}[2]{\Lambda^{#1'}_{\phantom{#1}#2}}
\newcommand{\iLor}[2]{\Lambda^{#1}_{\phantom{#1}#2'}}
%\newcommand{\D}[1]{\mathcal{D}_{#1}} % Fock-Ivanencko cov derivative
\newcommand{\D}[1]{\partial_{#1}} % Fock-Ivanencko cov derivative
\newcommand{\DW}[1]{\mathcal{D}_{#1}} % Fock-Ivanencko cov derivative
\newcommand{\Tors}[2]{T^{#1}_{\phantom{a}#2}}
\newcommand{\Supp}[2]{S_{#1}^{\phantom{a}#2}}	%supepotential
\newcommand{\Torsl}[1]{T_{#1}}
\newcommand{\ET}[2]{E^{#1}_{\phantom{#1}#2}}	%Torsion decomposition, analog of Electric field
\newcommand{\ETmix}[2]{E^{#1}_{#2}}	%Torsion decomposition, analog of Electric field
\newcommand{\eT}[2]{D_{#1}^{\phantom{#1}#2}}	%Torsion decomposition, analog of Electric field
\newcommand{\dT}[2]{\mathcal{D}_{#1}^{\phantom{#1}#2}}	%Torsion decomposition, analog of Electric 
%field
\newcommand{\BT}[2]{B^{#1#2}}	%Torsion decomposition, analog of Magnetic field
\newcommand{\BTmix}[2]{B^{#1}_{#2}}	%Torsion decomposition, analog of Electric field
\newcommand{\hT}[2]{H^{#1#2}}	%Torsion decomposition, analog of Magnetic field
\newcommand{\bT}[2]{\mathcal{B}^{#1#2}}	%Torsion decomposition, analog of Magnetic field
\newcommand{\W}[2]{\mathcal{W}^{#1}_{\phantom{#1}#2}}
\newcommand{\w}[2]{W^{#1}_{\phantom{#1}#2}}
\newcommand{\FI}{Fock-Ivanenko}
\newcommand{\We}{Weitzenb\"ock}
%\newcommand{\Lag}{\mathcal{L}}	% Lagrangian which depends on ordinary derivatives
\newcommand{\Lag}{\Lambda}	% Lagrangian which depends on ordinary derivatives
\newcommand{\Lagcov}{\pounds}% Lagrangian which depends on gauge covariant derivatives
\newcommand{\Laghodge}{L}% Lagrangian which depends on the Hodge dual of the torsion
\newcommand{\Lagtors}{\mathfrak{L}}% Lagrangian which depends on torsion
\newcommand{\LagBE}{\mathcal{L}}% Lagrangian which depends on the B and E fields
\newcommand{\LagST}{\mathcal{U}}% Final spacetime potetnial
\newcommand{\Lagtpo}{\mathcal{E}}% potential for 3+1
\newcommand{\veps}{\varepsilon}
\newcommand{\EM}[2]{\Sigma^{#1}_{\phantom{#1}#2}}
\newcommand{\LCsymb}{\bm{\in}}    % Levi-Civita symbol (tensor-density)
\newcommand{\LCtens}{\varepsilon} % Levi-Civita ordinary tensor

\newcommand{\tegr}{TEGR}
\newcommand{\HDT}[1]{\accentset{\star}{T}^{#1}}
\newcommand{\HDmix}{\accentset{\star}{T}}
\newcommand{\KD}[2]{\delta^{#1}_{\,\,#2}}
\newcommand{\NC}[2]{J^{\phantom{#1}#2}_{#1}}
\newcommand{\indlat}[1]{\hat{\mathsmaller{#1}}}

\newcommand{\TorsConj}[2]{\mathbb{T}_{#1}^{\phantom{#1}#2}}
\newcommand{\HTConj}[1]{\accentset{\star}{\mathbb{T}}_{#1}}
\newcommand{\Dbb}[2]{\mathbb{D}_{#1}^{\phantom{#1}#2}}
\newcommand{\Hbb}[2]{\mathbb{H}_{#1#2}}
\newcommand{\lapse}{\alpha}
\newcommand{\shift}[1]{\beta^{#1}}
\newcommand{\Tscal}{\mathbb{T}}		% torsion scalar


\hyphenation{Fortran hy-phen-ation} % Specify custom hyphenation points in 
%words with dashes where 
%you would like hyphenation to occur, or alternatively, don't put any dashes in a word to stop 
%hyphenation altogether


%----------------------------------------------------------------------------------------
%	TITLE AND AUTHOR(S)
%----------------------------------------------------------------------------------------

\title{\large\normalfont\spacedallcaps{On a first-order hyperbolic reduction of the pure tetrad 
teleparallel gravity}} % The article 
%title

%\subtitle{Subtitle} % Uncomment to display a subtitle

\author{\normalsize\textsc{Ilya Peshkov}$^1$ \& 
\normalsize\textsc{Evgeniy Romenski}$^{2,3}$
%\normalsize\textsc{Michael Dumbser}$^{1}$ \ldots
} % The article author(s) - author affiliations 
%need to be 
%specified in the 
%AUTHOR AFFILIATIONS block

\date{\small\today} % An optional date to appear under the author(s)

%----------------------------------------------------------------------------------------


\begin{document}

%----------------------------------------------------------------------------------------
%	HEADERS
%----------------------------------------------------------------------------------------

\renewcommand{\sectionmark}[1]{\markright{\spacedlowsmallcaps{#1}}} % The header for all pages 
%(oneside) or for even pages (twoside)
%\renewcommand{\subsectionmark}[1]{\markright{\thesubsection~#1}} % Uncomment when using the 
%%twoside option - this modifies the header on odd pages
\lehead{\mbox{\llap{\small\thepage\kern1em\color{halfgray} 
\vline}\color{halfgray}\hspace{0.5em}\rightmark\hfil}} % The header style

\pagestyle{scrheadings} % Enable the headers specified in this block

%----------------------------------------------------------------------------------------
%	TABLE OF CONTENTS & LISTS OF FIGURES AND TABLES
%----------------------------------------------------------------------------------------

\maketitle % Print the title/author/date block

\setcounter{tocdepth}{2} % Set the depth of the table of contents to show sections and subsections 
%only

\tableofcontents % Print the table of contents

% \listoffigures % Print the list of figures

% \listoftables % Print the list of tables

%----------------------------------------------------------------------------------------
%	ABSTRACT
%----------------------------------------------------------------------------------------

\section*{Abstract} % This section will not appear in the table of contents due to the star 
% (\section*)
We deduce a first-order reduction of the second-order $ f(T) $ teleparallel gravity field 
equations 
in the pure-tetrad formulation (no spin connection). The first-order reduction is then can be used 
in the 3+1 split of the governing equations and subsequently in the numerical simulations. 
%----------------------------------------------------------------------------------------
%	AUTHOR AFFILIATIONS
%----------------------------------------------------------------------------------------
\let\thefootnote\relax\footnotetext{* \textit{ilya.peshkov@unitn.it}}
\let\thefootnote\relax\footnotetext{\textsuperscript{1} \textit{University of Trento, Trento, 
Italy}}
\let\thefootnote\relax\footnotetext{\textsuperscript{2} \textit{Sobolev Institute of Mathematics, Novosibirsk, Russia}}
\let\thefootnote\relax\footnotetext{\textsuperscript{3} \textit{Novosibirsk State University, 
Novosibirsk, Russia}}
%----------------------------------------------------------------------------------------

%\newpage % Start the article content on the second page, remove this if you have a longer abstract 
%that goes onto the second page

% PARAGRAPH OPTIONS:
\setlength\parindent{10pt} % sets indent to zero
\setlength{\parskip}{5pt} % changes vertical space between paragraphs
% PARAGRAPH OPTIONS.

%----------------------------------------------------------------------------------------
%	INTRODUCTION
%----------------------------------------------------------------------------------------

\section{Introduction}

Why to study teleparallel gravity? Quickly recall the main arguments from Chapter\,18 of 
\cite{AldrovandiPereiraBook}, and then add that the same equations are, in fact, applicable to 
modeling of 
turbulence, micromorphic solids (acoustic metamaterials), dislocations 
\cite{PRD-Torsion2019}... 

\section{Definitions}

\subsection{Anholonomic tetrad field}

We use the following index convention. Greek letters $ \lambda,\mu,\nu,... =0,1,2,3
$ are used to index quantities related to the spacetime manifold, the Latin letters $ a,b,c,... 
=\hat{0},\hat{1},\hat{2},\hat{3}$ are used to index quantities related to the tangent Minkowski 
space.



Consider a spacetime manifold $ M $ equipped with a coordinate system $ x^\mu $. At each point of 
the spacetime there is a natural tangent $ T_{x}M $ space spanned by the frame, or tetrad, $ 
\dx{\mu} $ which is the standard coordinate basis. 
There is also the cotangent space $ T_x^*M $ spanned by the coframe $ \pdd{\mu} $.
The metric on $ M $ is a general Riemannian metric $ g_{\mu\nu} $. 
Recall that the frames $ \dx{\mu} $ and $ \pdd{\mu} $ are \emph{holonomic}.

In addition to $ T_{x}M $, we assume that at each point of $ M $, there is a soldered tangent space 
which is a Minkowski space spanned by tetrad $ \{ \bas{a} \}$ and equipped with the 
Minkowski metric $ g_{ab} $ with the signature $ 
(-,+,+,+) $. Similarly, there is the 
corresponding cotangent Minkowski space spanned by the co-frames $ \{ \cobas{a} \}$. It is assumed 
that the frames $ \bas{a} $ and $ \cobas{a} $ are non-holonomic in general.

The components of the non-holonomic frame $ \bas{a} $ in the holonomic coordinate basis $ \dx{\mu} 
$ are denoted by $ \itetr{\mu}{a} $, i.e. 
\begin{equation}
	\itetr{\mu}{a} \bas{a} = \dx{\mu}, \qquad \text{or} \qquad \bas{a} = \tetr{a}{\mu}\dx{\mu}
\end{equation}
with $ \tetr{a}{\mu} $ being the inverse of $ \itetr{\mu}{a} $, i.e.
\begin{equation}\label{eqn.inv.tetr}
	\tetr{a}{\mu} \itetr{\mu}{b} = \delta^a_{\ b},
	\qquad
	\tetr{a}{\mu} \itetr{\nu}{a} = \delta^\nu_{\ \mu}.
\end{equation}


\begin{equation}
	g_{\mu\nu} = g_{ab} \tetr{a}{\mu}\tetr{b}{\nu}
\end{equation}









\subsection{Torsion}

The torsion is introduced as
\begin{equation}\label{eqn.def.tors}
\Tors{a}{\mu\nu}:=\D{\mu}\tetr{a}{\nu} - \D{\nu}\tetr{a}{\mu} = 
\w{a}{\mu\nu} - \w{a}{\nu\mu},
\end{equation}
where $ \w{a}{\mu\nu} = \tetr{a}{\lambda}\w{\lambda}{\mu\nu}$ and 
\begin{equation}
\w{\lambda}{\mu\nu} := 
\itetr{\lambda}{a}\pd{\mu} \tetr{a}{\nu}
\end{equation}
is the \We\ connection \cite{AldrovandiPereiraBook,KleinertMultivalued}.


Note that while the spacetime derivatives commute
\begin{equation}\label{eqn.commut.D}
\D{\mu}(\D{\nu} v^\lambda) - \D{\nu}(\D{\mu} v^\lambda) = 0, 
\qquad 
\D{\mu}(\D{\nu} v^a) - \D{\nu}(\D{\mu} v^a) = 0,
\end{equation}
their tangent space counterparts $\D{a} = \itetr{\mu}{a}\D{\mu}$ do not
\begin{equation}
\D{b}(\D{c} v^a) - \D{c}(\D{b} v^a) = 
-\Tors{d}{b c}\D{d}v^a .
\end{equation}

%\subsection{Reference tetrad}
%
%We also define the \textit{reference} tetrad $ \rtetr{a}{\mu}$ as the one for which 
%the torsion 
%vanishes
%\begin{equation}
%\Tors{a}{\mu\nu}(\rtetr{a}{\mu},\spin{a}{\mu c}) = 0
%\end{equation}
%which means that this tetrad does not contain any gravity effect but only inertia.


\subsection{Levi-Civita tensor}

This \href{https://physics.stackexchange.com/questions/429434/lorentz-covariant-derivative-of-the-vielbein-determinant}{discussion} is very relevant.

We shall also need the Levi-Civita tensor
\begin{equation}\label{def.LeviCivita}
\LCtens^{\lambda\mu\nu\rho} = \deth^{-1} \LCsymb^{\lambda\mu\nu\rho}, 
\qquad 
\LCtens_{\lambda\mu\nu\rho} = 
\deth \, \LCsymb_{\lambda\mu\nu\rho},
\qquad
\LCtens_{0123} = \deth,
\end{equation}
where $ \deth = \det(\tetr{a}{\mu}) = \sqrt{-g}$, $ g = \det(g_{\mu\nu}) $, and $ 
\LCsymb^{{\lambda\mu\nu\rho}} $ is the totally antisymmetric Levi-Civita 
symbol~\cite{AldrovandiPereiraBook} with the reference $ \LCsymb_{0123} = 1 $, 
i.e. it is a \emph{tensor-density} of weight $ W=+1 $.
In particular, we remark that
\begin{equation}\label{eqn.diff.LeviCivita}
\D{\sigma}\LCtens^{\lambda\mu\nu\rho} = 
\pd{\sigma}(\deth^{-1}\LCsymb^{\lambda\mu\nu\rho}) = 
\LCsymb^{\lambda\mu\nu\rho}\pd{\sigma}\deth^{-1} = 
-\LCsymb^{\lambda\mu\nu\rho}\deth^{-1}\itetr{\eta}{a}\pd{\sigma}\tetr{a}{\eta} = 
-\LCtens^{\lambda\mu\nu\rho}\w{\eta}{\sigma\eta}. 
\end{equation}
Yet, for the Levi-Civita symbol, we have
\begin{equation}\label{eqn.diff.LCsymb}
\D{\sigma}\LCsymb^{\lambda\mu\nu\rho} = 0.
\end{equation}

%Similarly, we introduce the Levi-Civita tensor in the tangent Minkowski space
%\begin{equation}
%\LCtens^{abcd} =\frac{1}{ \sqrt{-\eta}}\LCsymb^{abcd} = \LCsymb^{abcd}, \qquad 
%\LCtens_{abcd} = 
%\sqrt{-\eta}\LCsymb_{abcd} = \LCsymb_{abcd}.
%\end{equation}
%
%It can be straightforwardly verified that the Levi-Civita tensors $ 
%\LCtens^{\lambda\mu\nu\rho} $ and 
%$ \LCtens^{abcd} $ are 
%related as
%\begin{equation}
%\LCtens^{abcd} = 
%\tetr{a}{\lambda}\tetr{b}{\mu}\tetr{c}{\nu}\tetr{d}{\rho}\LCtens^{\lambda\mu\nu\rho}.
%\end{equation}


\section{Variational formulation}

%\IP{Because it is not clear to me how to define derivatives $ \pd_a $ in the 
%tangent Minkowski 
%space it is then unclear how to do variation in the tangent space. Therefore, 
%let's try to do 
%variation in the spacetime 
%where the derivatives $ \pd_\mu $ are well defined...}

We consider a general Lagrangian (scalar-density) $ \Lag(\tetr{a}{\mu},\pd{\lambda}\tetr{a}{\nu}) $ 
of the teleparallel gravity which is a function of the frame field $ \tetr{a}{\mu} $ and its first 
gradients $ \pd{\lambda}\tetr{a}{\nu} $. For our porpoises, it is also convenient to treat $ 
\Lag $ as a function of a special combination of the gradients $ \pd{\lambda}\tetr{a}{\nu} $:
\begin{equation}\label{eqn.Lagrangians}
\Lag(\tetr{a}{\mu},\pd{\lambda}\tetr{a}{\nu}) = 
\Laghodge(\tetr{a}{\mu},\HDT{a\mu\nu}),
\end{equation}
where $ \HDT{a\mu\nu} $ is the Hodge dual to the 
torsion:
\begin{equation}\label{eqn.Hodge.def}
\HDT{a\mu\nu} := \frac{1}{2}\LCsymb^{\mu\nu\rho\sigma}\Tors{a}{\rho\sigma} = 
\LCsymb^{\mu\nu\rho\sigma}\D{\rho}\tetr{a}{\sigma}, \qquad \Tors{a}{\mu\nu} = 
-\frac{1}{2}\LCsymb_{\mu\nu\rho\sigma}\HDT{a\rho\sigma}.
\end{equation}
It is important to emphasize that we chose to define the Hodge dual be a 
\emph{tensor-density} of weight $ W=+1 $, that will be important later for 
the so-called integrability condition \eqref{integr.HT}.



The Euler-Lagrange equation for $ \Lag(\tetr{a}{\mu},\pd{\lambda}\tetr{a}{\nu}) $ is
\begin{equation}\label{eqn.EL}
\pd{\lambda}(\Lag_{\pd{\lambda}\tetr{a}{\mu}}) = \Lag_{\tetr{a}{\mu}},
\end{equation}
where $ \Lag_{\pd{\lambda}\tetr{a}{\mu}} = \frac{\partial 
\Lag}{\partial(\pd{\lambda}\tetr{a}{\mu})} $ and $ 
\Lag_{\tetr{a}{\mu}} = \frac{\partial \Lag}{\partial\tetr{a}{\mu}} $. Equations \eqref{eqn.EL} 
form a system of 16 second-order partial differential equations for 16 unknowns $ \tetr{a}{\mu} $.


\section{First-order extension}

From now on, we shall formally treat the frame field $ \tetr{a}{\mu} $ and its gradients $ 
\pd{\lambda}\tetr{a}{\mu} $ as independent variables and in what follows, we shall rewrite system 
of second-order PDEs \eqref{eqn.EL} as a larger system of first-order PDEs for the extended set of 
40 unknowns $ \{ \tetr{a}{\mu},\HDT{a\mu\nu}\} $.


In terms of the Lagrangian $ \Laghodge(\tetr{a}{\mu},\HDT{a\mu\nu}) $, using notations 
\eqref{eqn.Lagrangians} and definitions 
\eqref{eqn.Hodge.def}, we can instead
rewrite \eqref{eqn.EL} as
\begin{equation}\label{eqn.EM.Hodge}
\D{\nu}(\LCsymb^{\mu\nu\lambda\rho}\Laghodge_{\HDT{a\lambda\rho}}) 
=-\Laghodge_{\tetr{a}{\mu}}.
\end{equation}

The Euler-Lagrange equation \eqref{eqn.EM.Hodge} has to be supplemented by the integrability 
condition
\begin{equation}\label{integr.HT}
\D{\nu}\HDT{a\mu\nu} = 0,
\end{equation}
%\IP{this is WRONG, it can't be zero!!! :( because $ 
%\D{\nu}\LCtens^{\mu\nu\sigma\rho} \neq 0 $, see 
%\eqref{eqn.diff.LeviCivita}, and thus, we still NEED the integrability 
%condition. Most likely it is 
%only possible to get it in the tangent Minkowski space!}
which is a trivial consequence of the definition of the  Hodge dual \eqref{eqn.Hodge.def}, the 
commutativity of the standard spacetime derivative $ \D{\mu} $, and the 
identity \eqref{eqn.diff.LCsymb}.
We note that if the Hodge dual would be defined using the Levi-Civita tensor $ 
\LCtens^{\mu\nu\rho\sigma} $ instead of the Levi-Civita symbol, then one would 
have that $ \D{\mu}\HDT{a\mu\nu} \neq 0 $.

Another consequence of the commutative property of $ \pd{\mu} $ is the energy-momentum
conservation law
\begin{equation}\label{eqn.EM}
\D{\mu}\Laghodge_{\tetr{a}{\mu}} = 0,
\end{equation}
which also has to be supplemented by the integrability condition 
\begin{equation}\label{eqn.tetr}
\D{\mu}\tetr{a}{\nu} - \D{\nu}\tetr{a}{\mu} = \Tors{a}{\mu\nu}.
\end{equation}

Therefore, we have arrived at the following system of first-order PDEs 
\begin{subequations}\label{eqn.1st.order.TEGR0}
	\begin{align}	
	\D{\nu}(\LCsymb^{\mu\nu\lambda\rho}\Laghodge_{\HDT{a\lambda\rho}}) 
	&=-\Laghodge_{\tetr{a}{\mu}},\label{eqn.TEGR0.EL}\\[2mm]
	%		
	\D{\nu}\HDT{a\mu\nu} & = 0,\label{eqn.TEGR0.integr}\\[2mm]
	%		
	\D{\mu}\Laghodge_{\tetr{a}{\mu}} & = 0,\label{eqn.TEGR0.enermomen}\\[2mm]
	%		
	\D{\mu}\tetr{a}{\nu} - \D{\nu}\tetr{a}{\mu} &= \Tors{a}{\mu\nu},\label{eqn.TEGR0.tetrad}
	\end{align}
\end{subequations}
for the unknowns $ \{\tetr{a}{\mu},\HDT{a\mu\nu}\} $.

Energy-momentum conservation law \eqref{eqn.EM} can be rewritten in a pure spacetime form by adding 
to it 
$ 
0\equiv \Laghodge_{\tetr{a}{\mu}}\pd{\mu} \tetr{a}{\nu} - \Laghodge_{\tetr{a}{\mu}}\pd{\mu} 
\tetr{a}{\nu}  = \Laghodge_{\tetr{a}{\mu}}\pd{\mu} \tetr{a}{\nu} - \Laghodge_{\tetr{a}{\mu}} 
\tetr{a}{\lambda}\w{\lambda}{\mu\nu} $:
\begin{equation}\label{eqn.EM2}
	\pd{\mu}(\tetr{a}{\nu}L_{\tetr{a}{\mu}}) - \tetr{a}{\lambda}\Laghodge_{\tetr{a}{\mu}} 
	\w{\lambda}{\mu\nu} = 0,
\end{equation}
which, keeping in mind that $ \tetr{a}{\nu}L_{\tetr{a}{\mu}} $ is a tensor density of weight $ +1 
$, can be rewritten as a covariant divergence (for the \We\ connection)
\begin{equation}\label{eqn.EM.cov}
	\DW{\mu}(\tetr{a}{\nu}L_{\tetr{a}{\mu}}) = 0.
\end{equation}

Yet, another form of the energy-momentum conservation law can be obtained (see 
Appendix~\ref{app.sec.EM})
\begin{equation}\label{eqn.EM3}
	\pd{\mu}\left( 
		\tetr{a}{\nu} L_{\tetr{a}{\mu}} - 2 \HDT{a\lambda\mu}L_{\HDT{a\lambda\nu}} + 
		(\HDT{a\lambda\rho}L_{\HDT{a\lambda\rho}} - L) \delta^\mu_{\ \nu}
	\right) = 0,
\end{equation}
with 
\begin{equation}\label{eqn.EM4}
	 \EM{\mu}{\nu} :=
	\tetr{a}{\nu} L_{\tetr{a}{\mu}} - 2 \HDT{a\lambda\mu}L_{\HDT{a\lambda\nu}} + 
	(\HDT{a\lambda\rho}L_{\HDT{a\lambda\rho}} - L) \delta^\mu_{\ \nu}
\end{equation}
being the \emph{total} (gravity+matter) \emph{energy-momentum tensor density} 
(of weight $ +1 $).

Therefore, the first-order form of TEGR governing equations to be solved is
\begin{subequations}\label{eqn.1st.order.TEGR}
	\begin{empheq}[box={\mycreambox[2pt][2pt]}]{align}
		\D{\nu}(\LCsymb^{\mu\nu\lambda\rho}\Laghodge_{\HDT{a\lambda\rho}}) 
		&=-\Laghodge_{\tetr{a}{\mu}},\label{eqn.1st.order.EL}\\[2mm]
%		
		\D{\nu}\HDT{a\mu\nu} & = 0,\label{eqn.1st.order.integr}\\[2mm]
%		
			\pd{\mu}\left( 
		\tetr{a}{\nu} \Laghodge_{\tetr{a}{\mu}} - 2 \HDT{a\lambda\mu}\Laghodge_{\HDT{a\lambda\nu}} 
		+ 
		(\HDT{a\lambda\rho}\Laghodge_{\HDT{a\lambda\rho}} - \Laghodge) \delta^\mu_{\ \nu}
		\right) & = 0,\label{eqn.1st.order.enermomen}\\[2mm]
%		
		\D{\mu}\tetr{a}{\nu} - \D{\nu}\tetr{a}{\mu} &= \Tors{a}{\mu\nu}.\label{eqn.1st.order.tetrad}
	\end{empheq}
\end{subequations}



\section{Preparation to $ 3+1 $ split}


\subsection{Frame 4-velocity}

Let us introduce the frame 4-velocity as
\begin{equation}\label{eqn.4velocity}
u^\mu := \itetr{\mu}{\indlat{0}}, \qquad u_\mu = g_{\mu\nu}u^\nu,
\end{equation}
then, due to
\begin{equation}
u_\mu = g_{\mu\nu} u^\nu = \eta_{ab}\tetr{a}{\mu}\tetr{b}{\nu}\itetr{\nu}{\indlat{0}} = 
\eta_{a\indlat{0}}\tetr{a}{\mu} = -\tetr{\indlat{0}}{\mu},
\end{equation}\label{eqn.4velocity.cov}
the covariant components are
\begin{equation}
u_\mu = -\tetr{\indlat{0}}{\mu}.
\end{equation}


\subsection{Transformation of the torsion equations}\label{sec.transform.potential}


System \eqref{eqn.1st.order.TEGR} is not yet in a convenient form for the numerical treatment. It 
is necessary to perform a 3+1 split \cite{Alcubierre2008}. 


We then define (note that $ \ET{a}{\mu} $ is a tensor, while $ \BT{a}{\mu} $ is a tensor-density)
\begin{equation}
\ET{a}{\mu} := u^\nu \Tors{a}{\mu\nu}, \qquad  \BT{a}{\mu} := u_\nu\HDT{a\mu\nu}
\end{equation}


It is known that for any skew-symmetric tensor and a time-like vector $ u^\mu $ the following 
decompositions holds
\begin{subequations}\label{eqn.T.decompos}
\begin{align}
\HDT{a\mu\nu} &= u^\mu \BT{a}{\nu} - u^\nu \BT{a}{\mu} + 
\LCsymb^{\mu\nu\lambda\rho}u_\lambda 
\ET{a}{\rho},\\[2mm]
\Tors{a}{\mu\nu} &= u_\mu \ET{a}{\nu} - u_\nu \ET{a}{\mu} - 
\LCsymb_{\mu\nu\lambda\rho}u^\lambda 
\BT{a}{\rho},
\end{align}
\end{subequations}

We assume that the Lagrangian density can be \textit{equivalently} written 
\begin{equation}\label{eqn.Lagrangians2}
\Laghodge(\tetr{a}{\mu},\HDT{a\mu\nu}) = \Lagtors(\tetr{a}{\mu},\Tors{a}{\mu\nu}) = 
\LagBE(\tetr{a}{\mu},\BT{a}{\mu},\ET{a}{\nu}).
\end{equation}
It then can be shown that the derivatives of these different representations of the Lagrangian are related as
\begin{equation}
\Laghodge_{\HDT{a\mu\nu}}u^\nu = -\frac12\LagBE_{\BT{a}{\mu}}, 
\qquad 
\Lagtors_{\Tors{a}{\mu\nu}}u_\nu = -\frac12\LagBE_{\ET{a}{\mu}},
\end{equation}
and hence, the PDEs \eqref{eqn.1st.order.EL} and \eqref{eqn.1st.order.integr} 
can be written as (see Appendix~\eqref{app.sec.Deqn})
\begin{subequations}\label{eqn.tors.BE}
	\begin{align}
		\D{\nu}( u^\mu\LagBE_{\ET{a}{\nu}} - u^\nu \LagBE_{\ET{a}{\mu}} + 
		\LCsymb^{\mu\nu\lambda\rho}u_\lambda\LagBE_{\BT{a}{\rho}}) 
		&= \NC{a}{\mu}\label{eqn.tors.BE.a} \\[2mm]
%		
		\D{\nu}(u^\mu \BT{a}{\nu} - u^\nu\BT{a}{\mu} + 
		\LCsymb^{\mu\nu\lambda\rho}u_\lambda\ET{a}{\rho}) &= 0,
	\end{align}
\end{subequations}
where the source $ \NC{a}{\mu} = \Laghodge_{\tetr{a}{\mu}} $ yet to be developed.

Let us now introduce a new potential $ \LagST(\tetr{a}{\mu},\hT{a}{\mu},\eT{a}{\mu}) $ as a partial 
Legendre transform of the Lagrangian
\begin{equation}\label{eqn.Legandre1}
 \LagST(\tetr{a}{\mu},\hT{a}{\mu},\eT{a}{\mu}) := \ET{a}{\lambda}\LagBE_{\ET{a}{\lambda}} - \LagBE,
\end{equation}
and new state variables (note that both $ \eT{a}{\mu} $ and $ \hT{a}{\mu} $ are now 
\emph{tensor-densities})
\begin{equation}\label{eqn.Legandre2}
\eT{a}{\mu} = \LagBE_{\ET{a}{\mu}}, \qquad \hT{a}{\mu} = -\BT{a}{\mu}, \qquad \tetr{a}{\mu},
\end{equation}
such that we have
\begin{equation}\label{eqn.Legandre3}
\LagST_{\eT{a}{\mu}} = \ET{a}{\mu}, \qquad \LagST_{\hT{a}{\mu}} = \LagBE_{\BT{a}{\mu}},
\qquad \LagST_{\tetr{a}{\mu}} = - \LagBE_{\tetr{a}{\mu}}.
\end{equation}
This allows us to rewrite equations \eqref{eqn.1st.order.integr} and \eqref{eqn.1st.order.EL} in 
the form similar to the non-linear 
electrodynamics of moving medium~\cite{Obukhov2008,DPRZ2017,Hohmann2018a}
\begin{subequations}
	\begin{align}
		\D{\nu}(u^\mu\eT{a}{\nu} - u^\nu \eT{a}{\mu} + 
		\LCsymb^{\mu\nu\lambda\rho}u_\lambda 
		\LagST_{\hT{a}{\rho}})
		& =	\NC{a}{\mu},\\[2mm]
		\D{\nu}(u^\mu \hT{a}{\nu} - u^\nu \hT{a}{\mu} - 
		\LCsymb^{\mu\nu\lambda\rho}u_\lambda 
		\LagST_{\eT{a}{\rho}}) 
		& = 0,
\end{align}
\end{subequations}
with $\hT{a}{\mu}$ and $\eT{a}{\mu}$ being the analog of the magnetic and electric fields, 
accordingly.


Finally, we need to express also the Noether current $ \NC{a}{\mu} = \Laghodge_{\tetr{a}{\mu}} $ in 
terms of the 
new potential $ \LagST $ and the fields $ \eT{a}{\mu} $ and $ \hT{a}{\mu} $. One has (see details 
in Appendix~\ref{app.sec.NC})
\begin{equation}\label{eqn.J}
	\NC{a}{\mu} = \left\{
	\begin{array}{ll}
	-\LagST_{\tetr{\indlat{0}}{\mu}} 
	- (\hT{b}{\lambda} \LagST_{\hT{b}{\lambda}} 
	+ \eT{b}{\lambda} \LagST_{\eT{b}{\lambda}} )u^\mu
	+ \LCsymb^{\mu\alpha\beta\lambda} u_\alpha 
	\LagST_{\eT{b}{\beta}} \LagST_{\hT{b}{\lambda}},	& a=\hat{0},  \\[3mm] 
	-\LagST_{\tetr{a}{\mu}}	
	- \LCsymb_{\nu\alpha\beta\lambda}u^\alpha\eT{b}{\beta}\hT{b}{\lambda}\itetr{\nu}{a}u^\mu & a = 
	\hat{1},\hat{2},\hat{3}. \\ 
	\end{array} 
	\right.
\end{equation}

\subsection{Transformation of the tetrad equations}

%Let us introduce a pure spacetime energy-momentum tensor-density
%\begin{equation}\label{def.energymom.spacetime}
%\EM{\mu}{\nu} := \tetr{a}{\nu} \LagST_{\tetr{a}{\mu}} - \LagST \delta^\mu_{\ \nu}.
%\end{equation}
%
%
%\LagSTsing definition \eqref{def.energymom.spacetime}, the energy-momentum 
%conservation law \eqref{eqn.1st.order.enermomen} can be 
%rewritten in a pure spacetime form
%\begin{equation}
%\D{\mu} (\tetr{a}{\nu} \LagST_{\tetr{a}{\mu}} - \LagST \delta^\mu_{\ \nu}) - 
%\LagST_{\tetr{a}{\mu}} 
%\Tors{a}{\mu\nu} = 0.
%\end{equation}


Now, contracting \eqref{eqn.1st.order.tetrad} with the 4-velocity $ u^\mu $, and then after the 
change of variables and potentials described in Section\,\ref{sec.transform.potential}, the 
resulting equation reads as
\begin{equation}
	u^\mu(\D{\mu}\tetr{a}{\nu} - \D{\nu}\tetr{a}{\mu}) = \LagST_{\eT{a}{\nu}},
\end{equation}
Furthermore, using the identity $ \itetr{\mu}{b}\D{\nu}\tetr{a}{\nu} = - 
\tetr{a}{\nu}\D{\nu}\itetr{\mu}{b}$ and the definition $ u^\mu = \itetr{\mu}{\indlat{0}} 
$, the 
latter equation can be rewritten as
\begin{equation}
	u^\mu\D{\mu}\tetr{a}{\nu} + \tetr{a}{\mu}\D{\nu}u^\mu = \LagST_{\eT{a}{\nu}},
\end{equation}

Now, we express the energy-momentum $ \EM{\mu}{\nu} $ \eqref{eqn.EM4} in terms of new variables 
\eqref{eqn.Legandre2} and 
the potential $ \LagST(\tetr{a}{\mu},\hT{a}{\mu},\eT{a}{\mu}) $. It can be shown (see 
expression \eqref{app.eqn.Noether2} 
for $ \Laghodge_{\tetr{a}{\mu}} $ in 
Appendix~\ref{app.sec.NC}) that the tetrad part $ 
\tetr{a}{\nu}\Laghodge_{\tetr{a}{\mu}} $ expands as
\begin{align}
	\tetr{a}{\nu}\Laghodge_{\tetr{a}{\mu}} & =
    -\tetr{a}{\nu} \LagST_{\tetr{a}{\mu}} \nonumber\\
   & + u_\nu u^\mu(\hT{a}{\lambda} \LagST_{\hT{a}{\lambda}} + \eT{a}{\lambda} 
   \LagST_{\eT{a}{\lambda}}) 
   \nonumber\\
%   &- u^\mu u_\lambda \eT{a}{\lambda} \LagST_{\eT{a}{\nu}} - u_\nu u^\lambda 
%   \hT{a}{\mu}\LagST_{\hT{a}{\lambda}} \nonumber\\
   &-\LCsymb^{\mu\alpha\beta\lambda} u_\nu u_\alpha \LagST_{\eT{a}{\beta}} \LagST_{\hT{a}{\lambda}}
   -\LCsymb_{\nu\alpha\beta\lambda} u^\mu u^\alpha \eT{a}{\beta}\hT{a}{\lambda}
\end{align}

On the other hand, the torsion part $ - 2 \HDT{a\lambda\mu}L_{\HDT{a\lambda\nu}} + 
(\HDT{a\lambda\rho}L_{\HDT{a\lambda\rho}} - L) \delta^\mu_{\ \nu} $ reads
\begin{align}
	- 2 \HDT{a\lambda\mu}L_{\HDT{a\lambda\nu}} + 
	(\HDT{a\lambda\rho}L_{\HDT{a\lambda\rho}} - L) \delta^\mu_{\ \nu} & = -\LagST 
	\KD{\mu}{\nu}\nonumber\\
	& + \eT{a}{\mu}\LagST_{\eT{a}{\nu}}+ \hT{a}{\mu}\LagST_{\hT{a}{\nu}} \nonumber\\
	& - u_\nu u^\mu(\hT{a}{\lambda} \LagST_{\hT{a}{\lambda}} + \eT{a}{\lambda} 
	\LagST_{\eT{a}{\lambda}}) 
	\nonumber\\
	%   &- u^\mu u_\lambda \eT{a}{\lambda} \LagST_{\eT{a}{\nu}} - u_\nu u^\lambda 
	%   \hT{a}{\mu}\LagST_{\hT{a}{\lambda}} \nonumber\\
	&-\LCsymb^{\mu\alpha\beta\lambda} u_\nu u_\alpha \mathcal{\LagST}_{\eT{a}{\beta}} 
	\LagST_{\hT{a}{\lambda}}
	-\LCsymb_{\nu\alpha\beta\lambda} u^\mu u^\alpha \eT{a}{\beta}\hT{a}{\lambda}.
\end{align}

Therefore, the total energy-momentum  now reads
\begin{equation}
	\EM{\mu}{\nu} = 
	 \eT{a}{\mu}\LagST_{\eT{a}{\nu}}+ \hT{a}{\mu}\LagST_{\hT{a}{\nu}}
	 -\tetr{a}{\nu} \LagST_{\tetr{a}{\mu}} 
	- \LagST \KD{\mu}{\nu}.
\end{equation}

\subsection{Summary}

Finally, the resulting system of first-order governing equations for the 
unknowns $ \{\tetr{a}{\mu},\eT{a}{\mu},\hT{a}{\mu}\} $  reads
\begin{subequations}
	\begin{empheq}[box={\mycreambox[2pt][2pt]}]{align}
		\D{\nu}(u^\mu\eT{a}{\nu} - u^\nu \eT{a}{\mu} + 
		\LCsymb^{\mu\nu\lambda\rho}u_\lambda 
		\LagST_{\hT{a}{\rho}})
		& =	\NC{a}{\mu},\label{eT}\\[2mm]
%		
		\D{\nu}(u^\mu \hT{a}{\nu} - u^\nu \hT{a}{\mu} - 
		\LCsymb^{\mu\nu\lambda\rho}u_\lambda 
		\LagST_{\eT{a}{\rho}}) 
		& = 0,\label{hT}\\[2mm]
%		
			\pd{\mu}\left( \eT{a}{\mu}\LagST_{\eT{a}{\nu}}+ 
			\hT{a}{\mu}\LagST_{\hT{a}{\nu}}
			-\tetr{a}{\nu} \LagST_{\tetr{a}{\mu}} 
			- \LagST \KD{\mu}{\nu}
\right) & = 0,\\[2mm]
%		
		u^\mu\D{\mu}\tetr{a}{\nu} + \tetr{a}{\mu}\D{\nu}u^\mu & = \LagST_{\eT{a}{\nu}},
	\end{empheq}
\end{subequations}
with $ \NC{a}{\mu} $ given by \eqref{eqn.J}.


%\IP{Interestingly, the energy-momentum PDE is exactly like we use for the 
%symmetrization 
%in the non-relativistic settings, i.e. when we add the torsion multiplied by $ 
%\LagST_{\tetr{a}{\mu}} $ 
%to the momentum equation. However, when we usually du the summation, we don't 
%need this 
%non-conservative terms.... hm-m....}


%-------------------------------------------------------
\section{$ 3+1 $ split}	\label{sec.31}
%-------------------------------------------------------

The observer Eulerian 4-velocity $ n^\mu $ is associated with the $ \hat{0} $-th column of the 
inverse frame $ \itetr{\mu}{a} $, while the covariant components $ n_\mu $ of the 4-velocity with 
the $ 
\hat{0} $-th raw of the frame field:
\begin{subequations}
	\begin{align}
		n^{\mu} &= \itetr{\mu}{\indlat{0}}  = \lapse^{-1}(1,-\shift{i}) = \lapse^{-1}(1,v^i)\\
		n_\mu   &= - \tetr{\indlat{0}}{\mu} = (-\lapse,0,0,0) 
	\end{align}
%	\begin{equation}
%		\tetr{\hat{i}}{\mu} = 
%		(\shift{\hat{i}},\tetr{\hat{i}}{i})
%	\end{equation}
%	\begin{equation}
%		\itetr{\mu}{\hat{i}} = (0,\tetr{i}{\hat{i}})
%	\end{equation}
\end{subequations}
Here, $ \lapse $ is the \emph{lapse function}, and $ \shift{i} $ is the \emph{shift vector}.




%=========================================================
\subsection{$ 3+1 $ split of the torsion PDEs}	\label{ssec.31.tors}
%=========================================================

\subsubsection{Case $ \mu = i=1,2,3 $} 

Equations \eqref{eT} and \eqref{hT} read
\begin{subequations}\label{eqn.tpo.1}
	\begin{align}
		\pd{t} (\shift{i} \lapse^{-1} \eT{a}{0} + \lapse^{-1}\eT{a}{i}) + \pd{k}(\shift{i} 
		\lapse^{-1}\eT{a}{k} - \shift{k}\lapse^{-1}\eT{a}{i}  + \LCsymb^{ikj} \lapse \,
		\LagST_{\hT{a}{j}}) & 
		= -\NC{a}{i}, \\[2mm]
%
		\pd{t} (-\lapse^{-1}\hT{a}{i}) + \pd{k}(\shift{k} 
		\lapse^{-1}\hT{a}{i} - \shift{i}\lapse^{-1}\hT{a}{k}  + \LCsymb^{ikj} \lapse \,
		\LagST_{\eT{a}{j}}) & 
		= 0 .
	\end{align}
\end{subequations}


If we introduce the change of variables
\begin{equation}
	\dT{a}{\mu} := \lapse^{-1} \eT{a}{\mu}, \qquad \bT{a}{\mu} := -\lapse^{-1}\hT{a}{\mu}
\end{equation}
so that the derivatives of the potential $ \Lagtpo(\dT{a}{\mu},\bT{a}{\mu},\tetr{a}{\mu}) := 
\LagST(\eT{a}{\mu},\hT{a}{\mu},\tetr{a}{\mu})
$ 
transform as
\begin{equation}
	\Lagtpo_{\dT{a}{\mu}} =  \lapse \, \LagST_{\eT{a}{\mu}},
	\qquad
	\Lagtpo_{\bT{a}{\mu}} = -\lapse \, \LagST_{\hT{a}{\mu}}.
\end{equation}
Hence, \eqref{eqn.tpo.1} can be rewritten as
\begin{subequations}\label{eqn.tpo.2}
	\begin{align}
	\pd{t} (\dT{a}{i} + \shift{i} \dT{a}{0}) + \pd{k}(\shift{i} 
	\dT{a}{k} - \shift{k}\dT{a}{i}  - \LCsymb^{ikj} \,
	\Lagtpo_{\bT{a}{j}}) & 
	= -\NC{a}{i}, \\[2mm]
	%
	\pd{t} \bT{a}{i} + \pd{k}(\shift{i} 
	\bT{a}{k} - \shift{k}\bT{a}{i}  + \LCsymb^{ikj} 
	\Lagtpo_{\dT{a}{j}}) & 
	= 0 .
	\end{align}
\end{subequations}



\subsubsection{Case $ \mu = 0 $} 

The $ 0 $-th equations \eqref{eT} and \eqref{hT} are pure spacial constraints
\begin{equation}
	\pd{i} (\dT{a}{i} + \shift{i}\dT{a}{0}) = \NC{a}{0}, 
	\qquad
	\pd{i} \bT{a}{i} = 0.
\end{equation}





\section{Closure}



In the \tegr\ and its $ f(\Tscal) $-extensions the Lagrangian density is a function of the 
torsion scalar, e.g. in \tegr\ 
\begin{equation}
	\Lagtors(\tetr{a}{\mu},\Tors{a}{\mu\nu}) = \frac{\deth}{16\pi G} \Tscal,
\end{equation}
\begin{equation}
	\Tscal(\tetr{a}{\mu},\Tors{a}{\mu\nu}) := \frac14 g^{\beta\lambda} g^{\mu\gamma} g_{\alpha\eta} 
	\Tors{\alpha}{\lambda\gamma}
	\Tors{\eta}{\beta\mu} +
			  \frac12 g^{\mu\gamma} \Tors{\lambda}{\beta\mu} \Tors{\beta}{\lambda\gamma} - 
			  g^{\mu\lambda} \Tors{\rho}{\mu\rho} \Tors{\gamma}{\lambda\gamma},	  
\end{equation}
where $ \Tors{\lambda}{\mu\nu} = \itetr{\lambda}{a} \Tors{a}{\mu\nu} $, and the metric and it 
inverse must be computed as $ g_{\mu\nu} = 
\eta_{ab}\tetr{a}{\mu}\tetr{b}{\nu} $ and $ g^{\mu\nu} = \eta^{ab}\itetr{\mu}{a}\itetr{\nu}{b}$.

If one needs the Lagrangian density $ \Laghodge(\tetr{a}{\mu},\HDT{a\mu\nu}) $ as a function of the 
Hodge dual, then the torsion scalar $ \Tscal $ can be expressed as
\eqref{eqn.Lagrangians2}
\begin{equation}\label{eqn.Tscal.Hodge}
	\Tscal(\tetr{a}{\mu},\HDT{a\mu\nu}) = \frac12 \HDT{\mu\nu\lambda} (\HDmix_{\lambda\mu\nu} + 
	\HDmix_{\nu\lambda\mu} + g_{\mu\lambda} \HDmix^{\gamma}_{\phantom{\gamma\,} \nu\gamma})
\end{equation}


It is also possible to express the torsion scalar as $ 
\Tscal(\tetr{a}{\mu},\ET{a}{\mu},\BT{a}{\mu}) $:
\begin{multline}\label{eqn.Tscal.EB}
	\Tscal(\tetr{a}{\mu},\ET{a}{\mu},\BT{a}{\mu}) = 
	-\frac12 (\ETmix{\hat{0}\lambda}{}\ETmix{\hat{0}}{\ \,\lambda}   +
		      \ETmix{\ \,\lambda}{\beta} \ETmix{\beta}{\ \,\lambda}  +
			  \ETmix{\lambda}{\ \,\beta}\ETmix{\beta}{\ \,\lambda}   -
			  2\ETmix{\lambda}{\ \,\lambda}\ETmix{\beta}{\ \,\beta} )
			  \\
	+ \frac12 (  \BTmix{}{\hat{0}\lambda}\BTmix{\hat{0}\lambda}{}
	           - \BTmix{\ \,\lambda}{\lambda}\BTmix{\ \,\beta}{\beta}
	           + 2\BTmix{\lambda\beta}{}\BTmix{}{\beta\lambda}
	           )
	          \\
	          + \LCsymb_{\lambda\gamma\eta\rho} u^\eta 
	          (\ETmix{\lambda\gamma}{}\BTmix{\hat{0}\rho}{} + 2 
	          \ET{\hat{0}\lambda}{}\BTmix{\gamma\rho}{}).
\end{multline}


%-------------------------------------------------------
\appendix
%-------------------------------------------------------


%-------------------------------------------------------
\section{Spacetime form of the energy-momentum}\label{app.sec.EM}
%-------------------------------------------------------

Here, we show how the energy-momentum conservation law \eqref{eqn.EM} can be written in a pure 
spacetime form \eqref{eqn.EM3}.

Contracting $ \pd{\mu} \Laghodge_{\tetr{a}{\mu}} = 0 $ with $ \tetr{a}{\nu} $ and adding $ 0\equiv 
\Laghodge_{\tetr{a}{\mu}}\pd{\mu}\tetr{a}{\nu} -  \Laghodge_{\tetr{a}{\mu}}\pd{\mu}\tetr{a}{\nu} $ 
one gets
\begin{equation}
\pd{\mu}(\tetr{a}{\nu}\Laghodge_{\tetr{a}{\mu}}) - \Laghodge_{\tetr{a}{\mu}}\pd{\mu}\tetr{a}{\nu} = 
0.
\end{equation}
Replacing the last term with $ \pd{\mu}\tetr{a}{\nu} = \pd{\nu}\tetr{a}{\mu} + \Tors{a}{\mu\nu} $ 
yields
\begin{equation}
\pd{\mu}(\tetr{a}{\nu}\Laghodge_{\tetr{a}{\mu}}) - \Laghodge_{\tetr{a}{\mu}}\pd{\nu}\tetr{a}{\mu} - 
\Laghodge_{\tetr{a}{\mu}}\Tors{a}{\mu\nu} = 0.
\end{equation}
Then, using the fact that $ L = L(\tetr{a}{\mu},\HDT{a\mu\nu}) $, the second term can be 
substituted 
by $ \Laghodge_{\tetr{a}{\mu}}\pd{\nu}\tetr{a}{\mu} = \pd{\nu}\Laghodge - 
\Laghodge_{\HDT{b\lambda\rho}}\pd{\nu}\HDT{b\lambda\rho} $. This results in
\begin{equation}\label{app.eqn.EM1}
\pd{\mu}(\tetr{a}{\nu}\Laghodge_{\tetr{a}{\mu}} - L \delta^\mu_{\ \nu}) +
\Laghodge_{\HDT{b\lambda\rho}}\pd{\nu}\HDT{b\lambda\rho} -
\Laghodge_{\tetr{a}{\mu}}\Tors{a}{\mu\nu} = 0.
\end{equation} 
Now, the energy-momentum $ -\Laghodge_{\tetr{a}{\mu}} $ in the last term can be substituted by its 
expression from the Euler-Lagrange equation for the torsion \eqref{eqn.EM.Hodge}:
\begin{multline}\label{app.eqn.EM2}
	-\Laghodge_{\tetr{a}{\mu}}\Tors{a}{\mu\nu} = 
	\Tors{a}{\mu\nu}\pd{\lambda}(\LCsymb^{\mu\gamma\rho\lambda}\Laghodge_{\HDT{a\gamma\rho}}) =
	-\frac12\LCsymb_{\mu\nu\alpha\beta}\HDT{a\alpha\beta}(\LCsymb^{\mu\gamma\rho\lambda}\Laghodge_{\HDT{a\gamma\rho}})
	 =
	 \\
	2\HDT{a\alpha\beta}\pd{\beta}\Laghodge_{\HDT{a\nu\alpha}} + 
	\HDT{a\alpha\beta}\pd{\nu}\Laghodge_{\HDT{a\alpha\beta}},
\end{multline}
where we have used  $ 
-\LCsymb_{\mu\nu\alpha\beta}\LCsymb^{\mu\gamma\rho\lambda} = 
\KD{\gamma}{\nu}\KD{\rho}{\alpha}\KD{\lambda}{\beta} +
\KD{\lambda}{\nu}\KD{\gamma}{\alpha}\KD{\rho}{\beta} +
\KD{\rho}{\nu}\KD{\lambda}{\alpha}\KD{\gamma}{\beta} -
\KD{\gamma}{\nu}\KD{\lambda}{\alpha}\KD{\rho}{\beta} -
\KD{\lambda}{\nu}\KD{\rho}{\alpha}\KD{\gamma}{\beta} -
\KD{\rho}{\nu}\KD{\gamma}{\alpha}\KD{\lambda}{\beta}
$, e.g. see \cite{KleinertMultivalued}, p.45. Hence, using expression \eqref{app.eqn.EM2}, equation 
\eqref{app.eqn.EM1} can be written as
\begin{equation}
\pd{\mu}(\tetr{a}{\nu}\Laghodge_{\tetr{a}{\mu}} - L \delta^\mu_{\ \nu}) +
\Laghodge_{\HDT{b\lambda\rho}}\pd{\nu}\HDT{b\lambda\rho} +
2\HDT{a\alpha\beta}\pd{\beta}\Laghodge_{\HDT{a\nu\alpha}} + 
\HDT{a\alpha\beta}\pd{\nu}\Laghodge_{\HDT{a\alpha\beta}} = 0,
\end{equation} 
and then
\begin{equation}
\pd{\mu}\left(\tetr{a}{\nu}\Laghodge_{\tetr{a}{\mu}} + 
(\HDT{b\lambda\rho}\Laghodge_{\HDT{b\lambda\rho}}- 
L) \delta^\mu_{\ \nu} \right) +
2\HDT{a\alpha\mu}\pd{\mu}\Laghodge_{\HDT{a\nu\alpha}} = 0,
\end{equation} 
Finally, using the integrability condition \eqref{integr.HT}, this equation can be written in a 
fully conservative form
\begin{equation}
\pd{\mu}\left(\tetr{a}{\nu}\Laghodge_{\tetr{a}{\mu}} -
2\HDT{a\lambda\mu}\Laghodge_{\HDT{a\lambda\nu}}
+
(\HDT{b\lambda\rho}\Laghodge_{\HDT{b\lambda\rho}}- 
L) \delta^\mu_{\ \nu} \right) = 0.
\end{equation} 




\section{Transformation of the Noether's current $ \NC{a}{\mu} $}\label{app.sec.NC}

Here, we express the source terms $ \NC{a}{\mu} = \Laghodge_{\tetr{a}{\mu}} $ in terms of the 
potential $ \LagST $ and $ \eT{a}{\mu} $ and $ \hT{a}{\mu} $. One has
\begin{equation}\label{app.eqn.Noether1}
	\Laghodge_{\tetr{a}{\mu}} = \LagBE_{\tetr{a}{\mu}} 
	+ \LagBE_{\BT{b}{\lambda}} \frac{\partial \BT{b}{\lambda}}{\partial \tetr{a}{\mu}}
	+ \LagBE_{\ET{b}{\lambda}} \frac{\partial \ET{b}{\lambda}}{\partial \tetr{a}{\mu}}.
\end{equation}
Then, using the definitions of the frame 4-velocity $ u^\nu = \itetr{\nu}{\indlat{0}} $ and $ u_\nu 
= 
-\tetr{\indlat{0}}{\nu} $ and the torsion fields
$ \BT{b}{\lambda} = \HDT{b\lambda\nu} u_\nu = - \HDT{b\lambda\nu} \tetr{\indlat{0}}{\nu}$ and 
$ \ET{b}{\lambda} = \Tors{b}{\lambda\nu} u^\nu = 
-\frac12\LCsymb_{\lambda\nu\alpha\beta}\HDT{b\alpha\beta} \itetr{\nu}{\indlat{0}}$ and the simple 
fact 
that $ \partial\itetr{\lambda}{b}/\partial\tetr{a}{\mu} = -\itetr{\lambda}{a}\itetr{\mu}{b} $, we 
can rewrite \eqref{app.eqn.Noether1} as
\begin{multline}
	\NC{a}{\mu} = \LagBE_{\tetr{a}{\mu}} 
	+ \LagBE_{\BT{b}{\lambda}} \frac{\partial \BT{b}{\lambda}}{\partial \tetr{a}{\mu}}
	+ \LagBE_{\ET{b}{\lambda}} \frac{\partial \ET{b}{\lambda}}{\partial \tetr{a}{\mu}} = \\
	\LagBE_{\tetr{a}{\mu}} - \LagBE_{\BT{b}{\lambda}} \KD{\indlat{0}}{a}
	(u^\lambda \BT{b}{\mu} - u^\mu \BT{b}{\lambda}+\LCsymb^{\lambda\mu\alpha\beta} u_\alpha  
	\ET{b}{\beta}) \\
	-\LagBE_{\ET{b}{\lambda}} (u_\lambda \ET{b}{\nu} - u_\nu \ET{b}{\lambda} - 
	\LCsymb_{\lambda\nu\alpha\rho}u^\alpha\BT{b}{\rho})\itetr{\nu}{a}u^\mu.
\end{multline}
Using the definitions \eqref{eqn.Legandre2} and \eqref{eqn.Legandre3}, the latter can be rewritten 
as
\begin{multline}
	\NC{a}{\mu} =
	-\LagST_{\tetr{a}{\mu}} 
	- \LagST_{\hT{b}{\lambda}} \KD{\indlat{0}}{a}
	(-u^\lambda \hT{b}{\mu} + u^\mu \hT{b}{\lambda}+\LCsymb^{\lambda\mu\alpha\beta} u_\alpha 
	\LagST_{\eT{b}{\beta}}) \\
	-\eT{b}{\lambda} (u_\lambda \LagST_{\eT{b}{\nu}} - u_\nu \LagST_{\eT{b}{\lambda}} + 
	\LCsymb_{\lambda\nu\alpha\rho}u^\alpha\hT{b}{\rho})\itetr{\nu}{a}u^\mu.
\end{multline}
Note that the terms $ u^\lambda \LagST_{\hT{b}{\lambda}} $ and $ u_\lambda\eT{a}{\lambda} $ vanish 
because they are contractions of symmetric and anti-symmetric tensors: 
\begin{equation}
u^\lambda \LagST_{\hT{b}{\lambda}} = u^\lambda \LagBE_{\BT{b}{\lambda}} = -2 u^\lambda 
u^\gamma \Laghodge_{\HDT{b\lambda\gamma}} = 0, 
\qquad
u_\lambda\eT{b}{\lambda} = u_\lambda\LagBE_{\ET{b}{\lambda}} = -2 u_\lambda u_\gamma 
\Lagtors_{\Tors{b}{\lambda\gamma}} = 0.
\end{equation}
Hence, $ \NC{a}{\mu} $ becomes
\begin{multline}
\NC{a}{\mu} =
-\LagST_{\tetr{a}{\mu}} 
- \KD{\indlat{0}}{a}
(u^\mu \hT{b}{\lambda} \LagST_{\hT{b}{\lambda}} + \LCsymb^{\lambda\mu\alpha\beta} u_\alpha 
\LagST_{\eT{b}{\beta}} \LagST_{\hT{b}{\lambda}}) \\
+ ( u_\nu \eT{b}{\lambda} \LagST_{\eT{b}{\lambda}} - 
\LCsymb_{\nu\alpha\beta\lambda}u^\alpha\eT{b}{\beta}\hT{b}{\lambda})\itetr{\nu}{a}u^\mu,
\end{multline}
which, using the fact that $ \itetr{\nu}{a}u_\nu = \itetr{\nu}{a} \tetr{\indlat{0}}{\nu} = 
-\KD{\indlat{0}}{a} $, can be also rewritten as
\begin{multline}\label{app.eqn.Noether2}
\NC{a}{\mu} = 
-\LagST_{\tetr{a}{\mu}} 
- \KD{\indlat{0}}{a}
\left((\hT{b}{\lambda} \LagST_{\hT{b}{\lambda}} 
+ \eT{b}{\lambda} \LagST_{\eT{b}{\lambda}} )u^\mu
- \LCsymb^{\mu\alpha\beta\lambda} u_\alpha 
\LagST_{\eT{b}{\beta}} \LagST_{\hT{b}{\lambda}} \right) \\ 
- \LCsymb_{\nu\alpha\beta\lambda}u^\alpha\eT{b}{\beta}\hT{b}{\lambda}\itetr{\nu}{a}u^\mu.
\end{multline}
Finally, bearing in mind that, for $ a=\hat{0} $, the last term vanishes due to $ 
\LCsymb_{\nu\alpha\beta\lambda}u^\alpha\itetr{\nu}{a} =  
\LCsymb_{\nu\alpha\beta\lambda}u^\alpha u^\nu = 0$, the current $ \NC{a}{\mu} $ can be also written 
as
\begin{equation}
	\NC{a}{\mu} = \left\{
	\begin{array}{ll}
	-\LagST_{\tetr{\indlat{0}}{\mu}} 
	- (\hT{b}{\lambda} \LagST_{\hT{b}{\lambda}} 
	+ \eT{b}{\lambda} \LagST_{\eT{b}{\lambda}} )u^\mu
	+ \LCsymb^{\mu\alpha\beta\lambda} u_\alpha 
	\LagST_{\eT{b}{\beta}} \LagST_{\hT{b}{\lambda}},	& a=\hat{0},  \\[3mm] 
	-\LagST_{\tetr{a}{\mu}}	
	- \LCsymb_{\nu\alpha\beta\lambda}u^\alpha\eT{b}{\beta}\hT{b}{\lambda}\itetr{\nu}{a}u^\mu & a = 
	\hat{1},\hat{2},\hat{3}. \\ 
	\end{array} 
	\right.
\end{equation}

\section{Transformation of the torsion PDE}\label{app.sec.Deqn}

In this appendix, we demonstrate how the Euler-Lagrange equation \eqref{eqn.1st.order.EL} can be 
transformed to the form \eqref{eqn.tors.BE.a}.

First, let us introduce the notations
\begin{subequations}
	\begin{align}
			\TorsConj{a}{\mu\nu} := \Lagtors_{\Tors{a}{\mu\nu}},\hspace{1cm}
			&\HTConj{a\mu\nu} := \Laghodge_{\HDT{a\mu\nu}}, 
			\\[2mm]
			\Dbb{a}{\mu} := \TorsConj{a}{\mu\nu}u_\nu, \qquad
			&\Hbb{a}{\mu} := \HTConj{a\mu\nu}u^\nu,\label{app.eqn.DH}
	\end{align}
\end{subequations}
By a straightforward verification it then can be shown that
\begin{subequations}
	\begin{align}
	\TorsConj{a}{\mu\nu} &= u^\mu \Dbb{a}{\nu} - u^\nu \Dbb{a}{\mu} +
	\LCsymb^{\mu\nu\rho\sigma}u_\rho \Hbb{a}{\sigma},\\[2mm]
	\HTConj{a\mu\nu} &= u_\mu \Hbb{a}{\nu} - u_\nu \Hbb{a}{\mu} - 
	\LCsymb_{\mu\nu\rho\sigma}u^\rho \Dbb{a}{\sigma}\label{app.eqn.Deqn1},
	\end{align}
\end{subequations}
Hence, using \eqref{app.eqn.Deqn1}, Euler-Lagrange equation \eqref{eqn.1st.order.EL} can be written 
as
\begin{equation}
\D{\nu}(\LCsymb^{\mu\nu\alpha\beta}(u_\alpha \Hbb{a}{\beta} - u_\beta \Hbb{a}{\alpha} - 
\LCsymb_{\alpha\beta\rho\sigma}u^\rho \Dbb{a}{\sigma})) = -\NC{a}{\mu}.
\end{equation}
Then, using the fact that $ -\LCsymb_{\alpha\beta\rho\sigma}\LCsymb^{\mu\nu\alpha\beta} = 
-\LCsymb_{\alpha\beta\rho\sigma}\LCsymb^{\alpha\beta\mu\nu} = 2(\KD{\mu}{\rho}\KD{\nu}{\sigma} - 
\KD{\nu}{\rho}\KD{\mu}{\sigma})$, the latter equation can be re-written as
\begin{equation}\label{app.eqn.Deqn2}
	\D{\nu}(u^\mu \Dbb{a}{\nu} - u^\nu \Dbb{a}{\mu} + \LCsymb^{\mu\nu\alpha\beta}
	u_\alpha\Hbb{a}{\beta}) = -\frac12\NC{a}{\mu}.
\end{equation}
After substituting $ \Dbb{a}{\mu} $ and $ \Hbb{a}{\mu} $ by their definitions \eqref{app.eqn.DH}, 
equation \eqref{app.eqn.Deqn2} becomes
\begin{equation}
	\D{\nu}(u^\mu \Lagtors_{\Tors{a}{\nu\lambda}}u_\lambda - u^\nu \Lagtors_{\Tors{a}{\mu\lambda}} 
	u_\lambda + \LCsymb^{\mu\nu\alpha\beta}u_\alpha \Laghodge_{\HDT{a\beta\lambda}}u^\lambda) = 
	-\frac12 \NC{a}{\mu},
\end{equation}
which after using the following relations between the potentials $ 
\Laghodge(\tetr{a}{\mu},\HDT{a\mu\nu}) = 
\Lagtors(\tetr{a}{\mu},\Tors{a}{\mu\nu}) = 
\LagBE(\tetr{a}{\mu},\BT{a}{\mu},\ET{a}{\nu}) $ and their derivatives
\begin{equation}
	\Lagtors_{\Tors{a}{\mu\nu}} u_\nu = -\frac12\LagBE_{\ET{a}{\mu}},
	\qquad
	\Laghodge_{\HDT{a\mu\nu}} u^\nu= -\frac12\LagBE_{\BT{a}{\mu}},
\end{equation}
finally reads
\begin{equation}
	\D{\nu}( u^\mu\LagBE_{\ET{a}{\nu}} - u^\nu \LagBE_{\ET{a}{\mu}} + 
	\LCsymb^{\mu\nu\alpha
	\beta}u_\alpha\LagBE_{\BT{a}{\beta}}) 
	= \NC{a}{\mu}.
\end{equation}

\printbibliography

\end{document}








