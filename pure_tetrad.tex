%%%%%%%%%%%%%%%%%%%%%%%%%%%%%%%%%%%%%%%%%
% Arsclassica Article
% LaTeX Template
% Version 1.1 (1/8/17)
%
% This template has been downloaded from:
% http://www.LaTeXTemplates.com
%
% Original author:
% Lorenzo Pantieri (http://www.lorenzopantieri.net) with extensive modifications by:
% Vel (vel@latextemplates.com)
%
% License:
% CC BY-NC-SA 3.0 (http://creativecommons.org/licenses/by-nc-sa/3.0/)
%
%%%%%%%%%%%%%%%%%%%%%%%%%%%%%%%%%%%%%%%%%

%----------------------------------------------------------------------------------------
%	PACKAGES AND OTHER DOCUMENT CONFIGURATIONS
%----------------------------------------------------------------------------------------

\documentclass[
10pt, % Main document font size
a4paper, % Paper type, use 'letterpaper' for US Letter paper
oneside, % One page layout (no page indentation)
%twoside, % Two page layout (page indentation for binding and different headers)
headinclude,footinclude, % Extra spacing for the header and footer
BCOR5mm, % Binding correction
]{scrartcl}

%%%%%%%%%%%%%%%%%%%%%%%%%%%%%%%%%%%%%%%%%
% Arsclassica Article
% Structure Specification File
%
% This file has been downloaded from:
% http://www.LaTeXTemplates.com
%
% Original author:
% Lorenzo Pantieri (http://www.lorenzopantieri.net) with extensive modifications by:
% Vel (vel@latextemplates.com)
%
% License:
% CC BY-NC-SA 3.0 (http://creativecommons.org/licenses/by-nc-sa/3.0/)
%
%%%%%%%%%%%%%%%%%%%%%%%%%%%%%%%%%%%%%%%%%

%----------------------------------------------------------------------------------------
%	REQUIRED PACKAGES
%----------------------------------------------------------------------------------------

\usepackage[
nochapters, % Turn off chapters since this is an article        
beramono, % Use the Bera Mono font for monospaced text (\texttt)
%eulermath,% Use the Euler font for mathematics
pdfspacing, % Makes use of pdftex’ letter spacing capabilities via the microtype package
dottedtoc % Dotted lines leading to the page numbers in the table of contents
]{classicthesis} % The layout is based on the Classic Thesis style
%\usepackage{hyperref}

\usepackage{arsclassica} % Modifies the Classic Thesis package

\usepackage[T1]{fontenc} % Use 8-bit encoding that has 256 glyphs

\usepackage[utf8]{inputenc} % Required for including letters with accents

\usepackage{graphicx} % Required for including images
\graphicspath{{Figures/}} % Set the default folder for images

\usepackage{enumitem} % Required for manipulating the whitespace between and within lists

\usepackage{lipsum} % Used for inserting dummy 'Lorem ipsum' text into the template

\usepackage{subfig} % Required for creating figures with multiple parts (subfigures)

\usepackage{amsmath,amssymb,amsthm} % For including math equations, theorems, symbols, etc

\usepackage{varioref} % More descriptive referencing

\usepackage{accents}

\usepackage[title]{appendix}

\usepackage[symbol]{footmisc}

\usepackage{hyperref}

\usepackage{pgfplots}

%----------------------------------------------------------------------------------------
%	THEOREM STYLES
%---------------------------------------------------------------------------------------

\theoremstyle{definition} % Define theorem styles here based on the definition style (used for definitions and examples)
\newtheorem{definition}{Definition}

\theoremstyle{plain} % Define theorem styles here based on the plain style (used for theorems, lemmas, propositions)
\newtheorem{theorem}{Theorem}

\theoremstyle{remark} % Define theorem styles here based on the remark style (used for remarks and notes)


%----------------------------------------------------------------------------------------
%	BIBLATEX
%---------------------------------------------------------------------------------------

\usepackage[backend=biber,giveninits=true,url=false,doi=true,eprint=true,isbn=false,
backref,backrefstyle=none,sorting=nyt,maxbibnames=99]{biblatex}
\DefineBibliographyStrings{english}{%
  backrefpage = {Cited on p\adddot},%
  backrefpages = {Cited on pp\adddot}%
}

\bibliography{biblio}

\renewcommand*{\bibfont}{\footnotesize}

% in order to suppress 'In:'
\renewbibmacro{in:}{%
  \ifboolexpr{%
     test {\ifentrytype{article}}%
  }{}{\printtext{\bibstring{in}\intitlepunct}}%
}

%----------------------------------------------------------------------------------------
% these commands allow to put equations in a fancy boxes:
%----------------------------------------------------------------------------------------
\usepackage{empheq}
\newlength\mytemplen
\newsavebox\mytempbox
\makeatletter
\definecolor{cream}{rgb}{.81, .88, 1}
 \newcommand\mycreambox{%
     \@ifnextchar[%]
        {\@mycreambox}%
        {\@mycreambox[0pt]}}
 \def\@mycreambox[#1]{%
     \@ifnextchar[%]
        {\@@mycreambox[#1]}%
        {\@@mycreambox[#1][0pt]}}
 \def\@@mycreambox[#1][#2]#3{
     \sbox\mytempbox{#3}%
     \mytemplen\ht\mytempbox
     \advance\mytemplen #1\relax
     \ht\mytempbox\mytemplen
     \mytemplen\dp\mytempbox
     \advance\mytemplen #2\relax
     \dp\mytempbox\mytemplen
     \colorbox{cream}{\hspace{1em}\usebox{\mytempbox}\hspace{1em}}}
 \makeatother
 
 %----------------------------------------------------------------------------------------
 %	HYPERLINKS
 %---------------------------------------------------------------------------------------
 
 \hypersetup{
 	%draft, % Uncomment to remove all links (useful for printing in black and white)
 	colorlinks=true, breaklinks=true, bookmarks=true,bookmarksnumbered,
 	urlcolor=webbrown, linkcolor=RoyalBlue, citecolor=webgreen, % Link colors
 	pdftitle={}, % PDF title
 	pdfauthor={\textcopyright}, % PDF Author
 	pdfsubject={}, % PDF Subject
 	pdfkeywords={}, % PDF Keywords
 	pdfcreator={pdfLaTeX}, % PDF Creator
 	pdfproducer={LaTeX with hyperref and ClassicThesis} % PDF producer
 } % Include the structure.tex file which specified the document structure and 
%layout

\usepackage[hmarginratio=1:1,top=25mm,left=30mm,columnsep=25pt]{geometry}
\usepackage{relsize} % e.g. used for \mathsmaller
\usepackage{bm}
\usepackage{showlabels}


\newcommand{\xx}{\mathbf{x}}
\newcommand{\XX}{\mathbf{X}}
\newcommand{\XXX}{\mathbb{X}}
\newcommand{\diff}{\mathrm{d}}
\newcommand{\Id}{\mathbf{I}}
\newcommand{\Tr}{\mathrm{Tr}}
\newcommand{\CC}{\mathbf{C}}
\newcommand{\mm}{\mathbf{m}}
\newcommand{\vv}{\mathbf{v}}
\newcommand{\MM}{\mathbf{M}}
\newcommand{\OBig}{\mathcal{O}}
\newcommand{\FF}{\mathbf{F}}
\renewcommand{\AA}{\mathbf{A}}
\newcommand{\BB}{\mathbf{B}}
\newcommand{\qq}{\mathbf{q}}
\newcommand{\QQ}{\mathbf{Q}}
\newcommand{\LL}{\mathbf{L}}
\newcommand{\Lie}{\mathfrak{L}}

\newcommand{\MP}[1]{{\color{Green}MP:\ \ #1}}
\newcommand{\IP}[1]{{\color{Red}[IP:\ \ #1]}}
\newcommand{\MH}[1]{{\color{Red}MH:\ \ #1}}

\newcommand{\sA}{\mathsmaller A}
\newcommand{\sB}{\mathsmaller B}
\newcommand{\sC}{\mathsmaller C}
\newcommand{\sD}{\mathsmaller D}
\newcommand{\sM}{\mathsmaller M}
\newcommand{\sN}{\mathsmaller N}
\newcommand{\sL}{\mathsmaller L}
\newcommand{\sr}{\mathsmaller r}

\newcommand{\pd}[1]{\partial_{#1}}
\newcommand{\F}[2]{F^{\ #1}_{\mathsmaller#2}}
\newcommand{\hatF}[2]{\hat{F}^{\ #1}_{\mathsmaller#2}}
\newcommand{\A}[2]{A^{\mathsmaller#1}_{\ #2}}
\newcommand{\mg}[1]{\kappa_{#1}}			% Minkowski metric
\newcommand{\MG}[1]{\kappa^{#1}}			% inverse Minkowski metric

\newcommand{\pdd}[1]{{\bm{\partial}_{#1}}}
\newcommand{\dx}[1]{{\bm{\mathrm{d}x}^{#1}}}

\newcommand{\bas}[1]{\bm{h}^{#1}}
\newcommand{\cobas}[1]{\bm{h}_{#1}}

\newcommand{\tetrsymbol}{h}
\newcommand{\itetrsymbol}{\eta}
\newcommand{\itetr}[2]{\itetrsymbol^{#1}_{\phantom{#1}#2}}
\newcommand{\tetr}[2]{\tetrsymbol^{#1}_{\phantom{#1}#2}}
\newcommand{\detTetr}{\tetrsymbol}
\newcommand{\rtetr}[2]{h^{#1}_{\mathsmaller{(r)} #2}}
\newcommand{\spin}[2]{\omega^{#1}_{\phantom{#1}#2}}
\newcommand{\Lor}[2]{\Lambda^{#1'}_{\phantom{#1}#2}}
\newcommand{\iLor}[2]{\Lambda^{#1}_{\phantom{#1}#2'}}
%\newcommand{\D}[1]{\mathcal{D}_{#1}} % Fock-Ivanencko cov derivative
\newcommand{\D}[1]{\partial_{#1}} % Fock-Ivanencko cov derivative
\newcommand{\DW}[1]{\mathcal{D}_{#1}} % Fock-Ivanencko cov derivative
\newcommand{\Tors}[2]{T^{#1}_{\phantom{a}#2}}
\newcommand{\Supp}[2]{S_{#1}^{\phantom{a}#2}}	%supepotential
\newcommand{\Torsl}[1]{T_{#1}}
\newcommand{\ET}[2]{E^{#1}_{\phantom{#1}#2}}	%Torsion decomposition, analog of Electric field
\newcommand{\ETmix}[2]{E^{#1}_{#2}}	%Torsion decomposition, analog of Electric field
\newcommand{\dT}[2]{D_{#1}^{\phantom{#1}\,#2}}	%Torsion decomposition, analog of Electric field
\newcommand{\ddT}[2]{\mathcal{D}_{#1}^{\phantom{#1}\,#2}}	%Torsion decomposition, analog of 
\newcommand{\dddT}[2]{\mathtt{d}_{#1}^{\phantom{#1}\,#2}}	%Torsion decomposition, analog of 
%%%Electric 
%field
\newcommand{\BT}[2]{B^{#1#2}}	%Torsion decomposition, analog of Magnetic field
\newcommand{\BTmix}[2]{B^{#1}_{#2}}	%Torsion decomposition, analog of Electric field
\newcommand{\bT}[2]{B^{#1#2}}	%Torsion decomposition, analog of Magnetic field
\newcommand{\bbT}[2]{\mathcal{B}^{#1#2}}	%Torsion decomposition, analog of Magnetic field
\newcommand{\bbbT}[2]{\mathtt{b}^{#1#2}}	%Torsion decomposition, analog of Magnetic field
\newcommand{\bbbTmix}[2]{\mathtt{b}^{#1}_{#2}}	%Torsion decomposition, analog of Magnetic field
\newcommand{\bbTmix}[2]{\mathcal{B}^{#1}_{#2}}
\newcommand{\hT}[2]{H_{#1#2}}	%dual to B
\newcommand{\W}[2]{\mathcal{W}^{#1}_{\phantom{#1}#2}}
\newcommand{\w}[2]{W^{#1}_{\phantom{#1}#2}}
\newcommand{\FI}{Fock-Ivanenko}
\newcommand{\We}{Weitzenb\"ock}
%\newcommand{\Lag}{\mathcal{L}}	% Lagrangian which depends on ordinary derivatives
\newcommand{\Lag}{\Lambda}	% Lagrangian which depends on ordinary derivatives
\newcommand{\Lagcov}{\pounds}% Lagrangian which depends on gauge covariant derivatives
\newcommand{\Laghodge}{L}% Lagrangian which depends on the Hodge dual of the torsion
\newcommand{\Lagtors}{\mathfrak{L}}% Lagrangian which depends on torsion
\newcommand{\LagBE}{\mathcal{L}}% Lagrangian which depends on the B and E fields
\newcommand{\LagST}{\mathcal{U}}% Final spacetime potetnial
\newcommand{\Lagtpo}{\mathcal{E}}% potential for 3+1
\newcommand{\veps}{\varepsilon}
\newcommand{\EM}[2]{\Sigma^{#1}_{\phantom{#1}#2}}
\newcommand{\LCsymb}{\bm{\in}}    % Levi-Civita symbol (tensor-density)
\newcommand{\LCtens}{\varepsilon} % Levi-Civita ordinary tensor

\newcommand{\tegr}{TEGR}
\newcommand{\HDT}[1]{\accentset{\star}{T}^{#1}}
\newcommand{\HDmix}{\accentset{\star}{T}}
\newcommand{\KD}[2]{\delta^{#1}_{\,\,#2}}
\newcommand{\NC}[2]{J^{\phantom{#1}#2}_{#1}}
\newcommand{\indalg}[1]{\hat{\mathsmaller{#1}}}

\newcommand{\TorsConj}[2]{\mathbb{T}_{#1}^{\phantom{#1}#2}}
\newcommand{\HTConj}[1]{\accentset{\star}{\mathbb{T}}_{#1}}
\newcommand{\Dbb}[2]{\mathbb{D}_{#1}^{\phantom{#1}#2}}
\newcommand{\Hbb}[2]{\mathbb{H}_{#1#2}}
\newcommand{\lapse}{\alpha}
\newcommand{\shift}[1]{\beta^{#1}}
\newcommand{\Tscal}{\mathcal{T}}		% torsion scalar
\newcommand{\projector}[2]{\Delta^{#1}_{\ #2}}
\newcommand{\Hscal}{\mathcal{H}}		% torsion scalar


\newcommand{\ho}[1]{\textcolor{magenta}{HO: #1}}

\hyphenation{Fortran hy-phen-ation} % Specify custom hyphenation points in 
%words with dashes where 
%you would like hyphenation to occur, or alternatively, don't put any dashes in a word to stop 
%hyphenation altogether




%----------------------------------------------------------------------------------------
%	TITLE AND AUTHOR(S)
%----------------------------------------------------------------------------------------

\title{\large\normalfont\spacedallcaps{A first-order symmetric hyperbolic reduction of the pure 
tetrad 
teleparallel gravity}} % The article 
%title

%\subtitle{Subtitle} % Uncomment to display a subtitle

\author{\normalsize\textsc{Ilya Peshkov}$^1$ \& 
\normalsize\textsc{Evgeniy Romenski}$^{2,3}$
%\normalsize\textsc{Michael Dumbser}$^{1}$ \ldots
} % The article author(s) - author affiliations 
%need to be 
%specified in the 
%AUTHOR AFFILIATIONS block

\date{\small\today} % An optional date to appear under the author(s)

%----------------------------------------------------------------------------------------

\begin{document}

%----------------------------------------------------------------------------------------
%	HEADERS
%----------------------------------------------------------------------------------------

\renewcommand{\sectionmark}[1]{\markright{\spacedlowsmallcaps{#1}}} % The header for all pages 
%(oneside) or for even pages (twoside)
%\renewcommand{\subsectionmark}[1]{\markright{\thesubsection~#1}} % Uncomment when using the 
%%twoside option - this modifies the header on odd pages
\lehead{\mbox{\llap{\small\thepage\kern1em\color{halfgray} 
\vline}\color{halfgray}\hspace{0.5em}\rightmark\hfil}} % The header style

\pagestyle{scrheadings} % Enable the headers specified in this block

%----------------------------------------------------------------------------------------
%	TABLE OF CONTENTS & LISTS OF FIGURES AND TABLES
%----------------------------------------------------------------------------------------

\maketitle % Print the title/author/date block

\setcounter{tocdepth}{2} % Set the depth of the table of contents to show sections and subsections 
%only

\tableofcontents % Print the table of contents

% \listoffigures % Print the list of figures

% \listoftables % Print the list of tables

%----------------------------------------------------------------------------------------
%	ABSTRACT
%----------------------------------------------------------------------------------------

\section*{Abstract} % This section will not appear in the table of contents due to the star 
% (\section*)
Being motivated by solving the Einstein field equations numerically, we derive a first-order 
reduction of the second-order $ f(T) $ teleparallel gravity field 
equations 
in the pure-tetrad formulation (no spin connection). The first-order reduction is then can be used 
in the 3+1 split of the governing equations and subsequently in the numerical simulations. 
%----------------------------------------------------------------------------------------
%	AUTHOR AFFILIATIONS
%----------------------------------------------------------------------------------------
%\let\thee\relax\footnotetext{* \textit{ilya.peshkov@unitn.it}}
%\let\thefootnote\relax\footnotetext{\textsuperscript{1} \textit{University of Trento, Trento, 
%Italy}}
%\let\thefootnote\relax\footnotetext{\textsuperscript{2} \textit{Sobolev Institute of Mathematics, 
%Novosibirsk, Russia}}
%\let\thefootnote\relax\footnotetext{\textsuperscript{3} \textit{Novosibirsk State University, 
%Novosibirsk, Russia}}
\footnotetext{* \textit{ilya.peshkov@unitn.it}}
\footnotetext{\textsuperscript{1} \textit{University of Trento, Trento, 
		Italy}}
\footnotetext{\textsuperscript{2} \textit{Sobolev Institute of Mathematics, Novosibirsk, Russia}}
\footnotetext{\textsuperscript{3} \textit{Novosibirsk State University, 
		Novosibirsk, Russia}}
\renewcommand{\thefootnote}{\arabic{footnote}}
%----------------------------------------------------------------------------------------

%\newpage % Start the article content on the second page, remove this if you have a longer abstract 
%that goes onto the second page

% PARAGRAPH OPTIONS:
\setlength\parindent{10pt} % sets indent to zero
\setlength{\parskip}{5pt} % changes vertical space between paragraphs
% PARAGRAPH OPTIONS.

%----------------------------------------------------------------------------------------
%	INTRODUCTION
%----------------------------------------------------------------------------------------

\section{Introduction}

Why to study teleparallel gravity? Quickly recall the main arguments from Chapter\,18 of 
\cite{AldrovandiPereiraBook}, and then add that the same equations are, in fact, applicable to 
modeling of 
turbulence, micromorphic solids (acoustic metamaterials), dislocations 
\cite{Torsion2019}... 

\section{Definitions}

\subsection{Anholonomic tetrad field}

We use the following index convention. Greek letters $ \lambda,\mu,\nu,... =0,1,2,3
$ are used to index quantities related to the spacetime manifold, the Latin letters $ a,b,c,... 
=\hat{0},\hat{1},\hat{2},\hat{3}$ are used to index quantities related to the tangent Minkowski 
space.



Consider a spacetime manifold $ M $ equipped with a coordinate system $ x^\mu $. At each point of 
the spacetime there is a natural tangent $ T_{x}M $ space spanned by the frame, or tetrad, $ 
\dx{\mu} $ which is the standard coordinate basis. 
There is also the cotangent space $ T_x^*M $ spanned by the coframe $ \pdd{\mu} $.
The metric on $ M $ is a general Riemannian metric $ g_{\mu\nu} $. 
Recall that the frames $ \dx{\mu} $ and $ \pdd{\mu} $ are \emph{holonomic}.

\ho{I have always seen the tangent space $ T_{x}M $ spanned by the basis vectors
$ \pdd{\mu} $ and the cotangent space $ T_x^*M $ spanned by the basis 1-forms $ \dx{\mu} $.
However, I am not sure if it makes a difference, since I don't recall the definition of tangent
and co-tangent spaces. Also, in the literature, `tetrad' usually implies an orthonormal
basis, so it may cause some confusion to use it here for a coordinate frame.}

In addition to $ T_{x}M $, we assume that at each point of $ M $, there is a soldered tangent space 
which is a Minkowski space spanned by tetrad $ \{ \bas{a} \}$ and equipped with the 
Minkowski metric $ \mg{ab} $ with the signature $ 
(-,+,+,+) $. Similarly, there is the 
corresponding cotangent Minkowski space spanned by the co-frames $ \{ \cobas{a} \}$. It is assumed 
that the frames $ \bas{a} $ and $ \cobas{a} $ are non-holonomic in general.

The components of the non-holonomic frame $ \bas{a} $ in the holonomic coordinate basis $ \dx{\mu} 
$ are denoted by $ \itetr{\mu}{a} $, i.e. 
\begin{equation}
	\itetr{\mu}{a} \bas{a} = \dx{\mu}, \qquad \text{or} \qquad \bas{a} = \tetr{a}{\mu}\dx{\mu}
\end{equation}
with $ \tetr{a}{\mu} $ being the inverse of $ \itetr{\mu}{a} $, i.e.
\begin{equation}\label{eqn.inv.tetr}
	\tetr{a}{\mu} \itetr{\mu}{b} = \delta^a_{\ b},
	\qquad
	\tetr{a}{\mu} \itetr{\nu}{a} = \delta^\nu_{\ \mu}.
\end{equation}


\begin{equation}
	g_{\mu\nu} = \mg{ab} \tetr{a}{\mu}\tetr{b}{\nu}, \qquad \mg{ab} = \text{diag}(-1,1,1,1).
\end{equation}

\ho{Using the notation $\eta$ for the transformation coefficients can be
confusing, since in a lot of the literature $\eta_{\mu\nu}$ is used
for the Minkowski metric (e.g. Misner-Thorne-Wheeler and Aldrovandi \& Pereira).}





\subsection{Torsion}

The torsion is introduced as
\begin{equation}\label{eqn.def.tors}
\Tors{a}{\mu\nu}:=\D{\mu}\tetr{a}{\nu} - \D{\nu}\tetr{a}{\mu} = 
\w{a}{\mu\nu} - \w{a}{\nu\mu},
\end{equation}
where $ \w{a}{\mu\nu} = \tetr{a}{\lambda}\w{\lambda}{\mu\nu}$ and 
\begin{equation}
\w{\lambda}{\mu\nu} := 
\itetr{\lambda}{a}\pd{\mu} \tetr{a}{\nu}
\end{equation}
is the \We\ connection\footnote{Note that we use a different convention on the positioning of the 
lower indices of the \We\ connection $ \w{a}{\mu\nu} = \pd{\mu}\tetr{a}{\nu} $ than in 
\cite{AldrovandiPereiraBook}. Namely, the derivative index goes first.}
\cite{AldrovandiPereiraBook,KleinertMultivalued}.


Note that while the spacetime derivatives commute
\begin{equation}\label{eqn.commut.D}
\D{\mu}(\D{\nu} v^\lambda) - \D{\nu}(\D{\mu} v^\lambda) = 0, 
\qquad 
\D{\mu}(\D{\nu} v^a) - \D{\nu}(\D{\mu} v^a) = 0,
\end{equation}
their tangent space counterparts $\D{a} = \itetr{\mu}{a}\D{\mu}$ do not
\begin{equation}
\D{b}(\D{c} v^a) - \D{c}(\D{b} v^a) = 
-\Tors{d}{b c}\D{d}v^a .
\end{equation}

%\subsection{Reference tetrad}
%
%We also define the \textit{reference} tetrad $ \rtetr{a}{\mu}$ as the one for which 
%the torsion 
%vanishes
%\begin{equation}
%\Tors{a}{\mu\nu}(\rtetr{a}{\mu},\spin{a}{\mu c}) = 0
%\end{equation}
%which means that this tetrad does not contain any gravity effect but only inertia.


\subsection{Levi-Civita tensor}

%This 
%\href{https://physics.stackexchange.com/questions/429434/lorentz-covariant-derivative-of-the-vielbein-determinant}{discussion}
% is very relevant.

We shall also need the Levi-Civita symbol (tensor-density of weight\footnote{We 
	use the 
	sign convention for the tensor density weight according to \cite{Ryder2009,Grinfeld2013}, i.e.
	under a 
	general 
	coordinate change $ x^\mu \to x^{\mu'} $ the determinant $ \det(\tetr{a}{\mu}) = \detTetr $ 
	transforms as $ \detTetr' = \det \left(\frac{\partial x^\mu}{\partial x^{\mu'}} \right)^W 
	\cdot \detTetr $ with $ W=+1 $. Therefore, the tetrad's determinant $ \detTetr $ and the 
	square root of the metric determinant $ \detTetr = \sqrt{-g} $ have weights $ +1 $, as well as 
	the Lagrangian density in the action integral.} $ +1 $)
\begin{equation}\label{eqn.LCsymbol.def}
	\LCsymb^{\lambda\mu\nu\rho} = 
	\left\{ 
	\begin{array}{ll}
	 +1,	& \text{if \ }\lambda\mu\nu\rho \text{ is an even permutation of } 0123,\\[2mm]
     -1,	& \text{if \ }\lambda\mu\nu\rho \text{ is an odd \ permutation of } 0123,\\[2mm]
	  \phantom{-}0,	& \text{otehrwise}.
	\end{array}
	\right.
\end{equation}
Its covariant components $ \LCsymb_{\lambda\mu\nu\rho} $ define a tensor density of weight $ -1 $ 
with the reference value $ \LCsymb_{0123} = -1 $. One could define an absolute 
Levi-Civita 
contravariant $ \LCtens^{\lambda\mu\nu\rho} = h^{-1} \LCsymb^{\lambda\mu\nu\rho} $ 
and covariant $ \LCtens_{\lambda\mu\nu\rho} = h \LCsymb_{\lambda\mu\nu\rho} $ tensors  
but for our further considerations, it is important that 
(see Sec.\,\ref{sec.PDEs})
\begin{equation}\label{eqn.diff.LCsymb}
\D{\sigma}\LCsymb^{\lambda\mu\nu\rho} = 0,
\end{equation}
whereas for $ \LCtens^{\lambda\mu\nu\rho} $ one has
\begin{equation}\label{eqn.diff.LeviCivita}
\D{\sigma}\LCtens^{\lambda\mu\nu\rho} = 
\pd{\sigma}(\detTetr^{-1}\LCsymb^{\lambda\mu\nu\rho}) = 
\LCsymb^{\lambda\mu\nu\rho}\pd{\sigma}\detTetr^{-1} = 
-\LCsymb^{\lambda\mu\nu\rho}\detTetr^{-1}\itetr{\eta}{a}\pd{\sigma}\tetr{a}{\eta} = 
-\LCtens^{\lambda\mu\nu\rho}\w{\eta}{\sigma\eta}. 
\end{equation}



%Similarly, we introduce the Levi-Civita tensor in the tangent Minkowski space
%\begin{equation}
%\LCtens^{abcd} =\frac{1}{ \sqrt{-\eta}}\LCsymb^{abcd} = \LCsymb^{abcd}, \qquad 
%\LCtens_{abcd} = 
%\sqrt{-\eta}\LCsymb_{abcd} = \LCsymb_{abcd}.
%\end{equation}
%
%It can be straightforwardly verified that the Levi-Civita tensors $ 
%\LCtens^{\lambda\mu\nu\rho} $ and 
%$ \LCtens^{abcd} $ are 
%related as
%\begin{equation}
%\LCtens^{abcd} = 
%\tetr{a}{\lambda}\tetr{b}{\mu}\tetr{c}{\nu}\tetr{d}{\rho}\LCtens^{\lambda\mu\nu\rho}.
%\end{equation}


\section{Variational formulation}

%\IP{Because it is not clear to me how to define derivatives $ \pd_a $ in the 
%tangent Minkowski 
%space it is then unclear how to do variation in the tangent space. Therefore, 
%let's try to do 
%variation in the spacetime 
%where the derivatives $ \pd_\mu $ are well defined...}

We consider a general Lagrangian (scalar-density) $ \Lag(\tetr{a}{\mu},\pd{\lambda}\tetr{a}{\nu}) $ 
of the teleparallel gravity which is a function of the frame field $ \tetr{a}{\mu} $ and its first 
gradients $ \pd{\lambda}\tetr{a}{\nu} $. For our purposes, it is also convenient to treat $ 
\Lag $ as a function of a special combination of the gradients $ \pd{\lambda}\tetr{a}{\nu} $:
\begin{equation}\label{eqn.Lagrangians}
\Lag(\tetr{a}{\mu},\pd{\lambda}\tetr{a}{\nu}) = 
\Laghodge(\tetr{a}{\mu},\HDT{a\mu\nu}),
\end{equation}
where $ \HDT{a\mu\nu} $ is the Hodge dual to the 
torsion:
\begin{equation}\label{eqn.Hodge.def}
\HDT{a\mu\nu} := \frac{1}{2}\LCsymb^{\mu\nu\rho\sigma}\Tors{a}{\rho\sigma} = 
\LCsymb^{\mu\nu\rho\sigma}\D{\rho}\tetr{a}{\sigma}, \qquad \Tors{a}{\mu\nu} = 
-\frac{1}{2}\LCsymb_{\mu\nu\rho\sigma}\HDT{a\rho\sigma}.
\end{equation}
It is important to emphasize that we deliberately chose to define the Hodge dual using the 
Levi-Civita symbol $ 
\LCsymb^{\lambda\mu\nu\rho} $ and \emph{not} Levi-Civita tensor $ \LCtens^{\lambda\mu\nu\rho} = 
\detTetr^{-1} 
\LCsymb^{\lambda\mu\nu\rho} $ that will be important later for 
the so-called integrability condition \eqref{integr.HT}.
Remark that according to the definition \eqref{eqn.Hodge.def}, $ \HDT{a\mu\nu} $ is a  
\emph{tensor-density} of weight $ +1 $.



The Euler-Lagrange equation for $ \Lag(\tetr{a}{\mu},\pd{\lambda}\tetr{a}{\nu}) = \Lag(\tetr{a}{\mu},\w{a}{\lambda\nu}) $ is
\begin{equation}\label{eqn.EL}
\pd{\lambda}(\Lag_{\pd{\lambda}\tetr{a}{\mu}}) = \Lag_{\tetr{a}{\mu}},
\end{equation}
where $ \Lag_{\pd{\lambda}\tetr{a}{\mu}} = \frac{\partial 
\Lag}{\partial(\pd{\lambda}\tetr{a}{\mu})} $ and $ 
\Lag_{\tetr{a}{\mu}} = \frac{\partial \Lag}{\partial\tetr{a}{\mu}} $. Equations \eqref{eqn.EL} 
form a system of 16 second-order partial differential equations for 16 unknowns $ \tetr{a}{\mu} $.


\section{First-order extension}\label{sec.PDEs}

From now on, we shall formally treat the frame field $ \tetr{a}{\mu} $ and its gradients $ 
\pd{\lambda}\tetr{a}{\mu} $ as independent variables and in what follows, we shall rewrite system 
of second-order PDEs \eqref{eqn.EL} as a larger system of first-order PDEs for the extended set of 
40 unknowns $ \{ \tetr{a}{\mu},\HDT{a\mu\nu}\} $.


In terms of the Lagrangian $ \Laghodge(\tetr{a}{\mu},\HDT{a\mu\nu}) $, using notations 
\eqref{eqn.Lagrangians} and definitions 
\eqref{eqn.Hodge.def}, we can instead
rewrite \eqref{eqn.EL} as
\begin{equation}\label{eqn.EM.Hodge}
\D{\nu}(\LCsymb^{\mu\nu\lambda\rho}\Laghodge_{\HDT{a\lambda\rho}}) 
=-\Laghodge_{\tetr{a}{\mu}}.
\end{equation}

The Euler-Lagrange equation \eqref{eqn.EM.Hodge} has to be supplemented by the integrability 
condition
\begin{equation}\label{integr.HT}
\D{\nu}\HDT{a\mu\nu} = 0,
\end{equation}
%\IP{this is WRONG, it can't be zero!!! :( because $ 
%\D{\nu}\LCtens^{\mu\nu\sigma\rho} \neq 0 $, see 
%\eqref{eqn.diff.LeviCivita}, and thus, we still NEED the integrability 
%condition. Most likely it is 
%only possible to get it in the tangent Minkowski space!}
which is a trivial consequence of the definition of the  Hodge dual \eqref{eqn.Hodge.def}, the 
commutativity of the standard spacetime derivative $ \D{\mu} $, and the 
identity \eqref{eqn.diff.LCsymb}.
We note that if the Hodge dual would be defined using the Levi-Civita tensor $ 
\LCtens^{\mu\nu\rho\sigma} $ instead of the Levi-Civita symbol, then one would 
have that $ \D{\mu}\HDT{a\mu\nu} \neq 0 $.

Another consequence of the commutative property of $ \pd{\mu} $ and the definition of the Hodge 
dual (based on the Levi-Civita symbol) is the energy-momentum
conservation law
\begin{equation}\label{eqn.EM}
\D{\mu}\Laghodge_{\tetr{a}{\mu}} = 0,
\end{equation}

 \ho{It is not clear to me why $ \Laghodge_{\tetr{a}{\mu}} $ is the energy-momentum.
 Does it follow from Noether's theorem? Or does it coincide with
 a known expression? From eqs. 36 and 37,
 I get that it is not energy-momentum
 in the `traditional' sense, but a combination of gravitational plus matter
 energy-momentum. If so, which definition of gravitational
 energy-momentum is adopted (eg. Einstein or Landou-Lifshitz)?
 Or is it something completely different for TEGR?}

which also has to be supplemented by the integrability condition
\begin{equation}\label{eqn.tetr}
\D{\mu}\tetr{a}{\nu} - \D{\nu}\tetr{a}{\mu} = \Tors{a}{\mu\nu}.
\end{equation}

Therefore, we have arrived at the following system of first-order PDEs 
\begin{subequations}\label{eqn.1st.order.TEGR0}
	\begin{align}	
	\D{\nu}(\LCsymb^{\mu\nu\lambda\rho}\Laghodge_{\HDT{a\lambda\rho}}) 
	&=-\Laghodge_{\tetr{a}{\mu}},\label{eqn.TEGR0.EL}\\[2mm]
	%		
	\D{\nu}\HDT{a\mu\nu} & = 0,\label{eqn.TEGR0.integr}\\[2mm]
	%		
	\D{\mu}\Laghodge_{\tetr{a}{\mu}} & = 0,\label{eqn.TEGR0.enermomen}\\[2mm]
	%		
	\D{\mu}\tetr{a}{\nu} - \D{\nu}\tetr{a}{\mu} &= \Tors{a}{\mu\nu},\label{eqn.TEGR0.tetrad}
	\end{align}
\end{subequations}
for the unknowns $ \{\tetr{a}{\mu},\HDT{a\mu\nu}\} $.

Energy-momentum conservation law \eqref{eqn.EM} can be rewritten in a pure spacetime form by adding 
to it 
$ 
0\equiv \Laghodge_{\tetr{a}{\mu}}\pd{\mu} \tetr{a}{\nu} - \Laghodge_{\tetr{a}{\mu}}\pd{\mu} 
\tetr{a}{\nu}  = \Laghodge_{\tetr{a}{\mu}}\pd{\mu} \tetr{a}{\nu} - \Laghodge_{\tetr{a}{\mu}} 
\tetr{a}{\lambda}\w{\lambda}{\mu\nu} $:
\begin{equation}\label{eqn.EM2}
	\pd{\mu}(\tetr{a}{\nu}L_{\tetr{a}{\mu}}) - \tetr{a}{\lambda}\Laghodge_{\tetr{a}{\mu}} 
	\w{\lambda}{\mu\nu} = 0,
\end{equation}
which, keeping in mind that $ \tetr{a}{\nu}L_{\tetr{a}{\mu}} $ is a tensor density of weight $ +1 
$, can be rewritten as a covariant divergence (for the \We\ connection)
\begin{equation}\label{eqn.EM.cov}
	\DW{\mu}(\tetr{a}{\nu}L_{\tetr{a}{\mu}}) = 0.
\end{equation}

Yet, another form of the energy-momentum conservation law can be obtained (see 
Appendix~\ref{app.sec.EM})
\begin{equation}\label{eqn.EM3}
	\pd{\mu}\left( 
		\tetr{a}{\nu} L_{\tetr{a}{\mu}} - 2 \HDT{a\lambda\mu}L_{\HDT{a\lambda\nu}} + 
		(\HDT{a\lambda\rho}L_{\HDT{a\lambda\rho}} - L) \delta^\mu_{\ \nu}
	\right) = 0,
\end{equation}
with 
\begin{equation}\label{eqn.EM4}
	 \EM{\mu}{\nu} :=
	\tetr{a}{\nu} L_{\tetr{a}{\mu}} - 2 \HDT{a\lambda\mu}L_{\HDT{a\lambda\nu}} + 
	(\HDT{a\lambda\rho}L_{\HDT{a\lambda\rho}} - L) \delta^\mu_{\ \nu}
\end{equation}
being the \emph{total} (gravity+matter) \emph{energy-momentum tensor density} 
(of weight $ +1 $).

Therefore, the first-order form of the teleparallel gravity field equations to be written in the 3+1 form  
is
\begin{subequations}\label{eqn.1st.order.TEGR}
	\begin{empheq}[box={\mycreambox[2pt][2pt]}]{align}
		\D{\nu}(\LCsymb^{\mu\nu\lambda\rho}\Laghodge_{\HDT{a\lambda\rho}}) 
		&=-\Laghodge_{\tetr{a}{\mu}},\label{eqn.1st.order.EL}\\[2mm]
%		
		\D{\nu}\HDT{a\mu\nu} & = 0,\label{eqn.1st.order.integr}\\[2mm]
%		
			\pd{\mu}\left( 
		\tetr{a}{\nu} \Laghodge_{\tetr{a}{\mu}} - 2 \HDT{a\lambda\mu}\Laghodge_{\HDT{a\lambda\nu}} 
		+ 
		(\HDT{a\lambda\rho}\Laghodge_{\HDT{a\lambda\rho}} - \Laghodge) \delta^\mu_{\ \nu}
		\right) & = 0,\label{eqn.1st.order.enermomen}\\[2mm]
%		
		\D{\mu}\tetr{a}{\nu} - \D{\nu}\tetr{a}{\mu} &= \Tors{a}{\mu\nu}.\label{eqn.1st.order.tetrad}
	\end{empheq}
\end{subequations}



\section{Preparation to $ 3+1 $ split}


\subsection{Frame 4-velocity}

Let us introduce the observer 4-velocity as
\begin{equation}\label{eqn.4v}
u^\mu := \itetr{\mu}{\indalg{0}}, \qquad u_\mu = g_{\mu\nu}u^\nu,
\end{equation}
then, due to
\begin{equation}
u_\mu = g_{\mu\nu} u^\nu = \mg{ab}\tetr{a}{\mu}\tetr{b}{\nu}\itetr{\nu}{\indalg{0}} = 
\mg{a\indalg{0}}\tetr{a}{\mu} = -\tetr{\indalg{0}}{\mu},
\end{equation}\label{eqn.4v.cov}
the covariant components are
\begin{equation}
u_\mu = -\tetr{\indalg{0}}{\mu}.
\end{equation}

Also, we may need these velocities expressed in the frames themselves:
\begin{subequations}\label{eqn.4v.Lagr}
	\begin{gather}\label{eqn.4v.a}
		u^a := u^\mu \tetr{a}{\mu} = \itetr{\mu}{\indalg{0}}\tetr{a}{\mu} = \KD{a}{\indalg{0}} = 
		(1,0,0,0),
		\\[2mm]
		u_a := u_\mu \itetr{\mu}{a} =-\tetr{\indalg{0}}{\mu}\itetr{\mu}{a} =-\KD{a}{\indalg{0}} = 
		(-1,0,0,0).\label{eqn.4v.b}
	\end{gather}
\end{subequations}


\subsection{Transformation of the torsion equations}\label{sec.transform.potential}


System \eqref{eqn.1st.order.TEGR} is not yet in a convenient form for the numerical treatment. It 
is necessary to perform a 3+1 split \cite{Alcubierre2008}. 


We then define (note that $ \ET{a}{\mu} $ is a tensor, while $ \BT{a}{\mu} $ is a tensor-density)
\begin{equation}
\ET{a}{\mu} := \Tors{a}{\mu\nu} u^\nu, \qquad  \BT{a}{\mu} := \HDT{a\mu\nu} u_\nu
\end{equation}


It is known that for any skew-symmetric tensor and a time-like vector $ u^\mu $ the following 
decompositions holds
\begin{subequations}\label{eqn.T.decompos}
\begin{align}
\HDT{a\mu\nu} &= u^\mu \BT{a}{\nu} - u^\nu \BT{a}{\mu} + 
\LCsymb^{\mu\nu\lambda\rho}u_\lambda 
\ET{a}{\rho},\\[2mm]
\Tors{a}{\mu\nu} &= u_\mu \ET{a}{\nu} - u_\nu \ET{a}{\mu} - 
\LCsymb_{\mu\nu\lambda\rho}u^\lambda 
\BT{a}{\rho},
\end{align}
\end{subequations}

We assume that the Lagrangian density can be \textit{equivalently} written 
\begin{equation}\label{eqn.Lagrangians2}
\Laghodge(\tetr{a}{\mu},\HDT{a\mu\nu}) = \Lagtors(\tetr{a}{\mu},\Tors{a}{\mu\nu}) = 
\LagBE(\tetr{a}{\mu},\BT{a}{\mu},\ET{a}{\nu}).
\end{equation}
It then can be shown that the derivatives of these different representations of the Lagrangian are related as
\begin{equation}
\Laghodge_{\HDT{a\mu\nu}}u^\nu = -\frac12\left( \LagBE_{\BT{a}{\mu}} +u_\mu 
\LagBE_{\BT{a}{\lambda}} u^\lambda \right), 
\qquad 
\Lagtors_{\Tors{a}{\mu\nu}}u_\nu = -\frac12\left( \LagBE_{\ET{a}{\mu}} + u^\mu 
\LagBE_{\ET{a}{\lambda}} u_\lambda \right),
\end{equation}
and hence, the PDEs \eqref{eqn.1st.order.EL} and \eqref{eqn.1st.order.integr} 
can be written as (see Appendix~\eqref{app.sec.Deqn})
\begin{subequations}\label{eqn.tors.BE}
	\begin{align}
		\D{\nu}( u^\mu\LagBE_{\ET{a}{\nu}} - u^\nu \LagBE_{\ET{a}{\mu}} + 
		\LCsymb^{\mu\nu\lambda\rho}u_\lambda\LagBE_{\BT{a}{\rho}}) 
		&= \NC{a}{\mu}\label{eqn.tors.BE.a} \\[2mm]
%		
		\D{\nu}(u^\mu \BT{a}{\nu} - u^\nu\BT{a}{\mu} + 
		\LCsymb^{\mu\nu\lambda\rho}u_\lambda\ET{a}{\rho}) &= 0,
	\end{align}
\end{subequations}
where the source $ \NC{a}{\mu} = \Laghodge_{\tetr{a}{\mu}} $ yet to be developed.

Let us now introduce a new potential $ \LagST(\tetr{a}{\mu},\bT{a}{\mu},\dT{a}{\mu}) $ as a partial 
Legendre transform of the Lagrangian
\begin{equation}\label{eqn.Legandre1}
 \LagST(\tetr{a}{\mu},\bT{a}{\mu},\dT{a}{\mu}) := \ET{a}{\lambda}\LagBE_{\ET{a}{\lambda}} - \LagBE,
\end{equation}
and new state variables (note that both $ \dT{a}{\mu} $ and $ \bT{a}{\mu} $ are now 
\emph{tensor-densities})
\begin{equation}\label{eqn.Legandre2}
\dT{a}{\mu} = \LagBE_{\ET{a}{\mu}}, \qquad \bT{a}{\mu} = -\BT{a}{\mu}, \qquad 
\tetr{a}{\mu},
\end{equation}
({\color{blue}we use the same letter `$ B $' for the variable with the minus, it's not good})
such that we have
\begin{equation}\label{eqn.Legandre3}
\LagST_{\dT{a}{\mu}} = \ET{a}{\mu}, \qquad \LagST_{\bT{a}{\mu}} = \LagBE_{\BT{a}{\mu}},
\qquad \LagST_{\tetr{a}{\mu}} = - \LagBE_{\tetr{a}{\mu}}.
\end{equation}
This allows us to rewrite equations \eqref{eqn.1st.order.integr} and \eqref{eqn.1st.order.EL} in 
the form similar to the non-linear 
electrodynamics of moving medium~\cite{Obukhov2008,DPRZ2017,Hohmann2018a}
\begin{subequations}
	\begin{align}
		\D{\nu}(u^\mu\dT{a}{\nu} - u^\nu \dT{a}{\mu} + 
		\LCsymb^{\mu\nu\lambda\rho}u_\lambda 
		\LagST_{\bT{a}{\rho}})
		& =	\NC{a}{\mu},\\[2mm]
		\D{\nu}(u^\mu \bT{a}{\nu} - u^\nu \bT{a}{\mu} - 
		\LCsymb^{\mu\nu\lambda\rho}u_\lambda 
		\LagST_{\dT{a}{\rho}}) 
		& = 0,
\end{align}
\end{subequations}
with $\bT{a}{\mu}$ and $\dT{a}{\mu}$ being the analog of the magnetic and electric fields, 
accordingly.


Finally, we need to express also the Noether current $ \NC{a}{\mu} = \Laghodge_{\tetr{a}{\mu}} $ in 
terms of the 
new potential $ \LagST $ and the fields $ \dT{a}{\mu} $ and $ \bT{a}{\mu} $. One has (see details 
in Appendix~\ref{app.sec.NC})
\begin{equation}\label{eqn.J}
	\NC{a}{\mu} = \left\{
	\begin{array}{ll}
	-\LagST_{\tetr{\indalg{0}}{\mu}} 
	- (\bT{b}{\lambda} \LagST_{\bT{b}{\lambda}} 
	+ \dT{b}{\lambda} \LagST_{\dT{b}{\lambda}} )u^\mu
	+ \LCsymb^{\mu\alpha\beta\lambda} u_\alpha 
	\LagST_{\dT{b}{\beta}} \LagST_{\bT{b}{\lambda}},	& a=\hat{0},  \\[3mm] 
	-\LagST_{\tetr{a}{\mu}}	
	- \LCsymb_{\nu\alpha\beta\lambda}u^\alpha\dT{b}{\beta}\bT{b}{\lambda}\itetr{\nu}{a}u^\mu & a = 
	\hat{1},\hat{2},\hat{3}. \\ 
	\end{array} 
	\right.
\end{equation}

\subsection{Transformation of the tetrad equations}

%Let us introduce a pure spacetime energy-momentum tensor-density
%\begin{equation}\label{def.energymom.spacetime}
%\EM{\mu}{\nu} := \tetr{a}{\nu} \LagST_{\tetr{a}{\mu}} - \LagST \delta^\mu_{\ \nu}.
%\end{equation}
%
%
%\LagSTsing definition \eqref{def.energymom.spacetime}, the energy-momentum 
%conservation law \eqref{eqn.1st.order.enermomen} can be 
%rewritten in a pure spacetime form
%\begin{equation}
%\D{\mu} (\tetr{a}{\nu} \LagST_{\tetr{a}{\mu}} - \LagST \delta^\mu_{\ \nu}) - 
%\LagST_{\tetr{a}{\mu}} 
%\Tors{a}{\mu\nu} = 0.
%\end{equation}


Now, contracting \eqref{eqn.1st.order.tetrad} with the 4-velocity $ u^\mu $, and then after the 
change of variables and potentials described in Section\,\ref{sec.transform.potential}, the 
resulting equation reads as
\begin{equation}
	u^\mu(\D{\mu}\tetr{a}{\nu} - \D{\nu}\tetr{a}{\mu}) = \LagST_{\dT{a}{\nu}},
\end{equation}
Furthermore, using the identity $ \itetr{\mu}{b}\D{\nu}\tetr{a}{\nu} = - 
\tetr{a}{\nu}\D{\nu}\itetr{\mu}{b}$ and the definition $ u^\mu = \itetr{\mu}{\indalg{0}} 
$, the 
latter equation can be rewritten as
\begin{equation}
	u^\mu\D{\mu}\tetr{a}{\nu} + \tetr{a}{\mu}\D{\nu}u^\mu = \LagST_{\dT{a}{\nu}},
\end{equation}


\subsection{Transformation of the energy-momentum}
Now, we express the energy-momentum $ \EM{\mu}{\nu} $ \eqref{eqn.EM4} in terms of new variables 
\eqref{eqn.Legandre2} and 
the potential $ \LagST(\tetr{a}{\mu},\bT{a}{\mu},\dT{a}{\mu}) $. It now reads
\begin{equation}\label{eqn.EM.BD}
	\EM{\mu}{\nu} = 
	- \tetr{a}{\nu} \LagST_{\tetr{a}{\mu}}
	+ \dT{a}{\mu}\LagST_{\dT{a}{\nu}} + \bT{a}{\mu}\LagST_{\bT{a}{\nu}}
	- (
	\dT{a}{\lambda}\LagST_{\dT{a}{\lambda}}+ \bT{a}{\lambda}\LagST_{\bT{a}{\lambda}}
	-\LagST
	) \KD{\mu}{\nu}.
\end{equation}

\subsection{Summary}

Finally, the resulting system of first-order governing equations for the 
unknowns $ \{\tetr{a}{\mu},\dT{a}{\mu},\bT{a}{\mu}\} $  reads
\begin{subequations}\label{eq.PDE.4D}
	\begin{empheq}[box={\mycreambox[2pt][2pt]}]{align}
		\D{\nu}(u^\mu\dT{a}{\nu} - u^\nu \dT{a}{\mu} + 
		\LCsymb^{\mu\nu\lambda\rho}u_\lambda 
		\LagST_{\bT{a}{\rho}})
		& =	\NC{a}{\mu},\label{eT}\\[2mm]
%		
		\D{\nu}(u^\mu \bT{a}{\nu} - u^\nu \bT{a}{\mu} - 
		\LCsymb^{\mu\nu\lambda\rho}u_\lambda 
		\LagST_{\dT{a}{\rho}}) 
		& = 0,\label{hT}\\[2mm]
%		
		\pd{\mu}\left( 
			  \tetr{a}{\nu} \LagST_{\tetr{a}{\mu}}
        	- \dT{a}{\mu}\LagST_{\dT{a}{\nu}} - \bT{a}{\mu}\LagST_{\bT{a}{\nu}}
        	+ (
        	\dT{a}{\lambda}\LagST_{\dT{a}{\lambda}}+ \bT{a}{\lambda}\LagST_{\bT{a}{\lambda}}
        	-\LagST
        	) \KD{\mu}{\nu}
        \right) & = 0,\\[2mm]
%		
		u^\mu\D{\mu}\tetr{a}{\nu} + \tetr{a}{\mu}\D{\nu}u^\mu & = \LagST_{\dT{a}{\nu}},
		\label{tetr}
	\end{empheq}
\end{subequations}
with $ \NC{a}{\mu} $ given by \eqref{eqn.J}.


%\IP{Interestingly, the energy-momentum PDE is exactly like we use for the 
%symmetrization 
%in the non-relativistic settings, i.e. when we add the torsion multiplied by $ 
%\LagST_{\tetr{a}{\mu}} $ 
%to the momentum equation. However, when we usually du the summation, we don't 
%need this 
%non-conservative terms.... hm-m....}


%-------------------------------------------------------
\section{$ 3+1 $ split}	\label{sec.31}
%-------------------------------------------------------

WE shall use Latin indices $ i,j,k=1,2,3 $ to denote the spatial components of the spacetime 
tensors, and Latin indices $ \sA,\sB,\sC = \indalg{1},\indalg{2},\indalg{3} $ to denote the spatial 
components in the tangent Minkowski space. Additionally, we use $ \indalg{0} $ for time components 
in the Minkowski space in order to distinguish them from the time components $ \mu,\nu,\ldots
=0 $ of the spacetime tensors.  The observer Eulerian 4-velocity 
$ 
n^\mu $ is associated with the $ \hat{0} $-th column of the 
inverse frame $ \itetr{\mu}{a} $, while the covariant components $ n_\mu $ of the 4-velocity with 
the $ 
\hat{0} $-th raw of the frame field:
\begin{subequations}\label{eq.4vel}
	\begin{align}
		n^{\mu} &= \itetr{\mu}{\indalg{0}}  = \lapse^{-1}(1,-\shift{i}) = \lapse^{-1}(1,v^i)\\
		n_\mu   &= - \tetr{\indalg{0}}{\mu} = (-\lapse,0,0,0) 
	\end{align}
%	\begin{equation}
%		\tetr{\hat{i}}{\mu} = 
%		(\shift{\hat{i}},\tetr{\hat{i}}{i})
%	\end{equation}
%	\begin{equation}
%		\itetr{\mu}{\hat{i}} = (0,\tetr{i}{\hat{i}})
%	\end{equation}
\end{subequations}
Here, $ \lapse $ is the \emph{lapse function}, and $ \shift{i} $ is the \emph{shift vector}. Also, 
$ \beta^{\sA} = \tetr{\sA}{i}\beta^i$.
\begin{equation}
	\bm{\tetrsymbol} = \left(
	\begin{array}{cccc}
		\alpha          & 0 & 0 & 0 \\[1mm]
		\beta^{\indalg{1}} &  &  &  \\
		\beta^{\indalg{2}} &  & \left (\itetr{i}{\sA}\right )^{-1} &  \\
		\beta^{\indalg{3}} &  &  & 
	\end{array}
	\right) ,
	\quad
	\bm{h}^{-1} = \bm{\itetrsymbol} = \left(
	\begin{array}{rlcl}
		1/\alpha          & 0 & 0 & 0 \\[1mm]
		-\beta^{1}/\alpha &  &  &  \\
		-\beta^{2}/\alpha &  & \itetr{i}{\sA} &  \\
		-\beta^{3}/\alpha &  &  & 
	\end{array}
	\right) ,
\end{equation}
The metric tensor and its inverse are (e.g. see \cite{Gourgoulhon2012a})
\begin{equation}
	g_{\mu\nu} = \left(
	\begin{array}{cc}
		-\alpha^2 + \beta_i\beta^i & \beta_i \\[1mm]
		\beta_i & \gamma_{ij}  \\
	\end{array}
	\right) ,
	\quad
	g^{\mu\nu} = \left(
	\begin{array}{cc}
		-1/\alpha^2       & \beta^i/\alpha^2 \\[1mm]
		\beta^{i}/\alpha^2 & \gamma^{ij} - \beta^i\beta^j/\alpha^2  \\
	\end{array}
	\right) ,
\end{equation}
where $ \gamma_{ij} = \kappa_{\sA\sB} \tetr{\sA}{i}\tetr{\sB}{j}$, $ 
\gamma^{ij} = \left( \gamma_{ij} \right)^{-1}  $, and $ \beta_i =  \gamma_{ij}\beta^j$ 
according to (5.43) in \cite{Gourgoulhon2012a}.

\begin{equation}\label{eq.det}
	\detTetr_4 := \det(\tetr{a}{\mu}) = \alpha \detTetr_3, \qquad \detTetr_3 := \det(\tetr{\sA}{i}).
\end{equation}

%=========================================================
\subsection{$ 3+1 $ split of the torsion PDEs}	\label{ssec.31.tors}
%=========================================================

\IP{check the sign when you do replacement of $ \LCsymb^{0ijk} $ by $ \LCsymb^{ijk} $. It should be 
$ \LCsymb^{0ijk} = \LCsymb^{ijk} $.}

\subsubsection{Case $ \mu = i=1,2,3 $} 

Equations \eqref{eT} and \eqref{hT} read
\begin{subequations}\label{eqn.tpo.1}
	\begin{align}
		\pd{t} \lapse^{-1}\dT{a}{i} + \pd{k}(\shift{i} 
		\lapse^{-1}\dT{a}{k} - \shift{k}\lapse^{-1}\dT{a}{i}  - \LCsymb^{ikj} \lapse \,
		\LagST_{\bT{a}{j}}) & 
		= -\NC{a}{i}, \\[2mm]
%
		\pd{t} (-\lapse^{-1}\bT{a}{i}) + \pd{k}(\shift{k} 
		\lapse^{-1}\bT{a}{i} - \shift{i}\lapse^{-1}\bT{a}{k}  + \LCsymb^{ikj} \lapse \,
		\LagST_{\dT{a}{j}}) & 
		= 0 .
	\end{align}
\end{subequations}
Because of the factor $ \alpha^{-1} $ everywhere in these equations, it is convenient to rescale 
the 
variables 
\begin{equation}\label{eq.varDB.final}
	\ddT{a}{\mu} := \lapse^{-1} \dT{a}{\mu}, \qquad \bbT{a}{\mu} := -\lapse^{-1}\bT{a}{\mu}
\end{equation}
so that the derivatives of the potential $ \Lagtpo(\tetr{a}{\mu},\ddT{a}{\mu},\bbT{a}{\mu}) := 
\LagST(\tetr{a}{\mu},\dT{a}{\mu},\bT{a}{\mu})
$ 
transform as
\begin{equation}
	\Lagtpo_{\ddT{a}{\mu}} =  \lapse \, \LagST_{\dT{a}{\mu}},
	\qquad
	\Lagtpo_{\bbT{a}{\mu}} = -\lapse \, \LagST_{\bT{a}{\mu}}.
\end{equation}
Hence, \eqref{eqn.tpo.1} can be rewritten as
\begin{subequations}\label{eqn.tpo.2}
	\begin{align}
	\pd{t} \ddT{a}{i} + \pd{k}(\shift{i} 
	\ddT{a}{k} - \shift{k}\ddT{a}{i}  - \LCsymb^{ikj} \,
	\Lagtpo_{\bbT{a}{j}}) & 
	= -\NC{a}{i}, \label{eqn.tpo.D}\\[2mm]
	%
	\pd{t} \bbT{a}{i} + \pd{k}(\shift{i} 
	\bbT{a}{k} - \shift{k}\bbT{a}{i}  + \LCsymb^{ikj} 
	\Lagtpo_{\ddT{a}{j}}) & 
	= 0\label{eqn.tpo.B} .
	\end{align}
\end{subequations}



\subsubsection{Case $ \mu = 0 $} 

The $ 0 $-th equations \eqref{eT} and \eqref{hT} are pure spatial (stationary) constraints
\begin{equation}\label{eq.div.constr}
	\pd{k} \ddT{a}{k} = \NC{a}{0}, 
	\qquad
	\pd{k} \bbT{a}{k} = 0.
\end{equation}


\subsection{Tetrad PDE}

It seems that the components $ \tetr{a}{0} $ are abandoned (they are not used in any equations). 
Therefore, we are only interested in the components $ \tetr{a}{k} $. Thus, using the definition of 
the 4-velocity \eqref{eq.4vel}, equation \eqref{tetr} can be written as
\begin{equation}\label{eq.tetr.3+1}
	\pd{t} \tetr{a}{k} - \beta^i \pd{i} \tetr{a}{k} - \tetr{a}{i} \pd{k} \beta^i = 
	\Lagtpo_{\ddT{a}{k}}.
\end{equation}

This equation can be also written in a slightly different form. After adding $ 0\equiv - \beta^i 
\pd{k}\tetr{a}{i} + \beta^i \pd{k}\tetr{a}{i} $ to the left hand-side of \eqref{eq.tetr.3+1}, one 
has  
\begin{equation}\label{eq.tetr.3+1.2}
	\pd{t} \tetr{a}{k} - \pd{k} (\beta^i \tetr{a}{i}) - \beta^i (\pd{i}\tetr{a}{k} - 
	\pd{k}\tetr{a}{i}) = 
	\Lagtpo_{\ddT{a}{k}}.
\end{equation}
Finally, using the definition of $ \BT{a}{\mu} $ and $ n_\mu $, we have that 
\begin{equation}
	\BT{a}{\mu} = \HDT{a\mu\nu} u_\nu = u_\nu \LCsymb^{\mu\nu\alpha\beta} 
	\pd{\alpha}\tetr{a}{\beta} = -
	u_\nu \LCsymb^{\nu\mu\alpha\beta} \pd{\alpha}\tetr{a}{\beta} =
	\alpha \LCsymb^{0\mu\alpha\beta} \pd{\alpha}\tetr{a}{\beta} 
\end{equation}
and hence (use that $ \LCsymb^{0ijk} =\LCsymb^{ijk} $)
\begin{equation}
	\BT{a}{k} =  \alpha \LCsymb^{kij}\pd{i}\tetr{a}{j},
\end{equation}
or
\begin{equation}
	\alpha^{-1}\LCsymb_{klj}v^l \BT{a}{j} = \beta^i (\pd{i}\tetr{a}{k} - \pd{k}\tetr{a}{i}).
\end{equation}
Therefore, \eqref{eq.tetr.3+1.2} can be also rewritten as 
\begin{equation}\label{eq.tetr.3+1.3}
	\pd{t} \tetr{a}{k} - \pd{k} (\beta^i \tetr{a}{i}) =  
	\Lagtpo_{\ddT{a}{k}} - \LCsymb_{klj} \beta^l \bbT{a}{j}.
\end{equation}
It is not clear yet which form, \eqref{eq.tetr.3+1}, \eqref{eq.tetr.3+1.2}, or 
\eqref{eq.tetr.3+1.3}, is more suitable for numerical solution.

\subsection{3+1 summary}


\subsubsection{Evolution equations}
\begin{subequations}\label{eq.3+1.summary}
	\begin{empheq}[box={\mycreambox[2pt][2pt]}]{align}
		\pd{t} \ddT{a}{i} + \pd{k}(\shift{i} 
		\ddT{a}{k} - \shift{k}\ddT{a}{i}  - \LCsymb^{ikj} \,
		\Lagtpo_{\bbT{a}{j}}) 
		&= -\NC{a}{i},\tag{\ref{eqn.tpo.D}}\\[2mm]
		%		
		\pd{t} \bbT{a}{i} + \pd{k}(\shift{i} 
		\bbT{a}{k} - \shift{k}\bbT{a}{i}  + \LCsymb^{ikj} 
		\Lagtpo_{\ddT{a}{j}}) 
		&= 0,\tag{\ref{eqn.tpo.B}}\\[2mm]
		%		
		\pd{\mu}\left( \ddT{a}{\mu}\Lagtpo_{\ddT{a}{\nu}}+ 
		\bbT{a}{\mu}\Lagtpo_{\bbT{a}{\nu}}
		-\tetr{a}{\nu} \Lagtpo_{\tetr{a}{\mu}} 
		- \Lagtpo \KD{\mu}{\nu}
		\right) & = 0,\\[2mm]
		%		
		\pd{t} \tetr{a}{k} - \beta^i \pd{i} \tetr{a}{k} - \tetr{a}{i} \pd{k} \beta^i &= 
		\Lagtpo_{\ddT{a}{k}},
		\tag{\ref{eq.tetr.3+1}} 
	\end{empheq}
\end{subequations}

\subsubsection{Stationary constraints}

\begin{equation}
	\pd{k} \ddT{a}{k} = \NC{a}{0}, 
	\qquad
	\pd{k} \bbT{a}{k} = 0. \tag{\ref{eq.div.constr}}
\end{equation}

\subsubsection{Algebraic constraints}
As a consequence of our choice of the $ 3+1 $-split \eqref{eq.4vel} and definitions  and the fact 
that $ \ET{a}{\mu} u^\mu = 0 $ and $ \BT{a}{\mu} u_\mu = 0$, we have 
the following algebraic constraints 
\begin{subequations}\label{eq.constr.alg.EB}
	\begin{equation}
		\bbT{\indalg{0}}{\mu} = \bT{\indalg{0}}{\mu} = 0, \qquad \bbT{a}{0} = \bT{a}{0} = 0,
	\end{equation}
	\begin{equation}\label{eq.alg.constrE}
		\ET{\indalg{0}}{0} = W, \qquad \ET{\indalg{0}}{k} = N_k,\qquad  \ET{\sA}{0} = 
		\beta^j \ET{\sA}{j},
	\end{equation}
	\begin{equation}
		W := \beta^k N_k, \qquad 	N_k := \pd{k} 
		\ln(\alpha).
	\end{equation}
\end{subequations}
Also, from the expression \eqref{eqn.Tscal.EB1} for $ \LagBE $ we get the expressions of the time 
component of $ \ddT{a}{\mu} = \alpha^{-1} \dT{a}{\mu} = \alpha^{-1} \frac{\partial \LagBE}{\partial 
\ET{a}{\mu}} $ \ho{Here $\varkappa$ is used before its definition (until next section).
But also, why does it appear here? The Einstein constant gives the coupling between matter and gravity,
but here everything is gravity.}
\begin{equation}\label{eq.D0}
	\ddT{\indalg{0}}{0} = \frac{1}{2\alpha} \left(  \ddT{k}{k} - \frac{\detTetr_3}{2\alpha
	\varkappa} 
	\beta^k 
	N_k \right),
	\qquad
	\ddT{\indalg{0}}{k} = -\beta^k \ddT{\indalg{0}}{0} + \frac{\detTetr_3}{\varkappa} \LCsymb_{jli} 
	\gamma^{jk}\bbT{l}{i},
	\qquad
	\ddT{\sA}{0} = -\frac{\detTetr_3}{2\alpha\varkappa}\itetr{k}{\sA} N_k.
\end{equation}

\begin{equation}\label{eq.D0.tmp}
	\dT{\indalg{0}}{0} = \frac{1}{2 \alpha} \left(  \dT{k}{k} - \frac{\detTetr_3}{2 
		\varkappa} 
	\beta^k 
	N_k \right),
	\qquad
	\dT{\indalg{0}}{k} = -\beta^k \dT{\indalg{0}}{0} - \frac{\detTetr_3}{\varkappa} \LCsymb_{jli} 
	\gamma^{jk}\bT{l}{i},
	\qquad
	\dT{\sA}{0} = -\frac{\detTetr_3}{2\varkappa}\itetr{k}{\sA} N_k.
\end{equation}

\begin{equation}\label{eq.D0.tmp2}
	\dT{\indalg{0}}{0} = \frac{1}{2 \alpha} \tetr{\sA}{k}\left(  \dT{\sA}{k} + 
	\beta^k \dT{\sA}{0} \right),
	\qquad
	\dT{\indalg{0}}{k} = -\beta^k \dT{\indalg{0}}{0} - \frac{\detTetr_3}{\varkappa} \LCsymb_{jli} 
	\gamma^{jk}\bT{l}{i},
	\qquad
	\dT{\sA}{0} = -\frac{\detTetr_3}{2\varkappa}\itetr{k}{\sA} N_k.
\end{equation}

\section{Closure}



In the \tegr\ and its $ f(\Tscal) $-extensions the Lagrangian density is a function of the 
torsion scalar $ \Tscal $, e.g. in \tegr, the Lagrangian is
\begin{subequations}\label{eq.TEGR.Lagr}
	\begin{equation}
		\Lagtors(\tetr{a}{\mu},\Tors{a}{\mu\nu}) = \frac{\detTetr}{2 \, \varkappa} \Tscal,
	\end{equation}
	\begin{equation}\label{eqn.tors.scal0}
		\Tscal(\tetr{a}{\mu},\Tors{a}{\mu\nu}) := 
		\frac14 g^{\beta\lambda} g^{\mu\gamma} g_{\alpha\eta} \Tors{\alpha}{\lambda\gamma}
		\Tors{\eta}{\beta\mu} +
		%
		\frac12 g^{\mu\gamma} \Tors{\lambda}{\mu\beta} \Tors{\beta}{\gamma\lambda} - 
		%
		g^{\mu\lambda} \Tors{\rho}{\mu\rho} \Tors{\gamma}{\lambda\gamma},	  
	\end{equation}
\end{subequations}
where $ \varkappa = 8\pi G $ is the Einstein gravitational constant, $ \Tors{\lambda}{\mu\nu} = 
\itetr{\lambda}{a} \Tors{a}{\mu\nu} $, and the metric and its 
inverse must be computed as $ g_{\mu\nu} = 
\mg{ab}\tetr{a}{\mu}\tetr{b}{\nu} $ and $ g^{\mu\nu} = \MG{ab}\itetr{\mu}{a}\itetr{\nu}{b}$.

Denoting the scalars in the right-hand side of \eqref{eqn.tors.scal0} as
\begin{align}\label{eqn.tors.scal}
	\Tscal_1 = & \frac14 \mg{ab}g^{\beta\lambda }g^{\mu\gamma 
	}\Tors{a}{\lambda\gamma} \Tors{b}{\beta\mu },\\[2mm]
	%
	\Tscal_2 = & \frac12 g^{\mu\gamma} \itetr{\lambda}{a} \itetr{\beta}{b} \Tors{a}{\mu\beta} 
	\Tors{b}{\gamma\lambda},\\[2mm]
	%
	\quad
	\Tscal_3 = & -g^{\mu\lambda} \itetr{\rho}{a} \itetr{\gamma}{b} \Tors{a}{\mu\rho} 
	 \Tors{b}{\lambda\gamma},	  
\end{align}
we can also write 
\begin{align}\label{eqn.tors.scal.TQLC}
	\mathcal{T}_1 &= Q_3 + \frac12 C_4, \\
	\mathcal{T}_2 &=- Q_1 + Q_4 + L_1 -C_1 + C_2 + \frac12 C_4, \\[2mm]
	\mathcal{T}_3 &= 2 Q_1 + Q_2 + L_2 + 2 C_1 + C_3 - C_4,
\end{align}
where the scalars $ Q $, $ L $, and $ C $ are scalars made of $ \ET{a}{\mu} $ and 
$ \BT{a}{\mu} $.

Quadratic in $ \ET{a}{\mu} $:
\begin{align}\label{eqn.tors.scal2.Q}
	Q_1 &= -\frac12 \ET{\indalg{0}}{\alpha} g^{\alpha\beta} \ET{\indalg{0}}{\beta},\\[2mm]
	%
	Q_2 &= \itetr{\alpha}{a} \ET{a}{\alpha} \itetr{\beta}{b} \ET{b}{\beta} = 
	\ET{\alpha}{\alpha} \ET{\beta}{\beta},\\[2mm]
	%
	Q_3 &=-\frac12 \mg{ab} \ET{a}{\alpha} g^{\alpha\mu}
	\ET{b}{\mu},\\[2mm]
	%
	Q_4 &=-\frac12 \itetr{\lambda}{a} \ET{a}{\beta} \itetr{\beta}{b} \ET{b}{\lambda} = 
	\ET{\lambda}{\beta} \ET{\beta}{\lambda}.
\end{align}

Mixed scalars (linear in $ \ET{a}{\mu} $):
\begin{align}\label{eqb.ts.scal.L}
	L_1 &= \LCsymb_{\lambda\tau\varphi\gamma} u^\varphi g^{\tau\beta} \itetr{\lambda}{a} 
	\ET{a}{\beta} \BT{\indalg{0}}{\gamma},\\[2mm]
	%
	L_2 &= 2 \LCsymb_{\lambda\tau\varphi\gamma} u^\varphi g^{\lambda\beta} \itetr{\tau}{a} 
	\ET{\indalg{0}}{\beta} \BT{a}{\gamma}.
\end{align}

Quadratic in $ \BT{a}{\mu} $ (constant in  $ \ET{a}{\mu} $)
\begin{align}\label{eqn.tors.scal.C}
	C_1 &=-\frac12 h^{-2} g_{\lambda\rho} \BT{\indalg{0}}{\rho} \BT{\indalg{0}}{\lambda},\\[2mm]
	%
	C_2 &=-\frac12 h^{-2} g_{\rho\varphi} \itetr{\varphi}{a} \BT{a}{\rho} \itetr{\beta}{b} 
	\BT{b}{\lambda}g_{\lambda\beta} = -\frac12 h^{-2} g_{\rho\varphi} \BT{\varphi}{\rho} 
	\BT{\beta}{\lambda}g_{\lambda\beta},\\[2mm]
	%
	C_3 &= h^{-2} \itetr{\varphi}{a} \BT{a}{\sigma} g_{\varphi\beta} \itetr{\lambda}{b} 
	\BT{b}{\beta} g_{\lambda\sigma} = h^{-2} \BT{\varphi}{\sigma} g_{\varphi\beta} 
	\BT{\lambda}{\beta} g_{\lambda\sigma},\\[2mm]
	%
	C_4 &= h^{-2} \itetr{\varphi}{a} \BT{a}{\sigma} g_{\varphi\lambda} \itetr{\lambda}{b} 
	\BT{b}{\beta} g_{\beta\sigma} =h^{-2} \BT{\varphi}{\sigma} g_{\varphi\lambda} 
	\BT{\lambda}{\beta} g_{\beta\sigma}.
\end{align}

In terms of the Hodge dual $ \HDT{a\mu\nu} $, the torsion scalar $ \Tscal $ can be rewritten as
\begin{equation}\label{eqn.tors.scal.hodge}
	\Tscal = \Hscal_1 + \Hscal_2
\end{equation}
where 
\begin{align}\label{eqn.hodge.scal}
	\Hscal_1 &= \frac{1}{2\,h^2} g_{\alpha\mu}g_{\beta\nu}g_{\sigma\rho} 
	\itetr{\alpha}{a}\itetr{\beta}{b} \HDT{a\mu\rho}\HDT{b\nu\sigma} = 
	\frac{1}{2\,h^2} g_{\alpha\mu}g_{\beta\nu}g_{\sigma\rho} 
	\HDT{\alpha\mu\rho}\HDT{\beta\nu\sigma},
	\\[2mm]
	%
	\Hscal_2 &=-\frac{1}{h^2} g_{\alpha\mu}g_{\beta\nu}g_{\sigma\rho} 
	\itetr{\alpha}{a}\itetr{\beta}{b} \HDT{a\nu\sigma}\HDT{b\mu\rho} = 
	-\frac{1}{h^2} g_{\alpha\mu}g_{\beta\nu}g_{\sigma\rho} 
	\HDT{\alpha\nu\sigma}\HDT{\beta\mu\rho}
\end{align}



%If one needs the Lagrangian density $ \Laghodge(\tetr{a}{\mu},\HDT{a\mu\nu}) $ as a function of 
%the 
%Hodge dual, then the torsion scalar $ \Tscal $ can be expressed as
%\eqref{eqn.Lagrangians2}
%\begin{equation}\label{eqn.Tscal.Hodge}
%	\Tscal(\tetr{a}{\mu},\HDT{a\mu\nu}) = \frac12 \HDT{\mu\nu\lambda} (\HDmix_{\lambda\mu\nu} + 
%	\HDmix_{\nu\lambda\mu} + g_{\mu\lambda} \HDmix^{\gamma}_{\phantom{\gamma\,} \nu\gamma})
%\end{equation}


However, to close system \eqref{eq.3+1.summary}, we need not the Lagrangian $ 
\Lagtors(\tetr{a}{\mu},\Tors{a}{\mu\nu})  $ directly but we need to perform a sequence of variable 
and potential changes: $ \Lagtors(\tetr{a}{\mu},\Tors{a}{\mu\nu}) = 
\LagBE(\tetr{a}{\mu},\ET{a}{\mu},\BT{a}{\mu}) $ $ \longrightarrow $ $ \ET{a}{\mu} 
\dT{a}{\mu} - \LagBE = 
\LagST(\tetr{a}{\mu},\dT{a}{\mu},\bT{a}{\mu}) = 
\Lagtpo(\tetr{a}{\mu},\ddT{a}{\mu},\bbT{a}{\mu}) $. 
%in terms of 
%the state vector $ \vec{Q} = \{ 
%\tetr{a}{\mu}, \ddT{a}{i}, \bbT{a}{\mu} \} $ which can be obtained from the following expresion in 
%terms of $ \{ \tetr{a}{\mu},\ET{a}{\mu},\BT{a}{\mu}\} $:
Thus, we have
\begin{multline}\label{eqn.Tscal.EB}
	\LagBE(\tetr{a}{\mu},\ET{a}{\mu},\BT{a}{\mu}) = \frac{\detTetr}{2 \, \varkappa}
	\bigg(
	-\frac12 (\ETmix{\indalg{0}\lambda}{}\ETmix{\indalg{0}}{\ \lambda} -  
				2\ETmix{\lambda}{\ \,\lambda}\ETmix{\beta}{\ \,\beta} +
		      \ETmix{\ \,\lambda}{\beta} \ETmix{\beta}{\ \,\lambda}  +
			  \ETmix{\lambda}{\ \,\beta}\ETmix{\beta}{\ \,\lambda} )
	          \\
	          \hspace{4cm} + \LCsymb_{\lambda\gamma\eta\rho} u^\eta 
	          (\ETmix{\lambda\gamma}{}\BTmix{\indalg{0}\rho}{} + 2 
	          \ET{\indalg{0}\lambda}{}\BTmix{\gamma\rho}{})
  			  \\
	          - \frac12 h^{-2} ( \BTmix{\indalg{0}\lambda}{}\BTmix{\indalg{0}}{\ \lambda}
	          + \BTmix{\lambda}{\ \,\lambda}\BTmix{\beta}{\ \,\beta}
	          - 2\BTmix{\lambda}{\ \,\beta}\BTmix{\beta}{\ \,\lambda}
	          )
	          \bigg)
\end{multline}
or, using algebraic constraints \eqref{eq.constr.alg.EB}
\begin{multline}\label{eqn.Tscal.EB1}
	\LagBE(\tetr{a}{\mu},\ET{a}{\mu},\BT{a}{\mu}) = \frac{h}{2\,\varkappa} \bigg( 
	 \ETmix{k}{\ \,k}\ETmix{i}{\ \,i} 
	-\frac12 \ETmix{k}{\ \,i} \ETmix{i}{\ \,k}
	-\frac12 \ETmix{\ \,k}{i} \ETmix{i}{\ \,k} 
	%
	-\frac12 \detTetr^{-2} 
	  \BTmix{i}{\ \,i}\BTmix{k}{\ \,k}
	+ \detTetr^{-2} \BTmix{i}{\ \,k}\BTmix{k}{\ \,i} 
	\\
	%
	{\color{red}-} 2 \, \alpha^{-1} \LCsymb_{kli} N^k \BTmix{l i}{}
	\bigg),
\end{multline}
where  $ N^k = N_j \gamma^{jk} $, $ \ET{i}{j} = 
\itetr{i}{\sA} \ET{\sA}{j}$, $ \ETmix{\ \,k}{i} = \gamma_{ij} \ET{j}{l} \gamma^{lk} 
$, $ \BT{i}{j} = \itetr{i}{\sA} 
\BT{\sA}{j} $,  and $ \BTmix{i}{\ \,k} = 
\BT{i}{j} \gamma_{jk}$. Also, in \eqref{eqn.Tscal.EB1}, $ \LCsymb_{ijk} $ is the pure spatial 
Levi-Civita symbol with the reference $ \LCsymb_{123}=1 $ and $ \LCsymb_{ijk} = \LCsymb^{ijk} $ (in 
particular $ \LCsymb_{0123} = -\LCsymb_{123} $).

 
To close system \eqref{eq.3+1.summary}, we need however not the Lagrangian $ 
\LagBE(\tetr{\sA}{k},\ET{\sA}{k},\BT{\sA}{k}) $ but we need to express another 
potential $ \LagST(\tetr{\sA}{k},\dT{\sA}{k},\bT{\sA}{k}) = \ET{a}{\mu} 
\dT{a}{\mu} - \LagBE$
in terms of $ \{\tetr{\sA}{k},\dT{\sA}{k},\bT{\sA}{k}\} $. It reads
\begin{multline}\label{eq.Lagr.DB}
	 \LagST(\tetr{\sA}{k},\dT{\sA}{k},\bT{\sA}{k}) = 
	 \frac{1}{4\,h} \bigg( \varkappa \left(
	 \dT{i}{i}\dT{k}{k} - 2 \dT{i}{k}\dT{k}{i}
	 \right)
	 % 
 	+ \frac{1}{\varkappa} \left ( 
	 \BTmix{i}{\ \,i}\BTmix{k}{\ \,k}
	 - 2 \BTmix{i}{\ \,k}\BTmix{k}{\ \,i}
	 \right ) \\
	 	+ \varkappa (\beta^j \dT{j}{0} + 2 \beta^j \dT{j}{0}\dddT{k}{k} - 4 \beta^j \dT{k}{0} 
	 	\dddT{j}{k}) 
	 	\bigg).
%	 	 {\color{red}+} 18 \, \alpha^{-1} h \LCsymb_{kli} N^k \BTmix{l i}{}
\end{multline}
where we have introduced
\begin{equation}\label{eqn.db}
	\dddT{\sA}{k} = \dT{\sA}{k} + \beta^k \dT{\sA}{0}, \qquad \bbbT{\sA}{k} = \bT{\sA}{k}.
\end{equation}
But in fact, in terms of \eqref{eqn.db} this potential reads
\begin{equation}\label{eq.Lagr.db}
	\LagST(\tetr{\sA}{k},\dddT{\sA}{k},\bbbT{\sA}{k}) = 
	\frac{1}{4\,h} \bigg( \varkappa \left(
	\dddT{i}{i}\dddT{k}{k} - 2 \dddT{i}{k}\dddT{k}{i}
	\right)
	% 
	+ \frac{1}{\varkappa} \left ( 
	\bbbTmix{i}{\ \,i}\bbbTmix{k}{\ \,k}
	- 2 \bbbTmix{i}{\ \,k}\bbbTmix{k}{\ \,i}
	\right ) \bigg).
	%	 	 {\color{red}+} 18 \, \alpha^{-1} h \LCsymb_{kli} N^k \BTmix{l i}{}
\end{equation}

Finally, the potential $ \Lagtpo(\tetr{\sA}{k},\ddT{\sA}{k},\bbT{\sA}{k}) $ in 
the new variables \eqref{eq.varDB.final} reads as
\begin{equation}\label{eq.Lagr.DB.final}
	\Lagtpo(\tetr{\sA}{k},\ddT{\sA}{k},\bbT{\sA}{k}) = 
%	 \frac{1}{2\,\varkappa} \bigg(
%	-\frac78 \detTetr^{-1} \alpha^2 \left(  
%	\ddT{i}{i}\ddT{k}{k} - 2 \ddT{i}{k}\ddT{k}{i}
%	\right)
%	% 
%	+ \frac92 \detTetr^{-1} \alpha^2 \left ( 
%	\bbTmix{i}{\ \,i}\bbTmix{k}{\ \,k}
%	- 2 \bbTmix{i}{\ \,k}\bbTmix{k}{\ \,i}
%	\right )\\
%	%
%	{\color{red}-} 18 \, h \LCsymb_{kli} N^k \bbT{l}{i}
%	\bigg).
\end{equation}

%Also, because $ \LagST $ is a quadratic form in $ \dT{\sA}{k} $ and $ \bT{\sA}{k} $, 
%one can write 
%\begin{equation}\label{eqn.Tscal.HBDE}
%	\LagST(\tetr{\sA}{k},\dT{\sA}{k},\bT{\sA}{k}) = \frac12 
%\BT{a}{\mu}\LagBE_{\BT{a}{\mu}} - 
%	\frac12 \ET{a}{\mu}\LagBE_{\ET{a}{\mu}} 
%	=
%	\frac12 \BT{a}{\mu}\hT{a}{\mu} - \frac12 \ET{a}{\mu}\dT{a}{\mu},
%\end{equation}
%or
%\begin{equation}
%	\LagBE(\tetr{a}{\mu},\ET{a}{\mu},\BT{a}{\mu})
%	=
%	\frac12 \BT{a}{i}\hT{a}{i} - \frac12 \ET{a}{i}\dT{a}{i},
%\end{equation}
%where $ \hT{a}{\mu} = \LagBE_{\BT{a}{\mu}} $ and $ \dT{a}{\mu} = \LagBE_{\ET{a}{\mu}} $.

\ho{Hector Here}
\IP{Decide what variables are finally to be used $ \{\dT{}{},\bT{}{}\} $ or the re-scaled ones $ 
\{\ddT{}{},\bbT{}{}\} $ }

\IP{Not clear why we don't have a mixed invariant kinda $ \dT{i}{k} B^i_{\ k} $ in the 
Lagrangian $ \LagST $. Should we?}
%-------------------------------------------------------
\appendix
%-------------------------------------------------------


%-------------------------------------------------------
\section{Spacetime form of the energy-momentum}\label{app.sec.EM}
%-------------------------------------------------------

Here, we show how the energy-momentum conservation law \eqref{eqn.EM} can be written in a pure 
spacetime form \eqref{eqn.EM3}.

Contracting $ \pd{\mu} \Laghodge_{\tetr{a}{\mu}} = 0 $ with $ \tetr{a}{\nu} $ and adding $ 0\equiv 
\Laghodge_{\tetr{a}{\mu}}\pd{\mu}\tetr{a}{\nu} -  \Laghodge_{\tetr{a}{\mu}}\pd{\mu}\tetr{a}{\nu} $ 
one gets
\begin{equation}
\pd{\mu}(\tetr{a}{\nu}\Laghodge_{\tetr{a}{\mu}}) - \Laghodge_{\tetr{a}{\mu}}\pd{\mu}\tetr{a}{\nu} = 
0.
\end{equation}
Replacing the last term with $ \pd{\mu}\tetr{a}{\nu} = \pd{\nu}\tetr{a}{\mu} + \Tors{a}{\mu\nu} $ 
yields
\begin{equation}
\pd{\mu}(\tetr{a}{\nu}\Laghodge_{\tetr{a}{\mu}}) - \Laghodge_{\tetr{a}{\mu}}\pd{\nu}\tetr{a}{\mu} - 
\Laghodge_{\tetr{a}{\mu}}\Tors{a}{\mu\nu} = 0.
\end{equation}
Then, using the fact that $ L = L(\tetr{a}{\mu},\HDT{a\mu\nu}) $, the second term can be 
substituted 
by $ \Laghodge_{\tetr{a}{\mu}}\pd{\nu}\tetr{a}{\mu} = \pd{\nu}\Laghodge - 
\Laghodge_{\HDT{b\lambda\rho}}\pd{\nu}\HDT{b\lambda\rho} $. This results in
\begin{equation}\label{app.eqn.EM1}
\pd{\mu}(\tetr{a}{\nu}\Laghodge_{\tetr{a}{\mu}} - L \delta^\mu_{\ \nu}) +
\Laghodge_{\HDT{b\lambda\rho}}\pd{\nu}\HDT{b\lambda\rho} -
\Laghodge_{\tetr{a}{\mu}}\Tors{a}{\mu\nu} = 0.
\end{equation} 
Now, the energy-momentum $ -\Laghodge_{\tetr{a}{\mu}} $ in the last term can be substituted by its 
expression from the Euler-Lagrange equation for the torsion \eqref{eqn.EM.Hodge}:
\begin{multline}\label{app.eqn.EM2}
	-\Laghodge_{\tetr{a}{\mu}}\Tors{a}{\mu\nu} = 
	\Tors{a}{\mu\nu}\pd{\lambda}(\LCsymb^{\mu\gamma\rho\lambda}\Laghodge_{\HDT{a\gamma\rho}}) =
	-\frac12\LCsymb_{\mu\nu\alpha\beta}\HDT{a\alpha\beta}(\LCsymb^{\mu\gamma\rho\lambda}\Laghodge_{\HDT{a\gamma\rho}})
	 =
	 \\
	2\HDT{a\alpha\beta}\pd{\beta}\Laghodge_{\HDT{a\nu\alpha}} + 
	\HDT{a\alpha\beta}\pd{\nu}\Laghodge_{\HDT{a\alpha\beta}},
\end{multline}
where we have used  $ 
-\LCsymb_{\mu\nu\alpha\beta}\LCsymb^{\mu\gamma\rho\lambda} = 
\KD{\gamma\rho\lambda}{\nu\alpha\beta} +
\KD{\lambda\gamma\rho}{\nu\alpha\beta} +
\KD{\rho\lambda\gamma}{\nu\alpha\beta} -
\KD{\gamma\lambda\rho}{\nu\alpha\beta} -
\KD{\lambda\rho\gamma}{\nu\alpha\beta} -
\KD{\rho\gamma\lambda}{\nu\alpha\beta}
$, with $ \KD{\gamma\rho\lambda}{\nu\alpha\beta} = 
\KD{\gamma}{\nu}\KD{\rho}{\alpha}\KD{\lambda}{\beta} $, e.g. see \cite{KleinertMultivalued}, p.45. 
Hence, using expression \eqref{app.eqn.EM2}, equation 
\eqref{app.eqn.EM1} can be written as
\begin{equation}
\pd{\mu}(\tetr{a}{\nu}\Laghodge_{\tetr{a}{\mu}} - L \delta^\mu_{\ \nu}) +
\Laghodge_{\HDT{b\lambda\rho}}\pd{\nu}\HDT{b\lambda\rho} +
2\HDT{a\alpha\beta}\pd{\beta}\Laghodge_{\HDT{a\nu\alpha}} + 
\HDT{a\alpha\beta}\pd{\nu}\Laghodge_{\HDT{a\alpha\beta}} = 0,
\end{equation} 
and then
\begin{equation}
\pd{\mu}\left(\tetr{a}{\nu}\Laghodge_{\tetr{a}{\mu}} + 
(\HDT{b\lambda\rho}\Laghodge_{\HDT{b\lambda\rho}}- 
L) \delta^\mu_{\ \nu} \right) +
2\HDT{a\alpha\mu}\pd{\mu}\Laghodge_{\HDT{a\nu\alpha}} = 0,
\end{equation} 
Finally, using the integrability condition \eqref{integr.HT}, this equation can be written in a 
fully conservative form
\begin{equation}
\pd{\mu}\left(\tetr{a}{\nu}\Laghodge_{\tetr{a}{\mu}} -
2\HDT{a\lambda\mu}\Laghodge_{\HDT{a\lambda\nu}}
+
(\HDT{b\lambda\rho}\Laghodge_{\HDT{b\lambda\rho}}- 
L) \delta^\mu_{\ \nu} \right) = 0.
\end{equation} 


%It can be shown (see 
%expression \eqref{app.eqn.Noether2} 
%for $ \Laghodge_{\tetr{a}{\mu}} $ in 
%Appendix~\ref{app.sec.NC}) that the tetrad part $ 
%\tetr{a}{\nu}\Laghodge_{\tetr{a}{\mu}} $ expands as
%\begin{align}
%	\tetr{a}{\nu}\Laghodge_{\tetr{a}{\mu}} & =
%    -\tetr{a}{\nu} \LagST_{\tetr{a}{\mu}} \nonumber\\
%   & + u_\nu u^\mu(\bT{a}{\lambda} \LagST_{\bT{a}{\lambda}} + \dT{a}{\lambda} 
%   \LagST_{\dT{a}{\lambda}}) 
%   \nonumber\\
%%   &- u^\mu u_\lambda \dT{a}{\lambda} \LagST_{\dT{a}{\nu}} - u_\nu u^\lambda 
%%   \bT{a}{\mu}\LagST_{\bT{a}{\lambda}} \nonumber\\
%   &-\LCsymb^{\mu\alpha\beta\lambda} u_\nu u_\alpha \LagST_{\dT{a}{\beta}} \LagST_{\bT{a}{\lambda}}
%   -\LCsymb_{\nu\alpha\beta\lambda} u^\mu u^\alpha \dT{a}{\beta}\bT{a}{\lambda}
%\end{align}
%
%On the other hand, the torsion part $ - 2 \HDT{a\lambda\mu}L_{\HDT{a\lambda\nu}} + 
%(\HDT{a\lambda\rho}L_{\HDT{a\lambda\rho}} - L) \delta^\mu_{\ \nu} $ reads
%\begin{align}
%	- 2 \HDT{a\lambda\mu}L_{\HDT{a\lambda\nu}} + 
%	(\HDT{a\lambda\rho}L_{\HDT{a\lambda\rho}} - L) \delta^\mu_{\ \nu} & = -\LagST 
%	\KD{\mu}{\nu}\nonumber\\
%	& + \dT{a}{\mu}\LagST_{\dT{a}{\nu}}+ \bT{a}{\mu}\LagST_{\bT{a}{\nu}} \nonumber\\
%	& - u_\nu u^\mu(\bT{a}{\lambda} \LagST_{\bT{a}{\lambda}} + \dT{a}{\lambda} 
%	\LagST_{\dT{a}{\lambda}}) 
%	\nonumber\\
%	%   &- u^\mu u_\lambda \dT{a}{\lambda} \LagST_{\dT{a}{\nu}} - u_\nu u^\lambda 
%	%   \bT{a}{\mu}\LagST_{\bT{a}{\lambda}} \nonumber\\
%	&-\LCsymb^{\mu\alpha\beta\lambda} u_\nu u_\alpha \mathcal{\LagST}_{\dT{a}{\beta}} 
%	\LagST_{\bT{a}{\lambda}}
%	-\LCsymb_{\nu\alpha\beta\lambda} u^\mu u^\alpha \dT{a}{\beta}\bT{a}{\lambda}.
%\end{align}


\section{Transformation of the Noether's current $ \NC{a}{\mu} $}\label{app.sec.NC}

Here, we express the source terms $ \NC{a}{\mu} = \Laghodge_{\tetr{a}{\mu}} $ in terms of the 
potential $ \LagST $ and $ \dT{a}{\mu} $ and $ \bT{a}{\mu} $. One has
\begin{equation}\label{app.eqn.Noether1}
	\Laghodge_{\tetr{a}{\mu}} = \LagBE_{\tetr{a}{\mu}} 
	+ \LagBE_{\BT{b}{\lambda}} \frac{\partial \BT{b}{\lambda}}{\partial \tetr{a}{\mu}}
	+ \LagBE_{\ET{b}{\lambda}} \frac{\partial \ET{b}{\lambda}}{\partial \tetr{a}{\mu}}.
\end{equation}
Then, using the definitions of the frame 4-velocity $ u^\nu = \itetr{\nu}{\indalg{0}} $ and $ u_\nu 
= 
-\tetr{\indalg{0}}{\nu} $ and the torsion fields
$ \BT{b}{\lambda} = \HDT{b\lambda\nu} u_\nu = - \HDT{b\lambda\nu} \tetr{\indalg{0}}{\nu}$ and 
$ \ET{b}{\lambda} = \Tors{b}{\lambda\nu} u^\nu = 
-\frac12\LCsymb_{\lambda\nu\alpha\beta}\HDT{b\alpha\beta} \itetr{\nu}{\indalg{0}}$ and the simple 
fact 
that $ \partial\itetr{\lambda}{b}/\partial\tetr{a}{\mu} = -\itetr{\lambda}{a}\itetr{\mu}{b} $, we 
can rewrite \eqref{app.eqn.Noether1} as
\begin{multline}
	\Laghodge_{\tetr{a}{\mu}} = \LagBE_{\tetr{a}{\mu}} 
	+ \LagBE_{\BT{b}{\lambda}} \frac{\partial \BT{b}{\lambda}}{\partial \tetr{a}{\mu}}
	+ \LagBE_{\ET{b}{\lambda}} \frac{\partial \ET{b}{\lambda}}{\partial \tetr{a}{\mu}} = \\
	\LagBE_{\tetr{a}{\mu}} - \LagBE_{\BT{b}{\lambda}} \KD{\indalg{0}}{a}
	(u^\lambda \BT{b}{\mu} - u^\mu \BT{b}{\lambda}+\LCsymb^{\lambda\mu\alpha\beta} u_\alpha  
	\ET{b}{\beta}) \\
	-\LagBE_{\ET{b}{\lambda}} (u_\lambda \ET{b}{\nu} - u_\nu \ET{b}{\lambda} - 
	\LCsymb_{\lambda\nu\alpha\rho}u^\alpha\BT{b}{\rho})\itetr{\nu}{a}u^\mu.
\end{multline}
Using the definitions \eqref{eqn.Legandre2} and \eqref{eqn.Legandre3}, the latter can be rewritten 
as
\begin{multline}
	\Laghodge_{\tetr{a}{\mu}} =
	-\LagST_{\tetr{a}{\mu}} 
	- \LagST_{\bT{b}{\lambda}} \KD{\indalg{0}}{a}
	(-u^\lambda \bT{b}{\mu} + u^\mu \bT{b}{\lambda}+\LCsymb^{\lambda\mu\alpha\beta} u_\alpha 
	\LagST_{\dT{b}{\beta}}) \\
	-\dT{b}{\lambda} (u_\lambda \LagST_{\dT{b}{\nu}} - u_\nu \LagST_{\dT{b}{\lambda}} + 
	\LCsymb_{\lambda\nu\alpha\rho}u^\alpha\bT{b}{\rho})\itetr{\nu}{a}u^\mu,
\end{multline}
which, after some term rearrangements, reads
\begin{multline}\label{eqn.J.BD}
	\Laghodge_{\tetr{a}{\mu}} =
	-\LagST_{\tetr{a}{\mu}}
	% 
	+ \KD{\indalg{0}}{a}
	\left( 
	  u^\lambda \bT{b}{\mu} \LagST_{\bT{b}{\lambda}} 
	- u^\mu \bT{b}{\lambda} \LagST_{\bT{b}{\lambda}} 
	- u^\mu \dT{b}{\lambda} \LagST_{\dT{b}{\lambda}}
	+ \LCsymb^{\mu\lambda\rho\sigma} u_\rho \LagST_{\bT{b}{\lambda}}
	\LagST_{\dT{b}{\sigma}} 
	\right) \\
	%
	- \itetr{\nu}{a}u^\mu
	\left(
	u_\lambda \dT{b}{\lambda} \LagST_{\dT{b}{\nu}} 
	- \LCsymb_{\nu\lambda\rho\sigma}u^\rho\bT{b}{\sigma}\dT{b}{\lambda}
	\right).
\end{multline}


Finally, the Noether's current $ \NC{a}{\mu} = \Laghodge_{\tetr{a}{\mu}} $ reads
\begin{equation}
	\NC{a}{\mu} = \left\{
	\begin{array}{ll}
	-\LagST_{\tetr{\indalg{0}}{\mu}}
	+ u^\lambda \bT{b}{\mu} \LagST_{\bT{b}{\lambda}} 
	- u^\mu \bT{b}{\lambda} \LagST_{\bT{b}{\lambda}} 
	- u^\mu \dT{b}{\lambda} \LagST_{\dT{b}{\lambda}}
	+ \LCsymb^{\mu\lambda\rho\sigma} u_\rho \LagST_{\bT{b}{\lambda}}
	\LagST_{\dT{b}{\sigma}},	& a=\indalg{0},  \\[3mm] 
	%
	-\LagST_{\tetr{\sA}{\mu}}	
	- \itetr{\nu}{\sA}u^\mu
	\left(
	u_\lambda \dT{b}{\lambda} \LagST_{\dT{b}{\nu}} 
	- \LCsymb_{\nu\lambda\rho\sigma}u^\rho\bT{b}{\sigma}\dT{b}{\lambda}
	\right), & a = \sA. \\ 
	\end{array} 
	\right.
\end{equation}


\section{Expression for the energymomentum}\label{app.energymomentum}

In this section, we demonstrate how the following expression of the energymomentum tensor
\begin{equation}\label{eqn.EM.hodge}
	\EM{\mu}{\nu} :=
	\tetr{a}{\nu} L_{\tetr{a}{\mu}} - 2 \HDT{a\lambda\mu}L_{\HDT{a\lambda\nu}} + 
	(\HDT{a\lambda\rho}L_{\HDT{a\lambda\rho}} - L) \KD{\mu}{\nu}
\end{equation}
in terms of the Lagrangian $ \Laghodge(\tetr{a}{\mu},\HDT{a\mu\nu}) $ can be rewritten as
\begin{equation}
	\EM{\mu}{\nu} = 
	- \tetr{a}{\nu} \LagST_{\tetr{a}{\mu}}
	+ \dT{a}{\mu}\LagST_{\dT{a}{\nu}} + \bT{a}{\mu}\LagST_{\bT{a}{\nu}}
	- (
	\dT{a}{\lambda}\LagST_{\dT{a}{\lambda}}+ \bT{a}{\lambda}\LagST_{\bT{a}{\lambda}}
	-\LagST
	) \KD{\mu}{\nu},
\end{equation}
i.e. in terms of the new potential $ \LagST(\tetr{a}{\mu},\bT{a}{\mu},\dT{a}{\mu}) $.

The first term in \eqref{eqn.EM.hodge} is given by \eqref{eqn.J.BD}. Therefore, it remains to 
understand how to compute $ \HDT{a\lambda\mu}L_{\HDT{a\lambda\nu}} $. Because $ \HDT{a\mu\nu} $ is 
antisymmetric tensor, to compute the derivative $ \Laghodge_{\HDT{a\lambda\nu}} $ one needs to use 
its parametrization via the \We\ connection, which is not symmetric, i.e. $ \HDT{a\mu\nu} = 
\LCsymb^{\mu\nu\rho\sigma} \w{a}{\rho\sigma}$. Thus, for Lagrangians $ 
\Lag(\tetr{a}{\mu},\w{a}{\mu\nu}) = 
\Laghodge(\tetr{a}{\mu},\HDT{a\mu\nu})$ one can write 
\begin{equation}
	\Lag_{\w{a}{\lambda\mu}} = \LCsymb^{\lambda\mu\rho\sigma} \Laghodge_{\HDT{a\rho\sigma}}
\end{equation}
or using the identity $ \LCsymb_{\lambda\mu\alpha\beta}\LCsymb^{\lambda\mu\rho\sigma} = 
-2(\KD{\rho}{\alpha}\KD{\sigma}{\beta} - \KD{\rho}{\beta}\KD{\sigma}{\alpha}) $, 
\begin{equation}\label{eqn.L.T}
	\LCsymb_{\alpha\beta\rho\sigma}\Lag_{\w{a}{\lambda\mu}} = -4 \Laghodge_{\HDT{a\rho\sigma}}.
\end{equation}

On the other hand, using the definitions $ \ET{a}{\mu} = (\w{a}{\mu\nu} - \w{a}{\nu\mu})u^\nu $ and $ 
\BT{a}{\mu} = \LCsymb^{\mu\nu\rho\sigma}\w{a}{\rho\sigma} u_\nu $ and  the parametrization $ 
\Lag(\tetr{a}{\mu},\w{a}{\mu\nu}) = \LagBE(\tetr{a}{\mu},\BT{a}{\mu},\ET{a}{\mu}) $, one can write
\begin{equation}\label{eqn.Lambda.W}
	\Lag_{\w{b}{\lambda\gamma}} = 
	\LagBE_{\ET{a}{\mu}} \frac{\partial \ET{a}{\mu}}{\partial \w{b}{\lambda\gamma}} 
	+
	\LagBE_{\BT{a}{\mu}} \frac{\partial \BT{a}{\mu}}{\partial \w{b}{\lambda\gamma}} 
	=
	u^\gamma \LagBE_{\ET{b}{\lambda}} - u^\lambda \LagBE_{\ET{b}{\gamma}} - 
	\LCsymb^{\lambda\gamma\nu\mu} u_\nu \LagBE_{\BT{a}{\mu}}.
\end{equation}

From \eqref{eqn.L.T} and \eqref{eqn.Lambda.W}, one can deduce 
\begin{equation}
	\Laghodge_{\HDT{a\lambda\nu}} = -\frac{1}{2} 
	\left(
	u_\lambda \LagBE_{\BT{a}{\nu}} - u_\nu \LagBE_{\BT{a}{\lambda}} 
	-
	\LCsymb_{\lambda\nu\rho\sigma} u^\rho \LagBE_{\ET{a}{\sigma}} 
	\right).
\end{equation}
Then, after contracting the later equation with $ \HDT{a\lambda\mu} = u^\lambda \BT{a}{\mu} - u^\mu 
\BT{a}{\lambda} + \LCsymb^{\lambda\mu\alpha\beta} u_\alpha \ET{a}{\beta} $, one obtains
 \begin{align}\label{eqn.T.L.T} 
 	-2 \HDT{a\lambda\mu} \Laghodge_{\HDT{a\lambda\nu}} =
 	& -\BT{a}{\mu}\LagBE_{\BT{a}{\nu}} + \ET{a}{\nu}\LagBE_{\ET{a}{\mu}} \\
 	& - u^\lambda u_\nu \BT{a}{\mu} \LagBE_{\BT{a}{\lambda}} + u^\mu u_\nu \BT{a}{\lambda} 
 	\LagBE_{\BT{a}{\lambda}}				\nonumber\\
 	& + \LCsymb_{\nu\lambda\rho\sigma} u^\mu u^\sigma \BT{a}{\lambda} \LagBE_{\ET{a}{\rho}} 
 	  + \LCsymb^{\mu\lambda\rho\sigma} u_\nu u_\rho \ET{a}{\sigma} \LagBE_{\BT{a}{\lambda}}
 	\nonumber \\
 	& - (u^\mu u_\nu + \KD{\mu}{\nu}) \ET{a}{\lambda} \LagBE_{\ET{a}{\lambda}} + u^\mu u_\lambda 
 	\ET{a}{\nu} \LagBE_{\ET{a}{\lambda}}.
 	\nonumber
 \end{align}
This can be used to demonstrate that the full contraction  $ 	\HDT{a\lambda\rho} 
\Laghodge_{\HDT{a\lambda\rho}} $ results in
\begin{equation}\label{eqn.T.L.T.full}
	\HDT{a\lambda\rho} \Laghodge_{\HDT{a\lambda\rho}}
	=
	\BT{a}{\lambda} \LagBE_{\BT{a}{\lambda}} + \ET{a}{\lambda} \LagBE_{\ET{a}{\lambda}}.
\end{equation}

Collecting together \eqref{eqn.T.L.T} and \eqref{eqn.T.L.T.full} and using the change of 
variables and potential \eqref{eqn.Legandre1}--\eqref{eqn.Legandre3}, the torsion part of $ 
\EM{\mu}{\nu} $ reads
\begin{align}\label{eqn.Sigma.tors.part}
	- 2 \HDT{a\lambda\mu}L_{\HDT{a\lambda\nu}} + 
	(\HDT{a\lambda\rho}L_{\HDT{a\lambda\rho}} - L) \KD{\mu}{\nu} 
	& = \bT{a}{\mu}\LagST_{\bT{a}{\nu}} + \dT{a}{\nu}\LagST_{\dT{a}{\mu}} \\
	& + u^\lambda u_\nu \bT{a}{\mu} \LagBE_{\bT{a}{\lambda}} + u^\mu u_\lambda \dT{a}{\lambda} 
		\LagST_{\dT{a}{\nu}}				\nonumber\\
	& - \projector{\mu}{\nu} \bT{a}{\lambda} \LagST_{\bT{a}{\lambda}} 
	  - \projector{\mu}{\nu} \dT{a}{\lambda} \LagST_{\dT{a}{\lambda}}
	  \nonumber \\
	& - \LCsymb_{\nu\sigma\lambda\rho} u^\mu u^\sigma \bT{a}{\lambda} \dT{a}{\rho} 
	  - \LCsymb^{\mu\sigma\lambda\rho} u_\nu u_\sigma \LagST_{\bT{a}{\lambda}} \LagST_{\dT{a}{\rho}} 
	  \nonumber \\
	& + \LagST \KD{\mu}{\nu}, \nonumber
\end{align}
where $ \projector{\mu}{\nu} = u^\mu u_\nu + \KD{\mu}{\nu} $.

Finally, combining this result with the tetrad part given by \eqref{eqn.J.BD}, one arrives at
\begin{multline}
	\tetr{a}{\nu} \Laghodge_{\tetr{a}{\mu}}
		- 2 \HDT{a\lambda\mu}L_{\HDT{a\lambda\nu}} + 
	(\HDT{a\lambda\rho}L_{\HDT{a\lambda\rho}} - L) \KD{\mu}{\nu} = \\
	- \tetr{a}{\nu} \LagST_{\tetr{a}{\mu}}
	+ \dT{a}{\mu}\LagST_{\dT{a}{\nu}} + \bT{a}{\mu}\LagST_{\bT{a}{\nu}}
	- (
	\dT{a}{\lambda}\LagST_{\dT{a}{\lambda}}+ \bT{a}{\lambda}\LagST_{\bT{a}{\lambda}}
	-\LagST
	) \KD{\mu}{\nu}.	
\end{multline}






\section{Transformation of the torsion PDE}\label{app.sec.Deqn}

In this appendix, we demonstrate how the Euler-Lagrange equation \eqref{eqn.1st.order.EL} can be 
transformed to the form \eqref{eqn.tors.BE.a}.

First, let us introduce the notations
\begin{subequations}
	\begin{align}
			\TorsConj{a}{\mu\nu} := \Lagtors_{\Tors{a}{\mu\nu}},\hspace{1cm}
			&\HTConj{a\mu\nu} := \Laghodge_{\HDT{a\mu\nu}}, 
			\\[2mm]
			\Dbb{a}{\mu} := \TorsConj{a}{\mu\nu}u_\nu, \qquad
			&\Hbb{a}{\mu} := \HTConj{a\mu\nu}u^\nu,\label{app.eqn.DH}
	\end{align}
\end{subequations}
By a straightforward verification it then can be shown that
\begin{subequations}
	\begin{align}
	\TorsConj{a}{\mu\nu} &= u^\mu \Dbb{a}{\nu} - u^\nu \Dbb{a}{\mu} +
	\LCsymb^{\mu\nu\rho\sigma}u_\rho \Hbb{a}{\sigma},\\[2mm]
	\HTConj{a\mu\nu} &= u_\mu \Hbb{a}{\nu} - u_\nu \Hbb{a}{\mu} - 
	\LCsymb_{\mu\nu\rho\sigma}u^\rho \Dbb{a}{\sigma}\label{app.eqn.Deqn1},
	\end{align}
\end{subequations}
Hence, using \eqref{app.eqn.Deqn1}, Euler-Lagrange equation \eqref{eqn.1st.order.EL} can be written 
as
\begin{equation}
\D{\nu}(\LCsymb^{\mu\nu\alpha\beta}(u_\alpha \Hbb{a}{\beta} - u_\beta \Hbb{a}{\alpha} - 
\LCsymb_{\alpha\beta\rho\sigma}u^\rho \Dbb{a}{\sigma})) = -\NC{a}{\mu}.
\end{equation}
Then, using the fact that $ -\LCsymb_{\alpha\beta\rho\sigma}\LCsymb^{\mu\nu\alpha\beta} = 
-\LCsymb_{\alpha\beta\rho\sigma}\LCsymb^{\alpha\beta\mu\nu} = 2(\KD{\mu}{\rho}\KD{\nu}{\sigma} - 
\KD{\nu}{\rho}\KD{\mu}{\sigma})$, the latter equation can be re-written as
\begin{equation}\label{app.eqn.Deqn2}
	\D{\nu}(u^\mu \Dbb{a}{\nu} - u^\nu \Dbb{a}{\mu} + \LCsymb^{\mu\nu\alpha\beta}
	u_\alpha\Hbb{a}{\beta}) = -\frac12\NC{a}{\mu}.
\end{equation}
After substituting $ \Dbb{a}{\mu} $ and $ \Hbb{a}{\mu} $ by their definitions \eqref{app.eqn.DH}, 
equation \eqref{app.eqn.Deqn2} becomes
\begin{equation}
	\D{\nu}(u^\mu \Lagtors_{\Tors{a}{\nu\lambda}}u_\lambda - u^\nu \Lagtors_{\Tors{a}{\mu\lambda}} 
	u_\lambda + \LCsymb^{\mu\nu\alpha\beta}u_\alpha \Laghodge_{\HDT{a\beta\lambda}}u^\lambda) = 
	-\frac12 \NC{a}{\mu},
\end{equation}
which after using the following relations between the potentials $ 
\Laghodge(\tetr{a}{\mu},\HDT{a\mu\nu}) = 
\Lagtors(\tetr{a}{\mu},\Tors{a}{\mu\nu}) = 
\LagBE(\tetr{a}{\mu},\BT{a}{\mu},\ET{a}{\nu}) $ and their derivatives
\begin{equation}
	\Lagtors_{\Tors{a}{\mu\nu}} u_\nu = -\frac12\LagBE_{\ET{a}{\mu}},
	\qquad
	\Laghodge_{\HDT{a\mu\nu}} u^\nu= -\frac12\LagBE_{\BT{a}{\mu}},
\end{equation}
finally reads
\begin{equation}
	\D{\nu}( u^\mu\LagBE_{\ET{a}{\nu}} - u^\nu \LagBE_{\ET{a}{\mu}} + 
	\LCsymb^{\mu\nu\alpha
	\beta}u_\alpha\LagBE_{\BT{a}{\beta}}) 
	= \NC{a}{\mu}.
\end{equation}

\printbibliography

\end{document}








