%%%%%%%%%%%%%%%%%%%%%%%%%%%%%%%%%%%%%%%%%
% Arsclassica Article
% LaTeX Template
% Version 1.1 (1/8/17)
%
% This template has been downloaded from:
% http://www.LaTeXTemplates.com
%
% Original author:
% Lorenzo Pantieri (http://www.lorenzopantieri.net) with extensive modifications by:
% Vel (vel@latextemplates.com)
%
% License:
% CC BY-NC-SA 3.0 (http://creativecommons.org/licenses/by-nc-sa/3.0/)
%
%%%%%%%%%%%%%%%%%%%%%%%%%%%%%%%%%%%%%%%%%

%----------------------------------------------------------------------------------------
%	PACKAGES AND OTHER DOCUMENT CONFIGURATIONS
%----------------------------------------------------------------------------------------

\documentclass[
10pt, % Main document font size
a4paper, % Paper type, use 'letterpaper' for US Letter paper
oneside, % One page layout (no page indentation)
%twoside, % Two page layout (page indentation for binding and different headers)
twocolumn,
headinclude,footinclude, % Extra spacing for the header and footer
BCOR5mm, % Binding correction
]{scrartcl}

%%%%%%%%%%%%%%%%%%%%%%%%%%%%%%%%%%%%%%%%%
% Arsclassica Article
% Structure Specification File
%
% This file has been downloaded from:
% http://www.LaTeXTemplates.com
%
% Original author:
% Lorenzo Pantieri (http://www.lorenzopantieri.net) with extensive modifications by:
% Vel (vel@latextemplates.com)
%
% License:
% CC BY-NC-SA 3.0 (http://creativecommons.org/licenses/by-nc-sa/3.0/)
%
%%%%%%%%%%%%%%%%%%%%%%%%%%%%%%%%%%%%%%%%%

%----------------------------------------------------------------------------------------
%	REQUIRED PACKAGES
%----------------------------------------------------------------------------------------

\usepackage[
nochapters, % Turn off chapters since this is an article        
beramono, % Use the Bera Mono font for monospaced text (\texttt)
eulermath,% Use the Euler font for mathematics
pdfspacing, % Makes use of pdftex’ letter spacing capabilities via the microtype package
dottedtoc % Dotted lines leading to the page numbers in the table of contents
]{classicthesis} % The layout is based on the Classic Thesis style
\usepackage{hyperref}

\usepackage{arsclassica} % Modifies the Classic Thesis package

\usepackage[T1]{fontenc} % Use 8-bit encoding that has 256 glyphs

\usepackage[utf8]{inputenc} % Required for including letters with accents

\usepackage{graphicx} % Required for including images
\graphicspath{{Figures/}} % Set the default folder for images

\usepackage{enumitem} % Required for manipulating the whitespace between and within lists

\usepackage{lipsum} % Used for inserting dummy 'Lorem ipsum' text into the template

\usepackage{subfig} % Required for creating figures with multiple parts (subfigures)

\usepackage{amsmath,amssymb,amsthm} % For including math equations, theorems, symbols, etc

\usepackage{varioref} % More descriptive referencing

\usepackage{accents}


%----------------------------------------------------------------------------------------
%	THEOREM STYLES
%---------------------------------------------------------------------------------------

\theoremstyle{definition} % Define theorem styles here based on the definition style (used for definitions and examples)
\newtheorem{definition}{Definition}

\theoremstyle{plain} % Define theorem styles here based on the plain style (used for theorems, lemmas, propositions)
\newtheorem{theorem}{Theorem}

\theoremstyle{remark} % Define theorem styles here based on the remark style (used for remarks and notes)

%----------------------------------------------------------------------------------------
%	HYPERLINKS
%---------------------------------------------------------------------------------------

\hypersetup{
%draft, % Uncomment to remove all links (useful for printing in black and white)
colorlinks=true, breaklinks=true, bookmarks=true,bookmarksnumbered,
urlcolor=webbrown, linkcolor=RoyalBlue, citecolor=webgreen, % Link colors
pdftitle={}, % PDF title
pdfauthor={\textcopyright}, % PDF Author
pdfsubject={}, % PDF Subject
pdfkeywords={}, % PDF Keywords
pdfcreator={pdfLaTeX}, % PDF Creator
pdfproducer={LaTeX with hyperref and ClassicThesis} % PDF producer
}


%----------------------------------------------------------------------------------------
%	BIBLATEX
%---------------------------------------------------------------------------------------

\usepackage[backend=bibtex,giveninits=true,url=false,doi=true,eprint=true,isbn=false,
backref,backrefstyle=none,maxbibnames=99]{biblatex}
\DefineBibliographyStrings{english}{%
  backrefpage = {Cited on p\adddot},%
  backrefpages = {Cited on pp\adddot}%
}

\bibliography{library}

\renewcommand*{\bibfont}{\footnotesize}

% in order to suppress 'In:'
\renewbibmacro{in:}{%
  \ifboolexpr{%
     test {\ifentrytype{article}}%
  }{}{\printtext{\bibstring{in}\intitlepunct}}%
}

%----------------------------------------------------------------------------------------
% these commands allow to put equations in a fancy boxes:
%----------------------------------------------------------------------------------------
\usepackage{empheq}
\newlength\mytemplen
\newsavebox\mytempbox
\makeatletter
\definecolor{cream}{rgb}{.81, .88, 1}
 \newcommand\mycreambox{%
     \@ifnextchar[%]
        {\@mycreambox}%
        {\@mycreambox[0pt]}}
 \def\@mycreambox[#1]{%
     \@ifnextchar[%]
        {\@@mycreambox[#1]}%
        {\@@mycreambox[#1][0pt]}}
 \def\@@mycreambox[#1][#2]#3{
     \sbox\mytempbox{#3}%
     \mytemplen\ht\mytempbox
     \advance\mytemplen #1\relax
     \ht\mytempbox\mytemplen
     \mytemplen\dp\mytempbox
     \advance\mytemplen #2\relax
     \dp\mytempbox\mytemplen
     \colorbox{cream}{\hspace{1em}\usebox{\mytempbox}\hspace{1em}}}
 \makeatother % Include the structure.tex file which specified the document structure and 
%layout

\usepackage[hmarginratio=1:1,top=25mm,left=17mm,columnsep=20pt]{geometry}
\usepackage{relsize} % e.g. used for \mathsmaller
\usepackage{bm}
%\usepackage{showlabels}

\newcommand{\ERWBB}{{ERWBB}}
\newcommand{\xx}{\mathbf{x}}
\newcommand{\XX}{\mathbf{X}}
\newcommand{\XXX}{\mathbb{X}}
\newcommand{\diff}{\mathrm{d}}
\newcommand{\Id}{\mathbf{I}}
\newcommand{\Tr}{\mathrm{Tr}}
\newcommand{\CC}{\mathbf{C}}
\newcommand{\mm}{\mathbf{m}}
\newcommand{\vv}{\mathbf{v}}
\newcommand{\MM}{\mathbf{M}}
\newcommand{\OBig}{\mathcal{O}}
\newcommand{\FF}{\mathbf{F}}
\renewcommand{\AA}{\mathbf{A}}
\newcommand{\BB}{\mathbf{B}}
\newcommand{\qq}{\mathbf{q}}
\newcommand{\QQ}{\mathbf{Q}}
\newcommand{\LL}{\mathbf{L}}
\newcommand{\Lie}{\mathfrak{L}}

\newcommand{\MP}[1]{{\color{Green}MP:\ \ #1}}
\newcommand{\IP}[1]{{\color{Red}[IP:\ \ #1]}}
\newcommand{\MH}[1]{{\color{Red}MH:\ \ #1}}

\newcommand{\sA}{\mathsmaller A}
\newcommand{\sB}{\mathsmaller B}
\newcommand{\sC}{\mathsmaller C}
\newcommand{\sD}{\mathsmaller D}
\newcommand{\sE}{\mathsmaller E}
\newcommand{\sM}{\mathsmaller M}
\newcommand{\sN}{\mathsmaller N}
\newcommand{\sL}{\mathsmaller L}
\newcommand{\sr}{\mathsmaller r}

\newcommand{\pd}[1]{\partial_{#1}}
\newcommand{\F}[2]{F^{\ #1}_{\mathsmaller#2}}
\newcommand{\hatF}[2]{\hat{F}^{\ #1}_{\mathsmaller#2}}
\newcommand{\A}[2]{A^{\mathsmaller#1}_{\ #2}}
\newcommand{\mg}[1]{\kappa_{#1}}			% Minkowski metric
\newcommand{\MG}[1]{\kappa^{#1}}			% inverse Minkowski metric

\newcommand{\pdd}[1]{{\bm{\partial}_{#1}}}
\newcommand{\dx}[1]{{\bm{\mathrm{d}x}^{#1}}}


\newcommand{\tetrsymbol}{h}
\newcommand{\itetrsymbol}{\eta}
\newcommand{\itetr}[2]{\itetrsymbol^{#1}_{\phantom{#1}#2}}
\newcommand{\tetr}[2]{\tetrsymbol^{#1}_{\phantom{#1}#2}}
\newcommand{\stress}[2]{s_{\ #1}^{#2}}
\newcommand{\stressalt}[2]{\hat{s}_{\ #1}^{#2}}
\newcommand{\detTetr}{\tetrsymbol}
\newcommand{\rtetr}[2]{h^{#1}_{\mathsmaller{(r)} #2}}
\newcommand{\spin}[2]{\omega^{#1}_{\phantom{#1}#2}}
\newcommand{\Lor}[2]{\Lambda^{#1'}_{\phantom{#1}#2}}
\newcommand{\iLor}[2]{\Lambda^{#1}_{\phantom{#1}#2'}}
\newcommand{\cobas}[1]{\bm{\tetrsymbol}^{#1}}
\newcommand{\bas}[1]{\bm{\itetrsymbol}_{#1}}

%\newcommand{\D}[1]{\mathcal{D}_{#1}} % Fock-Ivanencko cov derivative
\newcommand{\D}[1]{\partial_{#1}} % Fock-Ivanencko cov derivative
\newcommand{\DW}[1]{\mathcal{D}_{#1}} % Fock-Ivanencko cov derivative
\newcommand{\Tors}[2]{T^{#1}_{\phantom{#1}#2}}
\newcommand{\Supp}[2]{S_{#1}^{\phantom{a}#2}}	%supepotential
\newcommand{\Torsl}[1]{T_{#1}}
\newcommand{\ET}[2]{E^{#1}_{\phantom{#1}#2}}	%Torsion decomposition, analog of Electric field
\newcommand{\ETmix}[2]{E^{#1}_{#2}}	%Torsion decomposition, analog of Electric field
\newcommand{\Dm}[2]{D_{\phantom{#2}#1}^{#2}}	%Torsion decomposition, analog of Electric field
\newcommand{\aD}[2]{\mathcal{D}_{\phantom{#2}#1}^{#2}}	%Torsion decomposition, analog of 
\newcommand{\Dfin}[2]{\mathtt{D}_{\phantom{#2}#1}^{#2}}	%Torsion decomposition, analog of 
\newcommand{\Hfin}[2]{\mathtt{H}_{#2#1}}	%Torsion decomposition, analog of 
\newcommand{\Hfinmix}[2]{\mathtt{H}^{#2}_{\phantom{#2}#1}}	%Torsion decomposition, analog of 
\newcommand{\Hfinnmix}[2]{\hat{\mathtt{H}}^{#2}_{\ #1}}	%Torsion decomposition, analog of 
\newcommand{\Efin}[2]{\mathtt{E}^{#1}_{\phantom{#1}#2}}	%Torsion decomposition, analog of 
\newcommand{\Ufin}{\mathtt{U}}
\newcommand{\Hfinn}[2]{\hat{\mathtt{H}}_{{#2}{#1}}}	%Torsion decomposition, analog of 
\newcommand{\Efinn}[2]{\hat{\mathtt{E}}^{#1}_{\ #2}}	%Torsion decomposition, analog of 
\newcommand{\Kbuch}[2]{\mathtt{K}_{{#1}{#2}}}	%Torsion decomposition, analog of 
\newcommand{\Nbuch}[2]{\mathtt{N}_{#1}^{\ \,#2}}	%Torsion decomposition, analog of 
\newcommand{\Nbuchdown}[2]{\mathtt{N}_{#1#2}}	%Torsion decomposition, analog of 
\newcommand{\Kbuchmix}[2]{\mathtt{K}^{#1}_{\ #2}}	%Torsion decomposition, analog of 
\newcommand{\Nbuchmix}[2]{\mathtt{N}^{#1}_{\ #2}}	%Torsion decomposition, analog of 
%%%Electric 
%field
\newcommand{\BT}[2]{B^{#1#2}}	%Torsion decomposition, analog of Magnetic field
\newcommand{\BTmix}[2]{B^{#1}_{#2}}	%Torsion decomposition, analog of Electric field
\newcommand{\Bmmix}[2]{B^{#1}_{#2}}	%Torsion decomposition, analog of Electric field
\newcommand{\Bm}[2]{B^{#1#2}}	%Torsion decomposition, analog of Magnetic field
\newcommand{\aB}[2]{\mathcal{B}^{#1#2}}	%Torsion decomposition, analog of Magnetic field
\newcommand{\Bfin}[2]{\mathtt{B}^{#1#2}}	%Torsion decomposition, analog of Magnetic field
\newcommand{\Bfinmix}[2]{\mathtt{B}^{#1}_{\phantom{#1}#2}}	%Torsion decomposition, analog of 
\newcommand{\Bfinmixx}[2]{\mathtt{B}_{#1}^{\phantom{#1}#2}}	%Torsion decomposition, analog of 
%Magnetic field
%\newcommand{\aBmix}[2]{\mathcal{B}^{#1}_{#2}}
\newcommand{\hT}[2]{H_{#1#2}}	%dual to B
\newcommand{\W}[2]{\mathcal{W}^{#1}_{\phantom{#1}#2}}
\newcommand{\w}[2]{W^{#1}_{\phantom{#1}#2}}
\newcommand{\FI}{Fock-Ivanenko}
\newcommand{\We}{Weitzenb\"ock}
%\newcommand{\Lag}{\mathcal{L}}	% Lagrangian which depends on ordinary derivatives
\newcommand{\Lag}{\Lambda}	% Lagrangian which depends on ordinary derivatives
\newcommand{\Lagcov}{\pounds}% Lagrangian which depends on gauge covariant derivatives
\newcommand{\Laghodge}{L}% Lagrangian which depends on the Hodge dual of the torsion
\newcommand{\Lagtors}{\mathfrak{L}}% Lagrangian which depends on torsion
\newcommand{\LagBE}{\mathcal{L}}% Lagrangian which depends on the B and E fields
\newcommand{\Um}{U}% Final spacetime potetnial
\newcommand{\aU}{\mathcal{U}}% potential for 3+1
\newcommand{\veps}{\varepsilon}
\newcommand{\EM}[2]{\Sigma^{#1}_{\phantom{#1}#2}}
\newcommand{\EMmat}[2]{\sigma^{#1}_{\ \,#2}}
\newcommand{\EMgrav}[2]{t^{#1}_{\phantom{#1}#2}}
\newcommand{\LCsymb}{\bm{\in}}    % Levi-Civita symbol (tensor-density)
\newcommand{\LCtens}{\varepsilon} % Levi-Civita ordinary tensor
\newcommand{\rhs}[1]{f_{#1}}
\newcommand{\mat}[1]{\prescript{\text{(m)}}{}{\hspace{-0.1cm}#1}}
\newcommand{\gra}[1]{\prescript{\text{(g)}}{}{\hspace{-0.1cm}#1}}

\newcommand{\tegr}{TEGR}
\newcommand{\HDT}[1]{\accentset{\star}{T}^{#1}}
\newcommand{\HDmix}{\accentset{\star}{T}}
\newcommand{\KD}[2]{\delta^{#1}_{\ #2}}
\newcommand{\NC}[2]{J^{#2}_{\phantom{#2}#1}}
\newcommand{\nc}[2]{j^{\phantom{#1}#2}_{#1}}
\newcommand{\indalg}[1]{\hat{\mathsmaller{#1}}}
\newcommand{\aE}[2]{\mathtt{E}^{#1}_{\phantom{#1}#2}}
\newcommand{\aH}[2]{\mathtt{H}_{#1#2}}

\newcommand{\TorsConj}[2]{\mathbb{T}_{#1}^{\phantom{#1}#2}}
\newcommand{\HTConj}[1]{\accentset{\star}{\mathbb{T}}_{#1}}
\newcommand{\Dbb}[2]{\mathbb{D}_{#1}^{\phantom{#1}#2}}
\newcommand{\Hbb}[2]{\mathbb{H}_{#1#2}}
\newcommand{\lapse}{\alpha}
\newcommand{\shift}[1]{\beta^{#1}}
\newcommand{\Tscal}{\mathcal{T}}		% torsion scalar
\newcommand{\projector}[2]{\Delta^{#1}_{\ #2}}
\newcommand{\Hscal}{\mathcal{H}}		% torsion scalar


\newcommand{\ho}[1]{\textcolor{magenta}{HO: #1}}

\hyphenation{Fortran hy-phen-ation} % Specify custom hyphenation points in 
%words with dashes where 
%you would like hyphenation to occur, or alternatively, don't put any dashes in a word to stop 
%hyphenation altogether




%----------------------------------------------------------------------------------------
%	TITLE AND AUTHOR(S)
%----------------------------------------------------------------------------------------

\title{\large\normalfont\spacedallcaps{
		First-order 
		hyperbolic 
		formulation of the \\
		%pure tetrad 
		teleparallel gravity theory}} % The article 
%title

%\subtitle{Subtitle} % Uncomment to display a subtitle

\author{\normalsize\textsc{Ilya Peshkov}$^{*,1}$,\ 
	\normalsize\textsc{Héctor Olivares}$^{2}$
	\& 
	\normalsize\textsc{Evgeniy Romenski}$^{3}$
	%\normalsize\textsc{Michael Dumbser}$^{1}$ \ldots
} % The article author(s) - author affiliations 
%need to be 
%specified in the 
%AUTHOR AFFILIATIONS block

\date{\small\today} % An optional date to appear under the author(s)

%----------------------------------------------------------------------------------------

\begin{document}
	
	%----------------------------------------------------------------------------------------
	%	HEADERS
	%----------------------------------------------------------------------------------------
	
	\renewcommand{\sectionmark}[1]{\markright{\spacedlowsmallcaps{#1}}} % The header for all pages 
	%(oneside) or for even pages (twoside)
	%\renewcommand{\subsectionmark}[1]{\markright{\thesubsection~#1}} % Uncomment when using the 
	%%twoside option - this modifies the header on odd pages
	\lehead{\mbox{\llap{\small\thepage\kern1em\color{halfgray} 
				\vline}\color{halfgray}\hspace{0.5em}\rightmark\hfil}} % The header style
	
	\pagestyle{scrheadings} % Enable the headers specified in this block
	
	%----------------------------------------------------------------------------------------
	%	TABLE OF CONTENTS & LISTS OF FIGURES AND TABLES
	%----------------------------------------------------------------------------------------
	
	\maketitle % Print the title/author/date block
	
	\setcounter{tocdepth}{2} % Set the depth of the table of contents to show sections and 
	%subsections 
	%only
	
	%\tableofcontents % Print the table of contents
	
	% \listoffigures % Print the list of figures
	
	% \listoftables % Print the list of tables
	
	%----------------------------------------------------------------------------------------
	%	ABSTRACT
	%----------------------------------------------------------------------------------------
	
	\section*{Abstract} % This section will not appear in the table of contents due to the star 
	% (\section*)
	\noindent
	\textit{
	Driven by the need for numerical solutions to the Einstein field equations,
	we derive a first-order reduction of the second-order $ f(T) $-teleparallel
	gravity equations in the pure-tetrad formulation (no spin connection).
	We then restrict our attention to the teleparallel equivalent of general
	relativity (TEGR) and propose a 3+1 decomposition of these equations
	suitable for computational implementation. We demonstrate that in vacuum 	(matter-free spacetime) the obtained system of first-order equations is
	equivalent to the tetrad reformulation of general relativity by Estabrook,
	Robinson, Wahlquist, and Buchman and Bardeen, and therefore also admits a
	symmetric hyperbolic formulation. However, the question of hyperbolicity of
	the 3+1 TEGR equations for arbitrary spacetimes remains unaddressed so far.
	Furthermore, the structure of the 3+1  equations resembles a lot of
	similarities with the equations of relativistic electrodynamics and the
	recently proposed dGREM tetrad-reformulation of general relativity.
	}
	%----------------------------------------------------------------------------------------
	%	AUTHOR AFFILIATIONS
	%----------------------------------------------------------------------------------------
	%\let\thee\relax\footnotetext{* \textit{ilya.peshkov@unitn.it}}
	%\let\thefootnote\relax\footnotetext{\textsuperscript{1} \textit{University of Trento, Trento, 
			%Italy}}
	%\let\thefootnote\relax\footnotetext{\textsuperscript{2} \textit{Sobolev Institute of 
	%Mathematics, 
			%Novosibirsk, Russia}}
	%\let\thefootnote\relax\footnotetext{\textsuperscript{3} \textit{Novosibirsk State University, 
			%Novosibirsk, Russia}}
	\footnotetext{* Corresponding author, \textit{ilya.peshkov@unitn.it}}
	\footnotetext{\textsuperscript{1} \textit{Department of Civil, Environmental and Mechanical 
			Engineering, University of Trento, Via
			Mesiano 77, 38123, Trento, 
			Italy}}
	\footnotetext{\textsuperscript{2} \textit{
			Department of Astrophysics/IMAPP, Radboud University Nijmegen, P.O. Box 9010, NL-6500 
			GL Nijmegen, 
			Netherlands}}
	\footnotetext{\textsuperscript{3} \textit{Sobolev Institute of Mathematics, Novosibirsk, 
	Russia}}
	%\footnotetext{\textsuperscript{4} \textit{Novosibirsk State University, 
			%		Novosibirsk, Russia}}
	\renewcommand{\thefootnote}{\arabic{footnote}}
	%----------------------------------------------------------------------------------------
	
	%\newpage % Start the article content on the second page, remove this if you have a longer 
	%%abstract 
	%that goes onto the second page
	
	% PARAGRAPH OPTIONS:
	\setlength\parindent{10pt} % sets indent to zero
	\setlength{\parskip}{5pt} % changes vertical space between paragraphs
	% PARAGRAPH OPTIONS.
	
	%----------------------------------------------------------------------------------------
	%	INTRODUCTION
	%----------------------------------------------------------------------------------------
	
	\section{Introduction}
	
	The class of teleparallel gravity theories is one of the alternative
	reformulations of Einstein's general relativity (GR)
	\cite{Hehl1976,AldrovandiPereiraBook,Cai2016}. While for GR gravitational
	interaction is a manifestation of the curvature of a torsion-free spacetime,
	for the teleparallel framework it is realized as a curvature-free linear
	connection with non-zero torsion (or/and non-zero nonmetricity which is not
	considered in this paper). Although different variables can be taken as the
	main dynamical fields in GR (tetrad fields, soldering forms, etc.), the most
	extended choice is the metric tensor accompanied by the Levi-Civita
	connection. In contrast, for teleparallel theory the metric is trivial and
	the main dynamical field is usually the space-time tetrad (or frame) field.
	%\ho{I changed the first paragraph. Using the metric and Levi-Civita connection
		%as main dynamical fields is not a feature of GR, but just the result of expressing
		%it in a coordinate basis.
		%There are infinite possible connections for GR, some with trivial metric or trivial
		%tetrad fields, but the features that really define GR are interpreting gravity as curvature
		%and making the torsion zero.}
	
	Despite the teleparallel geometries are considered as an alternative
	framework\footnote{Yet, it is guaranteed that simplest realizations of teleparallel geometries, such as TEGR for example, produce all the
	classical results of general relativity
	\cite{AldrovandiPereiraBook,Bahamonde2021a}.} to Einstein's gravity with
	several promising features missing in GR, e.g. see the discussion in
	\cite[Sec.18]{AldrovandiPereiraBook} and \cite{Cai2016}, in this paper, we
	are interested in the teleparallel gravity only from a pure computational
	viewpoint and our goal is to use its mathematical structure to develop an
	efficient computational framework for numerical relativity. Thus, the main
	goal of this paper is to derive a 3+1-split of the so-called
	\emph{teleparallel equivalent of general relativity} (TEGR)
	\cite{AldrovandiPereiraBook,Krssak2019} which is known to pass all standard
	tests of GR. Up to now, not many attempts have been done to obtain a $ 3+1
	$ formulation of the TEGR equations, e.g. \cite{Capozziello2021,Pati2022}. A
	Hamiltonian formulation of TEGR was used in \cite{Pati2022} to obtain
	evolution equations for the tetrads and conjugate momenta. In
	\cite{Capozziello2021}, the spatial tetrad and their first order Lie
	derivative along the normal vector to the foliations were chosen as the
	state variables. Here, we explore another line of deriving 3+1
	formulation of TEGR which is rather aligned with the relativistic
	electrodynamics. As the result, the obtained equations are complitelly
	different from the mentioned papers.

	A second objective of this paper is to develop a 3+1 formulation entirely
	independent of the Einstein's theory, i.e. our development avoids
	foundational GR tools like the Levi-Civita connection and the
	Hilbert-Einstein action. Instead, we start from an arbitrary Lagrangian
	density which is a function of the torsion scalar and obtain corresponding
	Euler-Lagrange equations coupled with some constraints (differential
	identities). The obtained equations are then reduced to the first-order form
	and the 3+1 decomposition is performed. However, despite this independent
	route, we demonstrate that the resulting 3+1 equations are equivalent to
	tetrad reformulations of the GR such as the one by
	Estabrook-Robinson-Wahlquist \cite{Estabrook1997} and Buchman-Bardeen
	\cite{Buchman2003} (the \ERWBB\ formulation for brevity), and recent dGREM formulation of GR \cite{Olivares2022}. 
	
	From the numerical view point, any 3+1 formulation has to have the
	well-posed initial value problem (i.e. a solution exists, the solution is
	unique and changes continuously with changes in the initial data) in order
	to compute stable evolution of the numerical solution. In other words, the
	system of governing equations has to be hyperbolic\footnote{Note that in the
	numerical relativity the term ``strong hyperbolicity'' is used emphasizing
	that not only eigenvalues must be real but that the full basis of
	eigenvectors must exist}. Therefore, the third objective of the paper is to
	test if the proposed 3+1 formulation of TEGR is hyperbolic. As is the
	case with other first-order reductions of the Einstein equations
	\cite{Baumgarte2003a}, the question of hyperbolicity of 3+1 TEGR
	equations considered here is not trivial and depends on the delicate use of
	multiple involution constraints (stationary identities), e.g. see
	\cite{FO-CCZ4}. In particular, for a vacuum space-time, we have found that
	the proposed 3+1 TEGR equations (if written in the tetrad frame) are
	equivalent to ERWBB \cite{Estabrook1997,Buchman2003} which is known to by
	symmetric hyperbolic for a certain choice of gauge conditions.
	
	We note that we do not consider the most general TEGR formulation as for
	example presented in \cite{AldrovandiPereiraBook}. The linear connection of
	TEGR is the sum of the two parts: the \We\ connection (which is the
	historical connection of TEGR) and the spin connection (parametrized by
	Lorentz matrices) representing the inertial content of the tetrad. The spin
	connection is necessary to separate the inertial effect of a chosen frame
	from its gravity content as well as for establishing the full covariance of
	the theory \cite{AldrovandiPereiraBook,Golovnev2017a,Krssak2019}, i.e. with
	respect to both the diffeomorphisms of the spacetime and the Lorentz
	transformations of tangent spaces. However, being important from the
	theoretical viewpoint and for extensions of teleparallel gravity
	\cite{Golovnev2017a}, the spin connection of TEGR introduces extra degrees
	of freedom which do not have evolution equations and therefore can be
	treated as parameters (not state variables) of the theory. Since we are
	interested in developing a computational framework for GR, the consideration
	of this paper is, therefore, restricted to the frames for which the spin
	connection is set to zero globally (\We\ gauge). We thus consider TEGR in
	its historical, or \emph{pure tetrad}, formulation \cite{Golovnev2017a}.
	
	
	Interestingly that another motivation to study the mathematical structure of
	the teleparallel gravity is coming from the continuum fluid and solid
	mechanics. In particular, the role of the torsion to describe defects in
	solids has been known for decades now, e.g. see
	\cite{VolovichKatanaev1992,Hehl2007,Yavari2012,NguyenLeMarrec2022,Bohmer2020,Lychev2022}.
	Moreover, the material tetrad field (called also the distortion field in our
	papers) is the key field for the unified hyperbolic model of fluid and solid
	mechanics \cite{HPR2016,DPRZ2016}. In such a theory, the concept of torsion
	can be connected with the inertial effect of small-scale eddies in turbulent
	flows and with dispersion effects in heterogeneous solids (e.g. acoustic
	metamaterials) as discussed in \cite{Torsion2019}. Interestingly, that the $
	3+1 $ equations we obtain in this paper resemble very closely the structure
	of the equations for continuum fluid and solid mechanics with torsion
	\cite{Torsion2019}. Furthermore, the unified theory of fluids and solids has
	been also extended in the general relativistic settings \cite{PTRSA2020} and
	therefore, as being a tetrad theory by its nature, it can be
	straightforwardly coupled with the 3+1 TEGR equations discussed in this
	paper.
	
	\section{Definitions}
	
	\subsection{Non-holonomic frame field}
	
	Throughout this paper, we use the following index convention. Greek letters $ \alpha, 
	\beta\,\gamma$, 
	$\ldots, \lambda,\mu,\nu,... 
	=0,1,2,3
	$ are used to index quantities related to the spacetime manifold, the Latin letters from the 
	beginning of the alphabet $ a,b,c,... 
	=\hat{0},\hat{1},\hat{2},\hat{3}$ are used to index quantities related to the tangent Minkowski 
	space.
	In the 3+1 split, the letters $ i,j,k,\ldots =1,2,3$ from the middle of the Latin alphabet 
	are 
	used 
	to denote spatial components of the spacetime tensors, and the upper case Latin letters $ 
	\sA,\sB,\sC,\ldots=\indalg{1},\indalg{2},\indalg{3} $ index spatial components of the tensors 
	written in a chosen frame of the 
	tangent space.
	
	
	
	We shall consider a spacetime manifold $ M $ equipped with a general Riemannian metric $ 
	g_{\mu\nu} 
	$ and a coordinate system $ x^\mu $. At each 
	point of 
	the spacetime, there is the tangent space $ T_{x}M $ spanned by the frame (or tetrad) $ 
	\pdd{\mu} $ which is the standard coordinate basis. 
	There is also the cotangent space $ T_x^*M $ spanned by the coframe field $ \dx{\mu} $. 
	Recall that the frames $ \pdd{\mu} $ and $ \dx{\mu} $ are \emph{holonomic} 
	\cite{AldrovandiPereiraBook}.
	
	%\ho{I have always seen the tangent space $ T_{x}M $ spanned by the basis vectors
		%$ \pdd{\mu} $ and the cotangent space $ T_x^*M $ spanned by the basis 1-forms $ \dx{\mu} $.
		%However, I am not sure if it makes a difference, since I don't recall the definition of 
		%tangent
		%and co-tangent spaces. Also, in the literature, `tetrad' usually implies an orthonormal
		%basis, so it may cause some confusion to use it here for a coordinate frame.}
	
	In addition to $ T_{x}M $, we assume that at each point of $ M $, there is a soldered tangent 
	space 
	which is a Minkowski space spanned by an orthonormal frame (tetrad) $ \bas{a} $ and equipped 
	with 
	the 
	Minkowski metric 
	\begin{equation}\label{eqn.mg}
		\mg{ab} = \text{diag}(-1,1,1,1).
	\end{equation}
	Similarly, there is the 
	corresponding cotangent space spanned by the co-frames $ \cobas{a} $. It is assumed 
	that the frames $ \cobas{a} $ and $ \bas{a} $ are independent of the coordinates $ x^\mu $ and, 
	therefore, are non-holonomic in general.
	
	The components of the non-holonomic frames $ \cobas{a} $ and $ \bas{a} $ in the holonomic 
	frames $ \dx{\mu} $ and $ \pdd{\mu} $ are denoted by $ \tetr{a}{\mu} $ and $ \itetr{\mu}{a} $, 
	i.e. 
	\begin{equation}
		\cobas{a} = \tetr{a}{\mu}\dx{\mu}, \qquad \text{or} \qquad \bas{a} = \itetr{\mu}{a}\pdd{\mu}
	\end{equation}
	with $ \tetr{a}{\mu} $ being the inverse of $ \itetr{\mu}{a} $, i.e.
	\begin{equation}\label{eqn.inv.tetr}
		\tetr{a}{\mu} \itetr{\mu}{b} = \KD{a}{b},
		\qquad
		\itetr{\nu}{a}\tetr{a}{\mu}  = \KD{\nu}{\mu},
	\end{equation}
	where $ \KD{a}{b} $ and $ \KD{\nu}{\mu} $ are the Kronecker deltas.
	
	The soldering of the spacetime and the tangent Minkowski space means that the metrics $ 
	g_{\mu\nu} 
	$ and $ \mg{ab} $ are related by  
	\begin{equation}
		g_{\mu\nu} = \mg{ab} \tetr{a}{\mu}\tetr{b}{\nu}.
	\end{equation}
	
	%\ho{Using the notation $\eta$ for the transformation coefficients can be
		%confusing, since in a lot of the literature $\eta_{\mu\nu}$ is used
		%for the Minkowski metric (e.g. Misner-Thorne-Wheeler and Aldrovandi \& Pereira).}
	
	Note that 
	\begin{equation}\label{eqn.det}
		\detTetr := \det(\tetr{a}{\mu}) = \sqrt{-g},
	\end{equation}
	if $ g = \det(g_{\mu\nu}) $.
	
	\subsection{Observer's 4-velocity}
	
	Observer's 4-velocity is associated with the $ 0 $-\textit{th} vector of the tetrad basis 
	\begin{equation}\label{eqn.4v}
		u^\mu := \itetr{\mu}{\indalg{0}}, \qquad u_\mu = g_{\mu\nu}u^\nu.
	\end{equation}
	Also, due to
	\begin{equation}
		u_\mu = g_{\mu\nu} u^\nu = \mg{ab}\tetr{a}{\mu}\tetr{b}{\nu}\itetr{\nu}{\indalg{0}} = 
		\mg{a\indalg{0}}\tetr{a}{\mu} = -\tetr{\indalg{0}}{\mu},
	\end{equation}\label{eqn.4v.cov}
	the covariant components of the 4-velocity equal to entries of the $ 0 $-\textit{th} vector of 
	the 
	co-basis with the opposite sign
	\begin{equation}
		u_\mu = -\tetr{\indalg{0}}{\mu}.
	\end{equation}
	%
	%Of course, we may need these velocities expressed in the frames themselves:
	%\begin{subequations}\label{eqn.4v.Lagr}
	%	\begin{gather}\label{eqn.4v.a}
		%		u^a := u^\mu \tetr{a}{\mu} = \itetr{\mu}{\indalg{0}}\tetr{a}{\mu} = 
		%\KD{a}{\indalg{0}} 
		%= 
		%		(1,0,0,0),
		%		\\[2mm]
		%		u_a := u_\mu \itetr{\mu}{a} =-\tetr{\indalg{0}}{\mu}\itetr{\mu}{a} 
		%=-\KD{a}{\indalg{0}} 
		%= 
		%		(-1,0,0,0).\label{eqn.4v.b}
		%	\end{gather}
	%\end{subequations}
	
	
	
	
	\subsection{Connection and torsion}
	
	Because we work in the framework of the pure-tetrad formulation of TEGR
	(\We\ gauge), the linear connection is set to the pure \We\
	connection\footnote{Note that we use a different convention on the
	positioning of the lower indices of the \We\ connection $ \w{a}{\mu\nu} =
	\pd{\mu}\tetr{a}{\nu} $ than in \cite{AldrovandiPereiraBook}. Precisely, the
	derivative index goes first.}
	\cite{AldrovandiPereiraBook,KleinertMultivalued}: 
	\begin{equation}\label{eqn.We}
		\w{a}{\mu\nu} := \pd{\mu}\tetr{a}{\nu}, 
		\qquad
		\text{or}
		\qquad
		\w{\lambda}{\mu\nu} := \itetr{\lambda}{a}\pd{\mu}\tetr{a}{\nu}.
	\end{equation}
	The torsion is then defined as
	\begin{equation}\label{eqn.def.tors}
		\Tors{a}{\mu\nu}:=\D{\mu}\tetr{a}{\nu} - \D{\nu}\tetr{a}{\mu} = 
		\w{a}{\mu\nu} - \w{a}{\nu\mu}.
	\end{equation}
	
	Note that while the spacetime derivatives commute
	\begin{align}\label{eqn.commut.D}
		\D{\mu}(\D{\nu} V^\lambda) - \D{\nu}(\D{\mu} V^\lambda) &= 0, 
		\\[2mm] 
		\D{\mu}(\D{\nu} V^a) - \D{\nu}(\D{\mu} V^a) &= 0,
	\end{align}
	their tangent space counterparts $\D{a} =  \itetr{\mu}{a}\D{\mu}$ do not (for non-vanishing 
	torsion)
	\begin{equation}
		\D{b}(\D{c} V^a) - \D{c}(\D{b} V^a) = 
		-\Tors{d}{b c}\D{d}V^a,
	\end{equation}
	where $  \Tors{d}{bc} = \Tors{d}{\mu\nu}\itetr{\mu}{b}\itetr{\nu}{c} $.
	
	%\subsection{Reference tetrad}
	%
	%We also define the \textit{reference} tetrad $ \rtetr{a}{\mu}$ as the one for which 
	%the torsion 
	%vanishes
	%\begin{equation}
	%\Tors{a}{\mu\nu}(\rtetr{a}{\mu},\spin{a}{\mu c}) = 0
	%\end{equation}
	%which means that this tetrad does not contain any gravity effect but only inertia.
	
	
	\subsection{Levi-Civita symbol (tensor density)}
	
	%This 
	%\href{https://physics.stackexchange.com/questions/429434/lorentz-covariant-derivative-of-the-vielbein-determinant}{discussion}
	% is very relevant.
	
	We shall also need the Levi-Civita symbol (tensor-density\footnote{We 
		use the 
		sign convention for the tensor density weight according to \cite{Ryder2009,Grinfeld2013}, 
		i.e.
		under a 
		general 
		coordinate change $ x^\mu \to x^{\mu'} $ the determinant $ \det(\tetr{a}{\mu}) = \detTetr $ 
		transforms as $ \detTetr' = \det \left(\frac{\partial x^\mu}{\partial x^{\mu'}} \right)^W 
		\cdot \detTetr $ with $ W=+1 $. Therefore, the tetrad's determinant $ \detTetr $ and the 
		square root of the metric determinant $ \detTetr = \sqrt{-g} $ have weights $ +1 $, as well 
		as 
		the Lagrangian density in the action integral.} of weight $ +1 $)
	\begin{equation}\label{eqn.LCsymbol.def}
		\LCsymb^{\lambda\mu\nu\rho} = 
		\left\{ 
		\begin{array}{ll}
			+1,	& \text{if \ }\lambda\mu\nu\rho \text{ is an even permutation of } 0123,\\[2mm]
			-1,	& \text{if \ }\lambda\mu\nu\rho \text{ is an odd \ permutation of } 0123,\\[2mm]
			\phantom{-}0,	& \text{otehrwise}.
		\end{array}
		\right.
	\end{equation}
	Its covariant components $ \LCsymb_{\lambda\mu\nu\rho} $ define a tensor density of weight $ -1 
	$ 
	with the reference value $ \LCsymb_{0123} = -1 $. One could define an absolute 
	Levi-Civita 
	contravariant $ \LCtens^{\lambda\mu\nu\rho} = h^{-1} \LCsymb^{\lambda\mu\nu\rho} $ 
	and covariant $ \LCtens_{\lambda\mu\nu\rho} = h \LCsymb_{\lambda\mu\nu\rho} $ ordinary tensors  
	but for our further considerations (see Sec.\,\ref{sec.PDEs}), it is important that 
	the derivatives $ \D{\sigma}\LCsymb^{\lambda\mu\nu\rho} $ vanish:
	\begin{equation}\label{eqn.diff.LCsymb}
		\D{\sigma}\LCsymb^{\lambda\mu\nu\rho} = 0,
	\end{equation}
	whereas for $ \LCtens^{\lambda\mu\nu\rho} $ one has
	\begin{multline}\label{eqn.diff.LeviCivita}
		\D{\sigma}\LCtens^{\lambda\mu\nu\rho} = 
		\pd{\sigma}(\detTetr^{-1}\LCsymb^{\lambda\mu\nu\rho}) = 
		\LCsymb^{\lambda\mu\nu\rho}\pd{\sigma}\detTetr^{-1}   = \\[2mm] 
		-\LCsymb^{\lambda\mu\nu\rho}\detTetr^{-1}\itetr{\eta}{a}\pd{\sigma}\tetr{a}{\eta} = 
		-\LCtens^{\lambda\mu\nu\rho}\w{\eta}{\sigma\eta},
	\end{multline}
	which is not zero in general.
	
	
	
	%Similarly, we introduce the Levi-Civita tensor in the tangent Minkowski space
	%\begin{equation}
	%\LCtens^{abcd} =\frac{1}{ \sqrt{-\eta}}\LCsymb^{abcd} = \LCsymb^{abcd}, \qquad 
	%\LCtens_{abcd} = 
	%\sqrt{-\eta}\LCsymb_{abcd} = \LCsymb_{abcd}.
	%\end{equation}
	%
	%It can be straightforwardly verified that the Levi-Civita tensors $ 
	%\LCtens^{\lambda\mu\nu\rho} $ and 
	%$ \LCtens^{abcd} $ are 
	%related as
	%\begin{equation}
	%\LCtens^{abcd} = 
	%\tetr{a}{\lambda}\tetr{b}{\mu}\tetr{c}{\nu}\tetr{d}{\rho}\LCtens^{\lambda\mu\nu\rho}.
	%\end{equation}
	
	
	
	
	\section{Variational formulation}
	
	We consider a general Lagrangian (scalar-density) $ 
	\Lag(\tetr{a}{\mu},\pd{\lambda}\tetr{a}{\nu}) $ 
	of the teleparallel gravity which is a function of the frame field $ \tetr{a}{\mu} $ and its 
	first 
	gradients $ \w{a}{\lambda\nu} = \pd{\lambda}\tetr{a}{\nu} $. In what follows, we shall not 
	explicitly split $ \Lag $ 
	into the gravity (g) and matter (m) parts (unless it is explicitly mentioned otherwise), i.e. 
	\begin{equation}\label{eqn.Lagr.split}
		\Lag = \Lag^\text{(m)}(\tetr{a}{\mu},\pd{\lambda}\tetr{a}{\mu},\ldots) + 
		\Lag^\text{(g)}(\tetr{a}{\mu},\pd{\lambda}\tetr{a}{\nu}) 
	\end{equation}
	but the 
	derivation will be performed for the total unspecified Lagrangian $ \Lag $, so that in 
	principle 
	the derivation 
	can be adopted for the extensions of the teleparallel gravity such as $ f(\Tscal) 
	$-teleparallel 
	gravity, 
	where $ f(\Tscal) $ is some function of the torsion scalar $ \Tscal $, see 
	Section\,\ref{sec.closure}. We shall utilize the explicit form 
	of the TEGR Lagrangian only in the last part of the paper.
	
	Varying the action of the teleparallel gravity $ \int 
	\Lag(\tetr{a}{\mu},\pd{\lambda}\tetr{a}{\mu}) 
	\bm{\dx{}}$ 
	with respect to 
	the tetrad, one obtains the Euler-Lagrange equations 
	\begin{equation}\label{eqn.EL}
		\frac{\delta \Lambda}{\delta\tetr{a}{\mu}} = \pd{\lambda}(\Lag_{\pd{\lambda}\tetr{a}{\mu}}) 
		- 
		\Lag_{\tetr{a}{\mu}} = 0,
	\end{equation}
	where $ \Lag_{\pd{\lambda}\tetr{a}{\mu}} = \frac{\partial 
		\Lag}{\partial(\pd{\lambda}\tetr{a}{\mu})} $ and $ 
	\Lag_{\tetr{a}{\mu}} = \frac{\partial \Lag}{\partial\tetr{a}{\mu}} $. Equations \eqref{eqn.EL} 
	form a system of 16 \emph{second-order} partial differential equations for 
	16 unknowns $ \tetr{a}{\mu} $. Our goal is to replace this second-order system by an equivalent 
	but 
	larger system of only first-order partial differential equations.
	
	\section{Equivalence to GR}
	
	Before writing system \eqref{eqn.EL} as a system of first-order equations let us first make a 
	comment on the equivalence of the GR and TEGR formulations. 
	
	The Lagrangian density of TEGR (and its extensions) is formed from the torsion scalar $ \Tscal 
	$, see \eqref{eqn.TEGR.Lagr}. As it is known, e.g. see \cite[Eq.(9.30)]{AldrovandiPereiraBook}, 
	the 
	torsion scalar can be written as
	\begin{equation}\label{eqn.TR}
		h \Tscal = -\sqrt{-g} R - \pd{\mu}(2 h \Tors{\nu}{\lambda\nu}g^{\lambda
			\mu}),
	\end{equation}
	where $ R $ is the Ricci scalar and $ \Tors{\nu}{\lambda\mu} = \itetr{\nu}{a} 
	\Tors{a}{\lambda\mu} 
	$. In other words, the Lagrangians of TEGR and GR differ by the four-divergence term (surface 
	term). The latter does not affect the Euler-Lagrange equations if there are no boundaries which 
	is 
	implied in this paper. 
	Therefore, Euler-Lagrange equations of TEGR \eqref{eqn.EL} in vacuum is nothing else but the 
	Euler-Lagrange 
	equations of GR written in terms of the tetrads and hence, their physical solutions must be 
	equivalent because the information about the physical interaction is contained in the 
	Euler-Lagrange equations of a theory. What is different in GR and TEGR is the 
	way one interprets the tetrads and their first derivatives, i.e. the way one defines the linear 
	connection of the spacetime from the gradients of tetrads, e.g. torsion-free Levi-Civita 
	connection 
	of GR and curvature-free \We\ connection of TEGR. These different geometrical 
	interpretations then define extra evolution equations (\emph{compatibility 
	constraints/identities}, 
	e.g. see \eqref{integr.HT}) 
	that are merely consequences of the geometrical definitions but do not define the physics of 
	the 
	gravitational interaction. % and must be solved simultaneously with the Euler-Lagrange 
	%equations. 
	The critical point for the numerical relativity, though, is that these extra evolution 
	equations 
	must be solved simultaneously with the Euler-Lagrange equations and may affect the mathematical 
	regularity (well-posedness of the Cauchy problem) of the resulting system. 
	
	%In particular, different 
	%from the  it has 
	%appeared that, thanks to changing the geometrical interpretation of the spacetime manifold, 
	%the 
	%governing 3+1
	%TEGR equations presented in this paper have more regular mathematical structure expressed only 
	%in 
	%terms of div, grad, and curl differential operators, similar to
	%the equations of electrodynamics of 
	%moving media \cite{Obukhov2008,DPRZ2017,Hohmann2018a} and Yang-Mills gauge theories 
	%\cite{RakotomananaBook}.
	
	%The mentioned theories usually have well posed Cauchy problem 
	%
	%Other tetrad formulations \cite{Buchman2003}
	%\IP{needs to be finished...}
	
	
	\section{First-order extension}\label{sec.PDEs}
	
	Our first goal is to replace second-order system \eqref{eqn.EL} by a larger but first-order 
	system. 
	This is achieved in this section.
	
	From now on, we shall treat the frame field $ \tetr{a}{\mu} $ and its 
	gradients (the 
	\We\ 
	connection) $ 
	\pd{\lambda}\tetr{a}{\mu} $ formally as independent variables and in what 
	follows, we shall rewrite 
	system 
	of second-order PDEs \eqref{eqn.EL} as a larger system of first-order PDEs for the extended set 
	of  
	unknowns $ \{ \tetr{a}{\mu},\pd{\lambda}\tetr{a}{\mu} \} $, or actually, for their equivalents.
	
	In the setting of the teleparallel gravity, $ \Lag $ is not a function of a general combination 
	of 
	the gradients $ \pd{\lambda}\tetr{a}{\nu} $ but of their special combination, that is torsion. 
	Yet, 
	we shall employ not the torsion directly but its Hodge dual, i.e. we assume that
	
	\begin{equation}\label{eqn.Lagrangians}
		\Lag(\tetr{a}{\mu},\pd{\lambda}\tetr{a}{\nu}) = 
		\Laghodge(\tetr{a}{\mu},\HDT{a\mu\nu}),
	\end{equation}
	where $ \HDT{a\mu\nu} $ is the Hodge dual to the 
	torsion, i.e.
	\begin{subequations}
		\begin{align}\label{eqn.Hodge.def}
			\HDT{a\mu\nu} &:= \frac{1}{2}\LCsymb^{\mu\nu\rho\sigma}\Tors{a}{\rho\sigma} = 
			\LCsymb^{\mu\nu\rho\sigma}\D{\rho}\tetr{a}{\sigma}, \\[2mm] 
			\Tors{a}{\mu\nu} &= 
			-\frac{1}{2}\LCsymb_{\mu\nu\rho\sigma}\HDT{a\rho\sigma}.
		\end{align}	
	\end{subequations}
	It is important to emphasize that we deliberately chose to define the Hodge dual using the 
	Levi-Civita symbol $ 
	\LCsymb^{\lambda\mu\nu\rho} $ and \emph{not} the Levi-Civita tensor $ 
	\LCtens^{\lambda\mu\nu\rho} = 
	\detTetr^{-1} 
	\LCsymb^{\lambda\mu\nu\rho} $ that will be important later for 
	the so-called integrability condition \eqref{integr.HT}.
	Remark that according to definition \eqref{eqn.Hodge.def}, $ \HDT{a\mu\nu} $ is a  
	\emph{tensor-density} of weight $ +1 $.
	
	
	In terms of the Lagrangian density $ \Laghodge(\tetr{a}{\mu},\HDT{a\mu\nu}) 
	$, using notations 
	\eqref{eqn.Lagrangians} and definitions 
	\eqref{eqn.Hodge.def}, we can instead
	rewrite Euler-Lagrange equations \eqref{eqn.EL} as
	\begin{equation}\label{eqn.EM.Hodge}
		\D{\nu}(\LCsymb^{\mu\nu\lambda\rho}\Laghodge_{\HDT{a\lambda\rho}}) 
		=-\Laghodge_{\tetr{a}{\mu}}.
	\end{equation}
	
	The latter has to be supplemented by the integrability 
	condition
	\begin{equation}\label{integr.HT}
		\D{\nu}\HDT{a\mu\nu} = 0,
	\end{equation}
	%\IP{this is WRONG, it can't be zero!!! :( because $ 
		%\D{\nu}\LCtens^{\mu\nu\sigma\rho} \neq 0 $, see 
		%\eqref{eqn.diff.LeviCivita}, and thus, we still NEED the integrability 
		%condition. Most likely it is 
		%only possible to get it in the tangent Minkowski space!}
	which is a trivial consequence of the definition of the  Hodge dual \eqref{eqn.Hodge.def}, i.e. 
	of 
	the 
	commutativity property of the standard spacetime derivative $ \D{\mu} $, and the 
	identity \eqref{eqn.diff.LCsymb}.
	We note that if the Hodge dual was defined using the Levi-Civita tensor $ 
	\LCtens^{\mu\nu\rho\sigma} $ instead of the Levi-Civita symbol, then one would 
	have that $ \D{\mu}\HDT{a\mu\nu} \neq 0 $.
	
	Another consequence of the commutative property of $ \pd{\mu} $ and the definition of the Hodge 
	dual (based on the Levi-Civita symbol) is that the Noether energy-momentum 
	current density
	\begin{equation}\label{eqn.Noether.current}
		\NC{a}{\mu} := \Laghodge_{\tetr{a}{\mu}}
	\end{equation}
	is conserved in the ordinary sense:
	\begin{equation}\label{eqn.Noether.cons}
		\D{\mu} \NC{a}{\mu} = 0.
	\end{equation}
	If  equations \eqref{eqn.EM.Hodge}, \eqref{integr.HT} and \eqref{eqn.Noether.cons} are 
	accompanied 
	with the 
	torsion definition
	\begin{subequations}
		\begin{align}\label{eqn.tetr}
			\D{\mu}\tetr{a}{\nu} - \D{\nu}\tetr{a}{\mu} &= \Tors{a}{\mu\nu},
			\qquad
			%		\\[2mm]
			%		\qquad
			%		\D{\mu}\tetr{a}{\nu} - \D{\nu}\tetr{a}{\mu} &=-\frac12 
			%		\LCsymb_{\mu\nu\rho\sigma} \HDT{a\rho\sigma},
		\end{align}	
	\end{subequations}
	they form the following system of \emph{first-order} partial differential
	equations (only first-order derivatives are involved)
	\begin{subequations}\label{eqn.1st.order.TEGR}
		\begin{align}	
			\D{\nu}(\LCsymb^{\mu\nu\lambda\rho}\Laghodge_{\HDT{a\lambda\rho}}) 
			&=-\Laghodge_{\tetr{a}{\mu}},\label{eqn.TEGR0.EL}\\[2mm]
			%		
			\D{\nu}\HDT{a\mu\nu} & = 0,\label{eqn.TEGR0.integr}\\[2mm]
			%		
			\D{\mu}\Laghodge_{\tetr{a}{\mu}} & = 0,\label{eqn.TEGR0.enermomen}\\[2mm]
			%		
			\D{\mu}\tetr{a}{\nu} - \D{\nu}\tetr{a}{\mu} &= \Tors{a}{\mu\nu},\label{eqn.TEGR0.tetrad}
		\end{align}
	\end{subequations}
	for the unknowns $ \{\tetr{a}{\mu},\HDT{a\mu\nu}\} $.
	
	This system forms a base on which we shall build our 3+1-split of TEGR in 
	Sections\,\ref{sec.31.prep} and \ref{sec.31}.
	
	
	
	
	
	%Therefore, the first-order form of the teleparallel gravity field equations to be written in 
	%the 
	%3+1 form  
	%is
	%\begin{subequations}\label{eqn.1st.order.TEGR}
	%	\begin{empheq}[box={\mycreambox[2pt][2pt]}]{align}
		%		\D{\nu}(\LCsymb^{\mu\nu\lambda\rho}\Laghodge_{\HDT{a\lambda\rho}}) 
		%		&=-\Laghodge_{\tetr{a}{\mu}},\label{eqn.1st.order.EL}\\[2mm]
		%%		
		%		\D{\nu}\HDT{a\mu\nu} & = 0,\label{eqn.1st.order.integr}\\[2mm]
		%%		
		%			\pd{\mu}\left( 
		%		\tetr{a}{\nu} \Laghodge_{\tetr{a}{\mu}} - 2 
		%\HDT{a\lambda\mu}\Laghodge_{\HDT{a\lambda\nu}} 
		%		+ 
		%		(\HDT{a\lambda\rho}\Laghodge_{\HDT{a\lambda\rho}} - \Laghodge) \delta^\mu_{\ \nu}
		%		\right) & = 0,\label{eqn.1st.order.enermomen}\\[2mm]
		%%		
		%		\D{\mu}\tetr{a}{\nu} - \D{\nu}\tetr{a}{\mu} &= 
		%\Tors{a}{\mu\nu}.\label{eqn.1st.order.tetrad}
		%	\end{empheq}
	%\end{subequations}
	
	\section{Energy-momentum balance laws}
	
	Any conservation law written as a  4-\emph{ordinary} divergence is a
	true conservation law, meaning that it yields a time-conserved 
	``charge'' \cite{AldrovandiPereiraBook}. Hence, the Noether current $ \NC{a}{\mu} 
	=\Laghodge_{\tetr{a}{\mu}}$ is a conserved charge in the ordinary sense, see 
	\eqref{eqn.Noether.cons}, \eqref{eqn.TEGR0.enermomen}. It expresses the conservation of the 
	total\footnote{The ``total'' here 
		means the gravitational + matter/electromagnetic energy-momentum current, i.e. $ 
		\NC{\mu}{a} = \Laghodge^\text{(g)}_{\tetr{a}{\mu}} + \Laghodge^\text{(m)}_{\tetr{a}{\mu}} 
		$.} 
	energy-momentum current density. 
	
	However, its spacetime counterpart 
	\begin{equation}\label{eqn.stress}
		\EMmat{\mu}{\nu}:=\tetr{a}{\nu} L_{\tetr{a}{\mu}},
	\end{equation}
	which can be called the total energy-momentum tensor density, does not
	conserved in the ordinary sense nor in the covariant one.
	
	Indeed, after contracting with $ \tetr{a}{\nu} $ and adding to it $ 0\equiv
	\Laghodge_{\tetr{a}{\mu}}\pd{\mu} \tetr{a}{\nu} -
	\Laghodge_{\tetr{a}{\mu}}\pd{\mu} \tetr{a}{\nu}  =
	\Laghodge_{\tetr{a}{\mu}}\pd{\mu} \tetr{a}{\nu} - \Laghodge_{\tetr{a}{\mu}}
	\tetr{a}{\lambda}\w{\lambda}{\mu\nu} $, Noether current conservation law
	\eqref{eqn.Noether.cons} can be rewritten in a pure spacetime form:	
	\begin{equation}\label{eqn.EM2}
		\pd{\mu}\EMmat{\mu}{\nu} = \EMmat{\mu}{\lambda} 
		\w{\lambda}{\mu\nu},
	\end{equation}
	which has a production term on the right hand-side and, therefore, $
	\EMmat{\mu}{\nu} $ is not a conserved quantity in the ordinary sense.
	Both forms \eqref{eqn.TEGR0.enermomen} and \eqref{eqn.EM2} will be put into a 3+1 form in Section \ref{sec.31}.

	On the other hand, if 
	\begin{subequations}
		\begin{align}\label{eqn.cov.W}
			\DW{\lambda} V^{\mu} &= \pd{\lambda} V^\mu + V^\rho \w{\mu}{\lambda\rho}, 
			\\[2mm] 
			\DW{\lambda} V_{\mu} &= \pd{\lambda} V_\mu - V_\rho \w{\rho}{\lambda\mu}
		\end{align}
	\end{subequations}
	is the covariant derivative of the \We\ connection, and 
	keeping in mind that $ \tetr{a}{\nu}\Laghodge_{\tetr{a}{\mu}} $ is a tensor density of weight $ 
	+1 
	$, balance law \eqref{eqn.EM2} can be rewritten as a covariant divergence with a production term
	\begin{equation*}\label{eqn.EM.cov}
		\DW{\mu}\EMmat{\mu}{\nu} = -\EMmat{\mu}{\nu} \Tors{\rho}{\mu\rho},
	\end{equation*}
	and hence, $ \EMmat{\mu}{\nu} $ does not conserved also in the covariant sense.
	
	%In what follows, it is always implied that $ \Laghodge(\tetr{a}{\mu},\HDT{a\mu\nu}) = 
	%\Laghodge^\text{(m)}(\tetr{a}{\mu},\HDT{a\mu\nu},\ldots) + 
	%\Laghodge^\text{(g)}(\tetr{a}{\mu},\HDT{a\mu\nu})$ and therefore, $ \EMmat{\mu}{\nu} = 
	%%(\EMmat{\mu}{\nu})^\text{(m)} + (\EMmat{\mu}{\nu})^\text{(g)} 
	%\mat{\EMmat{\mu}{\nu}} + \gra{\EMmat{\mu}{\nu}}$ with
	%\begin{equation}\label{eqn.sigmas}
	%	\mat{\EMmat{\mu}{\nu}} = \tetr{a}{\nu} L^\text{(m)}_{\tetr{a}{\mu}},
	%	\qquad
	%	\gra{\EMmat{\mu}{\nu}} = \tetr{a}{\nu} L^\text{(g)}_{\tetr{a}{\mu}}.
	%\end{equation}
	
	
	%t can be shown by cumbersome but straightforward calculations that for the TEGR 
	%Lagrangian\footnote{This is true 
		%	also for the case when $ \Laghodge^\text{(g)}$ is a generic 
		%	function of the torsion scalar $ \Laghodge^\text{(g)}(\tetr{a}{\mu},\HDT{a\mu\nu}) = 
		%f(\Tscal) 
		%	$.} $ \Laghodge^\text{(g)} $ (quadratic form 
	%in $ \HDT{a\mu\nu} $) the gravity part of $ \EMmat{\mu}{\nu} $ can be 
	%expressed as
	For later needs, the following expression of the energy momentum $ \EMmat{\mu}{\nu} =
	\tetr{a}{\nu} \Laghodge_{\tetr{a}{\mu}} $ is required
	\begin{equation}\label{eqn.TEGR.vacuum}
		\EMmat{\mu}{\nu} 
		= 
		2 \HDT{a\lambda\mu}\Laghodge_{\HDT{a\lambda\nu}} - 
		(\HDT{a\lambda\rho}\Laghodge_{\HDT{a\lambda\rho}} - \Laghodge ) 
		\delta^\mu_{\ \nu}.
	\end{equation}
	which is valid for the TEGR Lagrangian discussed in Sec.\,\ref{sec.closure}. 
	This formula will be used later in the 3+1-split and is analogous to
	\cite[Eq.(10.13)]{AldrovandiPereiraBook}. 
	
	%Yet, an interesting ordinary conservation law can be obtained  (see 
	%Appendix~\ref{app.sec.EM})
	%\begin{equation}\label{eqn.EM3}
	%	\pd{\mu}\left( 
	%	\tetr{a}{\nu} \Laghodge_{\tetr{a}{\mu}} - 2 \HDT{a\lambda\mu}\Laghodge_{\HDT{a\lambda\nu}} 
	%+ 
	%	(\HDT{a\lambda\rho}\Laghodge_{\HDT{a\lambda\rho}} - \Laghodge) \delta^\mu_{\ \nu}
	%	\right) = 0
	%\end{equation}
	%for the quantity
	%\begin{equation}\label{eqn.EM4}
	%	\EM{\mu}{\nu} := \EMmat{\mu}{\nu} + \EMgrav{\mu}{\nu},
	%\end{equation}
	%where 
	%\begin{equation}\label{eqn.EM.mat.grav}	
	%	\EMmat{\mu}{\nu}:=\tetr{a}{\nu} \Laghodge_{\tetr{a}{\mu}}, 
	%	\qquad 
	%	\EMgrav{\mu}{\nu} := -2 \HDT{a\lambda\mu}\Laghodge_{\HDT{a\lambda\nu}} + 
	%	(\HDT{a\lambda\rho}\Laghodge_{\HDT{a\lambda\rho}} - \Laghodge) \delta^\mu_{\ \nu}.
	%\end{equation}
	%However, it can be shown, that the gravity part of $ \EM{\mu}{\nu} $, i.e. that which is 
	%coming 
	%from the TEGR Lagrangian $ \Laghodge^\text{(g)} $ (quadratic form in torsion\footnote{This is 
	%true 
		%also for the case when $ \Laghodge^\text{(g)}$ is a generic 
		%	function of the torsion scalar $ \Laghodge^\text{(g)}(\tetr{a}{\mu},\HDT{a\mu\nu}) = 
		%f(\Tscal) 
		%	$.}) is 
	%identically $ 0 $, i.e.
	%\begin{equation}\label{eqn.TEGR.vacuum}
	%	\tetr{a}{\nu} \Laghodge^\text{(g)}_{\tetr{a}{\mu}} = 
	%	2 \HDT{a\lambda\mu}\Laghodge^\text{(g)}_{\HDT{a\lambda\nu}} - 
	%	(\HDT{a\lambda\rho}\Laghodge^\text{(g)}_{\HDT{a\lambda\rho}} - \Laghodge^\text{(g)}) 
	%	\delta^\mu_{\ \nu}.
	%\end{equation}
	%The later formula is analogous to equation (10.13) in \cite{AldrovandiPereiraBook}. 
	%
	%Therefore, the conservation law \eqref{eqn.EM3} makes sens only in the presence of matter, in 
	%which 
	%case it is nothing else
	%but the conservation of the matter energy-momentum
	%\begin{equation}\label{eqn.EM3.matter}
	%	\pd{\mu}\left( 
	%	\tetr{a}{\nu} L^\text{(m)}_{\tetr{a}{\mu}} - 2 
	%	\HDT{a\lambda\mu}L^\text{(m)}_{\HDT{a\lambda\nu}} + 
	%	(\HDT{a\lambda\rho}L^\text{(m)}_{\HDT{a\lambda\rho}} - L^\text{(m)}) \delta^\mu_{\ \nu}
	%	\right) = 0,
	%\end{equation}
	%and it expresses interaction of the matter and gravity fields.
	
	
	
	\section{Preliminaries for the 3+1 split}\label{sec.31.prep}
	
	
	
	\subsection{Transformation of the torsion equations}\label{sec.transform.potential}
	
	
	Before performing a 3+1-split \cite{Alcubierre2008} of system \eqref{eqn.1st.order.TEGR}, 
	we 
	need to do some preliminary transformations of every equation in \eqref{eqn.1st.order.TEGR}. 
	
	
	Similar to electromagnetism, we introduce the ``electric'' and ``magnetic''
	fields:
	\begin{equation}
		\ET{a}{\mu} := \Tors{a}{\mu\nu} u^\nu, \qquad  \BT{a}{\mu} := \HDT{a\mu\nu} u_\nu
	\end{equation}
	Note that $ \ET{a}{\mu} $ is a tensor, while $ \BT{a}{\mu}
	$ is a tensor-density.
	
	It is known that for any skew-symmetric tensor, its Hodge dual, and a time-like vector $ u^\mu 
	$ 
	the following 
	decompositions hold
	\begin{subequations}\label{eqn.T.decompos}
		\begin{align}
			\HDT{a\mu\nu} &= u^\mu \BT{a}{\nu} - u^\nu \BT{a}{\mu} + 
			\LCsymb^{\mu\nu\lambda\rho}u_\lambda 
			\ET{a}{\rho},\\[2mm]
			\Tors{a}{\mu\nu} &= u_\mu \ET{a}{\nu} - u_\nu \ET{a}{\mu} - 
			\LCsymb_{\mu\nu\lambda\rho}u^\lambda 
			\BT{a}{\rho}.
		\end{align}
	\end{subequations}
	
	Furthermore, we assume that the Lagrangian density can be \textit{equivalently} expressed in 
	different sets of variables, i.e. 
	\begin{equation}\label{eqn.Lagrangians2}
		\Laghodge(\tetr{a}{\mu},\HDT{a\mu\nu}) = \Lagtors(\tetr{a}{\mu},\Tors{a}{\mu\nu}) = 
		\LagBE(\tetr{a}{\mu},\BT{a}{\mu},\ET{a}{\nu}).
	\end{equation}
	It then can be shown that the derivatives of these different representations of the Lagrangian 
	are 
	related as
	\begin{subequations}
		\begin{align}
			\Laghodge_{\HDT{a\mu\nu}}u^\nu &= -\frac12\left( \LagBE_{\BT{a}{\mu}} +u_\mu 
			\LagBE_{\BT{a}{\lambda}} u^\lambda \right), \label{eqn.1st.order.EL}
			\\[2mm]
			\Lagtors_{\Tors{a}{\mu\nu}}u_\nu &= -\frac12\left( \LagBE_{\ET{a}{\mu}} + u^\mu 
			\LagBE_{\ET{a}{\lambda}} u_\lambda \right),
		\end{align}
	\end{subequations}
	and hence, \eqref{eqn.TEGR0.EL} and \eqref{eqn.TEGR0.integr} 
	can be written as (see Appendix~\eqref{app.sec.Deqn})
	\begin{subequations}\label{eqn.tors.BE}
		\begin{align}
			\D{\nu}( u^\mu\LagBE_{\ET{a}{\nu}} - u^\nu \LagBE_{\ET{a}{\mu}} + 
			\LCsymb^{\mu\nu\lambda\rho}u_\lambda\LagBE_{\BT{a}{\rho}}) 
			&= \NC{a}{\mu}\label{eqn.tors.BE.a} \\[2mm]
			%		
			\D{\nu}(u^\mu \BT{a}{\nu} - u^\nu\BT{a}{\mu} + 
			\LCsymb^{\mu\nu\lambda\rho}u_\lambda\ET{a}{\rho}) &= 0,
		\end{align}
	\end{subequations}
	where the source $ \NC{a}{\mu} = \Laghodge_{\tetr{a}{\mu}} $ has yet to be developed.
	
	Let us now introduce a new potential $ \Um(\tetr{a}{\mu},\Bm{a}{\mu},\Dm{a}{\mu}) $ as a 
	partial 
	Legendre transform of the Lagrangian $ \LagBE $
	\begin{equation}\label{eqn.Legendre1}
		\Um(\tetr{a}{\mu},\Bm{a}{\mu},\Dm{a}{\mu}) := \ET{a}{\lambda}\LagBE_{\ET{a}{\lambda}} - 
		\LagBE.
	\end{equation}
	By abusing a little bit notations for $ \BT{a}{\mu} $ (we shall use the same letter for $ 
	\BT{a}{\mu} $ and $ -\Bm{a}{\mu} $, this is an intermediate change 
	of variables and will not appear in the final formulation), we introduce the new state variables
	\begin{equation}\label{eqn.Legendre2}
		\Dm{a}{\mu} := \LagBE_{\ET{a}{\mu}}, \qquad \Bm{a}{\mu} := -\BT{a}{\mu}, \qquad 
		\tetr{a}{\mu} := \tetr{a}{\mu}.
	\end{equation}
	Note 
	that both $ \Dm{a}{\mu} $ and $ \Bm{a}{\mu} $ are 
	\emph{tensor-densities}. For derivatives of the new potential, we have the following relations
	\begin{equation}\label{eqn.Legendre3}
		\Um_{\Dm{a}{\mu}} = \ET{a}{\mu}, \quad \Um_{\Bm{a}{\mu}} = \LagBE_{\BT{a}{\mu}},
		\quad \Um_{\tetr{a}{\mu}} = - \LagBE_{\tetr{a}{\mu}}.
	\end{equation}
	This allows us to rewrite equations \eqref{eqn.tors.BE} in 
	the form similar to the non-linear 
	electrodynamics of moving media~\cite{Obukhov2008,DPRZ2017,Hohmann2018a}
	\begin{subequations}
		\begin{align}
			\D{\nu}(u^\mu\Dm{a}{\nu} - u^\nu \Dm{a}{\mu} + 
			\LCsymb^{\mu\nu\lambda\rho}u_\lambda 
			\Um_{\Bm{a}{\rho}})
			& =	\NC{a}{\mu},\\[2mm]
			\D{\nu}(u^\mu \Bm{a}{\nu} - u^\nu \Bm{a}{\mu} - 
			\LCsymb^{\mu\nu\lambda\rho}u_\lambda 
			\Um_{\Dm{a}{\rho}}) 
			& = 0,
		\end{align}
	\end{subequations}
	with $\Bm{a}{\mu}$ and $\Dm{a}{\mu}$ being the analogs of the magnetic and
	electric displacement fields, accordingly.
	
	
	Finally, we need to express also the Noether current $ \NC{a}{\mu} = \Laghodge_{\tetr{a}{\mu}} 
	$ in 
	terms of the 
	new potential $ \Um $ and the fields $ \Dm{a}{\mu} $ and $ \Bm{a}{\mu} $. One has (see details 
	in Appendix~\ref{app.sec.NC})
	\begin{subequations}\label{eqn.JiA}
		\begin{multline}\label{eqn.Ji0}
			\NC{\indalg{0}}{\mu} = 
				-\Um_{\tetr{\indalg{0}}{\mu}}
				+ u^\lambda \Bm{b}{\mu} \Um_{\Bm{b}{\lambda}} 
				- u^\mu \Bm{b}{\lambda} \Um_{\Bm{b}{\lambda}}\\ 
				- u^\mu \Dm{b}{\lambda} \Um_{\Dm{b}{\lambda}}
				+ \LCsymb^{\mu\lambda\rho\sigma} u_\rho \Um_{\Bm{b}{\lambda}}
				\Um_{\Dm{b}{\sigma}},
		\end{multline}
		\begin{multline}
				\NC{\sA}{\mu} = -\Um_{\tetr{\sA}{\mu}}	\\
				- \itetr{\nu}{\sA}u^\mu
				\left(
				u_\lambda \Dm{b}{\lambda} \Um_{\Dm{b}{\nu}} 
				- \LCsymb_{\nu\lambda\rho\sigma}u^\rho\Bm{b}{\sigma}\Dm{b}{\lambda}
				\right).
		\end{multline}
	\end{subequations}	


	\subsection{Transformation of the tetrad equations}
	
	%Let us introduce a pure spacetime energy-momentum tensor-density
	%\begin{equation}\label{def.energymom.spacetime}
	%\EM{\mu}{\nu} := \tetr{a}{\nu} \Um_{\tetr{a}{\mu}} - \Um \delta^\mu_{\ \nu}.
	%\end{equation}
	%
	%
	%\Umsing definition \eqref{def.energymom.spacetime}, the energy-momentum 
	%conservation law \eqref{eqn.1st.order.enermomen} can be 
	%rewritten in a pure spacetime form
	%\begin{equation}
	%\D{\mu} (\tetr{a}{\nu} \Um_{\tetr{a}{\mu}} - \Um \delta^\mu_{\ \nu}) - 
	%\Um_{\tetr{a}{\mu}} 
	%\Tors{a}{\mu\nu} = 0.
	%\end{equation}
	
	
	Contracting \eqref{eqn.TEGR0.tetrad} with the 4-velocity $ u^\mu $, and then after 
	change of variables \eqref{eqn.Legendre2} and \eqref{eqn.Legendre3}, the 
	resulting equation reads as
	\begin{equation}
		(\D{\mu}\tetr{a}{\nu} - \D{\nu}\tetr{a}{\mu}) u^\nu = \Um_{\Dm{a}{\mu}}.
	\end{equation}
	Furthermore, using the identity $ \itetr{\mu}{b}\D{\nu}\tetr{a}{\nu} = - 
	\tetr{a}{\nu}\D{\nu}\itetr{\mu}{b}$ and the definition $ u^\mu = \itetr{\mu}{\indalg{0}} 
	$, the 
	latter equation can be rewritten as
	\begin{equation}
		u^\nu\D{\nu}\tetr{a}{\mu} + \tetr{a}{\nu}\D{\mu}u^\nu =-\Um_{\Dm{a}{\mu}},
	\end{equation}
	that later will be used in the 3+1-split.
	
	\subsection{Transformation of the energy-momentum}
	Finally, we express the gravitational part of the energy-momentum tensor $ {\EMmat{\mu}{\nu}} $ 
	\eqref{eqn.TEGR.vacuum} 
	in 
	terms of 
	new 
	variables 
	\eqref{eqn.Legendre2} and 
	the potential $ \Um(\tetr{a}{\mu},\Bm{a}{\mu},\Dm{a}{\mu}) $, while we keep energy-momentum 
	equation \eqref{eqn.EM2} unchanged. It reads
	\begin{align}\label{eqn.sigma.BD}
		{\EMmat{\mu}{\nu}} =
		& -\Bm{a}{\mu}\Um_{\Bm{a}{\nu}} - \Dm{a}{\mu}\Um_{\Dm{a}{\nu}} \nonumber\\
		%
		& - u^\lambda u_\nu \Bm{a}{\mu} \Um_{\Bm{a}{\lambda}} 
		- u^\mu u_\lambda \Dm{a}{\lambda} \Um_{\Dm{a}{\nu}}
		\nonumber\\
		%
		& + u^\mu u_\nu \Bm{a}{\lambda} \Um_{\Bm{a}{\lambda}} 
		+ u^\mu u_\nu \Dm{a}{\lambda} \Um_{\Dm{a}{\lambda}}
		\nonumber \\
		%
		& + \LCsymb_{\nu\sigma\lambda\rho} u^\mu u^\sigma \Bm{a}{\lambda} \Dm{a}{\rho} 
		+ \LCsymb^{\mu\sigma\lambda\rho} u_\nu u_\sigma \Um_{\Bm{a}{\lambda}} 
		\Um_{\Dm{a}{\rho}} 
		\nonumber \\
		& + (\Bm{a}{\lambda} \Um_{\Bm{a}{\lambda}} + \Dm{a}{\lambda} \Um_{\Dm{a}{\lambda}} - 
		\Um) \KD{\mu}{\nu}. 
	\end{align}
	We shall need this expression for $ \EMmat{\mu}{\nu} $ in the last part of the derivation of 
	the $ 
	3+1 $ equations.
	
	%\begin{equation}\label{eqn.EM.BD}
	%	\EM{\mu}{\nu} = 
	%	- \tetr{a}{\nu} \Um_{\tetr{a}{\mu}}
	%	+ \Dm{a}{\mu}\Um_{\Dm{a}{\nu}} + \Bm{a}{\mu}\Um_{\Bm{a}{\nu}}
	%	- (
	%	\Dm{a}{\lambda}\Um_{\Dm{a}{\lambda}}+ \Bm{a}{\lambda}\Um_{\Bm{a}{\lambda}}
	%	-\Um
	%	) \KD{\mu}{\nu}.
	%\end{equation}
	
	Therefore, the TEGR system in its intermediate form for the 
	unknowns $ \{\tetr{a}{\mu},\Dm{a}{\mu},\Bm{a}{\mu}\} $  reads
	\begin{subequations}\label{eqn.PDE.4D}
		\begin{align}%[box={\mycreambox[2pt][2pt]}]{align}
			\D{\nu}(u^\mu\Dm{a}{\nu} - u^\nu \Dm{a}{\mu} + 
			\LCsymb^{\mu\nu\lambda\rho}u_\lambda 
			\Um_{\Bm{a}{\rho}})
			& =	\NC{a}{\mu},\label{eT}\\[2mm]
			%		
			\D{\nu}(u^\mu \Bm{a}{\nu} - u^\nu \Bm{a}{\mu} - 
			\LCsymb^{\mu\nu\lambda\rho}u_\lambda 
			\Um_{\Dm{a}{\rho}}) 
			& = 0,\label{bT}\\[2mm]
			%		
			%		\pd{\mu}\left( 
			%			  \tetr{a}{\nu} \Um_{\tetr{a}{\mu}}
			%        	- \Dm{a}{\mu}\Um_{\Dm{a}{\nu}} - \Bm{a}{\mu}\Um_{\Bm{a}{\nu}}
			%        	+ (
			%        	\Dm{a}{\lambda}\Um_{\Dm{a}{\lambda}}+ \Bm{a}{\lambda}\Um_{\Bm{a}{\lambda}}
			%        	-\Um
			%        	) \KD{\mu}{\nu}
			%        \right) & = 0
			%        \pd{\mu}\EMmat{\mu}{\nu} 
			%        & = \EMmat{\mu}{\lambda} 
			%		\w{\lambda}{\mu\nu},\\[2mm]
			\pd{\mu}\NC{a}{\mu}
			& = 0, \\[2mm] 
			%		
			u^\nu\D{\nu}\tetr{a}{\mu} + \tetr{a}{\nu}\D{\mu}u^\nu &=-\Um_{\Dm{a}{\mu}},
			\label{tetr}
		\end{align}
	\end{subequations}
	with $ \NC{a}{\mu} $ given by \eqref{eqn.JiA}. After we introduce a particular choice of the 
	observer's 4-velocity $ u^\mu $ at the beginning of the next section, we shall finalize 
	transformation of 
	system \eqref{eqn.PDE.4D} to 
	present the final 3+1 equations of TEGR.
	
	
	%\IP{Interestingly, the energy-momentum PDE is exactly like we use for the 
		%symmetrization 
		%in the non-relativistic settings, i.e. when we add the torsion multiplied by $ 
		%\Um_{\tetr{a}{\mu}} $ 
		%to the momentum equation. However, when we usually du the summation, we don't 
		%need this 
		%non-conservative terms.... hm-m....}
	
	
	%-------------------------------------------------------
	\section{3+1 split of the TEGR equations}	\label{sec.31}
	%-------------------------------------------------------
	
	
	In this section, we derive a 3+1 version of system \eqref{eqn.PDE.4D} that can be used in a 
	computational code for numerical relativity.
	
	We first recall that Latin indices from the middle of the alphabet $ i,j,k,\ldots=1,2,3 $ are 
	used 
	to 
	denote the spatial components of the 
	spacetime 
	tensors, and Latin indices $ \sA,\sB,\sC,\ldots = \indalg{1},\indalg{2},\indalg{3} $ to denote 
	the 
	spatial 
	directions in the tangent Minkowski space. Additionally, we use the hat on top of a number, 
	e.g. $ 
	\indalg{0} $, for the indices   marking the time and space direction in the tangent space in 
	order 
	to distinguish them 
	from the indices of the spacetime tensors.  Also recall that observer's 4-velocity 
	$ 
	u^\mu $ is associated with the $ \hat{0} $-th column of the 
	inverse tetrad $ \itetr{\mu}{a} $, while the covariant components $ u_\mu $ of the 4-velocity 
	with 
	the $ 
	\hat{0} $-th row of the frame field. For $ u^\mu $ and $ u_\mu $ we standardly assume 
	\cite{Alcubierre2008,RezzollaZanottiBook}:
	\begin{subequations}\label{eqn.4vel}
		\begin{align}
			u^{\mu} &= \itetr{\mu}{\indalg{0}}  = \lapse^{-1}(1,-\shift{i}), %= \lapse^{-1}(1,v^i)
			\\
			u_\mu   &= - \tetr{\indalg{0}}{\mu} = (-\lapse,0,0,0),
		\end{align}
		%	\begin{equation}
			%		\tetr{\hat{i}}{\mu} = 
			%		(\shift{\hat{i}},\tetr{\hat{i}}{i})
			%	\end{equation}
		%	\begin{equation}
			%		\itetr{\mu}{\hat{i}} = (0,\tetr{i}{\hat{i}})
			%	\end{equation}
	\end{subequations}
	with $ \lapse $ being the \emph{lapse function}, and $ \shift{i} $ being the \emph{shift 
	vector}. 
	One can write down $ \tetr{a}{\mu} $ and $ \itetr{\mu}{a} $ explicitly :
	\begin{subequations}\label{eqn.h.eta.matrix}
		\begin{equation}
			\bm{\tetrsymbol} = \left(
			\begin{array}{cccc}
				\alpha          & 0 & 0 & 0 \\[1mm]
				\beta^{\indalg{1}} &  &  &  \\
				\beta^{\indalg{2}} &  & \tetr{\sA}{i} &  \\
				\beta^{\indalg{3}} &  &  & 
			\end{array}
			\right) ,
		\end{equation}
		\begin{equation}
			\bm{h}^{-1} = \bm{\itetrsymbol} = \left(
			\begin{array}{rlcl}
				1/\alpha          & 0 & 0 & 0 \\[1mm]
				-\beta^{1}/\alpha &  &  &  \\
				-\beta^{2}/\alpha &  & \left (\tetr{\sA}{i}\right )^{-1} &  \\
				-\beta^{3}/\alpha &  &  & 
			\end{array}
			\right) ,
		\end{equation}
	\end{subequations}
	where $ \beta^{\sA} = \tetr{\sA}{i}\beta^i$.
	The metric tensor and its inverse are (e.g. see \cite{Gourgoulhon2012a})
	\begin{subequations}
		\begin{align}
			g_{\mu\nu} &= \left(
			\begin{array}{cc}
				-\alpha^2 + \beta_i\beta^i & \beta_i \\[1mm]
				\beta_i & \gamma_{ij}  \\
			\end{array}
			\right) ,
			\\[2mm]
			g^{\mu\nu} &= \left(
			\begin{array}{cc}
				-1/\alpha^2       & \beta^i/\alpha^2 \\[1mm]
				\beta^{i}/\alpha^2 & \gamma^{ij} - \beta^i\beta^j/\alpha^2  \\
			\end{array}
			\right) ,
		\end{align}
	\end{subequations}
	where $ \gamma_{ij} = \kappa_{\sA\sB} \tetr{\sA}{i}\tetr{\sB}{j}$, $ 
	\gamma^{ij} = \left( \gamma_{ij} \right)^{-1}  $, and $ \beta_i =  \gamma_{ij}\beta^j$.
	
	In the rest of the paper, $ \detTetr_3 $ stands for $ \det(\tetr{\sA}{i}) $ so that
	\begin{equation}\label{eqn.det}
		\detTetr := \det(\tetr{a}{\mu}) = \alpha \detTetr_3,
	\end{equation}
	and we use the following convention for the three-dimensional Levi-Civita symbol
	\begin{equation}\label{eqn.LC.3d}
		\LCsymb^{0ijk} = \LCsymb^{ijk}, 
		\qquad
		\LCsymb_{0ijk} =-\LCsymb_{ijk}.
	\end{equation}
	
	
	%Also, because $ \Um $ is a quadratic form in $ \Dm{\sA}{k} $ and $ \Bm{\sA}{k} $, 
	%one can write 
	%\begin{equation}\label{eqn.Tscal.HBDE}
	%	\Um(\tetr{\sA}{k},\Dm{\sA}{k},\Bm{\sA}{k}) = \frac12 
	%\BT{a}{\mu}\LagBE_{\BT{a}{\mu}} - 
	%	\frac12 \ET{a}{\mu}\LagBE_{\ET{a}{\mu}} 
	%	=
	%	\frac12 \BT{a}{\mu}\hT{a}{\mu} - \frac12 \ET{a}{\mu}\Dm{a}{\mu},
	%\end{equation}
	%or
	%\begin{equation}
	%	\LagBE(\tetr{a}{\mu},\ET{a}{\mu},\BT{a}{\mu})
	%	=
	%	\frac12 \BT{a}{i}\hT{a}{i} - \frac12 \ET{a}{i}\Dm{a}{i},
	%\end{equation}
	%where $ \hT{a}{\mu} = \LagBE_{\BT{a}{\mu}} $ and $ \Dm{a}{\mu} = \LagBE_{\ET{a}{\mu}} $.
	
	
	%\IP{Decide what variables are finally to be used $ \{\Dm{}{},\Bm{}{}\} $ or the re-scaled ones 
	%$ 
		%	\{\aD{}{},\aB{}{}\} $ }
	%
	%\IP{Not clear why we don't have a mixed invariant kinda $ \Dm{i}{k} B^i_{\ k} $ in the 
		%	Lagrangian $ \Um $. Should we?}
	
	%-------------------------------------------------------------------
	\subsection{3+1 split of the torsion PDEs}	\label{ssec.31.tors}
	%-------------------------------------------------------------------
	
	%\IP{check the sign when you do replacement of $ \LCsymb^{0ijk} $ by $ \LCsymb^{ijk} $. It 
	%should 
		%be 
		%$ \LCsymb^{0ijk} = \LCsymb^{ijk} $.}
	
	\paragraph{Case $ \mu = i=1,2,3 $.} 
	
	For our choice of observer's velocity \eqref{eqn.4vel}, equations \eqref{eT} 
	and \eqref{bT} read
	\begin{subequations}\label{eqn.tpo.1}
		\begin{multline}
			\pd{t} (\lapse^{-1}\Dm{a}{i}) + \pd{k}(\shift{i} 
			\lapse^{-1}\Dm{a}{k} - \shift{k}\lapse^{-1}\Dm{a}{i} \\
			- \LCsymb^{ikj} \lapse \,
			\Um_{\Bm{a}{j}}) 
			= -\NC{a}{i},
		\end{multline}
		\begin{multline}
			-\pd{t} (\lapse^{-1}\Bm{a}{i}) + \pd{k}(\shift{k} 
			\lapse^{-1}\Bm{a}{i} - \shift{i}\lapse^{-1}\Bm{a}{k}  \\
			+ \LCsymb^{ikj} \lapse \,
			\Um_{\Dm{a}{j}}) 
			= 0 .
		\end{multline}
	\end{subequations}
	Because of the factor $ \alpha^{-1} $ everywhere in these equations, it is convenient to 
	rescale 
	the 
	variables:
	\begin{equation}\label{eqn.varDB}
		\aD{a}{\mu} := \lapse^{-1} \Dm{a}{\mu}, \qquad \aB{a}{\mu} := -\lapse^{-1}\Bm{a}{\mu}
	\end{equation}
	so that the derivatives of the potential $ \aU(\tetr{a}{\mu},\aD{a}{\mu},\aB{a}{\mu}) := 
	\Um(\tetr{a}{\mu},\Dm{a}{\mu},\Bm{a}{\mu})
	$ 
	transform as
	\begin{equation}\label{eqn.change.alphaU}
		\aU_{\aD{a}{\mu}} =  \lapse \, \Um_{\Dm{a}{\mu}},
		\qquad
		\aU_{\aB{a}{\mu}} = -\lapse \, \Um_{\Bm{a}{\mu}}.
	\end{equation}
	Hence, \eqref{eqn.tpo.1} can be rewritten as
	\begin{subequations}\label{eqn.tpo.2}
		\begin{align}
			\pd{t} \aD{a}{i} + \pd{k}(\shift{i} 
			\aD{a}{k} - \shift{k}\aD{a}{i}  - \LCsymb^{ikj} \,
			\aU_{\aB{a}{j}}) & 
			= -\NC{a}{i},\\[2mm]
			%
			\pd{t} \aB{a}{i} + \pd{k}(\,\shift{i} 
			\aB{a}{k} - \shift{k}\aB{a}{i}  + \LCsymb^{ikj} 
			\aU_{\aD{a}{j}}) & 
			= 0.
		\end{align}
	\end{subequations}
	
	Finally, extending formally the shift vector as $ \bm{\beta}=(-1,\beta^{k}) 
	$ and introducing 
	the 
	change of variables
	\begin{equation}\label{eqn.varDB.final}
		\Dfin{a}{\mu} := \aD{a}{\mu} + \beta^{\mu} \aD{a}{0}, \qquad \Bfin{a}{\mu} := \aB{a}{\mu}
	\end{equation}
	which are essentially $ 3 $-by-$ 3 $ matrices ($ \Dfin{a}{0} = 0$, $ \Bfin{a}{0} = 
	\Bfin{\indalg{0}}{\mu} = 0 $, and $ \Dfin{\indalg{0}}{k} $ are given by \eqref{eqn.D0.tmp3}),
	we arrive at the final form of the 3+1 equations for the fields $ \Dfin{\sA}{i} $ and $ 
	\Bfin{\sA}{i} $
	\begin{subequations}\label{eqn.PDE.BD.final}
		\begin{align}
			\pd{t} \Dfin{\sA}{i} + \pd{k}(\shift{i} 
			\Dfin{\sA}{k} - \shift{k}\Dfin{\sA}{i}  - \LCsymb^{ikj} \,
			\Hfin{\sA}{j}) & 
			= -\NC{\sA}{i}, \label{eqn.tpo.D}\\[2mm]
			%
			\pd{t} \Bfin{\sA}{i} + \pd{k}(\,\shift{i} 
			\Bfin{\sA}{k} - \shift{k}\Bfin{\sA}{i}  + \LCsymb^{ikj} 
			\Efin{\sA}{j}) & 
			= 0\label{eqn.tpo.B} .
		\end{align}
	\end{subequations}
	where
	\begin{subequations}
		\begin{align}\label{eqn.HE.fin}
			&\Efin{\sA}{j} = \aU_{\aD{\sA}{j}} = \alpha \Um_{\Dm{\sA}{j}}= \alpha \ET{\sA}{j},
			\\
			&\Hfin{\sA}{j} = \aU_{\aB{\sA}{j}} =-\alpha \Um_{\Bm{\sA}{j}}.
		\end{align}
	\end{subequations}
	In terms of $ \{\Bfin{\sA}{k},\Dfin{\sA}{k}\} $ the potential $  
	\aU(\tetr{a}{\mu},\aD{a}{\mu},\aB{a}{\mu}) $ will be denoted as 
	$
	\Ufin(\tetr{a}{\mu},\Bfin{a}{\mu},\Dfin{a}{\mu})  
	$ and, in the case of TEGR, it reads (see \eqref{eqn.Ufin})
	\begin{align}\label{eqn.Ufin0}
		\Ufin(\tetr{\sA}{k},\Bfin{\sA}{k},\Dfin{\sA}{k}) & =  \nonumber\\
		 -\frac{\alpha}{2\detTetr_3} \bigg( 
		& \varkappa \left( \Dfin{\sB}{\sA} \Dfin{\sA}{\sB} - \frac{1}{2} 
		(\Dfin{\sA}{\sA})^2\right) \nonumber\\
		 +
		&\frac{1}{\varkappa} \left( \Bfinmix{\sA}{\sB} \Bfinmix{\sB}{\sA} - \frac{1}{2} 
		(\Bfinmix{\sA}{\sA})^2
		\right)
		\bigg),
	\end{align}
	Alternatively, it can be computed as
	\begin{equation}\label{eqn.U.bdeh}
		\Ufin = \frac{1}{2} ( \Dfin{a}{k} \Efin{a}{k} + \Bfin{a}{k} \Hfin{a}{k} ).
	\end{equation} 
	
	
	\paragraph{Case $ \mu = 0 $.} 
	
	The $ 0 $-th equations \eqref{eT} and \eqref{bT} are actually not time-evolution equations but  
	pure spatial (stationary) constraints
	\begin{equation}\label{eqn.div.constr}
		\pd{k} \Dfin{a}{k} = \NC{a}{0}, 
		\qquad
		\pd{k} \Bfin{a}{k} = 0.
	\end{equation}
	Their possible violation during the numerical integration of the time-evolution equations 
	\eqref{eqn.PDE.BD.final} is a well-known problem in the numerical analysis of hyperbolic 
	equations 
	with involution constraints. Different strategies to 
	preserve such stationary constraints   
	are known, e.g. constraint-cleaning approach
	\cite{Munz2000,Dedneretal,Dumbser2019} or constraint-preserving integrators
	\cite{Olivares2022,SIGPR2021,oliynyk2025}.
	
	
	\subsection{Tetrad PDE}
	
	Because of our choice of the tetrad \eqref{eqn.h.eta.matrix}, we are only interested in the  
	evolution equations for the components $ \tetr{\sA}{k} $. Thus, using the definition of 
	the 4-velocity \eqref{eqn.4vel}, equation \eqref{tetr} can be written as
	\begin{equation}\label{eqn.tetr.3+1}
		\pd{t} \tetr{\sA}{k} - \beta^i \pd{i} \tetr{\sA}{k} - \tetr{\sA}{i} \pd{k} \beta^i 
		= 
		-\Efin{\sA}{k}.
		% \aU_{\aD{\sA}{k}}.
	\end{equation}
	
	This equation can be also written in a slightly different form. After adding $ 0\equiv - 
	\beta^i 
	\pd{k}\tetr{\sA}{i} + \beta^i \pd{k}\tetr{\sA}{i} $ to the left hand-side of 
	\eqref{eqn.tetr.3+1}, 
	one 
	has  
	\begin{equation}\label{eqn.tetr.3+1.2}
		\pd{t} \tetr{\sA}{k} - \pd{k} (\beta^i \tetr{\sA}{i}) - \beta^i (\pd{i}\tetr{\sA}{k} - 
		\pd{k}\tetr{\sA}{i})
		= 
		-\Efin{\sA}{k}.
		%	\aU_{\aD{\sA}{k}}.
	\end{equation}
	Finally, using the definition of $ \BT{\sA}{\mu} $ and $ u_\mu $, we have that 
	\begin{multline}
		\BT{\sA}{\mu} = \HDT{\sA\mu\nu} u_\nu = u_\nu \LCsymb^{\mu\nu\alpha\beta} 
		\pd{\alpha}\tetr{\sA}{\beta} = \\
		-
		u_\nu \LCsymb^{\nu\mu\alpha\beta} \pd{\alpha}\tetr{\sA}{\beta} =
		\alpha \LCsymb^{0\mu\alpha\beta} \pd{\alpha}\tetr{\sA}{\beta} 
	\end{multline}
	and hence (use that $ \LCsymb^{0ijk} =\LCsymb^{ijk} $)
	\begin{equation}\label{eqn.B.curlh}
		\Bfin{\sA}{k} = \LCsymb^{kij}\pd{i}\tetr{\sA}{j},
	\end{equation}
	or
	\begin{equation}
		-\beta^i (\pd{i}\tetr{\sA}{k} - 
		\pd{k}\tetr{\sA}{i})
		%	=
		%	\alpha^{-1}\LCsymb_{klj}\beta^l \BT{\sA}{j}
		=
		%-
		\LCsymb_{klj}\beta^l \Bfin{\sA}{j}.
	\end{equation}
	Therefore, \eqref{eqn.tetr.3+1.2} can be also rewritten as 
	\begin{equation}\label{eqn.tetr.3+1.3}
		\pd{t} \tetr{\sA}{k} - \pd{k} (\beta^i \tetr{\sA}{i}) =  
		-\Efin{\sA}{k} - \LCsymb_{klj} \beta^l \Bfin{\sA}{j}.
	\end{equation}
	It is not clear yet which one of these equivalent forms
	\eqref{eqn.tetr.3+1}, \eqref{eqn.tetr.3+1.2}, or \eqref{eqn.tetr.3+1.3} at
	the continuous level is more suited for the numerical discretization.
	However, for a structure-preserving integrator, e.g.
	\cite{SIGPR2021,Olivares2022,oliynyk2025}, which is able  to preserve \eqref{eqn.B.curlh}
	up to the machine precision, all these forms are equivalent.
	
	
	
	
	
	\subsection{Energy-momentum PDE}\label{sec.energymomentum}
	
	We now turn to the energy-momentum evolution equation
	\begin{equation}\label{eqn.sigma.31}
		\pd{\mu}\EMmat{\mu}{\nu} 
		= \EMmat{\mu}{\lambda} 
		\w{\lambda}{\mu\nu}
	\end{equation}
	and provide its 3+1 version. Because the Lagrangian density in \eqref{eqn.Lagrangians} 
	represents the sum of the gravitational and matter Lagrangian, the energy-momentum tensor is 
	also 
	assumed to be the sum of the gravity and matter parts: $ \EMmat{\mu}{\nu} = 
	\mat{\EMmat{\mu}{\nu}} 
	+ 
	\gra{\EMmat{\mu}{\nu}} $. However, in what follows, we shall omit the matter part keeping in 
	mind 
	that the energy and momentum equations discussed below are equations for the total 
	(matter+gravity) 
	energy-momentum. 
	
	We first, explicitly split \eqref{eqn.sigma.31} into the three momentum and one energy equation:
	\begin{subequations}\label{eqn.EM.sigma}
		\begin{align}
			\pd{t}\EMmat{0}{i} + \pd{k}\EMmat{k}{i}
			& = \EMmat{\mu}{\lambda} 
			\w{\lambda}{\mu i},\\[2mm]
			%
			\pd{t}\EMmat{0}{0} + \pd{k}\EMmat{k}{0}
			& = \EMmat{\mu}{\lambda} 
			\w{\lambda}{\mu 0}.
		\end{align}
	\end{subequations}
	Applying the change of variables \eqref{eqn.varDB}, \eqref{eqn.change.alphaU}, and 
	\eqref{eqn.varDB.final}, expression \eqref{eqn.sigma.BD} for the energy-momentum reduces to the 
	following expressions
	\begin{itemize}
		\item 
		for the total momentum density, $ \sigma^0_{\ i} = \rho_i $ (the matter part is left 
		unspecified and omitted in this paper):
		\begin{equation}\label{eqn.sigma0i}
			\rho_i :=\LCsymb_{ijl}\Dfin{a}{j}\Bfin{a}{l},
		\end{equation}
		%	(with $ \rho^\text{(m)}_i $ being the total momentum of matter fields, e.g. see 
		%	\cite{PTRSA2020}, and $ \rho^\text{(g)}_i 
		%	$ 
		%	being the gravitational momentum)
		%
		\item
		for the momentum flux, $ \EMmat{k}{i}$:
		\begin{multline}\label{eqn.sigmaki}
			\EMmat{k}{i} = -\Bfin{a}{k} \Hfin{a}{i} - \Dfin{a}{k} \Efin{a}{i}  
			- \LCsymb_{ijl} 
			\beta^k \Dfin{a}{j}\Bfin{a}{l} \\
			+ (\Bfin{a}{j} \Hfin{a}{j} + \Dfin{a}{j} \Efin{a}{j} 
			- \Ufin)\KD{k}{i},
		\end{multline}
		%
		\item
		for the energy density:
		\begin{equation}\label{eqn.sigma00}
			\rho_0 :=\EMmat{0}{0} = \LCsymb_{ijl} 
			\beta^i\Dfin{a}{j}\Bfin{a}{l} - \Ufin = \beta^i 
			\rho_i 
			- \Ufin,
		\end{equation}
		%
		\item
		and for the energy flux, $ \EMmat{k}{0} $:
		\begin{multline}\label{eqn.sigmak0}
			{\EMmat{k}{0}} = -\beta^j( \Bfin{a}{k} \Hfin{a}{j} + \Dfin{a}{k} \Efin{a}{j}) \\
			+
			\beta^k( \Bfin{a}{j} \Hfin{a}{j} + \Dfin{a}{j} \Efin{a}{j}) \\
			-
			\LCsymb_{ijl}\beta^k\beta^i\Dfin{a}{j}\Bfin{a}{l}
			-
			\LCsymb^{kjl} \Efin{a}{j}\Hfin{a}{l}.
		\end{multline}
	\end{itemize}
	It is convenient to split the momentum flux $ \EMmat{k}{i} $ and energy
	flux $ \EMmat{k}{0} $ in \eqref{eqn.EM.sigma} into advective and
	constitutive parts, so that the finale form of the 3+1 equations for the
	total energy-momentum \eqref{eqn.EM.sigma} reads
	\begin{subequations}\label{eqn.EM.PDE}
		\begin{align}
			\pd{t}\rho_i & + \pd{k}(-\beta^k \rho_i + \stress{i}{k}) = 
			\rhs{i},\label{eqn.EM.m}\\[2mm]
			\pd{t}\rho_0 & + \pd{k}(-\beta^k \rho_0 + \beta^i \stress{i}{k} 
			- \LCsymb^{kjl} \Efin{a}{j}\Hfin{a}{l} ) = \rhs{0},\label{eqn.EM.e}
		\end{align}
		where the non-matter (gravity + inertia) part of $ \stress{i}{k} $ is given by
		\begin{gather}\label{eqn.flux.const}
			{\stress{i}{k}} = -\Bfin{a}{k} \Hfin{a}{i} - \Dfin{a}{k} \Efin{a}{i} + (\Bfin{a}{j} 
			\Hfin{a}{j} + 
			\Dfin{a}{j} \Efin{a}{j} 
			- \Ufin)\delta^k_{\ i}.
		\end{gather}
		Drawing parallels between electrodynamics and TEGR, one can note the
		presence of the Poynting vector in two forms in \eqref{eqn.EM.PDE}:
		$\Dfin{}{}\times\Bfin{}{}$ in the momentum density in \eqref{eqn.EM.m} and $\Efin{}{}\times\Hfin{}{}$ in the energy flux in \eqref{eqn.EM.e}.

		The source terms in \eqref{eqn.EM.m} and \eqref{eqn.EM.e} are given by
		\begin{align}\label{eqn.source.rhoi}
			\rhs{i} =&-\rho_k \itetr{k}{\sA} \Efin{\sA}{i} + \rho_j \pd{i}\shift{j}
			+\stress{j}{k} \w{j}{ki}
			%	- \LCsymb_{ilj} s^l_{\ k} \itetr{k}{\sA} 
			%	\Bfin{\sA}{j} + s^l_{\ k} \itetr{k}{\sA} \pd{i}\tetr{\sA}{l} 
			,\\[2mm]
			\rhs{0} =&- \left(\pd{t} \ln(\alpha) - \shift{k} \pd{k}\ln(\alpha) \right) \Ufin 
			\nonumber\\
			& + \left( \pd{t} \shift{\sA} - \shift{k} \pd{k}\shift{\sA} \right) \rho_j 
			\itetr{j}{\sA} \nonumber\\
			&- \LCsymb^{kjl} \Efin{a}{j}\Hfin{a}{l}\pd{k}\ln(\alpha) + \stress{j}{k} 
			\itetr{j}{\sA} 
			\pd{k}\shift{\sA}. \label{eqn.f0}
		\end{align}
	\end{subequations}
	%Interestingly, that in the right hand side of the energy equation \eqref{eqn.f0}, we have 
	%explicitly transport equations for the laps and the shift vector $ \shift{\sA} = \tetr{\sA}{i} 
	%\shift{i} $. 
	
	
	
	\subsubsection{Noether current}
	
	Because the energy-momentum $ \EMmat{k}{i} $ was defined as $ \EMmat{k}{i} = \tetr{a}{i} 
	\NC{a}{k} = \tetr{\sA}{i} \NC{\sA}{k} $, we can use the expression for $ \EMmat{k}{i} $ to have 
	an 
	explicit formula for the Noether current. Thus, we have
	\begin{equation}\label{eqn.J.new}
		\NC{\sA}{k} = \itetr{i}{\sA} \EMmat{k}{i} = \itetr{i}{\sA} (-\shift{k} \rho_i + 
		\stress{i}{k} ) 
		= 
		-\shift{k} \rho_{\sA} + \stress{\sA}{k}.
	\end{equation}
	
	In the same way we can find $ \NC{a}{0} $ necessary in \eqref{eqn.div.constr}: $ \NC{a}{0} = 
	\itetr{\lambda}{a} \EMmat{0}{\lambda} = \itetr{0}{a} \EMmat{0}{0} + \itetr{i}{a} \EMmat{0}{i}$, 
	and 
	hence
	\begin{equation}\label{eqn.Ja0}
		\NC{a}{0} = \left\{
		\begin{array}{rl}
			-\alpha^{-1} \Ufin=\rho_{\indalg{0}},	& a=\indalg{0},  \\[3mm] 
			%
			\itetr{i}{\sA} \rho_i = \rho_{\sA}, & \sA=\indalg{1},\indalg{2},\indalg{3}. \\ 
		\end{array} 
		\right.
	\end{equation}
	

	\subsubsection{Alternative form of the energy-momentum equations}
	
	In \eqref{eqn.sigma.31}, we deliberately use the energy-momentum tensor with
	both spacetime indices because we would like to use the SHTC and Hamiltonian
	structure \cite{SHTC-GENERIC-CMAT} of these equations for designing
	numerical schemes in the future, e.g. \cite{HTC2022,SPH_SHTC}. However, the
	resulting PDEs \eqref{eqn.EM.PDE} do not have a fully conservative form
	preferable for example when dealing with the shock waves in the matter
	fields. Therefore, one may want to use the true conservation law
	\eqref{eqn.Noether.cons} for the total (matter+gravity) Noether current $
	\NC{a}{\mu} $ instead of \eqref{eqn.EM.PDE}. Thus, in 	notations
	\eqref{eqn.J.new}, \eqref{eqn.Ja0}, four conservation laws $
	\pd{\mu}\NC{a}{\mu} = 0 $ now read
	\begin{subequations}
		\begin{align}
			\pd{t} \rho_\sA + & \pd{k} (-\shift{k} \rho_\sA + \stress{\sA}{k}) = 0,\\[2mm]
			\pd{t} \rho_{\indalg{0}} + & \pd{k} (-\shift{k} \rho_{\indalg{0}} 
			- \alpha^{-1} 
			\LCsymb^{kjl}\Efin{a}{j}\Hfin{a}{l} ) = 0.
		\end{align}
	\end{subequations}
	% with $ \rho_{\indalg{0}} = \NC{\indalg{0}}{0} = -\alpha^{-1} \Ufin $. 
	%This form, and therefore the entire system of the 
	%3+1 TEGR equations \eqref{eqn.3+1.summary},  is equivalent to the 
	%dGREM formulation obtained in \cite{Olivares2022} (there, however, we write evolution 
	%equations for the gravity and matter energy-momenta separately).
	
	
	
	\subsubsection{Evolution of the space volume}
	
	As in the computational Newtonian mechanics \cite{DPRZ2016,SIGPR2021}, the evolution of the 
	tetrad 
	field at the discrete level has to be performed consistently with the volume/mass conservation 
	law. 
	In 
	TEGR, the equivalent to the volume conservation in the Newtonian mechanics is 
	\begin{equation}\label{eqn.pde.det}
		\pd{\mu}(\detTetr u^\mu) = -\detTetr \ET{\mu}{\mu} 
	\end{equation} 
	which can be obtained after contracting \eqref{tetr} with $ \partial \detTetr/\partial 
	\tetr{a}{\mu} 
	= \detTetr
	\itetr{\mu}{a} $, and where $ \ET{\mu}{\mu} = \itetr{\mu}{a} \ET{a}{\mu}$.
	
	
	After using \eqref{eqn.4vel}, \eqref{eqn.h.eta.matrix}, and \eqref{eqn.E}, this balance law 
	becomes
	\begin{equation}\label{eqn.h.PDE}
		\pd{t}\detTetr_3 - \pd{k}(\detTetr_3\shift{k} ) =-\frac{\alpha \varkappa}{2} \Dfin{i}{i},
	\end{equation}
	where $ \detTetr_3 =\det(\tetr{\sA}{k}) = \sqrt{\det(\gamma_{ij})} $.
	
	
	
	
	\section{Summary of the 3+1 TEGR equations}
	Here, we summarize the 3+1 TEGR equations and give explicit expressions for the 
	constitutive 
	fluxes $ \Efin{\sA}{k} $ and
	$ \Hfin{\sA}{k} $ which then close the specification of the entire system.
	
	
	\subsection{Evolution equations}
	
	The system of 3+1 TEGR governing equations reads
	\begin{subequations}\label{eqn.3+1.summary}
		\begin{align}%[box={\mycreambox[2pt][2pt]}]{align}
			\pd{t} \Dfin{\sA}{i} + \pd{k}(\shift{i} 
			\Dfin{\sA}{k} - \shift{k}\Dfin{\sA}{i}  - \LCsymb^{ikj} \,
			\Hfin{\sA}{j}) & 
			= \shift{i} \rho_{\sA} - \stress{\sA}{i},
			\label{eqn.3+1.D}
			\\[2mm]
			%		
			\pd{t} \Bfin{\sA}{i} + \pd{k}(\,\shift{i} 
			\Bfin{\sA}{k} - \shift{k}\Bfin{\sA}{i}  + \LCsymb^{ikj} 
			\Efin{\sA}{j}) & 
			= 0,
			\label{eqn.3+1.B}
			\\[2mm]
			%		
			%		\pd{\mu}\left( \aD{a}{\mu}\aU_{\aD{a}{\nu}}+ 
			%		\aB{a}{\mu}\aU_{\aB{a}{\nu}}
			%		-\tetr{a}{\nu} \aU_{\tetr{a}{\mu}} 
			%		+ (
			%		\aD{a}{\lambda}\aU_{\aD{a}{\lambda}}+ \aB{a}{\lambda}\aU_{\aB{a}{\lambda}}
			%		-\aU
			%		) \KD{\mu}{\nu}
			%		\right) & = 0,
			\pd{t} \rho_\sA +  \pd{k} (-\shift{k} \rho_\sA + \stress{\sA}{k}) &= 
			0,\label{eqn.3+1.m}\\[2mm]
			\pd{t} \rho_{\indalg{0}} + \pd{k} (-\shift{k} \rho_{\indalg{0}} 
			- \alpha^{-1} 
			\LCsymb^{kjl}\Efin{a}{j}\Hfin{a}{l}) &= 0,\label{eqn.3+1.e}\\[2mm]
			%		\pd{t}\rho_i + \pd{k}(-\beta^k \rho_i + \stress{i}{k}) &= \rhs{i},
			%		\label{eqn.3+1.m}
			%		\\[2mm]
			%		\pd{t}\rho_0 + \pd{k}(-\beta^k \rho_0 + \beta^i \stress{i}{k} + \LCsymb^{klj} 
			%\Hfin{a}{l} 
			%		\Efin{a}{j}) &= \rhs{0},	
			%		\label{eqn.3+1.e}
			%		\\[2mm]
			%		
			\pd{t} \tetr{\sA}{k} - \beta^i \pd{i} \tetr{\sA}{k} - \tetr{\sA}{i} \pd{k} \beta^i 
			& = 
			-\Efin{\sA}{k},
			\label{eqn.3+1.h}
			%		\pd{t} \tetr{\sA}{k} - \pd{k} (\beta^i \tetr{\sA}{i}) - \beta^i 
			%(\pd{i}\tetr{\sA}{k} - 
			%		\pd{k}\tetr{\sA}{i})
			%		&= 
			%		-\Efin{\sA}{k}.
			%		\pd{t} \tetr{a}{k} - \beta^i \pd{i} \tetr{a}{k} - \tetr{a}{i} \pd{k} \beta^i &= 
			%		{\Efin{a}{k}},
			%		\tag{\ref{eqn.tetr.3+1}} 
		\end{align}
	\end{subequations}
	%\begin{subequations}
	%	\begin{align}
		%		\Hfin{\sA}{k} := %\Ufin_{\Bfin{\sA}{k}} =
		%		-\frac{\alpha}{\varkappa \detTetr_3} 
		%		\mg{\sA\sB}\mg{\sC\sD} 
		%		& \left( \tetr{\sD}{k} \tetr{\sB}{j} - \frac12 \tetr{\sD}{j} \tetr{\sB}{k} \right) 
		%		\Bfin{\sC}{j}
		%		- 
		%		\frac{1}{\varkappa \detTetr_3} \mg{\sA,\sB} \LCsymb^{ijl} \gamma_{ik} \tetr{\sB}{j} 
		%		\Efin{\indalg{0}}{l}
		%		%\frac{\detTetr_3}{\varkappa} \LCsymb_{kli} \itetr{l}{\sA} \gamma^{ij} 
		%		%\Efin{\indalg{0}}{j}
		%		,\\[2mm]
		%		\Efin{\sA}{k} := %\Ufin_{\Dfin{\sA}{k}}=
		%		-\frac{\alpha \varkappa}{\detTetr_3} 
		%		& \left( \tetr{\sA}{j} \tetr{\sB}{k} - \frac12 \tetr{\sA}{k} \tetr{\sB}{j} \right) 
		%		\Dfin{\sB}{j} 
		%		- 
		%		\frac{\varkappa}{\detTetr_3} \LCsymb^{ijl} 
		%		\gamma_{ik} \tetr{\sA}{j} \Hfin{\indalg{0}}{l}.
		%	\end{align}
	%\end{subequations}
	with the total (matter+gravity) momentum $ \rho_{\sA} = \itetr{i}{\sA}\rho_i $ and the total 
	energy density $ \rho_{\indalg{0}} = -\alpha^{-1} \Ufin = \NC{\indalg{0}}{0} $ computed from 
	\eqref{eqn.Ja0}, and the gravitational part of the momentum flux $ \stress{i}{\sA} = 
	\itetr{k}{\sA} \stress{i}{k}$  computed from \eqref{eqn.flux.const}.
%	\begin{gather*}
%		\stress{i}{k} := -\Bfin{a}{k} 
%		\Hfin{a}{i} - 
%		\Dfin{a}{k} \Efin{a}{i} + (\Bfin{a}{j} 
%		\Hfin{a}{j} + 
%		\Dfin{a}{j} \Efin{a}{j} 
%		- \Ufin)\delta^k_{\ i},
%	\end{gather*}
	%\begin{align*}
	%	\rhs{i} =&-\rho_k \itetr{k}{\sA} \Efin{\sA}{i} - \LCsymb_{ilj} s^l_{\ k} \itetr{k}{\sA} 
	%	\Bfin{\sA}{j} + s^l_{\ k} \itetr{k}{\sA} \pd{i}\tetr{\sA}{l} + \rho_j 
	%\pd{i}\shift{j},\\[2mm]
	%	\rhs{0} =&- \left(\pd{t} \ln(\alpha) - \shift{k} \pd{k}\ln(\alpha) \right) \Ufin
	%	+ \left( \pd{t} \shift{\sA} - \shift{k} \pd{k}\shift{\sA} \right) \rho_j 
	%	\itetr{j}{\sA}
	%	+ \LCsymb^{klj} \Hfin{a}{l}\Efin{a}{j}\pd{k}\ln(\alpha) + s^k_{\ j} \itetr{j}{\sA} 
	%	\pd{k}\shift{\sA}.
	%\end{align*}
	%while the source terms in the energy-momentum equations read
	%\begin{align*}
	%	\rhs{i} =&-\rho_k \itetr{k}{\sA} \Efin{\sA}{i} + \rho_j \pd{i}\shift{j}
	%	+\stress{j}{k} \w{j}{ki}\\[2mm]
	%%	-\rho_\sA  \Efin{\sA}{i} - \LCsymb_{ilj} s^l_{\ \sA} 
	%%	\Bfin{\sA}{j} + \rho_\sA \pd{i}\shift{\sA} + (-\shift{k} \rho_{\sA}+ s^k_{\ \sA})  
	%%	\pd{i}\tetr{\sA}{k} = 
	%%	-\rho_\sA  \Efin{\sA}{i} - \LCsymb_{ilj} s^l_{\ \sA} \Bfin{\sA}{j}
	%%	+ \rho_k\pd{i}\shift{k} + s^k_{\ j} \w{j}{ik},\\[2mm]
	%	%
	%	\rhs{0} =&
	%	- \left(\pd{t} \ln(\alpha) - \shift{k} \pd{k}\ln(\alpha) \right) \Ufin
	%	+ \left( \pd{t} \shift{\sA} - \shift{k} \pd{k}\shift{\sA} \right) \rho_{\sA}
	%	+ \LCsymb^{klj} \Hfin{\sA}{l}\Efin{\sA}{j}\pd{k}\ln(\alpha) + \stress{\sA}{k} 
	%	\pd{k}\shift{\sA}\\[2mm]
	%	=&
	%	- \left(\pd{t} \ln(\alpha) - \shift{k} \pd{k}\ln(\alpha) \right) \Ufin
	%	+ \rho_j \pd{t} \shift{j} + \stress{j}{k} \pd{k}\shift{j} + \rho_{\sA} \shift{j} 
	%\Efin{\sA}{j} 
	%	+ \LCsymb^{klj} \Hfin{\sA}{l}\Efin{\sA}{j}\pd{k}\ln(\alpha) + \shift{j} \stress{l}{k} 
	%\w{l}{kj},
	%\end{align*}
	%where $ \rho_{\sA} = \itetr{i}{\sA} \rho_i$, 
	%%$ s^k_{\ \sA} = \itetr{i}{\sA} s^k_{\ i} $, 
	%and $ 
	%\w{j}{ik} = \itetr{j}{\sA}\pd{i}\tetr{\sA}{k}$.
	
	In particular, the structure of this system resembles very much the structure of the nonlinear 
	electrodynamics of 
	moving media already solved numerically in \cite{DPRZ2017} for example, as well as the 
	structure of 
	the continuum mechanics
	equations with torsion \cite{Torsion2019}. Moreover,
	despite deep conceptual differences,
	it is identical to 
	the new dGREM formulation of the Einstein equations recently pushed forward in 
	\cite{Olivares2022}.
	However, it is important to mention that the Lagrangian approach adopted here
	in principle permits to generalize the formulation to other $f(\Tscal)$ theories.
	A detailed comparison of 
	these two formulations will be a subject of a future paper.
	
	It is also important to note that the structure of system \eqref{eqn.3+1.summary} remains the 
	same 
	independently of the 
	Lagrangian in use as it can be seen from system \eqref{eqn.PDE.4D} where all the constitutive 
	parts 
	are 
	defined through the derivatives of the potential \eqref{eqn.Legendre1}. If the Lagrangian 
	changes, 
	then only the constitutive functions $ \Efin{a}{\mu} $ and $ \Hfin{a}{\mu} $ have to be 
	recomputed.
	
	\subsection{Constitutive relations}
	
	For the TEGR Lagrangian \eqref{eqn.TEGR.Lagr}, $ \Efin{\sA}{k} $ and $ \Hfin{\sA}{k} $ read
	\begin{subequations}\label{eqn.H.E}
		\begin{multline}
			\Hfin{\sA}{k} := %\Ufin_{\Bfin{\sA}{k}} =
			-\frac{\alpha}{\varkappa \detTetr_3} 
			\mg{\sA\sB}\mg{\sC\sD} 
			\left( \tetr{\sD}{k} \tetr{\sB}{j}
			- \frac12 \tetr{\sD}{j} \tetr{\sB}{k} \right) 
			\Bfin{\sC}{j}
			\\
			- 
					\frac{1}{\varkappa \detTetr_3} \mg{\sA,\sB} \LCsymb^{ijl} \gamma_{ik} 
			\tetr{\sB}{j} 
					\Efin{\indalg{0}}{l}
			% \frac{\detTetr_3}{\varkappa} \LCsymb_{kjl} \itetr{j}{\sA} A^l,
		\end{multline}
		\begin{multline}
			\Efin{\sA}{k} := %\Ufin_{\Dfin{\sA}{k}}=
			-\frac{\alpha \varkappa}{\detTetr_3} 
			\left( \tetr{\sA}{j} \tetr{\sB}{k} - \frac12 \tetr{\sA}{k} \tetr{\sB}{j} \right) 
			\Dfin{\sB}{j} 
			\\
			- 
					\frac{\varkappa}{\detTetr_3} \LCsymb^{ijl} 
					\gamma_{ik} \tetr{\sA}{j} \Hfin{\indalg{0}}{l}.
			% \varkappa \detTetr_3 \LCsymb_{kjl} 
			% \itetr{j}{\sB} \MG{\sA\sB} \Omega^l.
			\label{eqn.E}
		\end{multline}
	\end{subequations}
	Note that these relations can also be written as
	\begin{align}\label{eqn.H.E.split}
		\Hfin{\sA}{k} &= %\Ufin_{\Bfin{\sA}{k}} =
		\frac{\pd{}\Ufin\hfill}{\pd{}\Bfin{\sA}{k}}
		{ 
			-
			\frac{1}{\varkappa}\detTetr_3 \LCsymb_{kjl} \itetr{j}{\sA} A^l},
		\\[2mm]
		\Efin{\sA}{k} &= %\Ufin_{\Dfin{\sA}{k}}=
		\frac{\pd{}\Ufin\hfill}{\pd{}\Dfin{\sA}{k}}
		{
			-
			\varkappa \detTetr_3 \LCsymb_{kjl} 
			\itetr{j}{\sB} \Omega^l}\MG{\sA\sB},
	\end{align}
	where $ A_k = \Efin{\indalg{0}}{k} $ and $ \Omega_k =\Hfin{\indalg{0}}{k}$, and with the potential $ \Ufin $ given by \eqref{eqn.Ufin0}.
	
	
	%Noether current \eqref{eqn.J} becomes:
	%\begin{subequations}
	%\begin{equation}\label{eqn.J_Ai}
	%	\NC{\sA}{i} =-\nc{\sA}{i} 
	%	-\frac{1}{2\alpha}  \beta^i \left( \tetr{\sB}{j} \itetr{k}{\sA} \Dfin{\sB}{j} 
	%	- \frac{\detTetr_3}{\alpha \varkappa} \itetr{j}{\sA} 
	%	\itetr{k}{\sB} \Efin{\sB}{j} \right) \Efin{\indalg{0}}{k}
	%	+ \beta^i \LCsymb_{jkl} \itetr{j}{\sA} \Bfin{\sB}{k} \Dfin{\sB}{l},
	%\end{equation}
	%
	%\begin{equation}\label{eqn.J_A0}
	%	\NC{\sA}{0} =-\nc{\sA}{0}
	%	+ \frac{1}{2\alpha} 
	%	\left(
	%	\itetr{i}{\sA} \tetr{\sB}{k} \Dfin{\sB}{k} - \frac{\detTetr_3}{\alpha \varkappa} 
	%\itetr{k}{\sA} 
	%	\itetr{i}{\sB} \Efin{\sB}{k} 
	%	\right) \Efin{\indalg{0}}{i} - \LCsymb_{ijl} \itetr{i}{\sA} \Bfin{\sB}{j} \Dfin{\sB}{l},
	%\end{equation}
	%
	%\begin{align}\label{eqn.J_0i}
	%	\NC{\indalg{0}}{i} =-\nc{\indalg{0}}{i}
	%	&+ \frac{1}{\alpha} 
	%	\left(
	%	  \Bfin{\sB}{i} \Hfin{\sB}{0}
	%	+ \beta^i \Bfin{\sB}{l} \Hfin{\sB}{l} 
	%	- \beta^l \Bfin{\sB}{i} \Hfin{\sB}{l} \right) \nonumber\\
	%	& + \frac{1}{\alpha} \beta^i
	%	\left(
	%	  \Dfin{\sB}{l} \Efin{\sB}{l} 
	%	+ \Dfin{\indalg{0}}{l} \Efin{\indalg{0}}{l} 
	%	\right) \nonumber\\ 
	%	& + \frac{\detTetr_3}{2\alpha\varkappa}
	%	\left( \itetr{k}{\sB} \gamma^{ij} - \itetr{i}{\sB} \gamma^{kj} \right) \Efin{\sB}{j} 
	%	\Efin{\indalg{0}}{k} \nonumber\\
	%	& + \frac{1}{\alpha} \LCsymb^{ilk} \Hfin{\sB}{l} \Efin{\sB}{k},
	%\end{align}
	%
	%\begin{equation}\label{eqn.J_00}
	%	\NC{\indalg{0}}{0} =-\nc{\indalg{0}}{0} 
	%	- \frac{1}{\alpha} 
	%	\left( 
	%	  \Bfin{\sA}{i} \Hfin{\sA}{i} 
	%	+ \Dfin{\sA}{i} \Efin{\sA}{i}
	%	\right)
	%	+ \frac{\detTetr_3}{\alpha \varkappa} \LCsymb_{ijk} \gamma^{il} \itetr{j}{\sA}\Bfin{\sA}{k} 
	%	\Efin{\indalg{0}}{l}
	%\end{equation}
	%
	%\end{subequations}
	%
	%\begin{subequations}
	%	\begin{align}\label{eqn.j_Ai}
		%		\nc{\sA}{i} =&
		%		\frac{\detTetr}{2\varkappa}\Bigg[
		%		\phantom{+} \frac{2}{\alpha^2} \Bigg( 
		%		\frac{1}{4} \left( \itetr{j}{\sA}\gamma^{ki} - 
		%		\itetr{k}{\sA} 
		%		\gamma^{ij} \right)  
		%		\Efin{\indalg{0}}{j}\Efin{\indalg{0}}{k} %\nonumber\\
		%		%
		%		%&\phantom{ \frac{2}{\alpha^2} \Bigg(} 
		%		+ 
		%		\frac{1}{\alpha} \shift{i} 
		%		\itetr{jk}{\sC\sA} \Efin{\sC}{j} \Efin{\indalg{0}}{k}
		%		-	
		%		\itetr{i}{\sC} \itetr{kj}{\sB\sA} \Efin{\sC}{j}\Efin{\sB}{k}
		%		+
		%		\frac{1}{2} \mg{\sC\sB}\gamma^{ik}\itetr{j}{\sA} \Efin{\sC}{j}\Efin{\sB}{k}
		%		\Bigg) 
		%		\nonumber\\
		%		%
		%		&\phantom{\phantom{+} \frac{2}{\alpha^2} \Bigg(}
		%		+ \frac{2}{\detTetr_3^2} 
		%		\Bigg( 
		%		\tetr{\sD}{k}\mg{\sA\sB\sC\sD} - \itetr{i}{\sA} \tetr{\sE\sD}{kj} 
		%\mg{\sC\sE}\mg{\sB\sD}
		%		\Bigg) \Bfin{\sB}{k}\Bfin{\sC}{i} 
		%		\nonumber\\
		%		%
		%		&\phantom{\phantom{+} \frac{2}{\alpha^2} \Bigg(}
		%		- \frac{2}{\alpha} \LCsymb_{jnk} 
		%		\left(
		%		\gamma^{in}\itetr{l}{\sA}\itetr{j}{\sC} +
		%		\gamma^{il}\itetr{n}{\sA}\itetr{j}{\sC} -
		%		\gamma^{jl}\itetr{n}{\sA}\itetr{i}{\sC}
		%		\right) \Bfin{\sC}{k} \Efin{\indalg{0}}{l} \Bigg]
		%		+ (\Dfin{a}{k} \Efin{a}{k} - \Ufin) \itetr{i}{\sA}
		%	\end{align}
	%where we have used the following notations
	%	\begin{equation}
		%		\itetr{jk}{\sC\sA} = \itetr{j}{\sC} \itetr{k}{\sA} - \frac12 \itetr{j}{\sA} 
		%\itetr{k}{\sC},
		%		\quad
		%		\tetr{\sC\sD}{kj} = \tetr{\sC}{k}\tetr{\sD}{j} - \frac12 \tetr{\sC}{j}\tetr{\sD}{k},
		%		\quad
		%		\mg{\sA\sB\sC\sD} = \mg{\sA\sB} \mg{\sC\sD} - \frac12 \mg{\sA\sC} \mg{\sB\sD}.
		%	\end{equation}
	%\end{subequations}
	
	
	
	\subsection{Stationary differential constraints}
	
	System \eqref{eqn.3+1.summary} is supplemented by several differential constraints. Thus, as 
	already was mentioned, the $ 0 $-\textit{th} equations ($ \mu=0 $) of \eqref{eT} are not 
	actually 
	time-evolution equations 
	but 
	reduce to the so-called Hamiltonian and momentum stationary divergence-type constraints that 
	must 
	hold on the 
	solution to \eqref{eqn.3+1.summary} at every time instant:
	\begin{equation}\label{eqn.div.constrD}
		\pd{k} \Dfin{\indalg{0}}{k} = \rho_{\indalg{0}},
		\qquad
		\pd{k} \Dfin{\sA}{k} = \rho_{\sA}, 
	\end{equation}
	with $ \rho_{\indalg{0}} = -\alpha^{-1} \Ufin = \NC{\indalg{0}}{0} $.
	
	Accordingly, the $ 0 $-\textit{th} equation of \eqref{bT} gives the divergence constraint on 
	the $ 
	\Bfin{\sA}{i} $ field
	\begin{equation}\label{eqn.div.constrB}
		\pd{k} \Bfin{\sA}{k} = 0,
	\end{equation}
	
	Finally, from the definition of $ \Bfin{\sA}{i} $, we also have a curl-type constraint on 
	the spatial components of the tetrad field:
	\begin{equation}\label{eqn.constrh}
		\LCsymb^{kij}\pd{i}\tetr{\sA}{j} = \Bfin{\sA}{k} .	
	\end{equation}
	
	%	\begin{equation}
		%		\LCsymb_{ikj}\pd{k} \aE{\indalg{0}}{j} = 0,
		%	\end{equation}
	
	
	
	\subsection{Algebraic constraints}
	As a consequence of our choice of observer's reference frame \eqref{eqn.4vel}, and the fact 
	that $ \ET{a}{\mu} u^\mu = 0 $ and $ \BT{a}{\mu} u_\mu = 0$, we have 
	the following algebraic constraints 
	\begin{subequations}\label{eqn.constr.alg.EB}
		\begin{equation}
			\Bfin{\indalg{0}}{\mu} = 0, \qquad \Bfin{a}{0} = 0,
		\end{equation}
		\begin{equation}\label{eqn.alg.constrE}
			\Efin{\indalg{0}}{0} = \beta^k \Efin{\indalg{0}}{k}, 
			\qquad 
			\Efin{\indalg{0}}{k} = \pd{k} \alpha,
			\qquad  
			\Efin{\sA}{0} = \beta^j \Efin{\sA}{j},
		\end{equation}
		%	\begin{equation}
			%		W := \beta^k N_k, \qquad 	N_k := \pd{k} \alpha.
			%	\end{equation}
	\end{subequations}
	
	\begin{subequations}
		\begin{gather}\label{eqn.D0.tmp3}
			\Dfin{a}{0} = 0,
			\qquad
			%		\Dfin{\indalg{0}}{k} =-\frac{\detTetr_3}{\varkappa} \LCsymb_{jli} 
			%		\gamma^{jk}\itetr{l}{\sA}\Bfin{\sA}{i},
			\Dfin{\indalg{0}}{k} =-\frac{1}{\varkappa\detTetr_3} \LCsymb^{kli} 
			\Bfinmix{}{li},
			\qquad
			\Dfin{ik}{} = \Dfin{ki}{},
			\\
			\Hfin{\sA}{0} = \shift{k} \Hfin{\sA}{k},
			%	\qquad
			%	\Hfin{\indalg{0}}{k} = \frac{\detTetr_3}{2\varkappa} \LCsymb_{\sA\sB\sC} 
			%\MG{\sC\sD} 
			%	\tetr{\sA}{i} 
			%	\itetr{j}{\sD} \Efin{\sB}{j}
		\end{gather}
	\end{subequations}
	where $ \Dfin{ik}{} = \gamma_{ij} \Dfin{\sA}{j} \tetr{\sA}{k}$.
	As already noted in \cite{AldrovandiPereiraBook}, the gravitational part of the energy-momentum 
	$ 
	\EMmat{\mu}{\nu} $  is trace-free:
	\begin{equation}%\label{key}
		{\EMmat{\mu}{\mu}} = 0.
	\end{equation}
	
	%\subsubsection{Free parameters}
	%
	%The lapse $\alpha$  an shift vector $\shift{k}$ are not subjected to any constraints and can 
	%be 
	%evolved in an arbitrary but meaningful way. 
	
	%$ \LCsymb_{\sA\sB\sC} = \LCsymb_{ijk} \itetr{i}{\sA}\itetr{j}{\sB}\itetr{k}{\sC} $
	
	%\subsection{Summary of change of variables and potentials}
	%
	%\begin{itemize}
	%	\item[$ \left\langle 1 \right\rangle $] 
	%	$ \Lag(\tetr{a}{\mu},\pd{\lambda}\tetr{a}{\nu}) = 
	%	\Laghodge(\tetr{a}{\mu},\HDT{a\mu\nu}) $,  
	%	\quad 
	%	$ \{\tetr{a}{\mu},\pd{\mu}\tetr{a}{\nu}\} \to \{\tetr{a}{\mu},\HDT{a\mu\nu}\}$,
	%	\quad
	%	eqs. \eqref{eqn.EL}$ \to $ \eqref{eqn.1st.order.TEGR}
	%	\\[1mm]
	%	%
	%	\item[$ \left\langle 2 \right\rangle $] 
	%	$ \Laghodge(\tetr{a}{\mu},\HDT{a\mu\nu}) = 
	%	\LagBE(\tetr{a}{\mu},\BT{a}{\mu},\ET{a}{\nu})$,
	%	\quad
	%	$ \{\tetr{a}{\mu},\HDT{a\mu\nu}\} \to \{\tetr{a}{\mu},\BT{a}{\mu},\ET{a}{\nu}\} $,
	%	\quad
	%	eqs. \eqref{eqn.1st.order.TEGR}$ \to $\eqref{eqn.tors.BE}
	%	\\[1mm]
	%	%
	%	\item[$ \left\langle 3 \right\rangle $] 
	%	$ \LagBE(\tetr{a}{\mu},\BT{a}{\mu},\ET{a}{\nu}) \to 
	%	\Um(\tetr{a}{\mu},\Bm{a}{\mu},\Dm{a}{\mu})$,
	%	\quad
	%	$ \{\tetr{a}{\mu},\BT{a}{\mu},\ET{a}{\nu}\} \to \{\tetr{a}{\mu},\Bm{a}{\mu},\Dm{a}{\mu}\}$,
	%	\quad
	%	eqs. \eqref{eqn.tors.BE}$ \to $\eqref{eqn.PDE.4D}
	%	\\[1mm]
	%	%
	%	\item[$ \left\langle 4 \right\rangle $] 
	%	$ \Um(\tetr{a}{\mu},\Bm{a}{\mu},\Dm{a}{\mu}) = \aU(\tetr{a}{\mu},\aB{a}{\mu},\aD{a}{\mu})$,
	%	\quad
	%	$\{\tetr{a}{\mu},\Bm{a}{\mu},\Dm{a}{\mu}\} \to 
	%	\{\tetr{a}{\mu},\aB{a}{\mu},\aD{a}{\mu}\}$,
	%	\quad
	%	eqs. \eqref{eqn.PDE.4D}$ \to $\eqref{eqn.tpo.2}
	%	\\[1mm]
	%	%
	%	\item[$ \left\langle 5 \right\rangle $] 
	%	$ \aU(\tetr{a}{\mu},\aB{a}{\mu},\aD{a}{\mu}) = 
	%	\Ufin(\tetr{a}{\mu},\Bfin{a}{\mu},\Dfin{a}{\mu})$,
	%	\quad
	%	$\{\tetr{a}{\mu},\aB{a}{\mu},\aD{a}{\mu}\} \to 
	%	\{\tetr{a}{\mu},\Bfin{a}{\mu},\Dfin{a}{\mu}\}$,
	%	\quad
	%	eqs. \eqref{eqn.tpo.2}$ \to $\eqref{eqn.3+1.summary}
	%\end{itemize}
	
	
	
	\section{Hyperbolicity of the vacuum 3+1 TEGR equations}

	The 3+1 TEGR system \eqref{eqn.3+1.summary} is an overdetermined system due
	the presence of the momentum and energy equations \eqref{eqn.3+1.m} and
	\eqref{eqn.3+1.e}. The question of hyperbolicity of \eqref{eqn.3+1.summary},
	therefore, requires considering a model for matter, which goes beyond the
	scope of this paper. However, we can still analyze the hyperbolicity of the
	vacuum equations \eqref{eqn.3+1.summary} without the momentum and energy
	equations which might be considered as a starting point of further research.
	In particular, we shall show that the vacuum 3+1 TEGR equations are
	equivalent to the tetrad reformulation of GR by Estabrook-Robinson-Wahlquist
	\cite{Estabrook1997} and Buchman-Bardeen \cite{Buchman2003} which is known
	to be symmetric hyperbolic, and will be referenced to as \ERWBB\
	formulation. We remind that solely hyperbolicity is not enough for
	well-posedness of the initial value problem for a general quasi-linear
	system of first-order equations. At least, to the best of our knowledge,
	there is no an existence and uniqueness theorem for such a class of
	equations in multiple dimensions\footnote{ Despite this, for a system
	describing causal propagation of 	signals, the hyperbolicity is considered
	as a minimum requirement for numerical discretization.}. In contrary, such a
	theorem exists for symmetric hyperbolic systems, e.g. see \cite{Serre2007}.
	
	Let us first introduce the main elements of the \ERWBB\ formulation which relies on the Ricci 
	rotation coefficients
	\begin{align}\label{eqn.Ricci.rot}
		\mathcal{R}_{abc} 
		:= & \bm{\itetrsymbol}_a\cdot\nabla_c \bm{\itetrsymbol}_b \nonumber\\
		= & \itetr{\mu}{a} g_{\mu\nu} \itetr{\lambda}{c} \pd{\lambda} \itetr{\nu}{b} 
		+   \itetr{\mu}{a} g_{\mu\nu} \itetr{\lambda}{c} \Gamma^{\nu}_{\ \lambda\rho} 
		\itetr{\rho}{b}	
	\end{align}
	where $ \nabla_c := \itetr{\lambda}{c} \nabla_{\lambda} $ and $ \nabla_{\lambda} $ is the 
	standard 
	covariant derivative of GR associated with the symmetric Levi-Civita connection, and $ 
	\Gamma^{\mu}_{\ 
		\nu\lambda} $ are the Christoffel symbols of the Levi-Civita connection.
	
	If $ \Tors{a}{bc} = \Tors{a}{\mu\nu}\itetr{\mu}{b}\itetr{\nu}{c} $ and $ \Tors{}{abc} = \mg{ad} 
	\Tors{d}{bc} $ is the torsion written in the tetrad basis, then the relation between the Ricci 
	rotation coefficient and torsion can be expressed by the formula
	\begin{equation}\label{eqn.Ricci.Tors}
		\mathcal{R}_{abc} = \frac{1}{2} 
		\left(  
			\Tors{}{abc} + \Tors{}{bca} - \Tors{}{cab}
		\right),
	\end{equation}
	which also shows that the Ricci rotation coefficients equal exactly to the so-called contortion 
	tensor with the opposite sign $ \mathcal{R}_{abc} =-K_{abc} $, e.g. see 
	\cite[Eq.(1.63)]{AldrovandiPereiraBook}.
	
	Then, 24 independent entries of $ \mathcal{R}_{abc} $ are organized into the following state 
	variables
	\begin{equation}\label{eqn.KN}
		\Kbuch{\sC}{\sA} := \mathcal{R}_{\sA\indalg{0}\sC},
		\qquad
		\Nbuch{\sB}{\sA} := \frac{1}{2} \LCtens^{\sA\sC\sD}\mathcal{R}_{\sC\sD\sB}
	\end{equation}
	and 
	\begin{equation}\label{eqn.aw}
		a_{\sA} = \mathcal{R}_{\sA\indalg{0}\indalg{0}},
		\qquad
		\omega^{\sA} = \frac{1}{2} \LCtens^{\sA\sB\sC} \mathcal{R}_{\sC\sB\indalg{0}}
	\end{equation}
	Here, $ \LCtens_{\sA\sB\sC} = 
	\detTetr_3 \LCsymb_{ijk}\itetr{i}{\sA}\itetr{j}{\sB}\itetr{k}{\sC} $.
	
	It can be shown that the following relations between the \ERWBB\ and TEGR state variables hold
	\begin{subequations}\label{eqn.Buchman.TEGR.KN}
		\begin{align}
			\Kbuch{\sA}{\sB} = & \frac{\varkappa}{\detTetr_3} 
			\left(
			\Dfin{\sA\sB}{}-\frac{1}{2} \Dfin{\sC}{\sC} \mg{\sA\sB}
			\right),\\[2mm]
			-\Nbuch{\sA}{\sB} =&\frac{1}{\detTetr_3}
			\left(
			\Bfinmixx{\sA}{\sB} - \frac{1}{2}\Bfinmix{\sC}{\sC}\KD{\sB}{\sA}
			\right).
			%		\\[2mm]
			%		a_{\sB} = & \frac{1}{\lapse} A_{\sB},\\[2mm]
			%		\omega^{\sB} = & \frac{\varkappa}{\lapse} \MG{\sB\sA} \Omega_{\sA}.
		\end{align}
	\end{subequations}
	We remark that due to our choice of the 3+1 split \eqref{eqn.h.eta.matrix}, i.e. that 
	observer's time vector, $ \bas{\indalg{0}} $, is align with the normal vector to the 
	spatial hypersurfaces, the matrices $ \Kbuch{\sA}{\sB} $ and $ \Dfin{\sA\sB}{} $ are 
	\textit{symmetric}.
	
	Additionally, we have the following relations between the \ERWBB\ vectors $
	a_{\sA} $, $ \omega^{\sA} $ and the \tegr\ vectors $
	A_{k}=\Efin{\indalg{0}}{k} = \pd{k}\lapse $ and $ \Omega_{k} =
	\Hfin{\indalg{0}}{k} $
	\begin{equation}\label{eqn.Buchman.TEGR.AW}
		a_{\sA} = \lapse^{-1} A_{\sA},
		\qquad
		\omega^{\sA} =-\varkappa \lapse^{-1} \MG{\sA\sB} \Omega_{\sB}.
	\end{equation}
	
	Finally, expression of the 
	constitutive fluxes $ \Efin{\sA}{k} $ and $ \Hfin{\sA}{k} $ in terms of $ 
	\{\Kbuch{\sA}{\sB},\Nbuchdown{\sA}{\sB},a_{\sA},\omega_{\sA}\} $ read
	\begin{subequations}\label{eqn.const.KN}
		\begin{align}
			\Efin{}{\sA\sB} &=
			- \lapse \,\Kbuch{\sA}{\sB} 
			+ \lapse \LCtens_{\sA\sB\sC} \omega^\sC ,
			\\[2mm]
			\Hfin{\sB}{\sA} &= 
			\phantom{-}\frac{\lapse}{\varkappa} \Nbuchdown{\sA}{\sB} 
			- \frac{\lapse}{\varkappa} 
			\LCtens_{\sA\sB\sC} a^\sC ,
		\end{align}
	\end{subequations}
	where $\Efin{}{\sA\sB} = \mg{\sA\sC}\itetr{k}{\sB}\Efin{\sC}{k}$ and $\Hfin{\sA}{\sB} = \itetr{k}{\sA}\Hfin{k}{\sB}$.
	
	% Unfortunately, system \eqref{eqn.3+1.summary} is not immediately hyperbolic
	% even for the Minkowski spacetime (it is actually only weakly hyperbolic
	% because some eigenvectors are missing), and therefore it can not be
	% equivalent to the symmetric hyperbolic \ERWBB\ formulation of GR right away.
	% In this section, we discuss what modifications of equations
	% \eqref{eqn.3+1.summary} are necessary in order to make it equivalent to the
	% \ERWBB\ formulation. These modifications rely on the use of Hamiltonian and
	% momentum constraints \eqref{eqn.div.constrD} and symmetry of $
	% \Dfin{\sA\sB}{} $. 
	
	% Because we are considering a vacuum space-time, the momentum and energy
	% equations \eqref{eqn.3+1.m}, \eqref{eqn.3+1.e} can be omitted. Moreover,
	% modifications concern only torsion equations \eqref{eqn.3+1.D},
	% \eqref{eqn.3+1.B}, and therefore for the sake of simplicity, we also omit
	% the tetrad equations \eqref{eqn.3+1.h}
	
	% If one considers 18 equations \eqref{eqn.3+1.D}, \eqref{eqn.3+1.B} written in a 
	% quasilinear form\footnote{Due to the rotational invariance, 
	% 	it is actually sufficient to consider the system in any preferred coordinate direction, say 
	% 	in $ x^1 
	% 	$, without lose of generality.}
	% \begin{equation}\label{eqn.quasi.lin}
	% 	\pd{t} \bm{Q} + \bm{M}^k(\bm{Q})\pd{k} \bm{Q} = \bm{S}
	% \end{equation}
	% for 18 unknowns $ \bm{Q}=\{ \Dfin{\sA}{i}, \Bfin{\sA}{i} \} $ then the matrix $ \bm{M} = \xi_k 
	% \bm{M}^k $ has all real eigenvalues but only 10 linearly independent eigenvectors already for 
	% the 
	% flat Minkowski spacetime, that is the 
	% system is only 
	% weakly hyperbolic. Here, $ \bm{M}^k =
	% \pd{}\bm{F}^k/\pd{}\bm{Q} $ is the Jacobian of the $ k $-\textit{th} flux vector.
	
	
	% Evolution equations \eqref{eqn.3+1.D} and \eqref{eqn.3+1.B} for the torsion
	% fields have a conservative form (i.e. their differential terms are the
	% 4-divergence) which is in particular convenient for developing of high-order
	% methods for hyperbolic equations.
	% %\begin{subequations}\label{eqn.BD.alt0}
	% %	\begin{align}
	% 	%		\pd{t} \Dfin{\sA}{i} + \pd{k}(\shift{i} 
	% 	%		\Dfin{\sA}{k} - \shift{k}\Dfin{\sA}{i}  - \LCsymb^{ikj} \,
	% 	%		\Hfin{\sA}{j}) & 
	% 	%		= \shift{i}\rho_\sA - \stress{\sA}{i},
	% 	%		\label{eqn.BD.alt0.D}
	% 	%		\\[2mm]
	% 	%		%		
	% 	%		\pd{t} \Bfin{\sA}{i} + \pd{k}(\,\shift{i} 
	% 	%		\Bfin{\sA}{k} - \shift{k}\Bfin{\sA}{i}  + \LCsymb^{ikj} 
	% 	%		\Efin{\sA}{j}) & 
	% 	%		= 0.
	% 	%		\label{eqn.BD.alt0.B}
	% 	%	\end{align}
	% %\end{subequations}
	% However, this flux-conservative form of equations has a flaw in the
	% eigenvalues already for the flat spacetime with non-zero shift vector $
	% \shift{k} $. Namely, some characteristic velocities are always $ 0 $ despite
	% the non-zero shift vector. The same issue is well known for the
	% magnetohydrodynamics equations \cite{Powell1999}. From our experience with
	% the non-relativistic equations of nonlinear electrodynamics of moving media
	% \cite{DPRZ2017} and continuum mechanics with torsion \cite{Torsion2019} that
	% have the same structure as \eqref{eqn.3+1.summary}. Instead, the following
	% non-conservative form does not have such a pathology
	% \begin{subequations}\label{eqn.BD.alt}
	% 	\begin{align}
	% 		\pd{t} \Dfin{\sA}{i} + \pd{k}(\shift{i} 
	% 		\Dfin{\sA}{k} - \shift{k}\Dfin{\sA}{i}  - \LCsymb^{ikj} \,
	% 		\Hfin{\sA}{j}) 
	% 		-\shift{i}\pd{k} \Dfin{\sA}{k}
	% 		& 
	% 		= - \stress{\sA}{i},
	% 		\label{eqn.BD.alt.D}
	% 		\\[2mm]
	% 		%		
	% 		\pd{t} \Bfin{\sA}{i} + \pd{k}(\,\shift{i} 
	% 		\Bfin{\sA}{k} - \shift{k}\Bfin{\sA}{i}  + \LCsymb^{ikj} 
	% 		\Efin{\sA}{j}) 
	% 		-\shift{i}\pd{k} \Bfin{\sA}{k}
	% 		& 
	% 		= 0,
	% 		\label{eqn.BD.alt.B}
	% 	\end{align}
	% \end{subequations}
	% and which is obtained from \eqref{eqn.3+1.D} and \eqref{eqn.3+1.B} by
	% turning the momentum $ \rho_\sA $ on the right hand-side of
	% \eqref{eqn.3+1.D} into a differential term according to the momentum
	% constraint $ \rho_\sA = \pd{k}\Dfin{\sA}{k} $ and adding formally a zero, $
	% 0 = \pd{k}\Bfin{\sA}{k} $, to \eqref{eqn.3+1.B}. These modifications are
	% legitimate for smooth solutions thanks to involution constraints
	% \eqref{eqn.div.constrD} and \eqref{eqn.Ja0}. After this, the characteristic
	% velocities have the desired values equal to $ \sim\shift{k} $ or $ \sim
	% \shift{k} \pm c $, if $ c $ is the light speed. However, this only allows to
	% fix the issue with the eigenvalues, while some eigenvectors are still
	% missing.
	
	
	
	
	
	Now, rewriting the 3+1 TEGR equations on $\Dfin{\sA}{i}$ and $\Bfin{\sA}{i}$
	(eqs.~\eqref{eqn.3+1.D} and \eqref{eqn.3+1.B}) in terms of $\Dfin{\sA}{\sB}
	= \tetr{\sB}{i}\Dfin{\sA}{i}$ and $\Bfin{\sA}{\sB} =
	\tetr{\sB}{i}\Bfin{\sA}{i}$, then applying transformations
	\eqref{eqn.Buchman.TEGR.KN} to these equations and using relations
	\eqref{eqn.Buchman.TEGR.KN}--\eqref{eqn.const.KN}, after a lengthy but
	rather straightforward sequence of transformations, one obtains the following equations
	\begin{subequations}\label{eqn.KN.pde}
		\begin{align}
			\pd{\indalg{0}} \Kbuch{\sA}{\sB}
			&
			- \alpha \mg{\sA\sC}\LCtens^{\sC\sD\sE} \pd{\sD} \Nbuchdown{\sE}{\sB}
			- \alpha \pd{\sA} a_{\sB} = \text{l.o.t.}
			\\[2mm]
			\pd{\indalg{0}} \Nbuchdown{\sA}{\sB}
			&
			+ \alpha \mg{\sB\sC}\LCtens^{\sC\sD\sE} \pd{\sD} \Kbuch{\sE}{\sA}
			+ \alpha \pd{\sA} \omega_{\sB} = \text{l.o.t.}
			\label{eqn.KN.pde.N}
		\end{align}
	\end{subequations}
	where 'l.o.t.' stands for 'low-order terms' (i.e. algebraic terms that do
	not contain space and time derivatives), $ \pd{\indalg{0}} =
	\itetr{\mu}{\indalg{0}}\pd{\mu} = u^\mu\pd{\mu} = \lapse^{-1}\pd{t} -
	\lapse^{-1}\shift{k}\pd{k}$, and $ \pd{\sA} = \itetr{k}{\sA}\pd{k} $. 
	We note the opposite order of the subscripts $\sA$ and $\sB$ in \eqref{eqn.KN.pde.N} in the second term with respect to \cite[eq.(40)]{Buchman2003}. This however does not affect the symmetric hyperbolicity of the system discussed in what follows.
	%The low-order terms are given in the appendix\,\ref{}.

	Similar to $ A_k $ and $ \Omega_k $ in TEGR, the vectors $ a_{\sB} $ and $
	\omega_{\sA} $ in the \ERWBB\ formulation of GR are not provided with
	particular evolution equations following from the variational formulation,
	and therefore they are considered as gauge conditions. Hence, one could try
	to choose their evolution equations in such a way as to guaranty the
	well-posedness of the enlarged system for the unknowns $
	\{\Kbuch{\sA}{\sB},\Nbuchdown{\sA}{\sB},a_\sB,\omega_\sB \}$. Thus, as shown
	in \cite{Estabrook1997,Buchman2003}, if coupled with the following gauge
	conditions on $a_\sA$ and $\omega_\sA$:
	\begin{subequations}\label{eqn.aw.pde}
		\begin{align}
			\pd{\indalg{0}} a_\sA - \alpha \MG{\sB\sC}\pd{\sB} \Kbuch{\sC}{\sA} &= 0,
			\\[2mm]
			\pd{\indalg{0}} \omega_\sA + \alpha \MG{\sB\sC}\pd{\sB} \Nbuchdown{\sC}{\sA} &= 0,
		\end{align}
	\end{subequations}
	the resulting system \eqref{eqn.KN.pde}--\eqref{eqn.aw.pde} is symmetric
	hyperbolic, i.e. it can be written in a quasi-linear form
	\begin{equation}\label{eqn.quasi.lin}
		\pd{t} \bm{Q} + \bm{M}^k\pd{k} \bm{Q} = \text{l.o.t.}
	\end{equation}
	with matrices $ \bm{M}^k = \bm{M}^k(\alpha,\itetr{i}{\sA})$ being symmetric for arbitrary $\itetr{i}{\sA}$.
	To see this, one needs to order the entries of
	$\Kbuchmix{\sA}{\sB}$ and $\Nbuchdown{\sA}{\sB}$ in the following way
	$\bm{Q} = \{\Kbuch{\indalg{1}}{\sA},\Kbuch{\indalg{2}}{\sA},\Kbuch{\indalg{3}}{\sA},\Nbuchdown{\sA}{\indalg{1}},\Nbuchdown{\sA}{\indalg{2}},\Nbuchdown{\sA}{\indalg{3}},a_\sA,\omega_\sA\}$.
	
	In summary, the 3+1 \tegr\ equations in vacuum, eqs. \eqref{eqn.3+1.D} and
	\eqref{eqn.3+1.B}, if coupled with the proper gauge conditions on $A_k$ and
	$\Omega_k$, are equivalent to a symmetric-hyperbolic system
	\eqref{eqn.KN.pde}--\eqref{eqn.aw.pde}. However, this is a very preliminary
	results for further investigation of hyperbolicity of the entire 3+1 TEGR
	system because even the time evolution on the tetrad field $\tetr{\sA}{i}$
	was excluded from our consideration.

	% \textbf{REWRITE THIS!!!!!}  However, we can not immediately conclude the
	% same for the original 3+1 TEGR equations \eqref{eqn.3+1.summary} because
	% the differential operators of \eqref{eqn.3+1.summary} and
	% \eqref{eqn.BD.alt2} are different even though the overall equations are
	% equivalent. We recall that the hyperbolicity (well-posedness of the initial
	% value problem) is defined by the leading-order terms of the differential
	% operator (all differential terms in the case of first-order systems).
	% Nevertheless, this is an important result demonstrating that inside of the
	% class of equations equivalent to TEGR equations \eqref{eqn.3+1.summary}
	% there is a possibility, at least one given by \eqref{eqn.BD.alt2},  to
	% rewrite the TEGR vacuum equations as a symmetric hyperbolic system.
	
	%The second step to recovering hyperbolicity of the TEGR equations consists in not modifying 
	%the 
	%equations but extending the set of state variables and the set of evolution equations.
	%Thus, following to \cite{Estabrook1997,Buchman2003}, we chose also to evolve the gauge 
	%freedoms $ 
	%A_k = 
	%\Efin{\indalg{0}}{k} $ and $ \Omega_k = \Hfin{\indalg{0}}{k} $ according to
	%\begin{subequations}\label{eqn.AOmega}
	%	\begin{align}
		%		&\pd{\indalg{0}}A_i      - \pd{k} \Efinn{k}{i} = 0, 
		%		\\[2mm]
		%		&\pd{\indalg{0}}\Omega_i + \pd{k} \Hfinnmix{i}{k} = 0.
		%	\end{align}
	%\end{subequations}
	%We set the right hand-side of these two equations to zero because it is not important for the 
	%hyperbolicity. In general, one may want to chose any other gauge conditions, e.g. see 
	%\cite{Buchman2003}. While increasing number of evolution equations, this modification 
	%drastically 
	%improves the situation with the missing eigenvectors. Now, the system of 24 equations 
	%\eqref{eqn.3+1.D}, 
	%\eqref{eqn.3+1.B}, and \eqref{eqn.AOmega} has only 2 eigenvectors missing, see the 
	%supplementary 
	%materials [??].
	
	
	
%	\section{Symmetrization}
%	
%	The energy is
%	\begin{equation}
%		\Ufin(\tetr{\sA}{k},\Dfin{\sA}{k},\Bfin{\sA}{k}) = 
%		-\frac{\alpha}{2\detTetr_3} \left( 
%		\varkappa \left( \Dfin{\sB}{\sA} \Dfin{\sA}{\sB} - \frac{1}{2} 
%(\Dfin{\sA}{\sA})^2\right)
%		+
%		\frac{1}{\varkappa} \left( \Bfinmix{\sA}{\sB} \Bfinmix{\sB}{\sA} - 
%\frac{1}{2} 
%		(\Bfinmix{\sA}{\sA})^2
%		\right)
%		\right).
%	\end{equation}
%	
%	Having this expression for the potential $ \Ufin $ and introducing notations
%	\begin{equation}
%		\Hfinn{\sA}{k} = \frac{\partial \Ufin\hfill}{\partial \Bfin{\sA}{k}},
%		\qquad
%		\Efinn{\sA}{k} = \frac{\partial \Ufin\hfill}{\partial \Dfin{\sA}{k}},	
%	\end{equation}
%	the constitutive relations \eqref{eqn.H.E} can be rewritten as
%	\begin{equation}\label{eqn.H.E.split}
%		\Hfin{\sA}{k} := %\Ufin_{\Bfin{\sA}{k}} =
%		\Hfinn{\sA}{k}
%		{ 
%			\color{blue}
%			-
%			\varkappa^{-1}\detTetr_3 \LCsymb_{kjl} \itetr{j}{\sA} A^l},
%		\qquad
%		\Efin{\sA}{k} := %\Ufin_{\Dfin{\sA}{k}}=
%		\Efinn{\sA}{k}
%		{
%			\color{blue}
%			-
%			\varkappa \detTetr_3 \LCsymb_{kjl} 
%			\itetr{j}{\sB} \MG{\sA\sB} \Omega^l}.
%	\end{equation}
%	
%	In turn, the equations for $ \Dfin{\sA}{k} $ and $ \Bfin{\sA}{k} $ can be 
%written as 
%	\begin{subequations}\label{eqn.BD.hat}
%		\begin{align}
%				\pd{t} \Dfin{\sA}{i} + \pd{k}(\shift{i} 
%				\Dfin{\sA}{k} - \shift{k}\Dfin{\sA}{i}  - \LCsymb^{ikj} \,
%				\Hfinn{\sA}{j}) 
%				&
%				{ + \color{red} \pd{k} 
%						\left(
%						\itetr{i}{\sA} \hat{A}^k - \itetr{k}{\sA} \hat{A}^i
%						\right) }
%				-\shift{i}\pd{k} \Dfin{\sA}{k}
%				= - \stress{\sA}{i},
%				\\[2mm]
%				\pd{t} \Bfin{\sA}{i} + \pd{k}(\,\shift{i} 
%				\Bfin{\sA}{k} - \shift{k}\Bfin{\sA}{i}  + \LCsymb^{ikj} 
%				\Efinn{\sA}{j}) 
%				&
%				{ - \color{red}\pd{k} 
%						\left(
%						\itetr{i}{\sB} \hat{\Omega}^k - \itetr{k}{\sB} 
%\hat{\Omega}^i
%						\right) \MG{\sA\sB}}
%				-\shift{i}\pd{k} \Bfin{\sA}{k}
%				= 0,
%			\end{align}
%	\end{subequations}
%	where {\color{blue}$ \hat{A}^i = \varkappa^{-1} \detTetr_3 A^i $, $ 
%\hat{\Omega}^i = \varkappa 
%		\detTetr_3 
%		\Omega^i $.} 
%	
%	\IP{ If we simply set $ A_i = 0 $ and $ \Omega_i = 0 $, this system of 18 
%equations has 
%		\textit{only 10 eigenvectors} for the flat spacetime.}
%	
%	Currently, the main natural idea, also exploited by Buchman et al 
%\cite{Buchman2003} (see 
%	eq.(44) 
%	or (47)), is to add 
%	two PDEs for the $ A_k = \Efin{\indalg{0}}{k} $  and $ \Omega_k = 
%\Hfin{\indalg{0}}{k} $. 
%	
%	\IP{The 
%		question is how to find these PDEs. So far, I tested the following 
%simple equations:}
%	\begin{subequations}
%		\begin{align}
%				&\pd{t}A_i - \shift{k}\pd{k}A_i + \itetr{k}{\sA} \pd{k} 
%\Efin{\sA}{i} = rhs, 
%				\\[2mm]
%				&\pd{t}\Omega_i -\shift{k}\pd{k}\Omega_i + \itetr{k}{\sA} \pd{k} 
%\Hfin{\sA}{i} = 
%		rhs,  
%			\end{align}
%	\end{subequations}
%	or in terms of \eqref{eqn.H.E.split}
%	\begin{subequations}\label{eqn.A.Omega}
%		\begin{align}
%				&\pd{t}A_i - \shift{k}\pd{k}A_i + {\color{red}\itetr{k}{\sA} 
%\pd{k} \Efinn{\sA}{i}}
%				{\color{blue}
%					-\itetr{k}{\sA} \pd{k}(	\LCsymb_{ijl} 
%					\itetr{j}{\sB} \hat{\Omega}^l) \MG{\sA\sB}
%					}
%				 = rhs, 
%				\\[2mm]
%				&\pd{t}\Omega_i -\shift{k}\pd{k}\Omega_i 
%				+ 
%				{\color{red}\itetr{k}{\sA} \pd{k} 
%					\Hfinn{\sB}{i} \MG{\sA\sB}
%					} 
%				{\color{blue}
%					- \itetr{k}{\sA} \pd{k}(\LCsymb_{ijl} \itetr{j}{\sB} 
%\hat{A}^l) \MG{\sA\sB}
%					}= rhs.
%			\end{align}
%	\end{subequations}
%	Now, \eqref{eqn.BD.hat} and \eqref{eqn.A.Omega} is a system for 24 unknowns  
%$ \{ 
%	\Dfin{\sA}{i},\Bfin{\sA}{i},A_i,\Omega_i\} $ which has 22 eigenvectors. 
%Therefore, 2 are 
%	missing.
%	
%	\IP{The goal is to try to symmetrize \eqref{eqn.BD.hat}, 
%\eqref{eqn.A.Omega}. The multipliers 
%	are 
%		$ 
%		\Efinn{\sA}{i} $, $ \Hfinn{\sA}{i} $, $ \hat{A}^i $, and $ 
%\hat{\Omega}^i $}
%	
	
	
	
	\section{Relation between the torsion and extrinsic curvature}
	
	It is useful to relate the state variables of TEGR to the conventional quantities used in 
	numerical 
	general relativity \cite{ADM2008,Baumgarte2003a,Gourgoulhon2012a}. In 
	what follows, we relate the spatial extrinsic curvature of GR to the $ \Dfin{\sA}{k} $ field. 
	We remark 
	that the two fields are conceptually different due to the different geometry interpretations in 
	GR and TEGR. Therefore, the obtained relation is possible only due to the equivalence of TEGR 
	and GR. 
	
	In GR, the evolution equation of the spatial metric $ \gamma_{ij} $ is (see 
	\cite[Eq.(7.64)]{RezzollaZanottiBook})
	%$\tilde{\gamma}_{ij} = 
	%\phi^{2}\gamma_{ij} $ reads \cite{CCZ4,FO-CCZ4}
	%\begin{equation}\label{eqn.gamma.CCZ4}
	%	\pd{t}\tilde{\gamma}_{ij} - \shift{k} \pd{k} \tilde{\gamma}_{ij} - \tilde{\gamma}_{ki} 
	%	\pd{j}\shift{k}
	%	-\tilde{\gamma}_{kj}\pd{i}\shift{k} + \frac{2}{3} \tilde{\gamma}_{ij} \pd{k}\shift{k} = -2 
	%	\alpha \phi^2 (K_{ij} - \frac13 K \gamma_{ij}),
	%\end{equation} 
	\begin{equation}\label{eqn.K}
		\pd{t} \gamma_{ij} - \shift{l}\pd{l}\gamma_{ij} - \gamma_{il}\pd{j}\shift{l}
		-\gamma_{jl}\pd{i}\shift{l} = - 2 \alpha K_{ij},
	\end{equation}
	%$ \phi = \det(\gamma_{ij})^{-1/6} = \detTetr_3^{-1/3} $,
	where  $ K_{ij} $ is the spatial extrinsic 
	curvature.
	On the other hand, in TEGR, contracting \eqref{eqn.tetr.3+1} with $ \mg{\sA\sB} \tetr{\sB}{j} 
	$, one 
	obtains 
	the following evolution equation for the spatial metric:
	\begin{multline}\label{eqn.gamma.PDE}
		\pd{t} \gamma_{ij} - \shift{l}\pd{l}\gamma_{ij} - \gamma_{il}\pd{j}\shift{l}
		-\gamma_{jl}\pd{i}\shift{l} = \\
		-\mg{\sA\sB}(\tetr{\sA}{i} \Efin{\sB}{j} + \tetr{\sA}{j} 
		\Efin{\sB}{i}).
	\end{multline} 
	Therefore, one can deduce an expression for the extrinsic curvature in terms of $ 
	\Efin{\sA}{i} $: 
	\begin{equation}\label{key}
		K_{ij} = \frac{1}{2\alpha}\mg{\sA\sB}(\tetr{\sA}{i} \Efin{\sB}{j} + \tetr{\sA}{j} 
		\Efin{\sB}{i}).
	\end{equation}
	
	To obtain another expression for $ K_{ij} $ in terms of the primary state variable $ 
	\Dfin{\sA}{i} 
	$, one needs to use the the constitutive relationship \eqref{eqn.E} to deduce
%	\begin{align}
%		&\mg{\sA\sB} (\tetr{\sA}{i}\Efin{\sB}{j} + \tetr{\sA}{j}\Efin{\sB}{i})
%		=\\
%		&-\frac{\alpha \varkappa}{\detTetr_3}  \mg{\sA\sB}  \bigg( 
%		\tetr{\sB}{k} (\tetr{\sA}{i}\tetr{\sC}{j} + \tetr{\sA}{j}\tetr{\sC}{i})
%		-
%		\frac12
%		(\tetr{\sA}{i}\tetr{\sB}{j} + \tetr{\sA}{j}\tetr{\sB}{i})\tetr{\sC}{k}
%		\bigg) \Dfin{\sC}{k}.
%	\end{align}
%	Therefore, the following expression can be obtained
%	\begin{equation}
%		K_{ij} = -\frac{\varkappa}{2 \detTetr_3}\mg{\sA\sB} \bigg( 
%		\tetr{\sB}{k} (\tetr{\sA}{i}\tetr{\sC}{j} + \tetr{\sA}{j}\tetr{\sC}{i})
%		-
%		\frac12
%		(\tetr{\sA}{i}\tetr{\sB}{j} + \tetr{\sA}{j}\tetr{\sB}{i})\tetr{\sC}{k}
%		\bigg) \Dfin{\sC}{k},
%	\end{equation}
	\begin{equation}
		K_{ij} = -\frac{\varkappa}{\detTetr_3} 
		\left ( 
		\Dfin{ij}{} - \frac{1}{2} \Dfin{k}{k} \gamma_{ij}
		\right ) ,
	\end{equation}
	which, if contracted, gives the relationship for the traces $ K^i_{\ i} = \gamma^{ij}K_{ji} $ 
	and $ 
	\Dfin{i}{i} =  \Dfin{\sA}{i}\tetr{\sA}{i} $
	\begin{equation}
		K^i_{\ i} = \frac{\varkappa}{2\detTetr_3}\Dfin{i}{i}.
	\end{equation}
	
	Remark that if written in the tetrad frame $ K_{\sA\sB} = 
	\itetr{i}{\sA}\itetr{j}{\sA} K_{ij}$ then the extrinsic curvature is exactly $ \Kbuch{\sA}{\sB} 
	$ 
	introduced in 
	\eqref{eqn.Buchman.TEGR.KN} apart from the opposite sign
	\begin{equation}
		K_{\sA\sB} = -\Kbuch{\sA}{\sB}.
	\end{equation}  
	
	
	
	
	\section{Torsion scalar}\label{sec.closure}
	
	
	
	In \tegr\ and its $ f(\Tscal) $-extensions, the Lagrangian density is a function of the 
	torsion scalar $ \Tscal $, e.g. in \tegr, the Lagrangian density is
	\begin{subequations}\label{eqn.TEGR.Lagr}
		\begin{equation}
			\Lagtors(\tetr{a}{\mu},\Tors{a}{\mu\nu}) = \frac{\detTetr}{2 \, \varkappa} \Tscal,
		\end{equation}
		\begin{align}\label{eqn.tors.scal0}
			\Tscal(\tetr{a}{\mu},\Tors{a}{\mu\nu}) &:= \\
			&\frac14 g^{\beta\lambda} g^{\mu\gamma} g_{\alpha\eta} \Tors{\alpha}{\lambda\gamma}
			\Tors{\eta}{\beta\mu} \\
			&+
			%
			\frac12 g^{\mu\gamma} \Tors{\lambda}{\mu\beta} \Tors{\beta}{\gamma\lambda} \\
			&- 
			%
			g^{\mu\lambda} \Tors{\rho}{\mu\rho} \Tors{\gamma}{\lambda\gamma},	  
		\end{align}
	\end{subequations}
	where $ \varkappa = 8\pi G c^{-4} $ is the Einstein gravitational constant, $ G $ is the 
	gravitational constant, $ c $ is the speed of light in vacuum.
	
	
	
	
	
	%If one needs the Lagrangian density $ \Laghodge(\tetr{a}{\mu},\HDT{a\mu\nu}) $ as a function 
	%of 
	%the 
	%Hodge dual, then the torsion scalar $ \Tscal $ can be expressed as
	%\eqref{eqn.Lagrangians2}
	%\begin{equation}\label{eqn.Tscal.Hodge}
	%	\Tscal(\tetr{a}{\mu},\HDT{a\mu\nu}) = \frac12 \HDT{\mu\nu\lambda} (\HDmix_{\lambda\mu\nu} + 
	%	\HDmix_{\nu\lambda\mu} + g_{\mu\lambda} \HDmix^{\gamma}_{\phantom{\gamma\,} \nu\gamma})
	%\end{equation}
	
	
	However, to close system \eqref{eqn.3+1.summary}, we need not the Lagrangian $ 
	\Lagtors(\tetr{a}{\mu},\Tors{a}{\mu\nu})  $ directly but we need to perform a sequence of 
	variable 
	and potential changes: $ \Lagtors(\tetr{a}{\mu},\Tors{a}{\mu\nu}) = 
	\LagBE(\tetr{a}{\mu},\ET{a}{\mu},\BT{a}{\mu}) $ $ \longrightarrow $ $ \ET{a}{\mu} 
	\Dm{a}{\mu} - \LagBE = 
	\Um(\tetr{a}{\mu},\Dm{a}{\mu},\Bm{a}{\mu}) = 
	\Ufin(\tetr{a}{\mu},\Dfin{a}{\mu},\Bfin{a}{\mu}) $. 
	%in terms of 
	%the state vector $ \vec{Q} = \{ 
	%\tetr{a}{\mu}, \aD{a}{i}, \aB{a}{\mu} \} $ which can be obtained from the following expresion 
	%in 
	%terms of $ \{ \tetr{a}{\mu},\ET{a}{\mu},\BT{a}{\mu}\} $:
	Thus, we have
	\begin{align}\label{eqn.Tscal.EB}
		&\LagBE(\tetr{a}{\mu},\ET{a}{\mu},\BT{a}{\mu}) = \nonumber\\
		&\frac{\detTetr}{2 \, \varkappa}
		\bigg(
		- \frac12(\ETmix{\indalg{0}\lambda}{}\ETmix{\indalg{0}}{\ \lambda} -  
		2\ETmix{\lambda}{\ \,\lambda}\ETmix{\beta}{\ \,\beta} +
		\ETmix{\ \,\lambda}{\beta} \ETmix{\beta}{\ \,\lambda}  +
		\ETmix{\lambda}{\ \,\beta}\ETmix{\beta}{\ \,\lambda} )
		\nonumber
		\\
		& + \LCsymb_{\lambda\gamma\eta\rho} u^\eta 
		(\ETmix{\lambda\gamma}{}\BTmix{\indalg{0}\rho}{} + 2 
		\ET{\indalg{0}\lambda}{}\BTmix{\gamma\rho}{})
		\nonumber
		\\
		&- \frac12 h^{-2} ( \BTmix{\indalg{0}\lambda}{}\BTmix{\indalg{0}}{\ \lambda}
		+ \BTmix{\lambda}{\ \,\lambda}\BTmix{\beta}{\ \,\beta}
		- 2\BTmix{\lambda}{\ \,\beta}\BTmix{\beta}{\ \,\lambda}
		)
		\bigg).
	\end{align}
	%or, using algebraic constraints \eqref{eqn.constr.alg.EB}
	%\begin{multline}\label{eqn.Tscal.EB1}
	%	\LagBE(\tetr{a}{\mu},\ET{a}{\mu},\BT{a}{\mu}) = \frac{h}{2\,\varkappa} \bigg( 
	%	\ETmix{k}{\ \,k}\ETmix{i}{\ \,i} 
	%	-\frac12 \ETmix{k}{\ \,i} \ETmix{i}{\ \,k}
	%	-\frac12 \ETmix{\ \,k}{i} \ETmix{i}{\ \,k} 
	%	%
	%	-\frac12 \detTetr^{-2} 
	%	\BTmix{i}{\ \,i}\BTmix{k}{\ \,k}
	%	+ \detTetr^{-2} \BTmix{i}{\ \,k}\BTmix{k}{\ \,i} 
	%	\\
	%	%
	%	{\color{red}-} 2 \, \alpha^{-1} \LCsymb_{kli} N^k \BTmix{l i}{}
	%	\bigg),
	%\end{multline}
	%where  $ N^k = N_j \gamma^{jk} $, $ \ET{i}{j} = 
	%\itetr{i}{\sA} \ET{\sA}{j}$, $ \ETmix{\ \,k}{i} = \gamma_{ij} \ET{j}{l} \gamma^{lk} 
	%$, $ \BT{i}{j} = \itetr{i}{\sA} 
	%\BT{\sA}{j} $,  and $ \BTmix{i}{\ \,k} = 
	%\BT{i}{j} \gamma_{jk}$. Also, in \eqref{eqn.Tscal.EB1}, $ \LCsymb_{ijk} $ is the pure spatial 
	%Levi-Civita symbol with the reference $ \LCsymb_{123}=1 $ and $ \LCsymb_{ijk} = \LCsymb^{ijk} 
	%$ 
	%(in 
	%particular $ \LCsymb_{0123} = -\LCsymb_{123} $).
	%
	
	In turn, if we perform the Legendre transform $ \Um(\tetr{\sA}{k},\Dm{\sA}{k},\Bm{\sA}{k}) = 
	\ET{a}{\mu} 
	\Dm{a}{\mu} - \LagBE$
	then the new potential $ \Um $ reads
	\begin{align}\label{eqn.Lagr.DB}
		\Um(\tetr{\sA}{k},\Dm{\sA}{k},\Bm{\sA}{k}) &= \nonumber\\ 
	  	\frac{1}{4\,h} \bigg( &\varkappa \left(
		\Dm{i}{i}\Dm{k}{k} - 2 \Dm{i}{k}\Dm{k}{i}
		\right)\nonumber\\
		% 
		 + &\frac{1}{\varkappa} \left ( 
		\Bmmix{i}{\ \,i}\Bmmix{k}{\ \,k}
		- 2 \Bmmix{i}{\ \,k}\Bmmix{k}{\ \,i}
		\right ) \nonumber\\
		+ \varkappa (\beta^j \Dm{j}{0} + &2 \beta^j \Dm{j}{0}\Dfin{k}{k} - 4 \beta^j \Dm{k}{0} 
		\Dfin{j}{k}) 
		\bigg),
	\end{align}
	where $ \Dm{k}{i} = \Dm{\sA}{i} \tetr{\sA}{k} $ and $ \Bmmix{k}{\ i} = \Bm{\sA}{j} 
	\itetr{k}{\sA} 
	\gamma_{ji} $, and in the last two terms, one should pay attention that the new field $ 
	\Dfin{\sA}{k} $ introduced in \eqref{eqn.varDB.final} appears there.
	
	Apparently, apart from the last terms in \eqref{eqn.Lagr.DB} depending on $ \Dm{k}{0} $, the 
	potential 
	$ \Um $ is more symmetric in the variables $ \Dm{a}{\mu} $ and $ 
	\Bm{a}{\mu} $ rather than the Lagrangian $ \LagBE $ in $ \ET{a}{\mu} $ and $ \BT{a}{\mu} $. 
	This in 
	particular, justifies the introduction of the new and final variables \eqref{eqn.varDB.final}
	\begin{equation}\label{eqn.db}
		\Dfin{\sA}{k} = \Dm{\sA}{k} + \beta^k \Dm{\sA}{0}, \qquad \Bfin{\sA}{k} = \Bm{\sA}{k},
	\end{equation}
	so that the potential $ \Um(\tetr{\sA}{k},\Dm{\sA}{k},\Bm{\sA}{k}) = 
	\Ufin(\tetr{\sA}{k},\Dfin{\sA}{k},\Bfin{\sA}{k})$ becomes just
	%\begin{subequations}
	\begin{align}\label{eqn.Ufin}
		\Ufin(\tetr{\sA}{k},\Dfin{\sA}{k},\Bfin{\sA}{k}) &= \nonumber\\
		-\frac{\alpha}{2\detTetr_3} \bigg( 
		&\varkappa \left( \Dfin{\sB}{\sA} \Dfin{\sA}{\sB} 
		- \frac{1}{2} (\Dfin{\sA}{\sA})^2\right)\nonumber\\
		+
		&\frac{1}{\varkappa} \left( \Bfinmix{\sA}{\sB} \Bfinmix{\sB}{\sA} - \frac{1}{2} 
		(\Bfinmix{\sA}{\sA})^2
		\right)
		\bigg).
		%		-\frac{\alpha}{2\detTetr_3} \left( 
		%		\varkappa \left( I_{\Dfin{}{}2} - \frac{1}{2} I_{\Dfin{}{}1}\right)
		%		+
		%		\frac{1}{\varkappa} \left( I_{\Bfin{}{}2} - \frac{1}{2} I_{\Bfin{}{}1}
		%		\right)
		%		\right) ,
	\end{align}
	
	
	
	\section{Conclusion and discussion}
	
	We have presented a 3+1-split of the TEGR equations in their historical pure
	tetrad version, i.e. with the spin connection set to zero. To the best of
	our knowledge, there were not many attempts to obtain a 3+1-split of the
	TEGR equations, e.g. \cite{Capozziello2021,Pati2022}, and we are not aware
	of any attempts to solve numerically the full TEGR system of equations for
	general spacetimes. This paper, therefore, may help to cover this gap.
	However, first attempts to solve similar equations for the dGREM equations
	\cite{Olivares2022} in vacuum were done recently \cite{oliynyk2025} with a
	structure compatible discretization.
	
	Our derivation started from the action integral of TEGR with an arbitrary
	Lagrangian denisty, and therefore extensions of TEGR such as $ f(\Tscal)
	$-teleparallel theories also can be covered in the future. After separating
	the spatial and time components of the torsion, the 3+1 governing partial
	differential equations have appeared to have the same structure as equations
	of nonlinear electrodynamics \cite{DPRZ2017} and equations for continuum
	fluid and solid mechanics with torsion \cite{Torsion2019} if, howeveer, the
	gauge vectors $A_k$ and $\Omega_k$ are set to $0$. For more general gauge
	conditions on $A_k$ and $\Omega_k$, the 3+1 equations have a structure of
	the Maxwell equations coupled with the acoustic-type equations, e.g. see
	\cite{MaxwellGLM}. Moreover, it has appeared that the derived equations are
	equivalent to the recently proposed dGREM tetrad formulation of GR
	\cite{Olivares2022}. However, we emphasize that that the starting point of
	\cite{Olivares2022} was different. The fundamental variables of dGREM
	formulation are the frame field, their exterior derivatives, and the
	Nester-Witten and Sparling forms.
	
	The derived 3+1 TEGR equations are not immediately hyperbolic as usually the
	case for many first-order reductions of the Einstein equations. We
	demonstrated that for the vacuum equations the differential operator of the
	equations can be transformed into a different but equivalent form which is
	equivalent to the symmetric hyperbolic tetrad reformulation of GR by
	Estabrook-Robinson-Wahlquist \cite{Estabrook1997} and Buchman-Bardeen
	\cite{Buchman2003}. The question of hyperbolicity of the full 3+1 TEGR
	equations coupled with matter is still open and requires further
	investigation.
	
	%Additionally, the equations have the structure of the so-called Symmetric Hyperbolic 
	%Thermodynamically 
	%Compatible (SHTC) class of equations \cite{God1961,SHTC-GENERIC-CMAT}, however the 
	%hyperbolicity 
	%of 
	%the proposed formulation is left unexplored in this paper. Furthermore, as any SHTC equations, 
	%the 
	%proposed model likely to have also a Hamiltonian structure and underlying Poisson bracket 
	%formulation \cite{SHTC-GENERIC-CMAT,PKG_Book2018}. This would be very important for developing 
	%of 
	%so-called structure-preserving integrators \cite{HTC2022} in order to respect various 
	%continuous 
	%properties of the 
	%governing equations, such as involutions \eqref{eqn.div.constr}, at the discrete level, see 
	%also 
	%the 
	%discussion on this topic in \cite{Olivares2022}.
	
	Despite it is argued that TEGR is fully equivalent to Einstein's general
	relativity, the proposed 3+1 TEGR equations have yet to be carefully tested
	in a computational code and have yet to be proved to pass all the standard
	benchmark tests of GR. Therefore, further research will concern
	implementation of the TEGR equations in a high-order discontinuous Galerkin
	code \cite{Dumbser2018a,Busto2020}, with a possibility of constraint
	cleaning  \cite{Dumbser2019} and well-balancing \cite{Gaburro2021}. This in
	particular would allow a direct comparison of the TEGR with other 	3+1
	equations of GR, such as Z4 formulations \cite{Alic2012} forwarded by Bona
	\textit{et al} in \cite{Z4}, and FO-CCZ4 by Dumbser \textit{et al}
	\cite{FO-CCZ4}, a strongly hyperbolic formulations of GR, within the same
	computational code.  Another numerical strategy would be to use staggered
	grids and to develop a structure-preserving discretization
	\cite{SIGPR2021,Olivares2022,Fambri2020a,oliynyk2025} that should allow to keep errors
	of div and curl-type involution constraints of TEGR at the machine
	precision.
	
	
	\paragraph{Acknowledgments}
	I.P. is grateful to A.~Golovnev for sharing his thoughts about various
	aspects of teleparallel gravity that were important for the appearance of
	this work. Also, I.P. would like to thank M.~Dumbser, O.~Zanotti, E.R. Most
	for their support and insightful discussions. I.P. is a member of the Gruppo
	Nazionale per il Calcolo Scientifico of the Istituto Nazionale di Alta
	Matematica (INdAM GNCS) and acknowledges the financial support received from
	the Italian Ministry of Education, University and Research (MIUR) in the
	frame of the Departments of Excellence Initiative 2018–2022 attributed to
	the Department of Civil, Environmental and Mechanical Engineering (DICAM) of
	the University of Trento (Grant No. L.232/2016) and in the frame of the
	Progetti di Rilevante Interesse Nazionale (PRIN) 2017, Project No.
	2017KKJP4X, “Innovative numerical methods for evolutionary partial
	differential equations and applications”.
	%
	Funding for H.O. comes from Radboud University Nijmegen through a Virtual
	Institute of Accretion (VIA) postdoctoral fellowship from the Netherlands
	Research School for Astronomy (NOVA).
	%
	The work of E.R. is supported by the Mathematical Center in Akademgorodok
	under the agreement No. 075-15-2025-348 with the Ministry of Science and
	Higher Education of the Russian Federation.
	
	
	%-------------------------------------------------------
	\appendix
	%-------------------------------------------------------
	
	%
	%%-------------------------------------------------------
	%\section{Spacetime form of the energy-momentum}\label{app.sec.EM}
	%%-------------------------------------------------------
	%
	%Here, we show how the energy-momentum conservation law \eqref{eqn.Noether.cons} can be written 
	%in 
	%a 
	%pure 
	%spacetime form \eqref{eqn.EM3}.
	%
	%Contracting 
	%\begin{equation}%\label{key}
	%	\pd{\mu} \Laghodge_{\tetr{a}{\mu}} = 0
	%\end{equation} with $ \tetr{a}{\nu} $ and then adding to it $ 0\equiv 
	%\Laghodge_{\tetr{a}{\mu}}\pd{\mu}\tetr{a}{\nu} -  
	%\Laghodge_{\tetr{a}{\mu}}\pd{\mu}\tetr{a}{\nu} $ 
	%one gets
	%\begin{equation}
	%\pd{\mu}(\tetr{a}{\nu}\Laghodge_{\tetr{a}{\mu}}) - 
	%\Laghodge_{\tetr{a}{\mu}}\pd{\mu}\tetr{a}{\nu} 
	%= 
	%0.
	%\end{equation}
	%Replacing $ \pd{\mu}\tetr{a}{\nu} $ with $ \pd{\mu}\tetr{a}{\nu} = \pd{\nu}\tetr{a}{\mu} + 
	%\Tors{a}{\mu\nu} $ 
	%yields
	%\begin{equation}
	%\pd{\mu}(\tetr{a}{\nu}\Laghodge_{\tetr{a}{\mu}}) - 
	%\Laghodge_{\tetr{a}{\mu}}\pd{\nu}\tetr{a}{\mu} 
	%- 
	%\Laghodge_{\tetr{a}{\mu}}\Tors{a}{\mu\nu} = 0.
	%\end{equation}
	%Then, using the fact that $ L = L(\tetr{a}{\mu},\HDT{a\mu\nu}) $, the second term can be 
	%substituted 
	%by $ \Laghodge_{\tetr{a}{\mu}}\pd{\nu}\tetr{a}{\mu} = \pd{\nu}\Laghodge - 
	%\Laghodge_{\HDT{a\lambda\rho}}\pd{\nu}\HDT{a\lambda\rho} $. This results in
	%\begin{equation}\label{app.eqn.EM1}
	%\pd{\mu}(\tetr{a}{\nu}\Laghodge_{\tetr{a}{\mu}} - L \delta^\mu_{\ \nu}) +
	%\Laghodge_{\HDT{a\lambda\rho}}\pd{\nu}\HDT{a\lambda\rho} -
	%\Laghodge_{\tetr{a}{\mu}}\Tors{a}{\mu\nu} = 0.
	%\end{equation} 
	%Now, the energy-momentum current $ -\Laghodge_{\tetr{a}{\mu}} $ in the last term can be 
	%substituted 
	%by its 
	%expression from the Euler-Lagrange equation \eqref{eqn.EM.Hodge}:
	%\begin{multline}\label{app.eqn.EM2}
	%	-\Laghodge_{\tetr{a}{\mu}}\Tors{a}{\mu\nu} = 
	%	\Tors{a}{\mu\nu}\pd{\lambda}(\LCsymb^{\mu\gamma\rho\lambda}\Laghodge_{\HDT{a\gamma\rho}}) =
	%	
	%-\frac12\LCsymb_{\mu\nu\alpha\beta}\HDT{a\alpha\beta}\pd{\lambda}(\LCsymb^{\mu\gamma\rho\lambda}\Laghodge_{\HDT{a\gamma\rho}})
	%	 =
	%	 \\
	%	2\HDT{a\alpha\beta}\pd{\beta}\Laghodge_{\HDT{a\nu\alpha}} + 
	%	\HDT{a\alpha\beta}\pd{\nu}\Laghodge_{\HDT{a\alpha\beta}},
	%\end{multline}
	%where we have used  $ 
	%-\LCsymb_{\mu\nu\alpha\beta}\LCsymb^{\mu\gamma\rho\lambda} = 
	%\KD{\gamma\rho\lambda}{\nu\alpha\beta} +
	%\KD{\lambda\gamma\rho}{\nu\alpha\beta} +
	%\KD{\rho\lambda\gamma}{\nu\alpha\beta} -
	%\KD{\gamma\lambda\rho}{\nu\alpha\beta} -
	%\KD{\lambda\rho\gamma}{\nu\alpha\beta} -
	%\KD{\rho\gamma\lambda}{\nu\alpha\beta}
	%$, with $ \KD{\gamma\rho\lambda}{\nu\alpha\beta} = 
	%\KD{\gamma}{\nu}\KD{\rho}{\alpha}\KD{\lambda}{\beta} $, e.g. see \cite{KleinertMultivalued}, 
	%p.45. 
	%Hence, using expression \eqref{app.eqn.EM2}, equation 
	%\eqref{app.eqn.EM1} can be written as
	%\begin{equation}
	%\pd{\mu}(\tetr{a}{\nu}\Laghodge_{\tetr{a}{\mu}} - L \delta^\mu_{\ \nu}) +
	%\Laghodge_{\HDT{b\lambda\rho}}\pd{\nu}\HDT{b\lambda\rho} +
	%2\HDT{a\alpha\beta}\pd{\beta}\Laghodge_{\HDT{a\nu\alpha}} + 
	%\HDT{a\alpha\beta}\pd{\nu}\Laghodge_{\HDT{a\alpha\beta}} = 0,
	%\end{equation} 
	%and then
	%\begin{equation}
	%\pd{\mu}\left(\tetr{a}{\nu}\Laghodge_{\tetr{a}{\mu}} + 
	%(\HDT{b\lambda\rho}\Laghodge_{\HDT{b\lambda\rho}}- 
	%L) \delta^\mu_{\ \nu} \right) -
	%2\HDT{a\alpha\mu}\pd{\mu}\Laghodge_{\HDT{a\alpha\nu}} = 0,
	%\end{equation} 
	%Finally, using the integrability condition \eqref{integr.HT}, this equation can be written in 
	%a 
	%fully conservative form
	%\begin{equation}
	%\pd{\mu}\left(\tetr{a}{\nu}\Laghodge_{\tetr{a}{\mu}} -
	%2\HDT{a\lambda\mu}\Laghodge_{\HDT{a\lambda\nu}}
	%+
	%(\HDT{b\lambda\rho}\Laghodge_{\HDT{b\lambda\rho}}- 
	%L) \delta^\mu_{\ \nu} \right) = 0.
	%\end{equation} 
	
	
	%It can be shown (see 
	%expression \eqref{app.eqn.Noether2} 
	%for $ \Laghodge_{\tetr{a}{\mu}} $ in 
	%Appendix~\ref{app.sec.NC}) that the tetrad part $ 
	%\tetr{a}{\nu}\Laghodge_{\tetr{a}{\mu}} $ expands as
	%\begin{align}
	%	\tetr{a}{\nu}\Laghodge_{\tetr{a}{\mu}} & =
	%    -\tetr{a}{\nu} \Um_{\tetr{a}{\mu}} \nonumber\\
	%   & + u_\nu u^\mu(\Bm{a}{\lambda} \Um_{\Bm{a}{\lambda}} + \Dm{a}{\lambda} 
	%   \Um_{\Dm{a}{\lambda}}) 
	%   \nonumber\\
	%%   &- u^\mu u_\lambda \Dm{a}{\lambda} \Um_{\Dm{a}{\nu}} - u_\nu u^\lambda 
	%%   \Bm{a}{\mu}\Um_{\Bm{a}{\lambda}} \nonumber\\
	%   &-\LCsymb^{\mu\alpha\beta\lambda} u_\nu u_\alpha \Um_{\Dm{a}{\beta}} \Um_{\Bm{a}{\lambda}}
	%   -\LCsymb_{\nu\alpha\beta\lambda} u^\mu u^\alpha \Dm{a}{\beta}\Bm{a}{\lambda}
	%\end{align}
	%
	%On the other hand, the torsion part $ - 2 \HDT{a\lambda\mu}L_{\HDT{a\lambda\nu}} + 
	%(\HDT{a\lambda\rho}L_{\HDT{a\lambda\rho}} - L) \delta^\mu_{\ \nu} $ reads
	%\begin{align}
	%	- 2 \HDT{a\lambda\mu}L_{\HDT{a\lambda\nu}} + 
	%	(\HDT{a\lambda\rho}L_{\HDT{a\lambda\rho}} - L) \delta^\mu_{\ \nu} & = -\Um 
	%	\KD{\mu}{\nu}\nonumber\\
	%	& + \Dm{a}{\mu}\Um_{\Dm{a}{\nu}}+ \Bm{a}{\mu}\Um_{\Bm{a}{\nu}} \nonumber\\
	%	& - u_\nu u^\mu(\Bm{a}{\lambda} \Um_{\Bm{a}{\lambda}} + \Dm{a}{\lambda} 
	%	\Um_{\Dm{a}{\lambda}}) 
	%	\nonumber\\
	%	%   &- u^\mu u_\lambda \Dm{a}{\lambda} \Um_{\Dm{a}{\nu}} - u_\nu u^\lambda 
	%	%   \Bm{a}{\mu}\Um_{\Bm{a}{\lambda}} \nonumber\\
	%	&-\LCsymb^{\mu\alpha\beta\lambda} u_\nu u_\alpha \mathcal{\Um}_{\Dm{a}{\beta}} 
	%	\Um_{\Bm{a}{\lambda}}
	%	-\LCsymb_{\nu\alpha\beta\lambda} u^\mu u^\alpha \Dm{a}{\beta}\Bm{a}{\lambda}.
	%\end{align}
	
	
	\section{Transformation of Noether's current $ \NC{a}{\mu} $}\label{app.sec.NC}
	
	Here, we express Noether's current $ \NC{a}{\mu} = \Laghodge_{\tetr{a}{\mu}} $ for the 
	gravitational part of the Lagrangian (i.e. the matter part is ignored in this section) in terms 
	of 
	the 
	potential $ \Um $ and new variables $ \Dm{a}{\mu} $ and $ \Bm{a}{\mu} $. 
	
	Thus, for the parametrization
	$ \Laghodge(\tetr{a}{\mu},\HDT{a\mu\nu}) = \LagBE(\tetr{a}{\mu},\BT{a}{\mu},\ET{a}{\mu}) $, one 
	has
	\begin{equation}\label{app.eqn.Noether1}
		\Laghodge_{\tetr{a}{\mu}} = \LagBE_{\tetr{a}{\mu}} 
		+ \LagBE_{\BT{b}{\lambda}} \frac{\partial \BT{b}{\lambda}}{\partial \tetr{a}{\mu}}
		+ \LagBE_{\ET{b}{\lambda}} \frac{\partial \ET{b}{\lambda}}{\partial \tetr{a}{\mu}}.
	\end{equation}
	Then, using the definitions of the frame 4-velocity $ u^\nu = \itetr{\nu}{\indalg{0}} $ and $ 
	u_\nu 
	= 
	-\tetr{\indalg{0}}{\nu} $ and the torsion fields
	$ \BT{b}{\lambda} = \HDT{b\lambda\nu} u_\nu = - \HDT{b\lambda\nu} \tetr{\indalg{0}}{\nu}$ and 
	$ \ET{b}{\lambda} = \Tors{b}{\lambda\nu} u^\nu = 
	-\frac12\LCsymb_{\lambda\nu\alpha\beta}\HDT{b\alpha\beta} \itetr{\nu}{\indalg{0}}$, and the 
	fact 
	that $ \partial\itetr{\lambda}{b}/\partial\tetr{a}{\mu} = -\itetr{\lambda}{a}\itetr{\mu}{b} $, 
	we 
	can rewrite \eqref{app.eqn.Noether1} as
	\begin{multline}
		\Laghodge_{\tetr{a}{\mu}} = \LagBE_{\tetr{a}{\mu}} 
		+ \LagBE_{\BT{b}{\lambda}} \frac{\partial \BT{b}{\lambda}}{\partial \tetr{a}{\mu}}
		+ \LagBE_{\ET{b}{\lambda}} \frac{\partial \ET{b}{\lambda}}{\partial \tetr{a}{\mu}} = \\
		\LagBE_{\tetr{a}{\mu}} - \LagBE_{\BT{b}{\lambda}} \KD{\indalg{0}}{a}
		(u^\lambda \BT{b}{\mu} - u^\mu \BT{b}{\lambda}+\LCsymb^{\lambda\mu\alpha\beta} u_\alpha  
		\ET{b}{\beta}) \\
		-\LagBE_{\ET{b}{\lambda}} (u_\lambda \ET{b}{\nu} - u_\nu \ET{b}{\lambda} - 
		\LCsymb_{\lambda\nu\alpha\rho}u^\alpha\BT{b}{\rho})\itetr{\nu}{a}u^\mu.
	\end{multline}
	Using the definitions \eqref{eqn.Legendre2} and \eqref{eqn.Legendre3}, the latter can be 
	rewritten 
	as
	\begin{align}
		&\Laghodge_{\tetr{a}{\mu}} =
		-\Um_{\tetr{a}{\mu}} \nonumber\\
		&- \Um_{\Bm{b}{\lambda}} \KD{\indalg{0}}{a}
		(-u^\lambda \Bm{b}{\mu} + u^\mu \Bm{b}{\lambda}+\LCsymb^{\lambda\mu\alpha\beta} u_\alpha 
		\Um_{\Dm{b}{\beta}}) \nonumber\\
		&-\Dm{b}{\lambda} \itetr{\nu}{a}u^\mu (u_\lambda \Um_{\Dm{b}{\nu}} - u_\nu 
		\Um_{\Dm{b}{\lambda}} + 
		\LCsymb_{\lambda\nu\alpha\rho}u^\alpha\Bm{b}{\rho}),
	\end{align}
	which, after some term rearrangements, reads
	\begin{align}\label{eqn.J.BD}
		&\Laghodge_{\tetr{a}{\mu}} =
		-\Um_{\tetr{a}{\mu}} 
		\nonumber\\
		% 
		&+ \KD{\indalg{0}}{a}
		\bigg( 
		u^\lambda \Bm{b}{\mu} \Um_{\Bm{b}{\lambda}} 
		- u^\mu \Bm{b}{\lambda} \Um_{\Bm{b}{\lambda}} 
		\nonumber\\
		&- u^\mu \Dm{b}{\lambda} \Um_{\Dm{b}{\lambda}}
		+ \LCsymb^{\mu\lambda\rho\sigma} u_\rho \Um_{\Bm{b}{\lambda}}
		\Um_{\Dm{b}{\sigma}} 
		\bigg) \nonumber\\
		%
		&- \itetr{\nu}{a}u^\mu
		\left(
		u_\lambda \Dm{b}{\lambda} \Um_{\Dm{b}{\nu}} 
		- \LCsymb_{\nu\lambda\rho\sigma}u^\rho\Bm{b}{\sigma}\Dm{b}{\lambda}
		\right),
	\end{align}
	and exactly is \eqref{eqn.JiA}.
		
	This formula, in particular, can be used to get the following expression for the 
	energy-momentum $ 
	\EMmat{\mu}{\nu} = \tetr{a}{\nu} \NC{a}{\mu}$:
	\begin{align}
		\EMmat{\mu}{\nu} =
		& - \tetr{a}{\nu} \Um_{\tetr{a}{\mu}} \nonumber\\
		& - u^\lambda u_\nu \Bm{a}{\mu} \Um_{\Bm{a}{\lambda}} - u^\mu u_\lambda \Dm{a}{\lambda} 
		\Um_{\Dm{a}{\nu}}				\nonumber\\
		& + u^\mu u_\nu \Bm{a}{\lambda} \Um_{\Bm{a}{\lambda}} 
		+ u^\mu u_\nu \Dm{a}{\lambda} \Um_{\Dm{a}{\lambda}}
		\nonumber \\
		& + \LCsymb_{\nu\sigma\lambda\rho} u^\mu u^\sigma \Bm{a}{\lambda} \Dm{a}{\rho} 
		\nonumber \\
		&+ \LCsymb^{\mu\sigma\lambda\rho} u_\nu u_\sigma \Um_{\Bm{a}{\lambda}} 
		\Um_{\Dm{a}{\rho}}.\label{eqn.sigma.tetr.part}
	\end{align}
	
	
	\section{Expression for the energy-momentum}\label{app.energymomentum}
	
	In this section, we derive expression \eqref{eqn.sigma.BD} for the gravitational part fof the 
	energy-momentum
	\begin{equation}\label{eqn.EM.hodge}
		\EMmat{\mu}{\nu} =
		2 \HDT{a\lambda\mu}L_{\HDT{a\lambda\nu}} - 
		(\HDT{a\lambda\rho}L_{\HDT{a\lambda\rho}} - L) \KD{\mu}{\nu}
	\end{equation}
	
	Because $ \HDT{a\mu\nu} $ is 
	antisymmetric tensor, to compute the derivative $ \Laghodge_{\HDT{a\lambda\nu}} $ one needs to 
	use 
	its parametrization via the \We\ connection, which is not symmetric, i.e. $ \HDT{a\mu\nu} = 
	\LCsymb^{\mu\nu\rho\sigma} \w{a}{\rho\sigma}$. Thus, for Lagrangians $ 
	\Lag(\tetr{a}{\mu},\w{a}{\mu\nu}) = 
	\Laghodge(\tetr{a}{\mu},\HDT{a\mu\nu})$ one can write 
	\begin{equation}
		\Lag_{\w{a}{\lambda\mu}} = \LCsymb^{\lambda\mu\rho\sigma} \Laghodge_{\HDT{a\rho\sigma}},
	\end{equation}
	or using the identity $ \LCsymb_{\lambda\mu\alpha\beta}\LCsymb^{\lambda\mu\rho\sigma} = 
	-2(\KD{\rho}{\alpha}\KD{\sigma}{\beta} - \KD{\rho}{\beta}\KD{\sigma}{\alpha}) $, 
	\begin{equation}\label{eqn.L.T}
		\LCsymb_{\alpha\beta\rho\sigma}\Lag_{\w{a}{\lambda\mu}} = -4 \Laghodge_{\HDT{a\rho\sigma}}.
	\end{equation}
	
	On the other hand, using the definitions $ \ET{a}{\mu} = (\w{a}{\mu\nu} - \w{a}{\nu\mu})u^\nu 
	$, $ 
	\BT{a}{\mu} = \LCsymb^{\mu\nu\rho\sigma}\w{a}{\rho\sigma} u_\nu $, and  the parametrization $ 
	\Lag(\tetr{a}{\mu},\w{a}{\mu\nu}) = \LagBE(\tetr{a}{\mu},\BT{a}{\mu},\ET{a}{\mu}) $, one can 
	write
	\begin{align}\label{eqn.Lambda.W}
		\Lag_{\w{b}{\lambda\gamma}} &= 
		\LagBE_{\ET{a}{\mu}} \frac{\partial \ET{a}{\mu}}{\partial \w{b}{\lambda\gamma}} 
		+
		\LagBE_{\BT{a}{\mu}} \frac{\partial \BT{a}{\mu}}{\partial \w{b}{\lambda\gamma}} 
		\nonumber\\
		&=
		u^\gamma \LagBE_{\ET{b}{\lambda}} - u^\lambda \LagBE_{\ET{b}{\gamma}} - 
		\LCsymb^{\lambda\gamma\nu\mu} u_\nu \LagBE_{\BT{a}{\mu}},
	\end{align}
	and hence, from \eqref{eqn.L.T} and \eqref{eqn.Lambda.W}, one can deduce 
	\begin{align}
		\Laghodge_{\HDT{a\lambda\nu}} &= \nonumber\\
		&-\frac{1}{2} 
		\left(
		u_\lambda \LagBE_{\BT{a}{\nu}} - u_\nu \LagBE_{\BT{a}{\lambda}} 
		-
		\LCsymb_{\lambda\nu\rho\sigma} u^\rho \LagBE_{\ET{a}{\sigma}} 
		\right).
	\end{align}
	Then, after contracting the later equation with $ \HDT{a\lambda\mu} = u^\lambda \BT{a}{\mu} - 
	u^\mu 
	\BT{a}{\lambda} + \LCsymb^{\lambda\mu\alpha\beta} u_\alpha \ET{a}{\beta} $, one obtains
	\begin{align}\label{eqn.T.L.T} 
		2 \HDT{a\lambda\mu} &\Laghodge_{\HDT{a\lambda\nu}}  = \nonumber\\
		&  \BT{a}{\mu}\LagBE_{\BT{a}{\nu}} 
		- \ET{a}{\nu}\LagBE_{\ET{a}{\mu}} \nonumber\\
		& + u^\lambda u_\nu \BT{a}{\mu} \LagBE_{\BT{a}{\lambda}} - u^\mu u_\nu \BT{a}{\lambda} 
		\LagBE_{\BT{a}{\lambda}}				\nonumber\\
		& - \LCsymb_{\nu\lambda\rho\sigma} u^\mu u^\sigma \BT{a}{\lambda} \LagBE_{\ET{a}{\rho}} 
		\nonumber\\
		& - \LCsymb^{\mu\lambda\rho\sigma} u_\nu u_\rho \ET{a}{\sigma} \LagBE_{\BT{a}{\lambda}}
		\nonumber \\
		& + (u^\mu u_\nu + \KD{\mu}{\nu}) \ET{a}{\lambda} \LagBE_{\ET{a}{\lambda}} - u^\mu 
		u_\lambda 
		\ET{a}{\nu} \LagBE_{\ET{a}{\lambda}}.
	\end{align}
	This can be used to demonstrate that the full contraction  $ 	\HDT{a\lambda\rho} 
	\Laghodge_{\HDT{a\lambda\rho}} $ results in
	\begin{equation}\label{eqn.T.L.T.full}
		\HDT{a\lambda\rho} \Laghodge_{\HDT{a\lambda\rho}}
		=
		\BT{a}{\lambda} \LagBE_{\BT{a}{\lambda}} + \ET{a}{\lambda} \LagBE_{\ET{a}{\lambda}}.
	\end{equation}
	
	Collecting together \eqref{eqn.T.L.T} and \eqref{eqn.T.L.T.full} and using the change of 
	variables and potential \eqref{eqn.Legendre1}--\eqref{eqn.Legendre3}, we arrive at
	\begin{align}\label{eqn.Sigma.tors.part}
		2 \HDT{a\lambda\mu}&L_{\HDT{a\lambda\nu}} - 
		(\HDT{a\lambda\rho}L_{\HDT{a\lambda\rho}} - L) \KD{\mu}{\nu} = \nonumber\\
		& - \Bm{a}{\mu}\Um_{\Bm{a}{\nu}} - \Dm{a}{\nu}\Um_{\Dm{a}{\mu}} \nonumber\\
		& - u^\lambda u_\nu \Bm{a}{\mu} \Um_{\Bm{a}{\lambda}} - u^\mu u_\lambda \Dm{a}{\lambda} 
		\Um_{\Dm{a}{\nu}}				\nonumber\\
		& + u^\mu u_\nu \Bm{a}{\lambda} \Um_{\Bm{a}{\lambda}} 
		+ u^\mu u_\nu \Dm{a}{\lambda} \Um_{\Dm{a}{\lambda}}
		\nonumber \\
		& + \LCsymb_{\nu\sigma\lambda\rho} u^\mu u^\sigma \Bm{a}{\lambda} \Dm{a}{\rho} 
		\nonumber\\
		& + \LCsymb^{\mu\sigma\lambda\rho} u_\nu u_\sigma \Um_{\Bm{a}{\lambda}} 
		\Um_{\Dm{a}{\rho}} 
		\nonumber \\
		& + (\Bm{a}{\lambda} \Um_{\Bm{a}{\lambda}} + \Dm{a}{\lambda} \Um_{\Dm{a}{\lambda}} - 
		\Um) \KD{\mu}{\nu}.
	\end{align}
	%where $ \projector{\mu}{\nu} = u^\mu u_\nu + \KD{\mu}{\nu} $.
	
	%Finally, combining this result with the tetrad part given by \eqref{eqn.J.BD}, one arrives at 
	%(many 
	%terms in \eqref{eqn.J.BD} and \eqref{eqn.Sigma.tors.part} are canceled out )
	%\begin{multline}
	%	\tetr{a}{\nu} \Laghodge_{\tetr{a}{\mu}}
	%		- 2 \HDT{a\lambda\mu}L_{\HDT{a\lambda\nu}} + 
	%	(\HDT{a\lambda\rho}L_{\HDT{a\lambda\rho}} - L) \KD{\mu}{\nu} = \\
	%	- \tetr{a}{\nu} \Um_{\tetr{a}{\mu}}
	%	+ \Dm{a}{\mu}\Um_{\Dm{a}{\nu}} + \Bm{a}{\mu}\Um_{\Bm{a}{\nu}}
	%	- (
	%	\Dm{a}{\lambda}\Um_{\Dm{a}{\lambda}}+ \Bm{a}{\lambda}\Um_{\Bm{a}{\lambda}}
	%	-\Um
	%	) \KD{\mu}{\nu}.	
	%\end{multline}
	
	
	
	
	
	
	\section{Transformation of the torsion PDE}\label{app.sec.Deqn}
	
	
	In this appendix, we demonstrate how the Euler-Lagrange equation \eqref{eqn.1st.order.EL} 
	\begin{equation}\label{eqn.EL.tors}
		\D{\nu}(\LCsymb^{\mu\nu\lambda\rho}\Laghodge_{\HDT{a\lambda\rho}}) 
		=-\Laghodge_{\tetr{a}{\mu}}
	\end{equation}
	can be 
	transformed to the form \eqref{eqn.tors.BE.a}.
	
	Based on the different parametrization of the Lagrangian (we omit for the moment dependence of 
	the 
	Lagrangian on the tetrad field) $ \Lag(\w{a}{\mu}) = 
	\Laghodge(\HDT{a\mu\nu}) = \Lagtors(\Tors{a}{\mu\nu}) $, one can obtain
	\begin{equation}\label{eqn.Lambda_W.Hodge}
		\Lag_{\w{a}{\lambda\mu}} = \LCsymb^{\lambda\mu\rho\sigma} \Laghodge_{\HDT{a\rho\sigma}},
		\quad
		\Lag_{\w{a}{\lambda\mu}} = \Lagtors_{\Tors{a}{\lambda\mu}} - 
		\Lagtors_{\Tors{a}{\mu\lambda}},
	\end{equation}
	from which it follows that the objects
	\begin{equation}
		\HTConj{a\mu\nu} := \Laghodge_{\HDT{a\mu\nu}}, 
		\qquad
		\TorsConj{a}{\mu\nu} := \frac12 \left( \Lagtors_{\Tors{a}{\mu\nu}} - 
		\Lagtors_{\Tors{a}{\nu\mu}} 
		\right)
	\end{equation}
	are Hodge duals of each others:
	\begin{subequations}
		\begin{align}
			\HTConj{a\mu\nu} &:=-\frac12\LCsymb_{\mu\nu\rho\sigma}\TorsConj{a}{\rho\sigma}, 
			\\
			\TorsConj{a}{\mu\nu} &:= \phantom{+}\frac12 
			\LCsymb^{\mu\nu\rho\sigma}\HTConj{a\rho\sigma}.
		\end{align}
	\end{subequations}
	Therefore, one can write the following identities
	\begin{subequations}
		\begin{align}
			\HTConj{a\mu\nu} &= u_\mu \Hbb{a}{\nu} - u_\nu \Hbb{a}{\mu} - 
			\LCsymb_{\mu\nu\rho\sigma}u^\rho \Dbb{a}{\sigma}\label{app.eqn.Deqn1},\\[2mm]
			\TorsConj{a}{\mu\nu} &= u^\mu \Dbb{a}{\nu} - u^\nu \Dbb{a}{\mu} +
			\LCsymb^{\mu\nu\rho\sigma}u_\rho \Hbb{a}{\sigma},\label{eqn.TorsConj}
		\end{align}
	\end{subequations}
	where 
	\begin{equation}\label{app.eqn.DH}
		\Hbb{a}{\mu} := \HTConj{a\mu\nu}u^\nu, 
		\qquad
		\Dbb{a}{\mu} := \TorsConj{a}{\mu\nu}u_\nu,
	\end{equation}
	
	Hence, the Euler-Lagrange equation \eqref{eqn.EL.tors} now reads
	\begin{equation}%\label{key}	
		\pd{\nu} \TorsConj{a}{\mu\nu} =-\frac12 \Laghodge_{\tetr{a}{\mu}},
	\end{equation}
	or, according to \eqref{eqn.TorsConj}, it can be written as
	\begin{equation}\label{app.eqn.Deqn2}
		\D{\nu}(u^\mu \Dbb{a}{\nu} - u^\nu \Dbb{a}{\mu} + \LCsymb^{\mu\nu\alpha\beta}
		u_\alpha\Hbb{a}{\beta}) = -\frac12 \Laghodge_{\tetr{a}{\mu}}.
	\end{equation}
	
	It remains to express $ \Dbb{a}{\mu} $ and $ \Hbb{a}{\mu} $ in terms of $ 
	\LagBE(\BT{a}{\mu},\ET{a}{\mu}) $. Thus, using the fact that $ \Lag_{\w{a}{\mu\nu}} = 2 
	\TorsConj{a}{\mu\nu} $ and $ \Lag_{\w{a}{\mu\nu}} = \LCsymb_{\mu\nu\rho\sigma} 
	\HTConj{a\rho\sigma} 
	$,  and the expression \eqref{eqn.Lambda.W}, one can derive that
	\begin{subequations}
		\begin{align}%\label{key}
			\Hbb{a}{\mu} &=-\frac12 \left( \LagBE_{\BT{a}{\mu}} + u^\lambda 
			\LagBE_{\BT{a}{\lambda}} 
			u_\mu 
			\right),
			\\
			\Dbb{a}{\mu} &=-\frac12 \left( \LagBE_{\ET{a}{\mu}} + u_\lambda 
			\LagBE_{\ET{a}{\lambda}} 
			u^\mu 
			\right).
		\end{align}
	\end{subequations}
	
	Finally, plugging these expressions in 
	\eqref{app.eqn.Deqn2}, one obtains the desired result 
	\begin{equation}
		\D{\nu}( u^\mu\LagBE_{\ET{a}{\nu}} - u^\nu \LagBE_{\ET{a}{\mu}} + 
		\LCsymb^{\mu\nu\rho\sigma}u_\rho\LagBE_{\BT{a}{\sigma}}) 
		= \Laghodge_{\tetr{a}{\mu}}.
	\end{equation}
	
	
	
	\section{Some expressions of the torsion scalar}
	
	Denoting the scalars in the right-hand side of \eqref{eqn.tors.scal0} as ($ \Tscal = \Tscal_1 + 
	\Tscal_2 + \Tscal_3 $)
	\begin{subequations}
		\begin{align}\label{eqn.tors.scal}
			\Tscal_1 = & \frac14 \mg{ab}g^{\beta\lambda }g^{\mu\gamma 
			}\Tors{a}{\lambda\gamma} \Tors{b}{\beta\mu },\\[2mm]
			%
			\Tscal_2 = & \frac12 g^{\mu\gamma} \itetr{\lambda}{a} \itetr{\beta}{b} 
			\Tors{a}{\mu\beta} 
			\Tors{b}{\gamma\lambda},\\[2mm]
			%
			\quad
			\Tscal_3 = & -g^{\mu\lambda} \itetr{\rho}{a} \itetr{\gamma}{b} \Tors{a}{\mu\rho} 
			\Tors{b}{\lambda\gamma},	  
		\end{align}
	\end{subequations}
	we can also write 
	\begin{subequations}
		\begin{align}\label{eqn.tors.scal.TQLC}
			\mathcal{T}_1 &= Q_3 + \frac12 C_4, \\
			\mathcal{T}_2 &=- Q_1 + Q_4 + L_1 -C_1 + C_2 + \frac12 C_4, \\[2mm]
			\mathcal{T}_3 &= 2 Q_1 + Q_2 + L_2 + 2 C_1 + C_3 - C_4,
		\end{align}
	\end{subequations}
	where the scalars $ Q $, $ L $, and $ C $ are scalars made of $ \ET{a}{\mu} $ and 
	$ \BT{a}{\mu} $ as follows.
	
	\begin{subequations}
	Quadratic in $ \ET{a}{\mu} $:
	\begin{align}\label{eqn.tors.scal2.Q}
		Q_1 &= -\frac12 \ET{\indalg{0}}{\alpha} g^{\alpha\beta} \ET{\indalg{0}}{\beta},\\[2mm]
		%
		Q_2 &= \itetr{\alpha}{a} \ET{a}{\alpha} \itetr{\beta}{b} \ET{b}{\beta} = 
		\ET{\alpha}{\alpha} \ET{\beta}{\beta},\\[2mm]
		%
		Q_3 &=-\frac12 \mg{ab} \ET{a}{\alpha} g^{\alpha\mu}
		\ET{b}{\mu},\\[2mm]
		%
		Q_4 &=-\frac12 \itetr{\lambda}{a} \ET{a}{\beta} \itetr{\beta}{b} \ET{b}{\lambda} = 
		\ET{\lambda}{\beta} \ET{\beta}{\lambda}.
	\end{align}
	
	Mixed scalars (linear in $ \ET{a}{\mu} $):
	\begin{align}\label{eqb.ts.scal.L}
		L_1 &= \LCsymb_{\lambda\tau\varphi\gamma} u^\varphi g^{\tau\beta} \itetr{\lambda}{a} 
		\ET{a}{\beta} \BT{\indalg{0}}{\gamma},\\[2mm]
		%
		L_2 &= 2 \LCsymb_{\lambda\tau\varphi\gamma} u^\varphi g^{\lambda\beta} \itetr{\tau}{a} 
		\ET{\indalg{0}}{\beta} \BT{a}{\gamma}.
	\end{align}
	
	Quadratic in $ \BT{a}{\mu} $ (constant in  $ \ET{a}{\mu} $)
	\begin{align}\label{eqn.tors.scal.C}
		C_1 &=-\frac12 h^{-2} g_{\lambda\rho} \BT{\indalg{0}}{\rho} \BT{\indalg{0}}{\lambda},\\[2mm]
		%
		C_2 &=-\frac12 h^{-2} g_{\rho\varphi} \itetr{\varphi}{a} \BT{a}{\rho} \itetr{\beta}{b} 
		\BT{b}{\lambda}g_{\lambda\beta} 
%		= -\frac12 h^{-2} g_{\rho\varphi} \BT{\varphi}{\rho} 
%		\BT{\beta}{\lambda}g_{\lambda\beta},
		,\\[2mm]
		%
		C_3 &= h^{-2} \itetr{\varphi}{a} \BT{a}{\sigma} g_{\varphi\beta} \itetr{\lambda}{b} 
		\BT{b}{\beta} g_{\lambda\sigma} 
%		= h^{-2} \BT{\varphi}{\sigma} g_{\varphi\beta} 
%		\BT{\lambda}{\beta} g_{\lambda\sigma},
		,\\[2mm]
		%
		C_4 &= h^{-2} \itetr{\varphi}{a} \BT{a}{\sigma} g_{\varphi\lambda} \itetr{\lambda}{b} 
		\BT{b}{\beta} g_{\beta\sigma} .
%		=h^{-2} \BT{\varphi}{\sigma} g_{\varphi\lambda} 
%		\BT{\lambda}{\beta} g_{\beta\sigma}.
	\end{align}
	\end{subequations}
	
	In terms of the Hodge dual $ \HDT{a\mu\nu} $, the torsion scalar $ \Tscal $ can be rewritten as
	\begin{subequations}
	\begin{equation}\label{eqn.tors.scal.hodge}
		\Tscal = \Hscal_1 + \Hscal_2 + \Hscal_3 + \Hscal_4,
	\end{equation}
	where 
		\begin{align}\label{eqn.hodge.scal}
			\Hscal_1 &= \frac{1}{2 h^2} \mg{ac}\mg{bd}\tetr{c}{\lambda}\tetr{d}{\sigma} 
			g_{\tau\rho} 
			\HDT{a\rho\sigma}\HDT{b\lambda\tau},
			\\[2mm]
			\Hscal_2 &= \frac{1}{2 h^2} \mg{ac}\mg{bd}\tetr{c}{\tau}\tetr{d}{\rho} 
			g_{\lambda\sigma} 
			\HDT{a\rho\sigma}\HDT{b\lambda\tau},
			\\[2mm]
			%
			\Hscal_3 &=-\frac{1}{4 h^2} \mg{ac}\mg{bd}\tetr{c}{\sigma}\tetr{d}{\lambda} 
			g_{\tau\rho} 
			\HDT{a\rho\sigma}\HDT{b\lambda\tau},
			\\[2mm] 
			\Hscal_4 &=-\frac{1}{4 h^2} \mg{ac}\mg{bd}\tetr{c}{\rho}\tetr{d}{\tau} 
			g_{\lambda\sigma} 
			\HDT{a\rho\sigma}\HDT{b\lambda\tau}.
		\end{align}
	\end{subequations}
	
	Finally, in terms of the \We\ connection, the torsion scalar reads:
	\begin{subequations}
		\begin{equation}
			\Tscal = \sum_{n=1}^{8}\mathcal{W}_n
		\end{equation}
		\begin{align}
			\mathcal{W}_1 &= \frac12 
			\mg{ab}g^{\beta\lambda}g^{\mu\gamma}\w{a}{\lambda\gamma}\w{b}{\beta\mu},
			\\
			\mathcal{W}_2 &=-\frac12 
			\mg{ab}g^{\beta\lambda}g^{\mu\gamma}\w{a}{\gamma\lambda}\w{b}{\beta\mu},
		\end{align}
		\begin{align}
			\mathcal{W}_3 &= \frac12 
			g^{\mu\gamma}\itetr{\beta}{a}\w{a}{\lambda\gamma}\itetr{\lambda}{b}\w{b}{\beta\mu},
			\\
			\mathcal{W}_4 &=\frac12 
			g^{\mu\gamma}\itetr{\beta}{a}\w{a}{\gamma\lambda}\itetr{\lambda}{b}\w{b}{\mu\beta},
		\end{align}
		\begin{equation}
			\mathcal{W}_5 =  
			-g^{\mu\gamma}\itetr{\beta}{a}\w{a}{\gamma\lambda}\itetr{\lambda}{b}\w{b}{\beta\mu},
		\end{equation}
		\begin{align}
			\mathcal{W}_6 &= -g^{\mu\lambda} w^{1}_{\lambda} w^{1}_{\mu},
			\\
			\mathcal{W}_7 &= -g^{\mu\lambda} w^{2}_{\lambda} w^{2}_{\mu},
			\\
			\mathcal{W}_8 &= 2 g^{\mu\lambda} w^{1}_{\mu} w^{2}_{\lambda},
		\end{align}
		where $ w^{1}_\mu = \itetr{\lambda}{a} \w{a}{\mu\lambda}$, $ w^{2}_\mu = \itetr{\lambda}{a} 
		\w{a}{\lambda\mu} $.
	\end{subequations}
	
	
	\printbibliography
	
\end{document}