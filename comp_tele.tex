%%%%%%%%%%%%%%%%%%%%%%%%%%%%%%%%%%%%%%%%%
% Arsclassica Article
% LaTeX Template
% Version 1.1 (1/8/17)
%
% This template has been downloaded from:
% http://www.LaTeXTemplates.com
%
% Original author:
% Lorenzo Pantieri (http://www.lorenzopantieri.net) with extensive modifications by:
% Vel (vel@latextemplates.com)
%
% License:
% CC BY-NC-SA 3.0 (http://creativecommons.org/licenses/by-nc-sa/3.0/)
%
%%%%%%%%%%%%%%%%%%%%%%%%%%%%%%%%%%%%%%%%%

%----------------------------------------------------------------------------------------
%	PACKAGES AND OTHER DOCUMENT CONFIGURATIONS
%----------------------------------------------------------------------------------------

\documentclass[
10pt, % Main document font size
a4paper, % Paper type, use 'letterpaper' for US Letter paper
oneside, % One page layout (no page indentation)
%twoside, % Two page layout (page indentation for binding and different headers)
headinclude,footinclude, % Extra spacing for the header and footer
BCOR5mm, % Binding correction
]{scrartcl}

%%%%%%%%%%%%%%%%%%%%%%%%%%%%%%%%%%%%%%%%%
% Arsclassica Article
% Structure Specification File
%
% This file has been downloaded from:
% http://www.LaTeXTemplates.com
%
% Original author:
% Lorenzo Pantieri (http://www.lorenzopantieri.net) with extensive modifications by:
% Vel (vel@latextemplates.com)
%
% License:
% CC BY-NC-SA 3.0 (http://creativecommons.org/licenses/by-nc-sa/3.0/)
%
%%%%%%%%%%%%%%%%%%%%%%%%%%%%%%%%%%%%%%%%%

%----------------------------------------------------------------------------------------
%	REQUIRED PACKAGES
%----------------------------------------------------------------------------------------

\usepackage[
nochapters, % Turn off chapters since this is an article        
beramono, % Use the Bera Mono font for monospaced text (\texttt)
eulermath,% Use the Euler font for mathematics
pdfspacing, % Makes use of pdftex’ letter spacing capabilities via the microtype package
dottedtoc % Dotted lines leading to the page numbers in the table of contents
]{classicthesis} % The layout is based on the Classic Thesis style
\usepackage{hyperref}

\usepackage{arsclassica} % Modifies the Classic Thesis package

\usepackage[T1]{fontenc} % Use 8-bit encoding that has 256 glyphs

\usepackage[utf8]{inputenc} % Required for including letters with accents

\usepackage{graphicx} % Required for including images
\graphicspath{{Figures/}} % Set the default folder for images

\usepackage{enumitem} % Required for manipulating the whitespace between and within lists

\usepackage{lipsum} % Used for inserting dummy 'Lorem ipsum' text into the template

\usepackage{subfig} % Required for creating figures with multiple parts (subfigures)

\usepackage{amsmath,amssymb,amsthm} % For including math equations, theorems, symbols, etc

\usepackage{varioref} % More descriptive referencing

\usepackage{accents}


%----------------------------------------------------------------------------------------
%	THEOREM STYLES
%---------------------------------------------------------------------------------------

\theoremstyle{definition} % Define theorem styles here based on the definition style (used for definitions and examples)
\newtheorem{definition}{Definition}

\theoremstyle{plain} % Define theorem styles here based on the plain style (used for theorems, lemmas, propositions)
\newtheorem{theorem}{Theorem}

\theoremstyle{remark} % Define theorem styles here based on the remark style (used for remarks and notes)

%----------------------------------------------------------------------------------------
%	HYPERLINKS
%---------------------------------------------------------------------------------------

\hypersetup{
%draft, % Uncomment to remove all links (useful for printing in black and white)
colorlinks=true, breaklinks=true, bookmarks=true,bookmarksnumbered,
urlcolor=webbrown, linkcolor=RoyalBlue, citecolor=webgreen, % Link colors
pdftitle={}, % PDF title
pdfauthor={\textcopyright}, % PDF Author
pdfsubject={}, % PDF Subject
pdfkeywords={}, % PDF Keywords
pdfcreator={pdfLaTeX}, % PDF Creator
pdfproducer={LaTeX with hyperref and ClassicThesis} % PDF producer
}


%----------------------------------------------------------------------------------------
%	BIBLATEX
%---------------------------------------------------------------------------------------

\usepackage[backend=bibtex,giveninits=true,url=false,doi=true,eprint=true,isbn=false,
backref,backrefstyle=none,maxbibnames=99]{biblatex}
\DefineBibliographyStrings{english}{%
  backrefpage = {Cited on p\adddot},%
  backrefpages = {Cited on pp\adddot}%
}

\bibliography{library}

\renewcommand*{\bibfont}{\footnotesize}

% in order to suppress 'In:'
\renewbibmacro{in:}{%
  \ifboolexpr{%
     test {\ifentrytype{article}}%
  }{}{\printtext{\bibstring{in}\intitlepunct}}%
}

%----------------------------------------------------------------------------------------
% these commands allow to put equations in a fancy boxes:
%----------------------------------------------------------------------------------------
\usepackage{empheq}
\newlength\mytemplen
\newsavebox\mytempbox
\makeatletter
\definecolor{cream}{rgb}{.81, .88, 1}
 \newcommand\mycreambox{%
     \@ifnextchar[%]
        {\@mycreambox}%
        {\@mycreambox[0pt]}}
 \def\@mycreambox[#1]{%
     \@ifnextchar[%]
        {\@@mycreambox[#1]}%
        {\@@mycreambox[#1][0pt]}}
 \def\@@mycreambox[#1][#2]#3{
     \sbox\mytempbox{#3}%
     \mytemplen\ht\mytempbox
     \advance\mytemplen #1\relax
     \ht\mytempbox\mytemplen
     \mytemplen\dp\mytempbox
     \advance\mytemplen #2\relax
     \dp\mytempbox\mytemplen
     \colorbox{cream}{\hspace{1em}\usebox{\mytempbox}\hspace{1em}}}
 \makeatother % Include the structure.tex file which specified the document structure and 
%layout

\usepackage[hmarginratio=1:1,top=25mm,left=40mm,columnsep=25pt]{geometry}
\usepackage{hyperref}
\usepackage{relsize} % e.g. used for \mathsmaller
\usepackage{bm}


\newcommand{\xx}{\mathbf{x}}
\newcommand{\XX}{\mathbf{X}}
\newcommand{\XXX}{\mathbb{X}}
\newcommand{\diff}{\mathrm{d}}
\newcommand{\Id}{\mathbf{I}}
\newcommand{\Tr}{\mathrm{Tr}}
\newcommand{\CC}{\mathbf{C}}
\newcommand{\dX}{\mathrm{d}\XX}
\newcommand{\dx}{\mathrm{d}\xx}
\newcommand{\mm}{\mathbf{m}}
\newcommand{\vv}{\mathbf{v}}
\newcommand{\MM}{\mathbf{M}}
\newcommand{\OBig}{\mathcal{O}}
\newcommand{\FF}{\mathbf{F}}
\renewcommand{\AA}{\mathbf{A}}
\newcommand{\BB}{\mathbf{B}}
\newcommand{\qq}{\mathbf{q}}
\newcommand{\QQ}{\mathbf{Q}}
\newcommand{\LL}{\mathbf{L}}
\newcommand{\Lie}{\mathfrak{L}}

\newcommand{\MP}[1]{{\color{Green}MP:\ \ #1}}
\newcommand{\IP}[1]{{\color{Red}IP:\ \ #1}}
\newcommand{\MH}[1]{{\color{Red}MH:\ \ #1}}

\newcommand{\sA}{\mathsmaller A}
\newcommand{\sB}{\mathsmaller B}
\newcommand{\sC}{\mathsmaller C}
\newcommand{\sD}{\mathsmaller D}
\newcommand{\sM}{\mathsmaller M}
\newcommand{\sN}{\mathsmaller N}
\newcommand{\sL}{\mathsmaller L}

\newcommand{\pd}{\partial}
\newcommand{\F}[2]{F^{\ #1}_{\mathsmaller#2}}
\newcommand{\hatF}[2]{\hat{F}^{\ #1}_{\mathsmaller#2}}
\newcommand{\A}[2]{A^{\mathsmaller#1}_{\ #2}}

\newcommand{\itetr}[2]{e^{\phantom{#2}#1}_{#2}}
\newcommand{\tetr}[2]{a^{#1}_{\phantom{#1}#2}}
\newcommand{\rtetr}[2]{a^{#1}_{(\text{r}) #2}}
\newcommand{\spin}[2]{\omega^{#1}_{\phantom{#1}#2}}
\newcommand{\Lor}[2]{\Lambda^{#1'}_{\phantom{#1}#2}}
\newcommand{\iLor}[2]{\Lambda^{\phantom{#2}#1}_{#2'}}
\newcommand{\D}[1]{\mathcal{D}_{#1}} % Fock-Ivanencko cov derivative
\newcommand{\Tors}[2]{T^{#1}_{\phantom{a}#2}}
\newcommand{\Supp}[2]{S_{#1}^{\phantom{a}#2}}	%supepotential
\newcommand{\Torsl}[1]{T_{#1}}
\newcommand{\ET}[2]{E^{#1}_{\phantom{#1}#2}}	%Torsion decomposition, analog of Electric field
\newcommand{\eT}[2]{D_{#1}^{\phantom{#1}#2}}	%Torsion decomposition, analog of Electric field
\newcommand{\BT}[2]{B^{#1#2}}	%Torsion decomposition, analog of Magnetic field
\newcommand{\hT}[2]{H^{#1#2}}	%Torsion decomposition, analog of Magnetic field
\newcommand{\W}[2]{\mathcal{W}^{#1}_{\phantom{#1}#2}}
\newcommand{\w}[2]{W^{#1}_{\phantom{#1}#2}}
\newcommand{\FI}{Fock-Ivanenko}
\newcommand{\We}{Weitzenb\"ock}
\newcommand{\Lag}{\mathcal{L}}	% Lagrangian which depends on ordinary derivatives
\newcommand{\Lagcov}{\pounds}% Lagrangian which depends on gauge covariant derivatives
\newcommand{\Laghodge}{L}% Lagrangian which depends on the Hodge dual of the torsion
\newcommand{\Lagtors}{\mathbb{L}}% Lagrangian which depends on torsion
\newcommand{\LagBE}{\mathfrak{L}}% Lagrangian which depends on the B and E fields
\newcommand{\veps}{\varepsilon}
\newcommand{\EM}[2]{\Sigma^{#1}_{\phantom{#1}#2}}

\newcommand{\tegr}{TEGR}
\newcommand{\HT}[1]{\accentset{\star}{T}^{#1}}

\hyphenation{Fortran hy-phen-ation} % Specify custom hyphenation points in words with dashes where 
%you would like hyphenation to occur, or alternatively, don't put any dashes in a word to stop 
%hyphenation altogether

%----------------------------------------------------------------------------------------
%	TITLE AND AUTHOR(S)
%----------------------------------------------------------------------------------------

\title{\large\normalfont\spacedallcaps{Computational aspects of teleparallel gravity}} % The article 
%title

%\subtitle{Subtitle} % Uncomment to display a subtitle

\author{\normalsize\textsc{Ilya Peshkov}$^1$ \& 
\normalsize\textsc{Evgeniy Romenski}$^{2,3}$ \&
\normalsize\textsc{Michael Dumbser}$^{4}$
} % The article author(s) - author affiliations 
%need to be 
%specified in the 
%AUTHOR AFFILIATIONS block

\date{\small\today} % An optional date to appear under the author(s)

%----------------------------------------------------------------------------------------


\begin{document}

%----------------------------------------------------------------------------------------
%	HEADERS
%----------------------------------------------------------------------------------------

\renewcommand{\sectionmark}[1]{\markright{\spacedlowsmallcaps{#1}}} % The header for all pages 
%(oneside) or for even pages (twoside)
%\renewcommand{\subsectionmark}[1]{\markright{\thesubsection~#1}} % Uncomment when using the 
%%twoside option - this modifies the header on odd pages
\lehead{\mbox{\llap{\small\thepage\kern1em\color{halfgray} 
\vline}\color{halfgray}\hspace{0.5em}\rightmark\hfil}} % The header style

\pagestyle{scrheadings} % Enable the headers specified in this block

%----------------------------------------------------------------------------------------
%	TABLE OF CONTENTS & LISTS OF FIGURES AND TABLES
%----------------------------------------------------------------------------------------

\maketitle % Print the title/author/date block

\setcounter{tocdepth}{2} % Set the depth of the table of contents to show sections and subsections 
%only

\tableofcontents % Print the table of contents

% \listoffigures % Print the list of figures

% \listoftables % Print the list of tables

%----------------------------------------------------------------------------------------
%	ABSTRACT
%----------------------------------------------------------------------------------------

\section*{Abstract} % This section will not appear in the table of contents due to the star 
% (\section*)
We discuss well-posedness properties of a first-order reformulation of the teleparallel gravity
theory.

%----------------------------------------------------------------------------------------
%	AUTHOR AFFILIATIONS
%----------------------------------------------------------------------------------------
\let\thefootnote\relax\footnotetext{* \textit{peshenator@gmail.com}}
\let\thefootnote\relax\footnotetext{\textsuperscript{1} \textit{Institut de mathématiques de Toulouse, Toulouse, France}}
\let\thefootnote\relax\footnotetext{\textsuperscript{2} \textit{Sobolev Institute of Mathematics, Novosibirsk, Russia}}
\let\thefootnote\relax\footnotetext{\textsuperscript{3} \textit{Novosibirsk State University, Novosibirsk, Russia}}
\let\thefootnote\relax\footnotetext{\textsuperscript{4} \textit{University of Trento, Trento, Italy}}
%----------------------------------------------------------------------------------------

%\newpage % Start the article content on the second page, remove this if you have a longer abstract 
%that goes onto the second page

% PARAGRAPH OPTIONS:
\setlength\parindent{10pt} % sets indent to zero
\setlength{\parskip}{5pt} % changes vertical space between paragraphs
% PARAGRAPH OPTIONS.

%----------------------------------------------------------------------------------------
%	INTRODUCTION
%----------------------------------------------------------------------------------------

\section{Introduction}

Why to study teleparallel gravity? Quickly recall the main arguments from Chapter\,18 of 
\cite{AldrovandiPereiraBook}, and then add that the same equations are, in fact, applicable to 
modeling of 
turbulence, micromorphic solids (acoustic metamaterials), dislocations \cite{PRD-Torsion2018}... 

\section{Definitions}

\subsection{Anholonomic tetrad field}

We use the following index convention. Greek letters $ \lambda,\mu,\nu,... =0,1,2,3
$ are used to index quantities related to the spacetime manifold, the Latin letters $ a,b,c,... 
=0,1,2,3$ are used to index quantities related to the tangent Minkowski space.



Consider a spacetime manifold $ M $ equipped with a coordinate system $ x^\mu $. At each point of 
the spacetime there is a natural tangent $ T_{x}M $ space spanned by the basis $ \bm{\pd}_\mu $. 
There is also the cotangent space $ T_x^*M $ spanned by the cobasis vectors $ \bm{\dx}^\mu $.
The metric on $ M $ is a general Riemannian metric $ g_{\mu\nu} $.

In addition to $ T_{x}M $, we assume that at each point of $ M $, there is a soldered tangent space 
which is a Minkowski space spanned by the orthogonal anholonomic 
frames $ \bm{e}^a $ and equipped with the Minkowski metric $ g_{ab} $. Similarly, there is the 
corresponding cotangent Minkowski space spanned by the co-frames $ \bm{e}_a $. 

\subsection{Spin connection}

We introduce the field of anholonomic basis tetrad $ \itetr{\mu}{a} $ with the inverse $ 
\tetr{a}{\mu} $ as
\begin{equation}
\itetr{\mu}{a} \bm{e}^a = \bm{\dx}^\mu, \qquad \text{or} \qquad \bm{e}_a = \tetr{a}{\mu}\bm{\dx}^\mu
\end{equation}


\begin{equation}
g_{\mu\nu} = g_{ab} \tetr{a}{\mu}\tetr{b}{\nu}
\end{equation}


The inertial spin connection is introduced as
\begin{equation}
\spin{a}{bc} := \iLor{a}{b}\pd_b\Lor{b}{c}
\end{equation}
where $ \Lor{a}{a} $ is the Lorentz transformation $ \bm{e}^{a'} = \Lor{a}{a} \bm{e}^a $ with $ 
\iLor{a}{a} $ being its inverse, and $ \pd_a := \itetr{\mu}{a} \pd_\mu $.

\subsection{Gauge covariant derivative}

The corresponding gauge covariant derivative is 
\begin{equation}
\D{a} v^b = \pd_a v^b + \spin{b}{ac} v^c, \qquad \D{a} v_b = \pd_a v_b - \spin{c}{ab} v_c
\end{equation}

We rewrite this covariant derivative in the spacetime basis
\begin{equation}
\D{\mu} v^b = \tetr{a}{\mu} \D{a} v^b = \pd_\mu v^b + \spin{b}{\mu c} v^c
\end{equation}
where $ \spin{b}{\mu c} = \tetr{a}{\mu}\spin{b}{ac} $.
This derivative still acts on the Minkowski indexes only. Sometimes, it is also called the \FI\ 
covariant derivative~\cite{AldrovandiPereiraBook}.

In particular remark that $ \D{\mu} $ acts on the spacetime tensors as ordinary derivative $ 
\pd_\mu $, e.g.
\begin{multline}\label{eqn.covD.pd}
\D{\mu}v^{\nu} = \D{\mu}(\itetr{\nu}{a}v^a) = \itetr{\nu}{a}\D{\mu}v^a + v^a\D{\mu}\itetr{\nu}{a} = 
\\
\itetr{\nu}{a}\pd_{\mu}v^a + \itetr{\nu}{a}\spin{a}{\mu c}v^c + 
v^a\pd_{\mu}\itetr{\nu}{a} - v^a\spin{c}{\mu c}\itetr{\nu}{c} = \pd_\mu(\itetr{\nu}{s}v^a) = 
\pd_\mu v^\nu.
\end{multline}



\subsection{Torsion}

The torsion is introduced as
\begin{equation}
\Tors{a}{\mu\nu}(\tetr{a}{\mu},\spin{a}{\mu c}):=\D{\mu}\tetr{a}{\nu} - \D{\nu}\tetr{a}{\mu} = 
\W{a}{\mu\nu} - \W{a}{\nu\mu},
\end{equation}
where $ \W{a}{\mu\nu} = \tetr{a}{\lambda}\W{\lambda}{\mu\nu}$ and $ \W{\lambda}{\mu\nu} = 
\w{\lambda}{\mu\nu} + \spin{\lambda}{\mu\nu}$ is the 
generalized \We\ connection \cite{AldrovandiPereiraBook}, $ \w{\lambda}{\mu\nu} = 
\itetr{\lambda}{a}\pd_\mu \tetr{a}{\nu}$ is 
the pure tetrad \We\ connection.


An important feature of this derivative is the commutative property
\begin{equation}\label{eqn.commut.D}
\D{\mu}(\D{\nu} v^a) - \D{\nu}(\D{\mu} v^a) = 0.
\end{equation}
Note that its tangent space counterpart does not have such a property
\begin{equation}
\D{b}(\D{c} v^a) - \D{c}(\D{b} v^a) = 
-\itetr{\mu}{b}\itetr{\nu}{c}\itetr{\lambda}{d}\Tors{d}{\mu\nu}\D{\lambda}v^a .
\end{equation}

\subsection{Reference tetrad}

We also define the \textit{reference} tetrad $ \rtetr{a}{\mu} $ as the one for which the torsion 
vanishes
\begin{equation}
\Tors{a}{\mu\nu}(\rtetr{a}{\mu},\spin{a}{\mu c}) = 0
\end{equation}
which means that this tetrad does not contain any gravity effect but only inertia.


\subsection{Levi-Civita tensor}
We shall also need the Levi-Civita tensor
\begin{equation}\label{def.LeviCivita}
\veps^{\mu\nu\lambda\rho} = a^{-1} \epsilon^{\mu\nu\lambda\rho}, \qquad \veps_{\mu\nu\lambda\rho} = 
a 
\epsilon_{\mu\nu\lambda\rho},
\end{equation}
where $ a = \det(\tetr{a}{\mu}) $, and $ 
\epsilon^{{\mu\nu\lambda\rho}} $ is the totally antisymmetric Levi-Civita 
symbol~\cite{AldrovandiPereiraBook}.

In particular, due to \eqref{eqn.covD.pd}, we remark that 
\begin{multline}
\D{\sigma}\veps^{\mu\nu\lambda\rho} = \D{\sigma}(a^{-1}\epsilon^{\mu\nu\lambda\rho}) = 
\epsilon\D{\sigma}a^{-1} = 
\\
-\epsilon^{\mu\nu\lambda\rho}a^{-1}\itetr{\eta}{a}\D{\sigma}\tetr{a}{\eta} = 
-\epsilon^{\mu\nu\lambda\rho}a^{-1}\W{\eta}{\sigma\eta} = 
-\veps^{\mu\nu\lambda\rho}\W{\eta}{\sigma\eta} . 
\end{multline}


\section{Variational formulation}
The Lagrangian is 
\begin{equation}
\Lambda = \Lag(\tetr{a}{\mu},\pd_\lambda\tetr{a}{\nu}) = 
\Lagcov(\tetr{a}{\mu},\D{\lambda}\tetr{a}{\nu})  = \Laghodge(\tetr{a}{\mu},\HT{a\mu\nu}),
\end{equation}
where $ \HT{a\mu\nu} $ is the Hodge dual to the torsion:
\begin{equation}\label{eqn.Hodge.def}
\HT{a\mu\nu} := \frac{1}{2}\veps^{\mu\nu\rho\sigma}\Tors{a}{\rho\sigma} = 
\veps^{\mu\nu\rho\sigma}\D{\rho}\tetr{a}{\sigma}, \qquad \Tors{a}{\mu\nu} = 
-\frac{1}{2}\veps_{\mu\nu\rho\sigma}\HT{a\rho\sigma}.
\end{equation}




The Euler-Lagrange equation is
\begin{equation}
\pd_\lambda(\Lag_{\pd_\lambda\tetr{a}{\mu}}) = \Lag_{\tetr{a}{\mu}},
\end{equation}
where $ \Lag_{\pd_\lambda\tetr{a}{\mu}} = \frac{\pd \Lag}{\pd(\pd_\lambda\tetr{a}{\mu})} $ and $ 
\Lag_{\tetr{a}{\mu}} = \frac{\pd \Lag}{\pd \tetr{a}{\mu}} $.


Using the results from \cite{Lewis2009,Lorce2013}, the Euler-Lagrange equation can be rewritten as
\begin{equation}\label{eqn.cov.field.PDE}
\D{\lambda}(\Lagcov_{\D{\lambda}\tetr{a}{\mu}}) = \Lagcov_{\tetr{a}{\mu}}.
\end{equation}
Note that $  \Lag_{\tetr{a}{\mu}} \neq  \Lagcov_{\tetr{a}{\mu}}. $

Eventually, using \eqref{eqn.Hodge.def}, we can rewrite \eqref{eqn.cov.field.PDE} as
\begin{equation}\label{eqn.cov.field.PDE.Hodge}
\D{\lambda}(\veps^{\mu\nu\rho\lambda}\Laghodge_{\HT{a\nu\rho}}) =-\Laghodge_{\tetr{a}{\mu}}.
\end{equation}

The Euler-Lagrange equations has to be supplemented by the integrability condition
\begin{equation}
\D{\nu}\HT{a\mu\nu} = 0,
\end{equation}
which is a trivial consequence of the Hodge dual and the commutative property \eqref{eqn.commut.D} 
of the gauge 
covariant derivative $ \D{\mu} $.

Another consequence of the commutative property is the energy-momentum covariant conservation law
\begin{equation}
\D{\mu}\Laghodge_{\tetr{a}{\mu}} = 0,
\end{equation}
which also has to be supplemented by the integrability condition 
\begin{equation}
\D{\mu}\tetr{a}{\nu} - \D{\nu}\tetr{a}{\mu} = \Tors{a}{\mu\nu}.
\end{equation}


\section{First-order field equations}

Therefore, the first-order form of TEGR governing equations to be solved is
\begin{subequations}\label{eqn.1st.order.TEGR}
	\begin{empheq}[box={\mycreambox[2pt][2pt]}]{align}
		\D{\lambda}(\veps^{\mu\nu\rho\lambda}\Laghodge_{\HT{a\nu\rho}}) 
		&=-\Laghodge_{\tetr{a}{\mu}},\label{eqn.1st.order.EL}\\[2mm]
		\D{\lambda}\HT{a\mu\lambda} &= 0,\label{eqn.1st.order.integr}\\[2mm]
		\D{\mu}\Laghodge_{\tetr{a}{\mu}} &= 0,\label{eqn.1st.order.enermomen}\\[2mm]
		\D{\mu}\tetr{a}{\nu} - \D{\nu}\tetr{a}{\mu} &= \Tors{a}{\mu\nu},\label{eqn.1st.order.tetrad}
	\end{empheq}
\end{subequations}
in which the unknowns are
\begin{equation}
(\tetr{a}{\mu},\HT{a\mu\nu}).
\end{equation}


\section{Preparation for 3+1 split}


\subsection{Frame 4-velocity}

Let us introduce the frame 4-velocity as
\begin{equation}\label{eqn.4velocity}
u^\mu := \itetr{\mu}{0}, \qquad u_\mu = g_{\mu\nu}u^\nu,
\end{equation}
then, due to
\begin{equation}
u_\mu = g_{\mu\nu} u^\nu = \eta_{ab}\tetr{a}{\mu}\tetr{b}{\nu}\itetr{\nu}{0} = 
\eta_{a0}\tetr{a}{\mu} = -\tetr{0}{\mu},
\end{equation}\label{eqn.4velocity.cov}
the covariant components are
\begin{equation}
u_\mu = -\tetr{0}{\mu}.
\end{equation}


\subsection{Transformation of the torsion equations}\label{sec.transform.potential}


System \eqref{eqn.1st.order.TEGR} is not yet in a convenient form for the numerical treatment. It 
is necessary to perform a 3+1 split \cite{Alcubierre2008}. 


We then define 
\begin{equation}
\ET{a}{\mu} := u^\nu \Tors{a}{\mu\nu}, \qquad  \BT{a}{\mu} := u_\nu\HT{a\mu\nu}
\end{equation}

It is known that for any skew-symmetric tensor and a time-like vector $ u^\mu $ the following 
decompositions holds
\begin{subequations}\label{eqn.T.decompos}
\begin{align}
\HT{a\mu\nu} &= u^\mu \BT{a}{\nu} - u^\nu \BT{a}{\mu} + \veps^{\mu\nu\lambda\rho}u_\lambda 
\ET{a}{\rho},\\[2mm]
\Tors{a}{\mu\nu} &= u_\mu \ET{a}{\nu} - u_\nu \ET{a}{\mu} - \veps^{\mu\nu\lambda\rho}u^\lambda 
\BT{a}{\rho},
\end{align}
\end{subequations}

We assume that the Lagrangian can be written 
\begin{equation}
\Lambda = \Laghodge(\tetr{a}{\mu},\HT{a\mu\nu}) = \Lagtors(\tetr{a}{\mu},\Tors{a}{\mu\nu}) = 
\LagBE(\tetr{a}{\mu},\BT{a}{\mu},\ET{a}{\nu}).
\end{equation}
It then can be shown that 
\begin{equation}
\Laghodge_{\HT{a\mu\nu}}u^\nu = -\LagBE_{\BT{a}{\mu}}, \qquad \Lagtors_{\Tors{a}{\mu\nu}}u_\nu = 
-\LagBE_{\ET{a}{\mu}},
\end{equation}
and hence, the PDEs \eqref{eqn.1st.order.integr} and \eqref{eqn.1st.order.EL} can be written as 
\begin{subequations}\label{eqn.tors.BE}
	\begin{align}
		\D{\nu}(u^\mu \LagBE_{\ET{a}{\nu}} - u^\nu\LagBE_{\ET{a}{\mu}} {\color{Red}\bm{+}} 
		\veps^{\mu\nu\lambda\rho}u_\lambda\LagBE_{\BT{a}{\rho}}) &=\LagBE_{\tetr{a}{\mu}}\\[2mm]
		\D{\mu}(u^\mu \BT{a}{\nu} - u^\nu\BT{a}{\mu} + 
		\veps^{\mu\nu\lambda\rho}u_\lambda\ET{a}{\rho}) &= 0,
	\end{align}
\end{subequations}

Let us now introduce a new potential $ U(\tetr{a}{\mu},\hT{a}{\mu},\eT{a}{\mu}) $ as a partial 
Legendre transform of the Lagrangian
\begin{equation}
 U(\tetr{a}{\mu},\hT{a}{\mu},\eT{a}{\mu}) := \ET{a}{\lambda}\LagBE_{\ET{a}{\lambda}} - \LagBE,
\end{equation}
and new state variables
\begin{equation}
\eT{a}{\mu} = \LagBE_{\ET{a}{\mu}}, \qquad \hT{a}{\mu} = -\BT{a}{\mu}, \qquad \tetr{a}{\mu},
\end{equation}
such that we have
\begin{equation}
U_{\eT{a}{\mu}} = \ET{a}{\mu}, \qquad U_{\hT{a}{\mu}} = \LagBE_{\BT{a}{\mu}},
\qquad U_{\tetr{a}{\mu}} = - \LagBE_{\tetr{a}{\mu}}.
\end{equation}
This allows us to rewrite equations \eqref{eqn.1st.order.integr} and \eqref{eqn.1st.order.EL} in 
the form similar to the non-linear 
electrodynamics of moving medium~\cite{Obukhov2008,DPRZ2017,Hohmann2018a}
\begin{subequations}
	\begin{align}
		\D{\mu}(u^\mu\eT{a}{\nu} - u^\nu\eT{a}{\mu} + \veps^{\mu\nu\lambda\rho}u_\lambda 
		U_{\hT{a}{\rho}})
		& ={\color{Red}\bm{-}}U_{\tetr{a}{\mu}},\\[2mm]
		\D{\mu}(u^\mu \hT{a}{\nu} - u^\nu\hT{a}{\mu} - 
		\veps^{\mu\nu\lambda\rho}u_\lambda 
		U_{\eT{a}{\rho}}) 
		& = 0,
\end{align}
\end{subequations}
with $\hT{a}{\mu}$ and $\eT{a}{\mu}$ being the analog of the magnetic and electric fields, 
accordingly.




\subsection{Transformation of the tetrad equations}

Let us introduce a pure spacetime energy-momentum tensor
\begin{equation}\label{def.energymom.spacetime}
\EM{\mu}{\nu} := \tetr{a}{\nu} U_{\tetr{a}{\mu}} - U \delta^\mu_{\ \nu}.
\end{equation}
Note that $ \D{\lambda} \EM{\mu}{\nu} = \pd_\lambda\EM{\mu}{\nu} $. 

Using definition \eqref{def.energymom.spacetime}, the energy-momentum conservation law can be 
rewritten in a pure spacetime form
\begin{equation}
\D{\mu} (\tetr{a}{\nu} U_{\tetr{a}{\mu}} - U \delta^\mu_{\ \nu}) - U_{\tetr{a}{\mu}} 
\Tors{a}{\mu\nu} = 0.
\end{equation}


Now, contracting \eqref{eqn.1st.order.tetrad} with the 4-velocity $ u^\mu $, and then after the 
change of variables and potentials described in Section\,\ref{sec.transform.potential}, the 
resulting equation reads as
\begin{equation}
	u^\mu(\D{\mu}\tetr{a}{\nu} - \D{\nu}\tetr{a}{\mu}) = U_{\eT{a}{\nu}},
\end{equation}
Furthermore, using the identity $ \itetr{\mu}{b}\D{\nu}\tetr{a}{\nu} = - 
\tetr{a}{\nu}\D{\nu}\itetr{\mu}{b}$ and the definition of 4-velocity \eqref{eqn.4velocity}, the 
latter equation can be rewritten as
\begin{equation}
	u^\mu\D{\mu}\tetr{a}{\nu} + \tetr{a}{\mu}\D{\nu}u^\mu = U_{\eT{a}{\nu}},
\end{equation}


\subsection{Summary}

Finally, the resulting system of first-order equations is
\begin{subequations}
	\begin{empheq}[box={\mycreambox[2pt][2pt]}]{align}
		\D{\mu}(u^\mu\eT{a}{\nu} - u^\nu\eT{a}{\mu} + \veps^{\mu\nu\lambda\rho}u_\lambda 
		U_{\hT{a}{\rho}})
		& = {\color{Red}\bm{-}}U_{\tetr{a}{\mu}},\\[2mm]
		\D{\mu}(u^\mu \hT{a}{\nu} - u^\nu\hT{a}{\mu} - 
		\veps^{\mu\nu\lambda\rho}u_\lambda 
		U_{\eT{a}{\rho}})
		& = 0,\\[2mm]
		\D{\mu} (\tetr{a}{\nu} U_{\tetr{a}{\mu}} - U \delta^\mu_{\ \nu}) - U_{\tetr{a}{\mu}} 
		\Tors{a}{\mu\nu} & = 0,\\[2mm]
		u^\mu\D{\mu}\tetr{a}{\nu} + \tetr{a}{\mu}\D{\nu}u^\mu & = U_{\eT{a}{\nu}},
	\end{empheq}
\end{subequations}
which has to be solved for the unknowns
\begin{equation}
(\tetr{a}{\mu},\eT{a}{\mu},\hT{a}{\mu}).
\end{equation}

{\color{Red} Interestingly, the energy-momentum PDE is exactly like we use for the symmetrization 
in the non-relativistic settings, i.e. when we add the torsion multiplied by $ U_{\tetr{a}{\mu}} $ 
to the momentum equation. However, when we usually du the summation, we don't need this 
non-conservative terms.... hm-m....}

\section{Reference tetrad}

Discuss here the choice of the reference tetrad $ \rtetr{a}{\mu} $ which is used to define the spin 
connection and 
subsequently, to define the gauge covariant derivative $ \D{\mu} $...


\section{Closure}



In the standard \tegr\ the Lagrangian density is %(e.g. \cite{Krssak2017})
\begin{equation}
\Lambda = \frac{a}{4\kappa} \Tors{a}{\mu\nu}\Supp{a}{\mu\nu},
\end{equation}
where $ \kappa = 8 \pi G $, $ a = \det(\tetr{a}{\mu}) $, and $ \Supp{a}{\mu\nu} $ is the so-called 
superpotential and is given by
\begin{equation}
\Supp{a}{\mu\nu} = -\Supp{a}{\nu\mu} = K^{\mu\nu}_{\phantom{\mu\nu}a} - \itetr{\nu}{a} T^\mu + 
\itetr{\mu}{a} 
T^\nu, \qquad T^\mu = T^{\lambda\mu}_{\phantom{\lambda\mu}\lambda}.
\end{equation}



\printbibliography

\end{document}
