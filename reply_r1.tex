\documentclass[3p,times]{article}
\usepackage{geometry}
\geometry{
 a4paper,
 total={160mm,257mm},
 left=30mm,
 top=25mm,
 bottom = 30mm,
 }
\usepackage{physics}
\usepackage{amsmath}
\usepackage{amsthm}
\usepackage{graphicx}
\usepackage{caption}
\usepackage{amsfonts}
\usepackage{bm}
\usepackage{mathbbol}
\usepackage{dsfont}
\usepackage{subcaption}
\usepackage[dvipsnames]{xcolor}
\usepackage{accents}

\newcommand{\HDT}[1]{\accentset{\star}{T}^{#1}}
\newcommand{\reply}[1]{\hfill \break \noindent {\color{blue}#1}}


\title{Reply to Reviewer \#1}

\begin{document}
\maketitle

\noindent
In this paper the authors develops the 3+1 decomposition of the equations of
motion of the teleparallel equivalent of general relativity as a 1st-order
partial differential system of equations. The system of equations is written in
terms of the tetrad and the Hodge dual to the torsion $\HDT{a\mu\nu}$. The
authors claim to have obtained an equivalent system of equations to the tetrad
reformulation of general relativity by Estabrook, Robinson, Wahlquist (ERW), and
Buchman and Bardeen (BB). The authors formulate in detail the first-order
extension from the second-order system that the equations of motion of
teleparallel gravity naturally appear. They also clearly specify the gauge
conditions (or algebraic constraints) in their treatment, and a valuable
discussion on the 3+1 decomposition including matter. In addition, they have
drawn analogies of the form of the equations of motion with electrodynamics. The
paper is a valuable contribution to the field and could be accepted for
publication, but the referee has detected a couple of issues throughout the text
that should be addressed before publication.

\reply{We would like to thank the referee for the careful reading of our manuscript and the constructive feedback provided.}
\\

\noindent
\textbf{(a)} One clear criticism of this manuscript is that they mention in the abstract
that they derive a 3+1 decomposition for the evolution equations for f(T)
gravity, when in reality they have only obtained those of TEGR. This is
misleading, and any reference to the 3+1 decomposition of the evolution
equations of f(T) gravity should be removed, unless the authors are willing to
provide some tentative procedure on how their results could be generalized to
this case. For example, the representation of f(T) gravity in the Jordan frame
could be useful for this purpose. If doing so, the authors should also be able
to explain whether the system has a well-posed initial value problem for all
possible field configurations. This is in relation to the well-known problem of
strong coupling in f(T) gravity, or in other words, the branching of the number
of degrees of freedom that has been identified through Dirac-Bergmann procedure
for constrained Hamiltonian systems. Some references relevant to this topic, and
to the usage of the Jordan frame in f(T) gravity, are

\begin{itemize}
    \item 
    Li, Miao, Miao (2011) https://arxiv.org/abs/1105.5934
    \item 
    Ferraro, Guzman (2018) https://arxiv.org/abs/1802.02130
   \item 
    Ferraro, Guzman (2018) https://arxiv.org/abs/1810.07171
   \item 
    Blagojevic, Nester (2020) https://arxiv.org/abs/2006.15303
   \item 
    Bajardi, Blixt, Capozziello (2025) https://arxiv.org/abs/2412.20592
\end{itemize}

\reply{We agree with the referee that the statement about the $f(T)$ gravity was misleading. We only intended to write that some elements of our technique could potentially be adapted to the $f(T)$ case. In the revised manuscript, we tried to make it clear that the $f(T)$ gravity is beyond the scope of the present work, e.g. see changes in the abstract, Section\,3, and conclusion.}

\reply{We appreciate the suggestions for relevant references regarding the $f(T)$ gravity and Hamiltonian formulations of teleparallel gravity. They helped us to clarify the differences between the existing 3+1 Hamiltonian formulations and our technique, see the changes in the introduction. We have added the references to the revised manuscript in relevant places, Section\,3 and conclusion. Apparently, $f(T)$ gravity will require some modifications to our approach, and we will address this in future work.}

\noindent
\textbf{(b)} The authors mention that they have obtained the same set of equations than
those obtained by ERW and BB, which can be expected if they have done the 3+1
splitting in the equations of motion. This is an important check, but in the
manuscript it is not explained in full detail whether the result is identical or
there are some redefinitions to be done in the variables in order to obtain the
equivalence. This is crucial, since if the result is identical to previous work,
then the numerical implementation might turn out trivial, since it has already
been developed in those references. Some equivalences have been established
between the ERWBB formalism and theirs, in particular in Eq.(94), but they are
probably not enough to convert from one set of equations to another.

\reply{...}

\noindent
\textbf{(c)} it is mentioned that this equivalence occurs in vacuum. I understand that
they have not derived the 3+1 equations for matter, but wouldn't it be expected
that they are equivalent too, provided that the decomposition of the matter
sector is identical?. I recommend the authors to elaborate in this point too.

\reply{...}

\noindent
Therefore, I consider the authors should address these points in a revised
version of their manuscript before proceeding with publication.

\end{document}